%% Generated by Sphinx.
\def\sphinxdocclass{report}
\IfFileExists{luatex85.sty}
 {\RequirePackage{luatex85}}
 {\ifdefined\luatexversion\ifnum\luatexversion>84\relax
  \PackageError{sphinx}
  {** With this LuaTeX (\the\luatexversion),Sphinx requires luatex85.sty **}
  {** Add the LaTeX package luatex85 to your TeX installation, and try again **}
  \endinput\fi\fi}
\documentclass[letterpaper,11pt,english]{sphinxmanual}
\ifdefined\pdfpxdimen
   \let\sphinxpxdimen\pdfpxdimen\else\newdimen\sphinxpxdimen
\fi \sphinxpxdimen=.75bp\relax
\ifdefined\pdfimageresolution
    \pdfimageresolution= \numexpr \dimexpr1in\relax/\sphinxpxdimen\relax
\fi
%% let collapsible pdf bookmarks panel have high depth per default
\PassOptionsToPackage{bookmarksdepth=5}{hyperref}
%% turn off hyperref patch of \index as sphinx.xdy xindy module takes care of
%% suitable \hyperpage mark-up, working around hyperref-xindy incompatibility
\PassOptionsToPackage{hyperindex=false}{hyperref}
%% memoir class requires extra handling
\makeatletter\@ifclassloaded{memoir}
{\ifdefined\memhyperindexfalse\memhyperindexfalse\fi}{}\makeatother

\PassOptionsToPackage{warn}{textcomp}

\catcode`^^^^00a0\active\protected\def^^^^00a0{\leavevmode\nobreak\ }
\usepackage{cmap}
\usepackage{fontspec}
\defaultfontfeatures[\rmfamily,\sffamily,\ttfamily]{}
\usepackage{amsmath,amssymb,amstext}
\usepackage{polyglossia}
\setmainlanguage{english}



\setmainfont{FreeSerif}[
  Extension      = .otf,
  UprightFont    = *,
  ItalicFont     = *Italic,
  BoldFont       = *Bold,
  BoldItalicFont = *BoldItalic
]
\setsansfont{FreeSans}[
  Extension      = .otf,
  UprightFont    = *,
  ItalicFont     = *Oblique,
  BoldFont       = *Bold,
  BoldItalicFont = *BoldOblique,
]
\setmonofont{FreeMono}[
  Extension      = .otf,
  UprightFont    = *,
  ItalicFont     = *Oblique,
  BoldFont       = *Bold,
  BoldItalicFont = *BoldOblique,
]



\usepackage[Bjarne]{fncychap}
\usepackage[,numfigreset=1,mathnumfig]{sphinx}

\fvset{fontsize=\small}
\usepackage{geometry}


% Include hyperref last.
\usepackage{hyperref}
% Fix anchor placement for figures with captions.
\usepackage{hypcap}% it must be loaded after hyperref.
% Set up styles of URL: it should be placed after hyperref.
\urlstyle{same}

\addto\captionsenglish{\renewcommand{\contentsname}{User Manual}}

\usepackage{sphinxmessages}
\setcounter{tocdepth}{2}



\title{OpihiExarata}
\date{Jun 23, 2022}
\release{2022.6.23}
\author{Kenji Sparrow Emerson}
\newcommand{\sphinxlogo}{\vbox{}}
\renewcommand{\releasename}{Release}
\usepackage[columns=1]{idxlayout}\makeindex
\begin{document}

\pagestyle{empty}
\sphinxmaketitle
\pagestyle{plain}
\sphinxtableofcontents
\pagestyle{normal}
\phantomsection\label{\detokenize{index::doc}}


\sphinxAtStartPar
OpihiExarata is a group of individual software packages, API services, and
executables designed to process data from the IRTF Opihi telescope. It serves
to be an astrometric solver, photometric calibrator, asteroid finder,
ephemeris computer, and astrometric monitor. We accomplish this by integrating
services and open source software collections for our purposes.

\sphinxAtStartPar
This software is specific to the IRTF Opihi telescope.

\begin{sphinxadmonition}{note}{Note:}
\sphinxAtStartPar
This documentation comes in two forms, a set of HTML webpages and a LaTeX
document. The HTML is given preference when formatting issues arise between
the two. We highly suggest that you read the
\sphinxhref{https://psmd-iberutaru.github.io/OpihiExarata}{HTML version}%
\begin{footnote}[1]\sphinxAtStartFootnote
\sphinxnolinkurl{https://psmd-iberutaru.github.io/OpihiExarata}
%
\end{footnote} as it is
better formatted, easier to navigate, and more likely to be up to date.
\end{sphinxadmonition}


\chapter{User Manual}
\label{\detokenize{index:user-manual}}\label{\detokenize{index:home-user-manual}}
\sphinxAtStartPar
The user manual is primarily for general users, such as telescope operators and
observing astronomers. It details how to use OpihiExarata in conjunction with
the Opihi telescope, primarily covering the usage of its GUI and command\sphinxhyphen{}line
interfaces. It can be accessed via the sidebar or by clicking
{\hyperref[\detokenize{user/index::doc}]{\sphinxcrossref{\DUrole{doc}{here}}}}.

\sphinxstepscope


\section{User Manual}
\label{\detokenize{user/index:user-manual}}\label{\detokenize{user/index:user-index}}\label{\detokenize{user/index::doc}}
\sphinxAtStartPar
This is the user manual portion of the OpihiExarata software package for
reducing and analyzing Opihi data.

\sphinxAtStartPar
The Opihi telescope is a smaller telescope mounted on the side of the IRTF. It
primarily provides asteroid view\sphinxhyphen{}finding and photometric calibration services.

\sphinxAtStartPar
It was conceived to assist the IRTF telescope and its instruments in finding
asteroids and other near Earth objects on the sky when the positional
uncertainty in the ephemeris of these objects are greater than the field of
view of the IRTF’s current acquisitional and tracking instruments (about 1
arcminuite). This can often be the case for newly discovered objects with
a small number of observations. With Opihi’s 32 arcminuite field of view, these
objects can be spotted and the telescope pointing can be corrected, thus
greatly reducing the overhead time of finding the target.

\sphinxAtStartPar
However, only a minority of time allocated by the IRTF are for these types of
targets. Having Opihi only operate a few times a semester where it is needed
for acquisition is inefficient. In addition to the asteroid view\sphinxhyphen{}finding
capabilities, the Opihi telescope also serves as a source for photometric
monitoring and calibration. By continuously taking pictures and determining the
photometric solution for each, the photometric conditions (as measured by
the zero point magnitude) can be monitored over time. Moreover, this
photometric data can also be used to assist with photometrically calibrating
science targets observed by other IRTF facility instruments (e.g. SpeX, MORIS,
SPECTRE, etc).

\sphinxAtStartPar
A more detailed description of the physical and optical specifications of the
Opihi telescope can be found in {\hyperref[\detokenize{user/opihi_telescope:user-opihi-telescope}]{\sphinxcrossref{\DUrole{std,std-ref}{Opihi Telescope}}}}.

\sphinxAtStartPar
It is typical for the asteroid view\sphinxhyphen{}finding to be done manually by either a
telescope operator or an observing astronomer; for this reason, we may refer
to this view\sphinxhyphen{}finding mode as the {\hyperref[\detokenize{user/manual_mode:user-manual-mode}]{\sphinxcrossref{\DUrole{std,std-ref}{Manual Mode}}}} of operation for
Opihi and OpihiExarata.

\sphinxAtStartPar
The photometric monitoring mode does not require the constant
input from a user, it only requires instructing the Opihi camera to
continuously take images and for the software (OpihiExarata) to continuously
solve for their respective photometric solutions; for this reason, we may
refer to this photometric monitoring mode as the {\hyperref[\detokenize{user/automatic_mode:user-automatic-mode}]{\sphinxcrossref{\DUrole{std,std-ref}{Automatic Mode}}}}
of operation for Opihi and OpihiExarata.

\sphinxAtStartPar
Photometric monitoring may be done manually if the user desires more control
over the process. However, asteroid view\sphinxhyphen{}finding cannot be done automatically.
It is currently beyond the scope of this software to implement automatic
asteroid/transient finding.

\sphinxAtStartPar
An astronomer or other user interacts with the Opihi telescope and the
OpihiExarata software by connecting to a VNC session; \sphinxhref{http://irtfweb.ifa.hawaii.edu/observing/computer/vnc.php}{this is common to all
IRTF instruments}%
\begin{footnote}[2]\sphinxAtStartFootnote
\sphinxnolinkurl{http://irtfweb.ifa.hawaii.edu/observing/computer/vnc.php}
%
\end{footnote}.
The Opihi camera controller has an interface and data viewer independent of
the interfaces of OpihiExarata.

\sphinxAtStartPar
For the general user, the OpihiExarata presents itself with helpful graphical
user interfaces (GUIs). However, a command\sphinxhyphen{}line interface is provided to
open these GUIs. It is likely that most users do not need to worry about this
interface as the GUI may already be open for them in the VNC session, however,
the command\sphinxhyphen{}line interface and its usage is documented in
{\hyperref[\detokenize{user/command_line:user-command-line}]{\sphinxcrossref{\DUrole{std,std-ref}{Command Line}}}}.

\sphinxAtStartPar
See {\hyperref[\detokenize{user/manual_mode:user-manual-mode}]{\sphinxcrossref{\DUrole{std,std-ref}{Manual Mode}}}} or {\hyperref[\detokenize{user/automatic_mode:user-automatic-mode}]{\sphinxcrossref{\DUrole{std,std-ref}{Automatic Mode}}}} for usage instructions
depending on the desired mode of operation. Both of these have some technical
jargon better described in {\hyperref[\detokenize{user/system_framework:user-system-framework}]{\sphinxcrossref{\DUrole{std,std-ref}{System Framework}}}}. See
{\hyperref[\detokenize{user/configuration:user-configuration}]{\sphinxcrossref{\DUrole{std,std-ref}{Configuration}}}} for available configuration options for both modes.
If you are having trouble, see {\hyperref[\detokenize{user/troubleshooting:user-troubleshooting}]{\sphinxcrossref{\DUrole{std,std-ref}{Troubleshooting}}}} for information and
possible solutions. If you believe you found an issue or bug with the software,
please report it to the appropriate IRTF staff member(s).

\sphinxAtStartPar
If you used the Opihi telescope or the OpihiExarata software, please
acknowledge your usage in any projects or publications; see
{\hyperref[\detokenize{user/citations:user-citations}]{\sphinxcrossref{\DUrole{std,std-ref}{Citations}}}} for assistance and for our own references.

\sphinxstepscope


\subsection{Opihi Telescope}
\label{\detokenize{user/opihi_telescope:opihi-telescope}}\label{\detokenize{user/opihi_telescope:user-opihi-telescope}}\label{\detokenize{user/opihi_telescope::doc}}
\begin{figure}[htbp]
\centering
\capstart

\noindent\sphinxincludegraphics{{opihi-telescope-no-shield}.png}
\caption{An image of the Opihi telescope during the commissioning process. An
aluminum weather proof shield has been since been added which covers the
back half of the system to cover the electronics.}\label{\detokenize{user/opihi_telescope:id1}}\label{\detokenize{user/opihi_telescope:figure-opihi-telescope-no-shield}}\end{figure}

\sphinxAtStartPar
We detail the physical specifications of the Opihi telescope itself. As the
{\hyperref[\detokenize{technical/index:technical-index}]{\sphinxcrossref{\DUrole{std,std-ref}{Technical Manual}}}} is more for detail about the software, and the
actual telescope’s physical quantities may be useful for users, we leave this
section here in the user manual.

\sphinxAtStartPar
See \hyperref[\detokenize{user/opihi_telescope:figure-opihi-telescope-no-shield}]{Fig.\@ \ref{\detokenize{user/opihi_telescope:figure-opihi-telescope-no-shield}}} for an image of the Opihi
telescope itself.

\sphinxAtStartPar
The Opihi telescope:
\begin{itemize}
\item {} 
\sphinxAtStartPar
Is a 0.43 meter corrected Dall\sphinxhyphen{}Kirkham telescope from PlaneWave Instruments.

\item {} 
\sphinxAtStartPar
It has a field of view of 32 arcminuites across.

\item {} 
\sphinxAtStartPar
The sensitivity is approximately 20 magnitudes in a one minute exposure.

\item {} 
\sphinxAtStartPar
Has a pixel scale of 0.94 arcseconds per pixel.

\item {} 
\sphinxAtStartPar
The filter wheel contains four SDSS filters (g’, r’, i’, z’), a clear position, and three (currently unused) slots for other filters.

\end{itemize}

\sphinxAtStartPar
The camera/detector on the Opihi telescope:
\begin{itemize}
\item {} 
\sphinxAtStartPar
Is a 2k x 2k Andor iKon\sphinxhyphen{}L 936 CCD Camera.

\item {} 
\sphinxAtStartPar
The operating temperature is \sphinxhyphen{}50 Celsius (223.15 K).

\item {} 
\sphinxAtStartPar
The dark current is about 1 electron/s.

\item {} 
\sphinxAtStartPar
The read noise is about 10.3 electron/s.

\item {} 
\sphinxAtStartPar
The well depth is about 150 000 electrons.

\end{itemize}

\begin{sphinxadmonition}{note}{Note:}
\sphinxAtStartPar
Most, if not all, of the information provided on this page/section is
taken from the Opihi telescope and OpihiExarata SPIE conference paper (see
{\hyperref[\detokenize{user/citations:user-citations}]{\sphinxcrossref{\DUrole{std,std-ref}{Citations}}}}) or the \sphinxhref{http://irtfweb.ifa.hawaii.edu/~opihi/}{IRTF Opihi website}%
\begin{footnote}[3]\sphinxAtStartFootnote
\sphinxnolinkurl{http://irtfweb.ifa.hawaii.edu/~opihi/}
%
\end{footnote}. You may find more
information from these resources.
\end{sphinxadmonition}

\sphinxstepscope


\subsection{System Framework}
\label{\detokenize{user/system_framework:system-framework}}\label{\detokenize{user/system_framework:user-system-framework}}\label{\detokenize{user/system_framework::doc}}
\sphinxAtStartPar
We briefly cover a few terms and processes necessary for usage of this software.
This does not go into more detail than is needed by a standard user; for more
information, see the {\hyperref[\detokenize{technical/index:technical-index}]{\sphinxcrossref{\DUrole{std,std-ref}{Technical Manual}}}}.


\subsubsection{Engines}
\label{\detokenize{user/system_framework:engines}}
\sphinxAtStartPar
For processing of Opihi data, there are four problems which the OpihiExarata
does not solve on its own (or does not have the data to do so). They are:
\begin{itemize}
\item {} 
\sphinxAtStartPar
Astrometric plate solving of an image.

\item {} 
\sphinxAtStartPar
Photometeric calibration.

\item {} 
\sphinxAtStartPar
Preliminary orbit determination.

\item {} 
\sphinxAtStartPar
Ephemeris computations.

\end{itemize}

\sphinxAtStartPar
Additionally, there is another problem which is implemented similarly as if
OpihiExarata could not solve it on its own:
\begin{itemize}
\item {} 
\sphinxAtStartPar
On\sphinxhyphen{}sky target propagation.

\end{itemize}

\sphinxAtStartPar
These five problems are solved by sending relevant data to other services,
utilizing their APIs to compute a solution. (See our {\hyperref[\detokenize{user/citations:user-citations}]{\sphinxcrossref{\DUrole{std,std-ref}{Citations}}}} for
more references.)

\sphinxAtStartPar
Each of these services are made by different organizations. We access the
capabilities of theses service by what is called an “engine”. Each service has
their own custom implemented engine. Selecting a given engine (as you use this
program) means that your data will be processed by the service the engine
corresponds to.

\sphinxAtStartPar
You can find out about the different available services (and thus engines) in
{\hyperref[\detokenize{technical/architecture/services_engines:technical-architecture-services-engines}]{\sphinxcrossref{\DUrole{std,std-ref}{Services and Engines}}}}.


\paragraph{Vehicles and Solutions}
\label{\detokenize{user/system_framework:vehicles-and-solutions}}
\sphinxAtStartPar
There may be multiple engines for solving one of the five problems. Different
datasets may be predisposed to work better with one engine than another. As an
implementation detail, we also built wrappers around the engines called
vehicle functions and solution classes for ease of use. More detail about these
wrappers can be found in {\hyperref[\detokenize{technical/architecture/vehicles_solutions:technical-architecture-vehicles-solutions}]{\sphinxcrossref{\DUrole{std,std-ref}{Vehicles and Solutions}}}}, but
it likely is particularly of little concern to the average user.

\sphinxAtStartPar
Conceptually, you may think of the engine as synonymous with the service; the
rest are just unnecessary details.

\sphinxstepscope


\subsection{Command Line}
\label{\detokenize{user/command_line:command-line}}\label{\detokenize{user/command_line:user-command-line}}\label{\detokenize{user/command_line::doc}}
\sphinxAtStartPar
The OpihiExarata software system should already be installed on the computer
Opihi operates off of. (If that is not the case, consult
{\hyperref[\detokenize{technical/installation/index:technical-installation}]{\sphinxcrossref{\DUrole{std,std-ref}{Installation}}}} for more information or contact a staff member.)
To test that OpihiExarata is installed on the system, you can run the
following in a terminal:

\begin{sphinxVerbatim}[commandchars=\\\{\}]
\PYG{n}{opihiexarata}
\end{sphinxVerbatim}

\sphinxAtStartPar
This should print a help menu of sorts.

\sphinxAtStartPar
The general overview of the command\sphinxhyphen{}line syntax of this command is:

\begin{sphinxVerbatim}[commandchars=\\\{\}]
\PYG{n}{opihiexarata} \PYG{p}{[}\PYG{n}{action}\PYG{p}{]} \PYG{p}{[}\PYG{n}{options}\PYG{p}{]}
\end{sphinxVerbatim}

\sphinxAtStartPar
Where {\hyperref[\detokenize{user/command_line:cmdoption-arg-action}]{\sphinxcrossref{\sphinxcode{\sphinxupquote{{[}action{]}}}}}} is the specified actions to take. There are many
actions which the command\sphinxhyphen{}line interface may execute. The currently available
actions are detailed in {\hyperref[\detokenize{user/command_line:user-command-line-available-actions}]{\sphinxcrossref{\DUrole{std,std-ref}{Available Actions}}}}.

\sphinxAtStartPar
Different command\sphinxhyphen{}line options, {\hyperref[\detokenize{user/command_line:cmdoption-arg-options}]{\sphinxcrossref{\sphinxcode{\sphinxupquote{{[}options{]}}}}}}, are detailed in {\hyperref[\detokenize{user/command_line:user-command-line-available-options}]{\sphinxcrossref{\DUrole{std,std-ref}{Available Options}}}}. The options detailed are all
optional. The only mandatory input is the required action to take.


\subsubsection{Available Actions}
\label{\detokenize{user/command_line:available-actions}}\label{\detokenize{user/command_line:user-command-line-available-actions}}\index{command line option@\spxentry{command line option}!{[}action{]}@\spxentry{{[}action{]}}}\index{{[}action{]}@\spxentry{{[}action{]}}!command line option@\spxentry{command line option}}

\begin{savenotes}\begin{fulllineitems}
\phantomsection\label{\detokenize{user/command_line:cmdoption-arg-action}}
\pysigstartsignatures
\pysigline{\sphinxbfcode{\sphinxupquote{{[}action{]}}}\sphinxcode{\sphinxupquote{}}}
\pysigstopsignatures
\end{fulllineitems}\end{savenotes}


\sphinxAtStartPar
Here we list the available actions for the command\sphinxhyphen{}line interface, ordered by
(approximately) the order of importance each command is to the average user.

\sphinxAtStartPar
Note that there are aliases for different actions for ease of use, they
are all listed here.


\paragraph{Manual}
\label{\detokenize{user/command_line:manual}}\label{\detokenize{user/command_line:user-command-line-available-actions-manual}}\index{command line option@\spxentry{command line option}!manual@\spxentry{manual}}\index{manual@\spxentry{manual}!command line option@\spxentry{command line option}}\index{command line option@\spxentry{command line option}!m@\spxentry{m}}\index{m@\spxentry{m}!command line option@\spxentry{command line option}}

\begin{savenotes}\begin{fulllineitems}
\phantomsection\label{\detokenize{user/command_line:cmdoption-arg-manual}}
\pysigstartsignatures
\pysigline{\sphinxbfcode{\sphinxupquote{manual}}\sphinxcode{\sphinxupquote{}}\sphinxcode{\sphinxupquote{,~}}\sphinxbfcode{\sphinxupquote{m}}\sphinxcode{\sphinxupquote{}}}
\pysigstopsignatures
\end{fulllineitems}\end{savenotes}


\sphinxAtStartPar
This opens up the manual mode GUI. This is used when the user wants to
utilize the manual (asteroid view\sphinxhyphen{}finding) of Opihi and OpihiExarata. Or,
alternatively, the user wants to manually operate the photometric monitoring
mode.


\paragraph{Automatic}
\label{\detokenize{user/command_line:automatic}}\label{\detokenize{user/command_line:user-command-line-available-actions-automatic}}\index{command line option@\spxentry{command line option}!automatic@\spxentry{automatic}}\index{automatic@\spxentry{automatic}!command line option@\spxentry{command line option}}\index{command line option@\spxentry{command line option}!auto@\spxentry{auto}}\index{auto@\spxentry{auto}!command line option@\spxentry{command line option}}\index{command line option@\spxentry{command line option}!a@\spxentry{a}}\index{a@\spxentry{a}!command line option@\spxentry{command line option}}

\begin{savenotes}\begin{fulllineitems}
\phantomsection\label{\detokenize{user/command_line:cmdoption-arg-automatic}}
\pysigstartsignatures
\pysigline{\sphinxbfcode{\sphinxupquote{automatic}}\sphinxcode{\sphinxupquote{}}\sphinxcode{\sphinxupquote{,~}}\sphinxbfcode{\sphinxupquote{auto}}\sphinxcode{\sphinxupquote{}}\sphinxcode{\sphinxupquote{,~}}\sphinxbfcode{\sphinxupquote{a}}\sphinxcode{\sphinxupquote{}}}
\pysigstopsignatures
\end{fulllineitems}\end{savenotes}


\sphinxAtStartPar
This opens up the automatic mode GUI. This is used when the user wants to
utilize the automatic (photometric monitoring) mode of Opihi and OpihiExarata.
Typically, this user would be a telescope operator.


\paragraph{Generate}
\label{\detokenize{user/command_line:generate}}\label{\detokenize{user/command_line:user-command-line-available-actions-generate}}\index{command line option@\spxentry{command line option}!generate@\spxentry{generate}}\index{generate@\spxentry{generate}!command line option@\spxentry{command line option}}\index{command line option@\spxentry{command line option}!g@\spxentry{g}}\index{g@\spxentry{g}!command line option@\spxentry{command line option}}

\begin{savenotes}\begin{fulllineitems}
\phantomsection\label{\detokenize{user/command_line:cmdoption-arg-generate}}
\pysigstartsignatures
\pysigline{\sphinxbfcode{\sphinxupquote{generate}}\sphinxcode{\sphinxupquote{}}\sphinxcode{\sphinxupquote{,~}}\sphinxbfcode{\sphinxupquote{g}}\sphinxcode{\sphinxupquote{}}}
\pysigstopsignatures
\end{fulllineitems}\end{savenotes}


\sphinxAtStartPar
This generates configuration files to allow a user to edit and customize the
functionality of OpihiExarata. There are two different configuration files.
Which type of configuration file, and where it is saved are determined by the
specification of the optional parameters
For more information about configuration, see {\hyperref[\detokenize{user/configuration:user-configuration}]{\sphinxcrossref{\DUrole{std,std-ref}{Configuration}}}}.


\paragraph{Help}
\label{\detokenize{user/command_line:help}}\label{\detokenize{user/command_line:user-command-line-available-actions-help}}\index{command line option@\spxentry{command line option}!help@\spxentry{help}}\index{help@\spxentry{help}!command line option@\spxentry{command line option}}\index{command line option@\spxentry{command line option}!h@\spxentry{h}}\index{h@\spxentry{h}!command line option@\spxentry{command line option}}

\begin{savenotes}\begin{fulllineitems}
\phantomsection\label{\detokenize{user/command_line:cmdoption-arg-help}}
\pysigstartsignatures
\pysigline{\sphinxbfcode{\sphinxupquote{help}}\sphinxcode{\sphinxupquote{}}\sphinxcode{\sphinxupquote{,~}}\sphinxbfcode{\sphinxupquote{h}}\sphinxcode{\sphinxupquote{}}}
\pysigstopsignatures
\end{fulllineitems}\end{savenotes}


\sphinxAtStartPar
This displays the help dialog in the terminal. It is identical to the
{\hyperref[\detokenize{user/command_line:cmdoption-help}]{\sphinxcrossref{\sphinxcode{\sphinxupquote{\sphinxhyphen{}\sphinxhyphen{}help}}}}} option or invoking \sphinxstyleliteralstrong{\sphinxupquote{opihiexarata}} without any
specified action.

\begin{sphinxadmonition}{warning}{Warning:}
\sphinxAtStartPar
If any action is specified that does not match any of the expected actions,
the program will raise a Python exception as opposed to ignoring it or
strictly a shell error.
\end{sphinxadmonition}


\subsubsection{Available Options}
\label{\detokenize{user/command_line:available-options}}\label{\detokenize{user/command_line:user-command-line-available-options}}\index{command line option@\spxentry{command line option}!{[}options{]}@\spxentry{{[}options{]}}}\index{{[}options{]}@\spxentry{{[}options{]}}!command line option@\spxentry{command line option}}

\begin{savenotes}\begin{fulllineitems}
\phantomsection\label{\detokenize{user/command_line:cmdoption-arg-options}}
\pysigstartsignatures
\pysigline{\sphinxbfcode{\sphinxupquote{{[}options{]}}}\sphinxcode{\sphinxupquote{}}}
\pysigstopsignatures
\end{fulllineitems}\end{savenotes}


\sphinxAtStartPar
Here we list the available actions for the command\sphinxhyphen{}line interface, ordered by
(approximately) the order of importance each command is to the average user.

\sphinxAtStartPar
Note that there are aliases for different actions for ease of use, they
are not listed here. Instead, the overall preferred syntax and formatting
is provided. To see the aliases, see the help screen for more information:
{\hyperref[\detokenize{user/command_line:user-command-line-available-actions-help}]{\sphinxcrossref{\DUrole{std,std-ref}{Help}}}}.


\paragraph{Help}
\label{\detokenize{user/command_line:id1}}\index{command line option@\spxentry{command line option}!\sphinxhyphen{}\sphinxhyphen{}help@\spxentry{\sphinxhyphen{}\sphinxhyphen{}help}}\index{\sphinxhyphen{}\sphinxhyphen{}help@\spxentry{\sphinxhyphen{}\sphinxhyphen{}help}!command line option@\spxentry{command line option}}

\begin{savenotes}\begin{fulllineitems}
\phantomsection\label{\detokenize{user/command_line:cmdoption-help}}
\pysigstartsignatures
\pysigline{\sphinxbfcode{\sphinxupquote{\sphinxhyphen{}\sphinxhyphen{}help}}\sphinxcode{\sphinxupquote{}}}
\pysigstopsignatures
\end{fulllineitems}\end{savenotes}


\sphinxAtStartPar
This overrides any specified actions and executes the help dialog. See
{\hyperref[\detokenize{user/command_line:user-command-line-available-actions-help}]{\sphinxcrossref{\DUrole{std,std-ref}{Help}}}}.


\paragraph{Manual}
\label{\detokenize{user/command_line:id2}}\index{command line option@\spxentry{command line option}!\sphinxhyphen{}\sphinxhyphen{}manual@\spxentry{\sphinxhyphen{}\sphinxhyphen{}manual}}\index{\sphinxhyphen{}\sphinxhyphen{}manual@\spxentry{\sphinxhyphen{}\sphinxhyphen{}manual}!command line option@\spxentry{command line option}}

\begin{savenotes}\begin{fulllineitems}
\phantomsection\label{\detokenize{user/command_line:cmdoption-manual}}
\pysigstartsignatures
\pysigline{\sphinxbfcode{\sphinxupquote{\sphinxhyphen{}\sphinxhyphen{}manual}}\sphinxcode{\sphinxupquote{}}}
\pysigstopsignatures
\end{fulllineitems}\end{savenotes}


\sphinxAtStartPar
This opens up the manual mode GUI regardless of the action specified. See {\hyperref[\detokenize{user/command_line:user-command-line-available-actions-manual}]{\sphinxcrossref{\DUrole{std,std-ref}{Manual}}}}.


\paragraph{Automatic}
\label{\detokenize{user/command_line:id3}}\index{command line option@\spxentry{command line option}!\sphinxhyphen{}\sphinxhyphen{}automatic@\spxentry{\sphinxhyphen{}\sphinxhyphen{}automatic}}\index{\sphinxhyphen{}\sphinxhyphen{}automatic@\spxentry{\sphinxhyphen{}\sphinxhyphen{}automatic}!command line option@\spxentry{command line option}}

\begin{savenotes}\begin{fulllineitems}
\phantomsection\label{\detokenize{user/command_line:cmdoption-automatic}}
\pysigstartsignatures
\pysigline{\sphinxbfcode{\sphinxupquote{\sphinxhyphen{}\sphinxhyphen{}automatic}}\sphinxcode{\sphinxupquote{}}}
\pysigstopsignatures
\end{fulllineitems}\end{savenotes}


\sphinxAtStartPar
This opens up the automatic mode GUI regardless of the action specified. See {\hyperref[\detokenize{user/command_line:user-command-line-available-actions-automatic}]{\sphinxcrossref{\DUrole{std,std-ref}{Automatic}}}}.


\paragraph{Configuration}
\label{\detokenize{user/command_line:configuration}}\label{\detokenize{user/command_line:user-command-line-available-options-configuration}}\index{command line option@\spxentry{command line option}!\sphinxhyphen{}\sphinxhyphen{}config@\spxentry{\sphinxhyphen{}\sphinxhyphen{}config}}\index{\sphinxhyphen{}\sphinxhyphen{}config@\spxentry{\sphinxhyphen{}\sphinxhyphen{}config}!command line option@\spxentry{command line option}}

\begin{savenotes}\begin{fulllineitems}
\phantomsection\label{\detokenize{user/command_line:cmdoption-config}}
\pysigstartsignatures
\pysigline{\sphinxbfcode{\sphinxupquote{\sphinxhyphen{}\sphinxhyphen{}config}}\sphinxcode{\sphinxupquote{=<path/to/config.yaml>}}}
\pysigstopsignatures
\end{fulllineitems}\end{savenotes}


\sphinxAtStartPar
This specifies the path of the configuration file. The configuration file is
in a YAML format. If the action specified is {\hyperref[\detokenize{user/command_line:cmdoption-arg-generate}]{\sphinxcrossref{\sphinxcode{\sphinxupquote{generate}}}}}, then this is
the path where the generated configuration fill will be saved. Otherwise, the
program will read the configuration file at this path and use its values
instead of the program’s defaults, where they differ.

\sphinxAtStartPar
See {\hyperref[\detokenize{user/configuration:user-configuration-standard-configuration-file}]{\sphinxcrossref{\DUrole{std,std-ref}{Standard Configuration File}}}} for the
specifications of the configuration file.


\paragraph{Secrets}
\label{\detokenize{user/command_line:secrets}}\label{\detokenize{user/command_line:user-command-line-available-options-secrets}}\index{command line option@\spxentry{command line option}!\sphinxhyphen{}\sphinxhyphen{}secret@\spxentry{\sphinxhyphen{}\sphinxhyphen{}secret}}\index{\sphinxhyphen{}\sphinxhyphen{}secret@\spxentry{\sphinxhyphen{}\sphinxhyphen{}secret}!command line option@\spxentry{command line option}}

\begin{savenotes}\begin{fulllineitems}
\phantomsection\label{\detokenize{user/command_line:cmdoption-secret}}
\pysigstartsignatures
\pysigline{\sphinxbfcode{\sphinxupquote{\sphinxhyphen{}\sphinxhyphen{}secret}}\sphinxcode{\sphinxupquote{=<path/to/secret.yaml>}}}
\pysigstopsignatures
\end{fulllineitems}\end{savenotes}


\sphinxAtStartPar
This specifies the path of the secrets file. The secrets file is
in a YAML format. If the action specified is {\hyperref[\detokenize{user/command_line:cmdoption-arg-generate}]{\sphinxcrossref{\sphinxcode{\sphinxupquote{generate}}}}}, then this is
the path where the generated secrets fill will be saved. Otherwise, the
program will read the secrets file at this path and use its values
instead of the program’s defaults, where they differ.

\sphinxAtStartPar
See {\hyperref[\detokenize{user/configuration:user-configuration-secrets-configuration-file}]{\sphinxcrossref{\DUrole{std,std-ref}{Secrets Configuration File}}}} for the
specifications of the secrets file.


\paragraph{Overwrite}
\label{\detokenize{user/command_line:overwrite}}\index{command line option@\spxentry{command line option}!\sphinxhyphen{}\sphinxhyphen{}overwrite@\spxentry{\sphinxhyphen{}\sphinxhyphen{}overwrite}}\index{\sphinxhyphen{}\sphinxhyphen{}overwrite@\spxentry{\sphinxhyphen{}\sphinxhyphen{}overwrite}!command line option@\spxentry{command line option}}

\begin{savenotes}\begin{fulllineitems}
\phantomsection\label{\detokenize{user/command_line:cmdoption-overwrite}}
\pysigstartsignatures
\pysigline{\sphinxbfcode{\sphinxupquote{\sphinxhyphen{}\sphinxhyphen{}overwrite}}\sphinxcode{\sphinxupquote{}}}
\pysigstopsignatures
\end{fulllineitems}\end{savenotes}


\sphinxAtStartPar
This option allows for the specification of what to do when a provided
file at a path already exists. This is typically used when generating new
configuration files. When provided, any pre\sphinxhyphen{}existing files are overwritten.
The default, when this option is not provided, is to raise an error because a
file already exists at a given path.


\paragraph{Keep Temporary}
\label{\detokenize{user/command_line:keep-temporary}}\index{command line option@\spxentry{command line option}!\sphinxhyphen{}\sphinxhyphen{}keep\sphinxhyphen{}temporary@\spxentry{\sphinxhyphen{}\sphinxhyphen{}keep\sphinxhyphen{}temporary}}\index{\sphinxhyphen{}\sphinxhyphen{}keep\sphinxhyphen{}temporary@\spxentry{\sphinxhyphen{}\sphinxhyphen{}keep\sphinxhyphen{}temporary}!command line option@\spxentry{command line option}}

\begin{savenotes}\begin{fulllineitems}
\phantomsection\label{\detokenize{user/command_line:cmdoption-keep-temporary}}
\pysigstartsignatures
\pysigline{\sphinxbfcode{\sphinxupquote{\sphinxhyphen{}\sphinxhyphen{}keep\sphinxhyphen{}temporary}}\sphinxcode{\sphinxupquote{}}}
\pysigstopsignatures
\end{fulllineitems}\end{savenotes}


\sphinxAtStartPar
The normal operation of OpihiExarata requires the writing of temporary files.
A temporary directory is created (as specified by the configuration file) and
is then purged and deleted. Using this flag prevents the cleanup of the
temporary directory on the program’s exit. However, as the software itself often
cleans up before exiting, this option is not very useful for the end user. It
is more for debugging purposes.

\begin{sphinxadmonition}{warning}{Warning:}
\sphinxAtStartPar
If any option is specified that does not match any of the expected option,
the program will raise a shell error.
\end{sphinxadmonition}

\sphinxstepscope


\subsection{Manual Mode}
\label{\detokenize{user/manual_mode:manual-mode}}\label{\detokenize{user/manual_mode:user-manual-mode}}\label{\detokenize{user/manual_mode::doc}}
\sphinxAtStartPar
The manual mode of OpihiExarata is its asteroid view\sphinxhyphen{}finding mode. Although,
photometric monitoring can be done manually, and the manual mode is well
equipped for said use case, it is not the primary use case for the manual mode.

\sphinxAtStartPar
The user uses the Opihi instrument to take images of asteroids. They then
specify the location of the asteroid using a GUI (usually with help from
comparing it to a reference image). An astrometric solution can convert the
asteroid’s pixel location to an on\sphinxhyphen{}sky location, and using previous
observations, its path across the sky can be determined. This information is
used to properly point the IRTF telescope to the desired asteroid target.

\sphinxAtStartPar
We present the procedure for operating OpihiExarata in its manual mode,
please also reference the GUI figure provided for reference. We also summarize
the procedure and process of the manual mode via a flowchart.


\subsubsection{Graphical User Interface}
\label{\detokenize{user/manual_mode:graphical-user-interface}}\label{\detokenize{user/manual_mode:user-manual-mode-graphical-user-interface}}
\begin{figure}[htbp]
\centering
\capstart

\noindent\sphinxincludegraphics{{manual-mode-gui-all}.png}
\caption{The GUI of the manual mode of OpihiExarata. This is what you see when
the interface is freshly loaded. Most of the text is filler text, they will
change as the program is executed. Note that there are different tabs for
the summary, astrometry, photometry, orbit, ephemeris, and propagation.
Each of the sections covers their GUIs, this is just the general GUI.}\label{\detokenize{user/manual_mode:id2}}\label{\detokenize{user/manual_mode:figure-manual-mode-gui-all}}\end{figure}

\sphinxAtStartPar
The manual mode GUI contains a main data window and a few tabs
compartmentalizing a lot of the functionality of OpihiExarata, see
\hyperref[\detokenize{user/manual_mode:figure-manual-mode-gui-all}]{Fig.\@ \ref{\detokenize{user/manual_mode:figure-manual-mode-gui-all}}}.

\sphinxAtStartPar
In the manual mode general GUI, you have three buttons which control file
and target acquisition. The \sphinxguilabel{New Target} button specifies that you
desire to work on a new target (typically a new asteroid). The
\sphinxguilabel{Manual New Image} button opens a file dialog so that you can
select a new Opihi image FITS file to load into OpihiExarata. The
\sphinxguilabel{Automatic New Image} button pulls the most recent FITS image from
the same directory as the last image specified manually.

\sphinxAtStartPar
There is also a data viewer. Relevant information from the solutions will be
plotted here along with the image data. There is a navigation bar for
manipulating the image, including zooming, panning, saving, and configuring
other options. This functionality makes use of matplotlib, see
\sphinxhref{https://matplotlib.org/3.2.2/users/navigation\_toolbar.html}{Interactive navigation (outdated)}%
\begin{footnote}[4]\sphinxAtStartFootnote
\sphinxnolinkurl{https://matplotlib.org/3.2.2/users/navigation\_toolbar.html}
%
\end{footnote}
or \sphinxhref{https://matplotlib.org/stable/users/explain/interactive.html}{Interactive figures}%
\begin{footnote}[5]\sphinxAtStartFootnote
\sphinxnolinkurl{https://matplotlib.org/stable/users/explain/interactive.html}
%
\end{footnote}.

\sphinxAtStartPar
There is also a \sphinxguilabel{Refresh Window} button to redraw the image in the
viewer and to also refresh the information in the tabs. This should not be
needed too much as the program should automatically refresh itself if there
is new information.


\subsubsection{Procedure}
\label{\detokenize{user/manual_mode:procedure}}
\begin{figure}[htbp]
\centering
\capstart

\noindent\sphinxincludegraphics{{manual-mode-flowchart}.pdf}
\caption{A flowchart summary of the procedure of the manual mode. It includes
the actions of the user along with the program’s flow afterwards.}\label{\detokenize{user/manual_mode:id3}}\label{\detokenize{user/manual_mode:figure-manual-mode-flowchart}}\end{figure}

\sphinxAtStartPar
We describe the procedure for utilizing the asteroid view\sphinxhyphen{}finding (manual)
mode of OpihiExarata. See \hyperref[\detokenize{user/automatic_mode:figure-automatic-mode-flowchart}]{Fig.\@ \ref{\detokenize{user/automatic_mode:figure-automatic-mode-flowchart}}} for a
flowchart summary of this procedure.


\paragraph{Start and Open GUI}
\label{\detokenize{user/manual_mode:start-and-open-gui}}
\sphinxAtStartPar
You will want to open the OpihiExarata manual mode GUI, typically via the
command\sphinxhyphen{}line interface with:

\begin{sphinxVerbatim}[commandchars=\\\{\}]
\PYG{n}{opihiexarata} \PYG{n}{manual} \PYG{o}{\PYGZhy{}}\PYG{o}{\PYGZhy{}}\PYG{n}{config}\PYG{o}{=}\PYG{n}{config}\PYG{o}{.}\PYG{n}{yaml} \PYG{o}{\PYGZhy{}}\PYG{o}{\PYGZhy{}}\PYG{n}{secret}\PYG{o}{=}\PYG{n}{secrets}\PYG{o}{.}\PYG{n}{yaml}
\end{sphinxVerbatim}

\sphinxAtStartPar
Please replace the configuration parameters with the correct path to your
configuration and secrets file; see {\hyperref[\detokenize{user/configuration:user-configuration}]{\sphinxcrossref{\DUrole{std,std-ref}{Configuration}}}} for more
information.


\paragraph{Specify New Target Name}
\label{\detokenize{user/manual_mode:specify-new-target-name}}\label{\detokenize{user/manual_mode:user-manual-mode-procedure-specify-new-target-name}}
\sphinxAtStartPar
The OpihiExarata manual mode software groups sequential images as belonging to
a set of observations of a single target. In order to tell the software what
target/asteroid you are observing you need to provide it.

\sphinxAtStartPar
Use the \sphinxguilabel{New Target} button, it will open a small prompt for you to enter the
name of the asteroid/target you are observing. From this point on, all images
loaded (either manually or automatically fetched) will be assumed to be of the
target you provided (until you change it by clicking the button again).

\begin{sphinxadmonition}{note}{Note:}
\sphinxAtStartPar
If you are observing an asteroid, this target name should also be the name
of the asteroid. The software will attempt to use this name to obtain
historical observations from the Minor Planet Center.
\end{sphinxadmonition}


\paragraph{Take and Image}
\label{\detokenize{user/manual_mode:take-and-image}}
\sphinxAtStartPar
The first step in processing data is to take an image using the camera
controller.

\begin{sphinxadmonition}{note}{Note:}
\sphinxAtStartPar
It is highly suggested that you take two images if this is the first image
you are taking of a given specified asteroid. This allows you to have a
reference image which makes asteroid finding much easier. You process the
first image normally (using the second image as the reference image) then
process the second image (using the first image as the reference image).
It does not matter which is the reference image for more images taken. (See
{\hyperref[\detokenize{user/manual_mode:user-manual-mode-procedure-find-asteroid-location}]{\sphinxcrossref{\DUrole{std,std-ref}{Find Asteroid Location}}}}.)
\end{sphinxadmonition}


\paragraph{Find Asteroid Location}
\label{\detokenize{user/manual_mode:find-asteroid-location}}\label{\detokenize{user/manual_mode:user-manual-mode-procedure-find-asteroid-location}}
\sphinxAtStartPar
You will need to find and specify the location of the asteroid in the image.
It is beyond the scope of this software and procedure to implement this
automatically. (If you are not observing asteroids, you may skip this step
and just click \sphinxguilabel{Submit} in the
{\hyperref[\detokenize{user/manual_mode:user-manual-mode-procedure-find-asteroid-location-target-selector-gui}]{\sphinxcrossref{\DUrole{std,std-ref}{Target Selector GUI}}}}.)

\sphinxAtStartPar
Use the {\hyperref[\detokenize{user/manual_mode:user-manual-mode-procedure-find-asteroid-location-target-selector-gui}]{\sphinxcrossref{\DUrole{std,std-ref}{Target Selector GUI}}}}
to find the asteroid pixel location.


\subparagraph{Target Selector GUI}
\label{\detokenize{user/manual_mode:target-selector-gui}}\label{\detokenize{user/manual_mode:user-manual-mode-procedure-find-asteroid-location-target-selector-gui}}
\begin{figure}[htbp]
\centering
\capstart

\noindent\sphinxincludegraphics{{target-selector-gui}.png}
\caption{The GUI for finding the pixel location of a target in the image. The
targets are typically asteroids.}\label{\detokenize{user/manual_mode:id4}}\label{\detokenize{user/manual_mode:figure-target-selector-gui}}\end{figure}

\sphinxAtStartPar
The target selector GUI allows you to select a specific target location in
an image, see \hyperref[\detokenize{user/manual_mode:figure-target-selector-gui}]{Fig.\@ \ref{\detokenize{user/manual_mode:figure-target-selector-gui}}}.

\sphinxAtStartPar
The current file which you are determining the location of a target in is
given by \sphinxguilabel{Current:}. A reference image (if provided) to compare against is
given by \sphinxguilabel{Reference:}. Both of these files can be changed using their
respective \sphinxguilabel{Change} buttons; a file dialog will be opened so you can specify
the new FITS files.

\sphinxAtStartPar
There is a data viewer similar to the one specified in
{\hyperref[\detokenize{user/manual_mode:user-manual-mode-graphical-user-interface}]{\sphinxcrossref{\DUrole{std,std-ref}{Graphical User Interface}}}}. However, in addition, if you
drag a box (left click and hold, drag, then release) without any tool selected
in toolbar, then the software will search within the drawn (blue) box and
extract the brightest object within the box. It will mark this target with a
(red) triangle. It will assume that this is the desired target and update the
\sphinxguilabel{Target X} and \sphinxguilabel{Target Y} fields with its pixel coordinates.

\begin{sphinxadmonition}{note}{Note:}
\sphinxAtStartPar
This box drawing method  finds the brightest object in the current image. It ignores the
subtractive comparison method and its result as such comparisons do not
affect the actual current image.
\end{sphinxadmonition}

\sphinxAtStartPar
You can compare your current image file \$C\$ with your reference image \$R\$ file
in two subtractive ways using the two labeled buttons under
\sphinxguilabel{Subtraction Method}. (There are also buttons for simply viewing the images.)
These four ways are:
\begin{itemize}
\item {} 
\sphinxAtStartPar
\sphinxguilabel{None}, \(C-0\): The current image is not compared with the reference image.

\item {} 
\sphinxAtStartPar
\sphinxguilabel{Reference}, \(R-0\): The reference image is shown rather than the current image.

\item {} 
\sphinxAtStartPar
\sphinxguilabel{Sidereal}, \(C-R\): The two images are subtracted assuming the IRTF is doing sidereal tracking. Because of this assumption, no shifting is done.

\item {} 
\sphinxAtStartPar
\sphinxguilabel{Non\sphinxhyphen{}sidereal}, \(C-T_v(R)\): The two images are subtracted assuming the IRTF is doing non\sphinxhyphen{}sidereal tracking. Because of this assumption, the images are shifted based on the non\sphinxhyphen{}sidereal rates of the current image and the time between the two images.

\end{itemize}

\sphinxAtStartPar
The displayed image’s color bar scale can be modified manually by entering
values into the boxes accompanying \sphinxguilabel{Scale {[}Low High{]}}, the left and
right being the lower and higher bounds of the color bar respectively as
indicated. The scale can also be automatically set so that the lower bound is
the 1 percentile and the higher bound is the 99 percentile by clicking the
\sphinxguilabel{1 \sphinxhyphen{} 99 \%} button. If the \sphinxguilabel{Auto} checkbox is enabled,
this autoscaling is done whenever a new operation is done the image (i.e.
using the tools in the toolbar, changing the comparison method, among others).

\sphinxAtStartPar
Select your target, either from the box method or by manually entering the
coordinates in the \sphinxguilabel{Target X} and \sphinxguilabel{Target Y} boxes, and
click \sphinxguilabel{Submit}. The location of your target will be recorded.


\paragraph{Compute Astrometric Solution}
\label{\detokenize{user/manual_mode:compute-astrometric-solution}}\label{\detokenize{user/manual_mode:user-manual-mode-procedure-find-asteroid-location-compute-astrometric-solution}}
\begin{figure}[htbp]
\centering
\capstart

\noindent\sphinxincludegraphics{{manual-mode-gui-astrometry}.png}
\caption{The astrometry GUI tab for customizing and executing astrometric solutions.}\label{\detokenize{user/manual_mode:id5}}\label{\detokenize{user/manual_mode:figure-manual-mode-gui-astrometry}}\end{figure}

\sphinxAtStartPar
The astrometric solution of the image is next to be solved. The pattern of
stars within the image is compared with known patterns in astrometric star
databases to derive the \sphinxhref{https://fits.gsfc.nasa.gov/fits\_wcs.html}{WCS}%
\begin{footnote}[6]\sphinxAtStartFootnote
\sphinxnolinkurl{https://fits.gsfc.nasa.gov/fits\_wcs.html}
%
\end{footnote}
astrometric solution of the image. See \hyperref[\detokenize{user/manual_mode:figure-manual-mode-gui-astrometry}]{Fig.\@ \ref{\detokenize{user/manual_mode:figure-manual-mode-gui-astrometry}}}
for the interface for astrometric solutions.

\sphinxAtStartPar
To solve for the astrometric solution of the image, you will need to select
the desired astrometric engine from the drop down menu then click on the
\sphinxguilabel{Solve Astrometry} button to solve.
(See {\hyperref[\detokenize{technical/architecture/services_engines:technical-architecture-services-engines}]{\sphinxcrossref{\DUrole{std,std-ref}{Services and Engines}}}} for more information on
the available engines.)

\sphinxAtStartPar
The pixel location (X,Y) of the center of the image, given by \sphinxguilabel{Opihi Center},
and the specified target, given by \sphinxguilabel{Target/Asteroid}, is provided
with or without an astrometric solution. When the astrometric solution is
provided, the right ascension and declination of these will also be provided.

\sphinxAtStartPar
Custom pixel coordinate (X,Y) can be provided in the boxes to be translated to
the sky coordinates that they correspond to. Alternatively, if sky coordinates
are provided (in sexagesimal form, RA hours and DEC degrees, delimitated by
colons), the pixel coordinates of the sky coordinates can also be determined;
the pixel coordinate boxes must be empty as the solving gives preference to
pixel to on\sphinxhyphen{}sky solving. Enter in either pixel or sky coordinates as described
and click the \sphinxguilabel{Custom Solve} button to convert it to the other. The
button does nothing without a valid astrometric solution.


\paragraph{Compute Photometric Solution}
\label{\detokenize{user/manual_mode:compute-photometric-solution}}\label{\detokenize{user/manual_mode:user-manual-mode-procedure-find-asteroid-location-compute-photometric-solution}}
\begin{figure}[htbp]
\centering
\capstart

\noindent\sphinxincludegraphics{{manual-mode-gui-photometry}.png}
\caption{The photometry GUI tab for customizing and executing photometric solutions.}\label{\detokenize{user/manual_mode:id6}}\label{\detokenize{user/manual_mode:figure-manual-mode-gui-photometry}}\end{figure}

\sphinxAtStartPar
The photometric solution of the image is next to be solved. The brightness of
the stars in the image is compared to known filter magnitudes from a
photometric database to derive a photometric calibration solution.
See \hyperref[\detokenize{user/manual_mode:figure-manual-mode-gui-photometry}]{Fig.\@ \ref{\detokenize{user/manual_mode:figure-manual-mode-gui-photometry}}} for the interface for photometric
solutions.

\sphinxAtStartPar
This is an optional step and is not related to asteroid finding in of itself.
This operation can be skipped entirely if a photometric solution is not
necessary.

\sphinxAtStartPar
To solve for the photometric solution of the image, you will need to select
the desired photometric engine from the drop down menu then click on the
\sphinxguilabel{Solve Photometry} button to solve.
(See {\hyperref[\detokenize{technical/architecture/services_engines:technical-architecture-services-engines}]{\sphinxcrossref{\DUrole{std,std-ref}{Services and Engines}}}} for more information on
the available engines.)

\sphinxAtStartPar
The filter that the image was taken in is noted by \sphinxguilabel{Filter}, this is
determined by the FITS file header.

\sphinxAtStartPar
Once a photometric solution has been solved, the corresponding filter zero
point magnitude of the image is provided by \sphinxguilabel{Zero Point}.

\begin{sphinxadmonition}{note}{Note:}
\sphinxAtStartPar
Execution of the photometric solution requires a completed astrometric
solution from
{\hyperref[\detokenize{user/manual_mode:user-manual-mode-procedure-find-asteroid-location-compute-astrometric-solution}]{\sphinxcrossref{\DUrole{std,std-ref}{Compute Astrometric Solution}}}}.
\end{sphinxadmonition}


\paragraph{Asteroid On\sphinxhyphen{}Sky Position}
\label{\detokenize{user/manual_mode:asteroid-on-sky-position}}\label{\detokenize{user/manual_mode:user-manual-mode-procedure-asteroid-on-sky-position}}
\sphinxAtStartPar
The asteroid pixel location is derived from the procedure in
{\hyperref[\detokenize{user/manual_mode:user-manual-mode-procedure-find-asteroid-location-target-selector-gui}]{\sphinxcrossref{\DUrole{std,std-ref}{Target Selector GUI}}}}
and the corresponding on\sphinxhyphen{}sky location is derived from the procedure in
{\hyperref[\detokenize{user/manual_mode:user-manual-mode-procedure-find-asteroid-location-compute-astrometric-solution}]{\sphinxcrossref{\DUrole{std,std-ref}{Compute Astrometric Solution}}}}.


\paragraph{Historical Observations}
\label{\detokenize{user/manual_mode:historical-observations}}\label{\detokenize{user/manual_mode:user-manual-mode-procedure-historical-observations}}
\sphinxAtStartPar
The software will attempt to use the target/asteroid name provided in
{\hyperref[\detokenize{user/manual_mode:user-manual-mode-procedure-specify-new-target-name}]{\sphinxcrossref{\DUrole{std,std-ref}{Specify New Target Name}}}}
to obtain the set of historical observations from the Minor Planet Center.

\sphinxAtStartPar
Older images will also be considered part of the set of historical observations.


\paragraph{Asteroid Observation Record}
\label{\detokenize{user/manual_mode:asteroid-observation-record}}\label{\detokenize{user/manual_mode:user-manual-mode-procedure-asteroid-observation-record}}
\sphinxAtStartPar
The combination of both {\hyperref[\detokenize{user/manual_mode:user-manual-mode-procedure-historical-observations}]{\sphinxcrossref{\DUrole{std,std-ref}{Historical Observations}}}}
and {\hyperref[\detokenize{user/manual_mode:user-manual-mode-procedure-asteroid-on-sky-position}]{\sphinxcrossref{\DUrole{std,std-ref}{Asteroid On\sphinxhyphen{}Sky Position}}}} makes up the
sum total of the asteroid observation record. Using this asteroid observation
record, the future path of the asteroid on the sky can be determined to
eventually allow for the proper acquisition.

\sphinxAtStartPar
There are two different procedures for determining the future track of the
asteroid:
\begin{itemize}
\item {} 
\sphinxAtStartPar
Propagating the on\sphinxhyphen{}sky motion of the asteroid into the future.

\item {} 
\sphinxAtStartPar
Solving for the orbital elements and deriving an ephemeris.

\end{itemize}

\sphinxAtStartPar
Both options are sufficient but we recommend
{\hyperref[\detokenize{user/manual_mode:user-manual-mode-procedure-asteroid-position-propagation}]{\sphinxcrossref{\DUrole{std,std-ref}{Asteroid Position Propagation}}}}.


\paragraph{Asteroid Position Propagation}
\label{\detokenize{user/manual_mode:asteroid-position-propagation}}\label{\detokenize{user/manual_mode:user-manual-mode-procedure-asteroid-position-propagation}}
\begin{figure}[htbp]
\centering
\capstart

\noindent\sphinxincludegraphics{{manual-mode-gui-propagate}.png}
\caption{The propagation GUI tab for customizing and executing propagation solutions.}\label{\detokenize{user/manual_mode:id7}}\label{\detokenize{user/manual_mode:figure-manual-mode-gui-propagate}}\end{figure}

\sphinxAtStartPar
Propagating the on\sphinxhyphen{}sky motion of the asteroid is done by taking the
observational record from
{\hyperref[\detokenize{user/manual_mode:user-manual-mode-procedure-asteroid-observation-record}]{\sphinxcrossref{\DUrole{std,std-ref}{Asteroid Observation Record}}}} and propagating
only the most recent observations forward in time. See
\hyperref[\detokenize{user/manual_mode:figure-manual-mode-gui-propagate}]{Fig.\@ \ref{\detokenize{user/manual_mode:figure-manual-mode-gui-propagate}}} for the interface for propagation solutions.

\sphinxAtStartPar
To solve for the propagation solution from the observations, you will need to
select the desired propagation engine from the drop down menu then click on the
\sphinxguilabel{Solve Propagation} button to solve.
(See {\hyperref[\detokenize{technical/architecture/services_engines:technical-architecture-services-engines}]{\sphinxcrossref{\DUrole{std,std-ref}{Services and Engines}}}} for more information on
the available engines.)

\sphinxAtStartPar
If a propagation solution is done, the on\sphinxhyphen{}sky rates will be provided under
\sphinxguilabel{Propagate Rate {[} “/s | “/s²{]}}. Both the first order (velocity) and
second order (acceleration) on\sphinxhyphen{}sky rates in RA and DEC are given in arcseconds
per second, or arcseconds per second squared. The RA is given on the right and
DEC on the left within the first or second order pairs.

\sphinxAtStartPar
You may also provide a custom date and time, in
(\sphinxhref{https://www.iso.org/standard/70907.html}{ISO\sphinxhyphen{}8601 like formatting}%
\begin{footnote}[7]\sphinxAtStartFootnote
\sphinxnolinkurl{https://www.iso.org/standard/70907.html}
%
\end{footnote}) the
provided dialog box. You can specify the timezone that the provided date and
time corresponds to using the dropdown menu. When you click
\sphinxguilabel{Custom Solve}, the displayed RA and DEC coordinate is the estimated
sky coordinates for the asteroid at the provided time.

\begin{sphinxadmonition}{note}{Note:}
\sphinxAtStartPar
Execution of the propagation solution requires a completed astrometric
solution from
{\hyperref[\detokenize{user/manual_mode:user-manual-mode-procedure-find-asteroid-location-compute-astrometric-solution}]{\sphinxcrossref{\DUrole{std,std-ref}{Compute Astrometric Solution}}}}.
\end{sphinxadmonition}


\paragraph{Orbital Elements}
\label{\detokenize{user/manual_mode:orbital-elements}}\label{\detokenize{user/manual_mode:user-manual-mode-procedure-orbital-elements}}
\begin{figure}[htbp]
\centering
\capstart

\noindent\sphinxincludegraphics{{manual-mode-gui-orbit}.png}
\caption{The orbit GUI tab for customizing and executing orbital solutions.}\label{\detokenize{user/manual_mode:id8}}\label{\detokenize{user/manual_mode:figure-manual-mode-gui-orbit}}\end{figure}

\sphinxAtStartPar
Provided a list of historical observations, we can solve for the Keplerian
orbital elements using preliminary orbit determination for osculating elements.
See \hyperref[\detokenize{user/manual_mode:figure-manual-mode-gui-orbit}]{Fig.\@ \ref{\detokenize{user/manual_mode:figure-manual-mode-gui-orbit}}} for the interface for orbital
solutions.

\sphinxAtStartPar
To solve for the orbital solution from the observations, you will need to
select the desired orbit engine from the drop down menu then click on the
\sphinxguilabel{Solve Orbit} button to solve.
(See {\hyperref[\detokenize{technical/architecture/services_engines:technical-architecture-services-engines}]{\sphinxcrossref{\DUrole{std,std-ref}{Services and Engines}}}} for more information on
the available engines.)

\sphinxAtStartPar
The six Keplerian orbital elements (plus the epoch) are provided after the
orbital solutions is solved. They are:
\begin{itemize}
\item {} 
\sphinxAtStartPar
\sphinxguilabel{Ecc.}: The eccentricity of the orbit, this is unit\sphinxhyphen{}less.

\item {} 
\sphinxAtStartPar
\sphinxguilabel{Incli.}: The inclination of the orbit, in degrees.

\item {} 
\sphinxAtStartPar
\sphinxguilabel{As\sphinxhyphen{}Node}: The longitude of the ascending node, in degrees.

\item {} 
\sphinxAtStartPar
\sphinxguilabel{Peri.}: The argument of perihelion, in degrees.

\item {} 
\sphinxAtStartPar
\sphinxguilabel{M\sphinxhyphen{}Anom.}: The mean anomaly, in degrees.

\item {} 
\sphinxAtStartPar
\sphinxguilabel{Epoch}: The epoch of these of these osculating orbital elements, in Julian days.

\end{itemize}

\sphinxAtStartPar
If the Engine provided is \sphinxguilabel{Custom}, then you are trying to provide a
custom orbit. You provide your Keplerian orbital parameters in the boxes. You
may also specify the error in these elements by providing another number
delimitated from the first by a letter. (Note, scientific notation is not
supported, especially E\sphinxhyphen{}notation based entries.) After you provide your
orbital parameters, you can click \sphinxguilabel{Solve Orbit} to \sphinxstyleemphasis{solve} for your
orbital solution.

\begin{sphinxadmonition}{note}{Note:}
\sphinxAtStartPar
Execution of the orbital solution requires a completed astrometric
solution from
{\hyperref[\detokenize{user/manual_mode:user-manual-mode-procedure-find-asteroid-location-compute-astrometric-solution}]{\sphinxcrossref{\DUrole{std,std-ref}{Compute Astrometric Solution}}}}.
\end{sphinxadmonition}


\paragraph{Ephemeris}
\label{\detokenize{user/manual_mode:ephemeris}}\label{\detokenize{user/manual_mode:user-manual-mode-procedure-ephemeris}}
\begin{figure}[htbp]
\centering
\capstart

\noindent\sphinxincludegraphics{{manual-mode-gui-ephemeris}.png}
\caption{The ephemeris GUI tab for customizing and executing ephemeris solutions.}\label{\detokenize{user/manual_mode:id9}}\label{\detokenize{user/manual_mode:figure-manual-mode-gui-ephemeris}}\end{figure}

\sphinxAtStartPar
The orbital elements derived in {\hyperref[\detokenize{user/manual_mode:user-manual-mode-procedure-orbital-elements}]{\sphinxcrossref{\DUrole{std,std-ref}{Orbital Elements}}}}
are then used to derive the ephemeris of the asteroid. See
\hyperref[\detokenize{user/manual_mode:figure-manual-mode-gui-ephemeris}]{Fig.\@ \ref{\detokenize{user/manual_mode:figure-manual-mode-gui-ephemeris}}} for the interface for ephemeris
solutions.

\sphinxAtStartPar
To solve for the ephemeris solution from the orbital elements, you will need to
select the desired ephemeris engine from the drop down menu then click on the
\sphinxguilabel{Solve Ephemeris} button to solve.
(See {\hyperref[\detokenize{technical/architecture/services_engines:technical-architecture-services-engines}]{\sphinxcrossref{\DUrole{std,std-ref}{Services and Engines}}}} for more information on
the available engines.)

\sphinxAtStartPar
If an ephemeris solution is done, the on\sphinxhyphen{}sky rates will be provided under
\sphinxguilabel{Ephemeris Rate {[} “/s | “/s²{]}}. Both the first order (velocity) and
second order (acceleration) on\sphinxhyphen{}sky rates in RA and DEC are given in arcseconds
per second, or arcseconds per second squared. The RA is given on the right and
DEC on the left within the first or second order pairs.

\sphinxAtStartPar
You may also provide a custom date and time, in
(\sphinxhref{https://www.iso.org/standard/70907.html}{ISO\sphinxhyphen{}8601 like formatting}%
\begin{footnote}[8]\sphinxAtStartFootnote
\sphinxnolinkurl{https://www.iso.org/standard/70907.html}
%
\end{footnote}) the
provided dialog box. You can specify the timezone that the provided date and
time corresponds to using the dropdown menu. When you click
\sphinxguilabel{Custom Solve}, the displayed RA and DEC coordinate is the estimated
sky coordinates for the asteroid at the provided time.

\begin{sphinxadmonition}{note}{Note:}
\sphinxAtStartPar
Execution of the ephemeris solution requires a completed orbital
solution from {\hyperref[\detokenize{user/manual_mode:user-manual-mode-procedure-orbital-elements}]{\sphinxcrossref{\DUrole{std,std-ref}{Orbital Elements}}}} which
itself depends on a completed astrometric solution from
{\hyperref[\detokenize{user/manual_mode:user-manual-mode-procedure-find-asteroid-location-compute-astrometric-solution}]{\sphinxcrossref{\DUrole{std,std-ref}{Compute Astrometric Solution}}}}.
\end{sphinxadmonition}


\paragraph{Asteroid On\sphinxhyphen{}Sky Future Track}
\label{\detokenize{user/manual_mode:asteroid-on-sky-future-track}}\label{\detokenize{user/manual_mode:user-manual-procedure-asteroid-on-sky-future-track}}
\sphinxAtStartPar
Regardless of which method you use to derive the future track of the asteroid
(either from {\hyperref[\detokenize{user/manual_mode:user-manual-mode-procedure-asteroid-position-propagation}]{\sphinxcrossref{\DUrole{std,std-ref}{Asteroid Position Propagation}}}} or
from {\hyperref[\detokenize{user/manual_mode:user-manual-mode-procedure-orbital-elements}]{\sphinxcrossref{\DUrole{std,std-ref}{Orbital Elements}}}} and
{\hyperref[\detokenize{user/manual_mode:user-manual-mode-procedure-ephemeris}]{\sphinxcrossref{\DUrole{std,std-ref}{Ephemeris}}}}), the future position of the
asteroid and the on\sphinxhyphen{}sky rates are determined (see the respective sections
for details).


\paragraph{Telescope Control Software: Update}
\label{\detokenize{user/manual_mode:telescope-control-software-update}}
\sphinxAtStartPar
The new asteroid on\sphinxhyphen{}sky future track (position and on\sphinxhyphen{}sky rates) derived from
{\hyperref[\detokenize{user/manual_mode:user-manual-procedure-asteroid-on-sky-future-track}]{\sphinxcrossref{\DUrole{std,std-ref}{Asteroid On\sphinxhyphen{}Sky Future Track}}}} can be sent to the
telescope control software to slew the telescope to the correct location of
the asteroid.

\sphinxstepscope


\subsection{Automatic Mode}
\label{\detokenize{user/automatic_mode:automatic-mode}}\label{\detokenize{user/automatic_mode:user-automatic-mode}}\label{\detokenize{user/automatic_mode::doc}}
\sphinxAtStartPar
The automatic mode of OpihiExarata is its atmospheric monitoring mode. It is
best operated by its graphical user interface.

\sphinxAtStartPar
Images, continuously taken by the Opihi telescope and camera, are
photometrically solved such that their zero points are calculated. The
program does this automatically after the initial specifications and trigger.

\sphinxAtStartPar
We present the procedure for operating OpihiExarata in its automatic mode,
please also reference the GUI figure provided for reference. We also summarize
the procedure and process of the automatic mode via a flowchart.


\subsubsection{Graphical User Interface}
\label{\detokenize{user/automatic_mode:graphical-user-interface}}
\begin{figure}[htbp]
\centering
\capstart

\noindent\sphinxincludegraphics{{automatic-mode-gui}.png}
\caption{The GUI of the automatic mode of OpihiExarata. This is what you see when
the interface is freshly loaded. Most of the text is filler text, they will
change as the program is executed.}\label{\detokenize{user/automatic_mode:id1}}\label{\detokenize{user/automatic_mode:figure-automatic-mode-gui}}\end{figure}

\sphinxAtStartPar
The graphical user interface of the automatic mode of OpihiExarata, see
\hyperref[\detokenize{user/automatic_mode:figure-automatic-mode-gui}]{Fig.\@ \ref{\detokenize{user/automatic_mode:figure-automatic-mode-gui}}}. More information is detailed in the
respective subsections of this section.


\subsubsection{Procedure}
\label{\detokenize{user/automatic_mode:procedure}}
\begin{figure}[htbp]
\centering
\capstart

\noindent\sphinxincludegraphics{{automatic-mode-flowchart}.pdf}
\caption{A flowchart summary of the procedure of the automatic mode. It includes
the actions of the user along with the program’s flow afterwards.}\label{\detokenize{user/automatic_mode:id2}}\label{\detokenize{user/automatic_mode:figure-automatic-mode-flowchart}}\end{figure}

\sphinxAtStartPar
We describe the procedure to configuring and starting and stopping the
automatic mode of OpihiExarata. See \hyperref[\detokenize{user/automatic_mode:figure-automatic-mode-flowchart}]{Fig.\@ \ref{\detokenize{user/automatic_mode:figure-automatic-mode-flowchart}}}
for a flowchart summary of this procedure.


\paragraph{Start and Open GUI}
\label{\detokenize{user/automatic_mode:start-and-open-gui}}
\sphinxAtStartPar
You will want to open the OpihiExarata automatic mode GUI, typically via the
command\sphinxhyphen{}line interface with:

\begin{sphinxVerbatim}[commandchars=\\\{\}]
\PYG{n}{opihiexarata} \PYG{n}{automatic} \PYG{o}{\PYGZhy{}}\PYG{o}{\PYGZhy{}}\PYG{n}{config}\PYG{o}{=}\PYG{n}{config}\PYG{o}{.}\PYG{n}{yaml} \PYG{o}{\PYGZhy{}}\PYG{o}{\PYGZhy{}}\PYG{n}{secret}\PYG{o}{=}\PYG{n}{secrets}\PYG{o}{.}\PYG{n}{yaml}
\end{sphinxVerbatim}

\sphinxAtStartPar
Please replace the configuration parameters with the correct path to your
configuration and secrets file; see {\hyperref[\detokenize{user/configuration:user-configuration}]{\sphinxcrossref{\DUrole{std,std-ref}{Configuration}}}} for more
information.


\paragraph{Specify Data Directory}
\label{\detokenize{user/automatic_mode:specify-data-directory}}
\sphinxAtStartPar
You will need to specify the data directory that the OpihiExarata program will
pull the most recent images from. This should be the same directory where the
Opihi camera will be saving the images it is taking constantly.

\sphinxAtStartPar
You can specify this directory by clicking the \sphinxguilabel{Change} button to
bring up a directory selecting dialog. The exact form of your dialog is
operating system dependent as we defer to its implementation. Select the
directory using that dialog.

\sphinxAtStartPar
Once submitted, the \sphinxguilabel{Fetch Directory} entry should change to match
your selection. This display uses absolute paths; ensure that it is correct.


\paragraph{Specify Solving Engines}
\label{\detokenize{user/automatic_mode:specify-solving-engines}}
\sphinxAtStartPar
You will need to specify the solving engines that OpihiExarata will use to in
order to solve the images. For this mode, only the AstrometryEngine(s) and
PhotometryEngine(s) is considered as it does not use any other engine. The
selected engines will constantly be used until a different one is specified.

\sphinxAtStartPar
Specify the desired AstrometryEngine using the left drop\sphinxhyphen{}down combo box.
Specify the desired PhotometryEngine using the right drop\sphinxhyphen{}down combo box.
The \sphinxguilabel{Engines (A, P)} label has also small key to tell you which is
which. For more information on the available engines, see
{\hyperref[\detokenize{technical/architecture/services_engines:technical-architecture-services-engines}]{\sphinxcrossref{\DUrole{std,std-ref}{Services and Engines}}}}


\paragraph{Trigger Continuous Images}
\label{\detokenize{user/automatic_mode:trigger-continuous-images}}
\sphinxAtStartPar
You will need to trigger the Opihi camera to continuously take images.
Please consult the Opihi camera controller documentation for more information.

\sphinxAtStartPar
(A little supporting information about operating the camera controller will
be posted here when more information is available.)

\sphinxAtStartPar
Ensure that the directory that these continuous images are being saved to is
the same directory you specified for this program to fetch from.


\paragraph{Trigger Automatic Solving}
\label{\detokenize{user/automatic_mode:trigger-automatic-solving}}
\sphinxAtStartPar
You will need to start automatic solving by triggering it via the
\sphinxguilabel{Start} button. Once it starts, it will do the following, in order,
automatically, without need for user intervention:
\begin{enumerate}
\sphinxsetlistlabels{\arabic}{enumi}{enumii}{}{.}%
\item {} 
\sphinxAtStartPar
It will fetch the most recent fits image as determined by the modification timestamp of all fits files within the specified directory as provided by the operating system.

\item {} 
\sphinxAtStartPar
It will pre\sphinxhyphen{}process the fetched image according to the preprocessing algorithm, see {\hyperref[\detokenize{technical/algorithms/preprocessing:technical-algorithms-preprocessing}]{\sphinxcrossref{\DUrole{std,std-ref}{Preprocessing}}}}.

\item {} 
\sphinxAtStartPar
It will solve for the astrometric solution of the pre\sphinxhyphen{}processed image via the specified AstrometryEngine. The results of the solution will be displayed in the GUI.

\item {} 
\sphinxAtStartPar
It will solve for the photometric solution of the pre\sphinxhyphen{}processed image via the specified PhotometryEngine. The results of the solution will be displayed in the GUI.

\item {} 
\sphinxAtStartPar
It will add this result (the filter zero point measurement) to the archive of observations. The monitoring webpage uses this archive to derive its figures.

\item {} 
\sphinxAtStartPar
The results from the pre\sphinxhyphen{}processing of the raw image and the subsequent engine solves will be saved to disk under a similar name to the original raw file and in the data directory specified.

\item {} 
\sphinxAtStartPar
It will repeat this process until stopped.

\end{enumerate}


\subparagraph{Trigger Once Manually}
\label{\detokenize{user/automatic_mode:trigger-once-manually}}
\sphinxAtStartPar
You may also trigger the automatic solving procedure manually once via the
\sphinxguilabel{Trigger} button. This will do the entire process as elaborated above,
except for repeating it.


\paragraph{Displayed Status and Results}
\label{\detokenize{user/automatic_mode:displayed-status-and-results}}
\sphinxAtStartPar
As each image is solved, the results of the solve will be displayed. We
describe the fields which change as the automatic solving runs.

\sphinxAtStartPar
The \sphinxguilabel{Working} field details the file that is currently being (or
was last) worked on by the automatic solving algorithms. The
\sphinxguilabel{Results} field details the file that has already been worked on
and has been solved without failure. When the working file, undergoing the
solving, is solved successfully, it becomes a file with results and the
program designates it as such. If the working file failed to solve, it is not
transferred over.

\sphinxAtStartPar
The astrometric and photometric results of the \sphinxguilabel{Results} file is
displayed as well.

\sphinxAtStartPar
The \sphinxguilabel{Coordinates} felids specify the on\sphinxhyphen{}sky right ascension and
declination of the center of the image along with the UTC time of when this
image was taken.

\sphinxAtStartPar
The \sphinxguilabel{Zero Point} value, calculated via the photometric solver, of
the image is provided, along with the \sphinxguilabel{Filter} that the image was
taken in specified in said field. (The filter term is based on the fits
metadata.)

\sphinxAtStartPar
The status of the automatic solving will be displayed under
\sphinxguilabel{Loop Status}. The possible statuses, and their meanings, are:
\begin{itemize}
\item {} 
\sphinxAtStartPar
\sphinxguilabel{Running}: The automatic loop of fetching and solving images is currently running.

\item {} 
\sphinxAtStartPar
\sphinxguilabel{Stopped}: The automatic loop of fetching and solving images is stopped.

\item {} 
\sphinxAtStartPar
\sphinxguilabel{Triggered}: The solving of a single image has been triggered and it is being worked on.

\item {} 
\sphinxAtStartPar
\sphinxguilabel{Failed}: An image in the automatic loop failed to solve, but this does not stop the loop.

\item {} 
\sphinxAtStartPar
\sphinxguilabel{Halted}: The loop has been stopped via an alternative method than the \sphinxguilabel{Stop} button.

\item {} 
\sphinxAtStartPar
\sphinxguilabel{Default}: This is filler text when the GUI is first opened. This should not reappear throughout usage.

\end{itemize}


\paragraph{Stop Automatic Solving}
\label{\detokenize{user/automatic_mode:stop-automatic-solving}}
\sphinxAtStartPar
When you want to stop the automatic solving, you can click the \sphinxguilabel{Stop} button
at any time. This will finish the current image it is working on and stop the
automatic loop from fetching another image from the data directory. Because of
the nature of sending information to other services (i.e. the backends to the
engines), the solving of an image cannot gracefully stopped mid\sphinxhyphen{}way and so the
process must finish and we only prevent it from continuing.

\sphinxAtStartPar
If you want to stop the solving immediately for whatever reason, it is
suggested to cancel or crash the process that OpihiExarata is running on.

\sphinxAtStartPar
The \sphinxguilabel{Stop} button will not prevent a manual trigger from being executed via
the \sphinxguilabel{Trigger} button.

\sphinxAtStartPar
If the infinite automatic loop continues fetching images even after the stop
button is pressed, this likely means something was changed in the code and
the original logic failed. A solution to stop the loop is detailed in
{\hyperref[\detokenize{user/troubleshooting:user-troubleshooting-automatic-mode-stop-button-not-working}]{\sphinxcrossref{\DUrole{std,std-ref}{Automatic Mode Stop Button Not Working}}}}.

\sphinxstepscope


\subsection{Configuration}
\label{\detokenize{user/configuration:configuration}}\label{\detokenize{user/configuration:user-configuration}}\label{\detokenize{user/configuration::doc}}
\sphinxAtStartPar
There are many different configuration options available to the user to
modify the OpihiExarata software. The configuration options are split into
two different configuration files, detailed here.

\sphinxAtStartPar
If these configuration files to not exist on the system, you can generate a
new configuration file, which contains the defaults, using the
{\hyperref[\detokenize{user/command_line:cmdoption-arg-generate}]{\sphinxcrossref{\sphinxcode{\sphinxupquote{generate}}}}} action from the command\sphinxhyphen{}line interface, see {\hyperref[\detokenize{user/command_line:user-command-line-available-actions-generate}]{\sphinxcrossref{\DUrole{std,std-ref}{Generate}}}}.


\subsubsection{Standard Configuration File}
\label{\detokenize{user/configuration:standard-configuration-file}}\label{\detokenize{user/configuration:user-configuration-standard-configuration-file}}
\sphinxAtStartPar
A standard configuration file (simplified to configuration file), typically
called something like \sphinxcode{\sphinxupquote{configuration.yaml}}, is where most of the
configuration parameters for all aspects of the OpihiExarata software package
exists.

\sphinxAtStartPar
The configuration file is a YAML formatted file with relatively verbose names.
A newly generated configuration file is a copy of the default file being used.

\sphinxAtStartPar
To detail all of the configuration parameters here is not particularly
efficient. The parameters are documented in the configuration file itself.
It can be found at \sphinxcode{\sphinxupquote{OpihiExarata/src/opihiexarata/configuration.yaml}} in
the directory tree. Alternatively, it can also be found
\sphinxhref{https://github.com/psmd-iberutaru/OpihiExarata/blob/master/src/opihiexarata/configuration.yaml}{here on the OpihiExarata Github, configuration.yaml}%
\begin{footnote}[9]\sphinxAtStartFootnote
\sphinxnolinkurl{https://github.com/psmd-iberutaru/OpihiExarata/blob/master/src/opihiexarata/configuration.yaml}
%
\end{footnote}.


\subsubsection{Secrets Configuration File}
\label{\detokenize{user/configuration:secrets-configuration-file}}\label{\detokenize{user/configuration:user-configuration-secrets-configuration-file}}
\sphinxAtStartPar
A secrets configuration file (simplified to secrets file), typically
called something like \sphinxcode{\sphinxupquote{secrets.yaml}}, is where configuration parameters
which should not be released to the public (via open source) are kept, hence a
secrets file. These are often software and API keys.

\sphinxAtStartPar
The secrets file is a YAML formatted file with relatively verbose names.
A newly generated secrets file is a copy of the default file being used. But,
there are no defaults and the secrets are blank as that is the whole point of
a secrets file. They need to be filled in with a user provided secrets file.

\sphinxAtStartPar
To detail all of the secrets parameters here is not particularly efficient.
The secrets themselves are documented in the secrets file itself. It can be
found at \sphinxcode{\sphinxupquote{OpihiExarata/src/opihiexarata/secrets.yaml}} in the directory tree.
Alternatively, it can also be found \sphinxhref{https://github.com/psmd-iberutaru/OpihiExarata/blob/master/src/opihiexarata/secrets.yaml}{here on the OpihiExarata Github, secrets.yaml}%
\begin{footnote}[10]\sphinxAtStartFootnote
\sphinxnolinkurl{https://github.com/psmd-iberutaru/OpihiExarata/blob/master/src/opihiexarata/secrets.yaml}
%
\end{footnote}.

\sphinxstepscope


\subsection{Troubleshooting}
\label{\detokenize{user/troubleshooting:troubleshooting}}\label{\detokenize{user/troubleshooting:user-troubleshooting}}\label{\detokenize{user/troubleshooting::doc}}
\sphinxAtStartPar
For all of your troubleshooting needs. Here we detail some of the common
problems encountered and the solutions for these problems.


\subsubsection{Automatic Mode Stop Button Not Working}
\label{\detokenize{user/troubleshooting:automatic-mode-stop-button-not-working}}\label{\detokenize{user/troubleshooting:user-troubleshooting-automatic-mode-stop-button-not-working}}
\sphinxAtStartPar
If the \sphinxguilabel{Stop} button in the automatic mode GUI is not working and
fails to stop the loop as designed, then this is likely because of an oversight
when the code was changed. The loop will run forever until stopped as per the
design of the automatic mode.

\sphinxAtStartPar
You can force the loop to stop gracefully by creating a file named
\sphinxcode{\sphinxupquote{opihiexarata\_automatic.stop}} in the same data directory that the automatic
loop is fetching the most recent images from. The loop checks for the
existence of this file and it will stop in a similar way to if the stop button
was pressed. The loop will be considered \sphinxguilabel{Halted}.

\sphinxAtStartPar
If this fails, then the suggested remedy is to ungracefully stop or crash the
process that OpihiExarata is running in.

\sphinxAtStartPar
In either case, this is an error with the code of OpihiExarata and the
maintainers of the software should be contacted.


\subsubsection{PySide GUI Does Not Match Documentation or UI Files}
\label{\detokenize{user/troubleshooting:pyside-gui-does-not-match-documentation-or-ui-files}}\label{\detokenize{user/troubleshooting:user-troubleshooting-pyside-gui-does-not-match-documentation-or-ui-files}}
\sphinxAtStartPar
This problem can be caused by one of two things.

\sphinxAtStartPar
If the documentation and the GUI differs, then the documentation may be out of
date. Otherwise, the Python GUI files built by PySide6 are out of date and
need to be built again.

\sphinxAtStartPar
If the documentation is out of date, see
{\hyperref[\detokenize{technical/installation/documentation:technical-installation-documentation}]{\sphinxcrossref{\DUrole{std,std-ref}{Optional: Documentation}}}} for rebuilding up to date
documentation. If information is missing, please contact the maintainers to
update the documentation.

\sphinxAtStartPar
If the GUI differs from the Qt Designer UI files, then the Python UI needs to
be built from these files. See
{\hyperref[\detokenize{technical/architecture/graphical_user_interface:technical-architecture-graphical-user-interface-building-ui-files}]{\sphinxcrossref{\DUrole{std,std-ref}{Building UI Files}}}} for
more information.

\sphinxstepscope


\subsection{Citations}
\label{\detokenize{user/citations:citations}}\label{\detokenize{user/citations:user-citations}}\label{\detokenize{user/citations::doc}}
\sphinxAtStartPar
If you use the Opihi telescope or its accompanying OpihiExarata software, please cite your usage using the following citation information:

\sphinxAtStartPar
(The citation information, here, will be updated as more publication
information is provided.)


\subsubsection{References}
\label{\detokenize{user/citations:references}}
\sphinxAtStartPar
We detail the other projects that OpihiExarata uses to accomplish its goals,
in no particular order.


\paragraph{NASA Infrared Telescope Facility}
\label{\detokenize{user/citations:nasa-infrared-telescope-facility}}
\sphinxAtStartPar
This work utilizes, is utilized by, is built for, and is funded by the NASA
Infrared Telescope Facility (IRTF), which is operated by the University of
Hawaii under Cooperative Agreement no. NNX\sphinxhyphen{}08AE38A with the National
Aeronautics and Space Administration, Science Mission Directorate, Planetary
Astronomy Program.


\paragraph{MAST}
\label{\detokenize{user/citations:mast}}
\sphinxAtStartPar
Some/all of the data presented in this work were obtained from the Mikulski
Archive for Space Telescopes (MAST). STScI is operated by the Association of
Universities for Research in Astronomy, Inc., under NASA contract NAS5\sphinxhyphen{}26555.
Support for MAST for non\sphinxhyphen{}HST data is provided by the NASA Office of Space
Science via grant NNX13AC07G and by other grants and contracts.


\paragraph{Minor Planet Center}
\label{\detokenize{user/citations:minor-planet-center}}
\sphinxAtStartPar
This work makes use of the Minor Planet \& Comet Ephemeris Service (IAU Minor
Planet Center)


\paragraph{Scipy}
\label{\detokenize{user/citations:scipy}}
\sphinxAtStartPar
This work makes use of SciPy
(\sphinxhref{https://doi.org/10.1038/s41592-019-0686-2}{Virtanen et. al. 2020}%
\begin{footnote}[11]\sphinxAtStartFootnote
\sphinxnolinkurl{https://doi.org/10.1038/s41592-019-0686-2}
%
\end{footnote}).


\paragraph{Matplotlib}
\label{\detokenize{user/citations:matplotlib}}
\sphinxAtStartPar
This work makes use of matplotlib, a Python library for publication quality graphics
(\sphinxhref{https://doi.ieeecomputersociety.org/10.1109/MCSE.2007.55}{Hunter 2007}%
\begin{footnote}[12]\sphinxAtStartFootnote
\sphinxnolinkurl{https://doi.ieeecomputersociety.org/10.1109/MCSE.2007.55}
%
\end{footnote}).


\paragraph{Astropy}
\label{\detokenize{user/citations:astropy}}
\sphinxAtStartPar
This work makes use of Astropy, a community\sphinxhyphen{}developed core Python package for
Astronomy
(\sphinxhref{https://ui.adsabs.harvard.edu/abs/2018AJ....156..123A}{Astropy Collaboration et. al. 2018}%
\begin{footnote}[13]\sphinxAtStartFootnote
\sphinxnolinkurl{https://ui.adsabs.harvard.edu/abs/2018AJ....156..123A}
%
\end{footnote}, \sphinxhref{https://ui.adsabs.harvard.edu/abs/2013A\&A...558A..33A}{Astropy Collaboration et. al. 2013}%
\begin{footnote}[14]\sphinxAtStartFootnote
\sphinxnolinkurl{https://ui.adsabs.harvard.edu/abs/2013A\&A...558A..33A}
%
\end{footnote}).


\paragraph{Numpy}
\label{\detokenize{user/citations:numpy}}
\sphinxAtStartPar
This work makes use of NumPy
(\sphinxhref{https://doi.org/10.1038/s41586-020-2649-2\}}{Harris et. al. 2020}%
\begin{footnote}[15]\sphinxAtStartFootnote
\sphinxnolinkurl{https://doi.org/10.1038/s41586-020-2649-2\}}
%
\end{footnote}).


\paragraph{Gaia}
\label{\detokenize{user/citations:gaia}}
\sphinxAtStartPar
This work makes use of data from the European Space Agency (ESA) mission \sphinxstyleemphasis{Gaia}
(\sphinxhref{https://www.cosmos.esa.int/gaia}{website}%
\begin{footnote}[16]\sphinxAtStartFootnote
\sphinxnolinkurl{https://www.cosmos.esa.int/gaia}
%
\end{footnote}), processed by the \sphinxstyleemphasis{Gaia} Data
Processing and Analysis Consortium
(\sphinxhref{https://www.cosmos.esa.int/web/gaia/dpac/consortium}{DPAC}%
\begin{footnote}[17]\sphinxAtStartFootnote
\sphinxnolinkurl{https://www.cosmos.esa.int/web/gaia/dpac/consortium}
%
\end{footnote}). Funding for
the DPAC has been provided by national institutions, in particular the
institutions participating in the \sphinxstyleemphasis{Gaia} Multilateral Agreement.


\paragraph{Pan\sphinxhyphen{}STARRS1}
\label{\detokenize{user/citations:pan-starrs1}}
\sphinxAtStartPar
The Pan\sphinxhyphen{}STARRS1 Surveys (PS1) have been made possible through contributions of
the Institute for Astronomy, the University of Hawaii, the Pan\sphinxhyphen{}STARRS Project
Office, the Max\sphinxhyphen{}Planck Society and its participating institutes, the Max Planck
Institute for Astronomy, Heidelberg and the Max Planck Institute for
Extraterrestrial Physics, Garching, The Johns Hopkins University, Durham
University, the University of Edinburgh, Queen’s University Belfast, the
Harvard\sphinxhyphen{}Smithsonian Center for Astrophysics, the Las Cumbres Observatory
Global Telescope Network Incorporated, the National Central University of
Taiwan, the Space Telescope Science Institute, the National Aeronautics and
Space Administration under Grant No. NNX08AR22G issued through the Planetary
Science Division of the NASA Science Mission Directorate, the National
Science Foundation under Grant No. AST\sphinxhyphen{}1238877, the University of Maryland,
and Eotvos Lorand University (ELTE).


\paragraph{JPL Horizons}
\label{\detokenize{user/citations:jpl-horizons}}
\sphinxAtStartPar
This work makes use of the JPL Horizons API service
(\sphinxhref{https://ssd.jpl.nasa.gov/horizons/}{Giorgini, JD and JPL Solar System Dynamics Group, NASA/JPL Horizons On\sphinxhyphen{}Line Ephemeris System https://ssd.jpl.nasa.gov/horizons/}%
\begin{footnote}[18]\sphinxAtStartFootnote
\sphinxnolinkurl{https://ssd.jpl.nasa.gov/horizons/}
%
\end{footnote};
\sphinxhref{https://ui.adsabs.harvard.edu/abs/2001DPS....33.5813G}{Giorgini et. al. 2001}%
\begin{footnote}[19]\sphinxAtStartFootnote
\sphinxnolinkurl{https://ui.adsabs.harvard.edu/abs/2001DPS....33.5813G}
%
\end{footnote};
\sphinxhref{https://web.archive.org/web/20220620101354/https://www.techbriefs.com/component/content/article/tb/pub/briefs/software/30057}{Giorgini et. al. 1999}%
\begin{footnote}[20]\sphinxAtStartFootnote
\sphinxnolinkurl{https://web.archive.org/web/20220620101354/https://www.techbriefs.com/component/content/article/tb/pub/briefs/software/30057}
%
\end{footnote};
\sphinxhref{https://ui.adsabs.harvard.edu/abs/1996DPS....28.2504G}{Giorgini et. al. 1996}%
\begin{footnote}[21]\sphinxAtStartFootnote
\sphinxnolinkurl{https://ui.adsabs.harvard.edu/abs/1996DPS....28.2504G}
%
\end{footnote})


\paragraph{Astrometry.net}
\label{\detokenize{user/citations:astrometry-net}}
\sphinxAtStartPar
This work makes use of the Astrometry.net Nova web service and installable
programs
(\sphinxhref{https://ui.adsabs.harvard.edu/abs/2010AJ....139.1782L}{Lang et. al 2010}%
\begin{footnote}[22]\sphinxAtStartFootnote
\sphinxnolinkurl{https://ui.adsabs.harvard.edu/abs/2010AJ....139.1782L}
%
\end{footnote},
\sphinxhref{https://ui.adsabs.harvard.edu/abs/2008ASPC..394...27H}{Hogg et. al. 2008}%
\begin{footnote}[23]\sphinxAtStartFootnote
\sphinxnolinkurl{https://ui.adsabs.harvard.edu/abs/2008ASPC..394...27H}
%
\end{footnote}).

\sphinxstepscope


\subsection{Trivia}
\label{\detokenize{user/trivia:trivia}}\label{\detokenize{user/trivia::doc}}

\subsubsection{Etymology of OpihiExarata}
\label{\detokenize{user/trivia:etymology-of-opihiexarata}}
\sphinxAtStartPar
The IRTF Opihi telescope is named after the \sphinxhref{https://www.waikikiaquarium.org/experience/animal-guide/invertebrates/molluscs/opihi/}{opihi}%
\begin{footnote}[24]\sphinxAtStartFootnote
\sphinxnolinkurl{https://www.waikikiaquarium.org/experience/animal-guide/invertebrates/molluscs/opihi/}
%
\end{footnote}, a limpet which lives on
rocky shores by sticking to the rocks. In a similar vein, the Opihi telescope
sticks hard onto the side of the main IRTF telescope. One of the three opihi
species endemic to the Hawaii islands is the \sphinxhref{https://en.wikipedia.org/wiki/Cellana\_exarata}{Hawaiian blackfoot opihi}%
\begin{footnote}[25]\sphinxAtStartFootnote
\sphinxnolinkurl{https://en.wikipedia.org/wiki/Cellana\_exarata}
%
\end{footnote}. The
Opihi telescope is similar to this species and for this reason, this software
is named after it. (The similarities: the opihi species and the telescope are
both black; the Opihi telescope is rather small compared to the IRTF, the
blackfoot is a small opihi; and the blackfoot opihi is a high delicacy,
similar to this software, it has a rather small audience.)

\sphinxAtStartPar
The binomial taxonomical name of the Hawaiian blackfoot opihi is Cellana
exarata. The genus Cellana describes the group of limpets including the opihi,
but “exarata” is the identifying part for the Hawaiian blackfoot, thus the
name of this software, we decided, is a hybrid: OpihiExarata. A backronym for
it, which describes its function: Ephemeris with eXtra Atmospheric Response
and Asteroid Trajectory Analysis.


\chapter{Technical Manual}
\label{\detokenize{index:technical-manual}}\label{\detokenize{index:home-technical-manual}}
\sphinxAtStartPar
The technical manual is primarily for software maintainers and other IRTF
staff. (For general use purposes, see the {\hyperref[\detokenize{index:home-user-manual}]{\sphinxcrossref{\DUrole{std,std-ref}{User Manual}}}}).
It details the overall architectural design and thought process of the
OpihiExarata software in much greater detail to assist in development. (For
function and class documentation of the raw API itself, see the
{\hyperref[\detokenize{index:home-technical-manual}]{\sphinxcrossref{\DUrole{std,std-ref}{Technical Manual}}}}.) It can be accessed via the sidebar or by
clicking {\hyperref[\detokenize{technical/index::doc}]{\sphinxcrossref{\DUrole{doc}{here}}}}.

\sphinxstepscope


\section{Technical Manual}
\label{\detokenize{technical/index:technical-manual}}\label{\detokenize{technical/index:technical-index}}\label{\detokenize{technical/index::doc}}
\sphinxAtStartPar
The technical manual here is made for the those looking for more detail about
the inner\sphinxhyphen{}workings of the OpihiExarata software. This is typically staff and
maintainers for the OpihiExarata package, but it may be useful for some other
audiences.

\sphinxstepscope


\subsection{Installation}
\label{\detokenize{technical/installation/index:installation}}\label{\detokenize{technical/installation/index:technical-installation}}\label{\detokenize{technical/installation/index::doc}}
\sphinxAtStartPar
The installation instructions are detailed here. For convenience, installation
scripts have been written to install the software for choice operating systems.
Currently, the supported operating systems are, in order, Rocky Linux, Windows
and Ubuntu. Future compatibility with other operating systems is unlikely to
be supported in the future.


\subsubsection{Automated Scripts}
\label{\detokenize{technical/installation/index:automated-scripts}}
\sphinxAtStartPar
As there are many parts to the OpihiExarata software, for convenience, an
installation script has been written for convenience and to help facilitate
installation. Please note that the automated script installs everything, if
you want to customize your install, please see
{\hyperref[\detokenize{technical/installation/index:technical-installation-manual-installation}]{\sphinxcrossref{\DUrole{std,std-ref}{Manual Installation}}}}.

\sphinxAtStartPar
The installation script is written in Powershell Core and thus require the
installation of Powershell Core. Instructions for the installation and usage
of \sphinxhref{https://docs.microsoft.com/en-us/powershell/}{Powershell Core}%
\begin{footnote}[26]\sphinxAtStartFootnote
\sphinxnolinkurl{https://docs.microsoft.com/en-us/powershell/}
%
\end{footnote} can be
found on its website. The installation of the OpihiExarata software does not
depend on Powershell Core after installation, but the convenience scripts
would be unavailable. If this is the case, please follow the other
installation instructions as usual.

\sphinxAtStartPar
In addition to installation script, an auxillary script has been written in
Powershell Core. The purpose of the auxillary script is to do tasks not
strictly related to installation. These tasks include code formatting,
testing, and documentation building. You may use this as well but it is not
necessary for installation.


\paragraph{Script Usage}
\label{\detokenize{technical/installation/index:script-usage}}
\sphinxAtStartPar
A few elements of the scripts must be tuned to your current operating system.
Editing a few lines in the beginning of each script file before running them
is all that is needed.

\sphinxAtStartPar
For the installation script \sphinxcode{\sphinxupquote{install.ps1}}:
\begin{itemize}
\item {} 
\sphinxAtStartPar
You need to specify the version\sphinxhyphen{}specific Python command that Powershell Core uses to enter the correct Python interpreter. It is typically something like \sphinxcode{\sphinxupquote{python3.9}} for Python 3.9, or something similar for other versions. (Windows operating systems, for example, is often just \sphinxcode{\sphinxupquote{python}}.) For example, the line should be something like:

\begin{sphinxVerbatim}[commandchars=\\\{\}]
\PYG{n}{Set}\PYG{o}{\PYGZhy{}}\PYG{n}{Alias} \PYG{o}{\PYGZhy{}}\PYG{n}{Name} \PYG{n}{pyox} \PYG{o}{\PYGZhy{}}\PYG{n}{Value} \PYG{n}{python3}\PYG{l+m+mf}{.9}
\end{sphinxVerbatim}

\item {} 
\sphinxAtStartPar
If you are using a Linux system, you need to specify the primary flavor as different systems have different ways of installing packages. Set either \sphinxcode{\sphinxupquote{\$isCentOS}} or \sphinxcode{\sphinxupquote{\$isUbuntu}} to True based on what system you have. If the system detects that you have a Linux system and neither is specified, an error will be raised.

\end{itemize}

\sphinxAtStartPar
For the auxillary script \sphinxcode{\sphinxupquote{auxillary.ps1}}:
\begin{itemize}
\item {} 
\sphinxAtStartPar
Like the install script, you need to specify the version\sphinxhyphen{}specific Python command that Powershell Core uses. Typically it is typically something like \sphinxcode{\sphinxupquote{python3.9}} for Python 3.9. It is advised that you use the same entry as in the installation script:

\begin{sphinxVerbatim}[commandchars=\\\{\}]
\PYG{n}{Set}\PYG{o}{\PYGZhy{}}\PYG{n}{Alias} \PYG{o}{\PYGZhy{}}\PYG{n}{Name} \PYG{n}{pyox} \PYG{o}{\PYGZhy{}}\PYG{n}{Value} \PYG{n}{python3}\PYG{l+m+mf}{.9}
\end{sphinxVerbatim}

\end{itemize}


\subsubsection{Manual Installation}
\label{\detokenize{technical/installation/index:manual-installation}}\label{\detokenize{technical/installation/index:technical-installation-manual-installation}}
\sphinxAtStartPar
Instructions are present for those who prefer a manual install, either because
the installation script does not work or they want to customize their
installation.

\sphinxAtStartPar
The installation instructions must be followed in order. There are optional
parts to the installation instructions, typically they are not needed for a
feature complete install of OpihiExarata, but they are highly suggested. The
installation order is as follows, we note what is optional and why:
\begin{enumerate}
\sphinxsetlistlabels{\arabic}{enumi}{enumii}{}{.}%
\item {} 
\sphinxAtStartPar
{\hyperref[\detokenize{technical/installation/download:technical-installation-download}]{\sphinxcrossref{\DUrole{std,std-ref}{Install: Download}}}}: You will download the OpihiExarata package from its Github repository in this step.

\item {} 
\sphinxAtStartPar
{\hyperref[\detokenize{technical/installation/python:technical-installation-python-part}]{\sphinxcrossref{\DUrole{std,std-ref}{Install: Python Part}}}}: You will download a supported Python version and build then install the Python part of OpihiExarata and its dependencies. This part being the primary part.

\item {} 
\sphinxAtStartPar
{\hyperref[\detokenize{technical/installation/documentation:technical-installation-documentation}]{\sphinxcrossref{\DUrole{std,std-ref}{Optional: Documentation}}}}: (Optional) You will build and process the documentation files to build the documentation for OpihiExarata. This is helpful if this set of documentation that you are reading is outdated or if you just want the documentation.

\item {} 
\sphinxAtStartPar
{\hyperref[\detokenize{technical/installation/windows:technical-installation-windows-compatibility}]{\sphinxcrossref{\DUrole{std,std-ref}{Optional: Windows Compatibility}}}}: (Situational) Some of the services that OpihiExarata depends on only works for Linux. In order to have supported functionality on Windows, the OpihiExarata software leverages the Windows Subsystem for Linux. It is required that Windows users download and install this feature. The instructions are detailed here.

\item {} 
\sphinxAtStartPar
{\hyperref[\detokenize{technical/installation/orbfit:technical-installation-orbfit}]{\sphinxcrossref{\DUrole{std,std-ref}{Install: Orbfit}}}}: This installs the Orbfit preliminary orbit determiner program which OpihiExarata uses for orbit determination.

\end{enumerate}

\sphinxstepscope


\paragraph{Install: Download}
\label{\detokenize{technical/installation/download:install-download}}\label{\detokenize{technical/installation/download:technical-installation-download}}\label{\detokenize{technical/installation/download::doc}}

\subparagraph{Via git (Recommended)}
\label{\detokenize{technical/installation/download:via-git-recommended}}
\sphinxAtStartPar
The best way to download OpihiExarata is to clone the repository:

\begin{sphinxVerbatim}[commandchars=\\\{\}]
\PYG{n}{git} \PYG{n}{clone} \PYG{n}{https}\PYG{p}{:}\PYG{o}{/}\PYG{o}{/}\PYG{n}{github}\PYG{o}{.}\PYG{n}{com}\PYG{o}{/}\PYG{n}{psmd}\PYG{o}{\PYGZhy{}}\PYG{n}{iberutaru}\PYG{o}{/}\PYG{n}{OpihiExarata}\PYG{o}{.}\PYG{n}{git}
\end{sphinxVerbatim}

\sphinxAtStartPar
The location where you download OpihiExarata is relatively irrelevant.

\sphinxAtStartPar
Throughout the installation tutorial we refer to the internal OpihiExarata
directory as \sphinxcode{\sphinxupquote{OpihiExarata/}}. This allows for the instructions to be
general. Please adapt any absolute paths as needed.


\subparagraph{Via .zip Archive}
\label{\detokenize{technical/installation/download:via-zip-archive}}
\sphinxAtStartPar
There are alternative ways to download the software. Although, the only other
method that is worthwhile to document is downloading the software as an
archive and extracting it to the desired location.

\sphinxAtStartPar
A zip archive of the git repository can be downloaded via:

\begin{sphinxVerbatim}[commandchars=\\\{\}]
\PYG{n}{curl} \PYG{o}{\PYGZhy{}}\PYG{n}{O} \PYG{o}{\PYGZhy{}}\PYG{n}{L} \PYG{n}{https}\PYG{p}{:}\PYG{o}{/}\PYG{o}{/}\PYG{n}{github}\PYG{o}{.}\PYG{n}{com}\PYG{o}{/}\PYG{n}{psmd}\PYG{o}{\PYGZhy{}}\PYG{n}{iberutaru}\PYG{o}{/}\PYG{n}{OpihiExarata}\PYG{o}{/}\PYG{n}{archive}\PYG{o}{/}\PYG{n}{refs}\PYG{o}{/}\PYG{n}{heads}\PYG{o}{/}\PYG{n}{master}\PYG{o}{.}\PYG{n}{zip}
\PYG{n}{unzip} \PYG{n}{master}\PYG{o}{.}\PYG{n}{zip}
\PYG{c+c1}{\PYGZsh{} Optional; to follow the documentation naming conventions.}
\PYG{n}{mv} \PYG{n}{OpihiExarata}\PYG{o}{\PYGZhy{}}\PYG{n}{master} \PYG{n}{OpihiExarata}
\PYG{c+c1}{\PYGZsh{} The zip is no longer needed.}
\PYG{n}{rm} \PYG{n}{master}\PYG{o}{.}\PYG{n}{zip}
\end{sphinxVerbatim}

\sphinxAtStartPar
Throughout the installation tutorial we refer to the internal OpihiExarata
directory as \sphinxcode{\sphinxupquote{OpihiExarata/}}. This allows for the instructions to be
general. Please adapt any absolute paths as needed.

\sphinxstepscope


\paragraph{Install: Python Part}
\label{\detokenize{technical/installation/python:install-python-part}}\label{\detokenize{technical/installation/python:technical-installation-python-part}}\label{\detokenize{technical/installation/python::doc}}
\sphinxAtStartPar
The Python part of OpihiExarata is the primary part of OpihiExarata. Luckily,
it is likely also the simplest.

\sphinxAtStartPar
You will build the Python wheel from the source and install it.


\subparagraph{Download}
\label{\detokenize{technical/installation/python:download}}
\sphinxAtStartPar
Download and install a current release of Python if you have not already. You
can find the latest releases at: \sphinxhref{https://www.python.org/downloads/}{Python Releases}%
\begin{footnote}[27]\sphinxAtStartFootnote
\sphinxnolinkurl{https://www.python.org/downloads/}
%
\end{footnote}.

\sphinxAtStartPar
Please note you must install/have Python 3.9+.


\subparagraph{Build}
\label{\detokenize{technical/installation/python:build}}
\sphinxAtStartPar
You should have have the repository code in \sphinxcode{\sphinxupquote{OpihiExarata/}}. You can
run the build command (while in the directory) based on the operation system
you are using:

\sphinxAtStartPar
Windows: \sphinxstyleliteralstrong{\sphinxupquote{python \sphinxhyphen{}m build}}

\sphinxAtStartPar
Linux: \sphinxstyleliteralstrong{\sphinxupquote{python39 \sphinxhyphen{}m build}}

\sphinxAtStartPar
The package will build into a distributable wheel, note in the output the
version that you installed. The version is likely to be the day you built it,
it affects the name of the wheel file for the next step.


\subparagraph{Install}
\label{\detokenize{technical/installation/python:install}}
\sphinxAtStartPar
Install the wheel package.

\sphinxAtStartPar
You will need to modify this command to the proper wheel that you generated
from before. Because this project using date\sphinxhyphen{}based versioning, your package
will likely be that of the date that you build the wheel.

\sphinxAtStartPar
(The \sphinxcode{\sphinxupquote{\sphinxhyphen{}\sphinxhyphen{}upgrade}} option is to ensure that you have the most up to date
version.)

\sphinxAtStartPar
You can install the wheel using while in \sphinxcode{\sphinxupquote{OpihiExarata/}}:

\begin{sphinxVerbatim}[commandchars=\\\{\}]
\PYG{n}{pip} \PYG{n}{install} \PYG{o}{.}\PYG{o}{/}\PYG{n}{dist}\PYG{o}{/}\PYG{n}{OpihiExarata}\PYG{o}{\PYGZhy{}}\PYG{n}{YYYY}\PYG{o}{.}\PYG{n}{MM}\PYG{o}{.}\PYG{n}{DD}\PYG{o}{\PYGZhy{}}\PYG{n}{py3}\PYG{o}{\PYGZhy{}}\PYG{n}{none}\PYG{o}{\PYGZhy{}}\PYG{n+nb}{any}\PYG{o}{.}\PYG{n}{whl} \PYG{o}{\PYGZhy{}}\PYG{o}{\PYGZhy{}}\PYG{n}{upgrade}
\end{sphinxVerbatim}

\sphinxstepscope


\paragraph{Optional: Documentation}
\label{\detokenize{technical/installation/documentation:optional-documentation}}\label{\detokenize{technical/installation/documentation:technical-installation-documentation}}\label{\detokenize{technical/installation/documentation::doc}}
\sphinxAtStartPar
Building the documentation is relatively simple as we leverage Sphinx to
build it.


\subparagraph{Prerequisites}
\label{\detokenize{technical/installation/documentation:prerequisites}}

\subparagraph{Sphinx}
\label{\detokenize{technical/installation/documentation:sphinx}}
\sphinxAtStartPar
To build the Sphinx documentation, Python needs to be installed (the Python
version installed in this installation procedure can be used).

\sphinxAtStartPar
The following packages must be installed:

\begin{sphinxVerbatim}[commandchars=\\\{\}]
\PYG{n}{pip} \PYG{n}{install} \PYG{n}{sphinx} \PYG{n}{sphinx}\PYG{o}{\PYGZhy{}}\PYG{n}{rtd}\PYG{o}{\PYGZhy{}}\PYG{n}{theme}
\end{sphinxVerbatim}


\subparagraph{LaTeX}
\label{\detokenize{technical/installation/documentation:latex}}
\sphinxAtStartPar
If a PDF version of the documentation is desired, Sphinx can build it via
LaTeX. LaTeX must be installed. We suggest installing the
\sphinxhref{https://tug.org/texlive/}{TeX Live distribution}%
\begin{footnote}[28]\sphinxAtStartFootnote
\sphinxnolinkurl{https://tug.org/texlive/}
%
\end{footnote} and installing the full install. Instructions on
how to install TeX Live is beyond the scope of this documentation.


\subparagraph{Directory}
\label{\detokenize{technical/installation/documentation:directory}}
\sphinxAtStartPar
Change your directory to the \sphinxcode{\sphinxupquote{OpihiExarata/docs}} directory, run all of
the commands while within this directory.


\subparagraph{Build}
\label{\detokenize{technical/installation/documentation:build}}
\sphinxAtStartPar
\sphinxstylestrong{First}, the Python docstrings need to be processed into documentation. This
can be done via running:

\begin{sphinxVerbatim}[commandchars=\\\{\}]
\PYG{n}{sphinx}\PYG{o}{\PYGZhy{}}\PYG{n}{apidoc} \PYG{o}{\PYGZhy{}}\PYG{n}{f} \PYG{o}{\PYGZhy{}}\PYG{n}{e} \PYG{o}{\PYGZhy{}}\PYG{n}{o} \PYG{o}{.}\PYG{o}{/}\PYG{n}{code}\PYG{o}{/} \PYG{o}{.}\PYG{o}{/}\PYG{o}{.}\PYG{o}{.}\PYG{o}{/}\PYG{n}{src}\PYG{o}{/}\PYG{n}{opihiexarata}\PYG{o}{/}
\end{sphinxVerbatim}

\sphinxAtStartPar
\sphinxstylestrong{Second}, it is helpful to removed the cached versions of the documentation
files, you can do this via the commands: (This just removes the build directory
along with other goodies.)
\begin{itemize}
\item {} 
\sphinxAtStartPar
Windows: \sphinxstyleliteralstrong{\sphinxupquote{.make.bat clean}}

\item {} 
\sphinxAtStartPar
Linux: \sphinxstyleliteralstrong{\sphinxupquote{make clean}}

\end{itemize}

\sphinxAtStartPar
\sphinxstylestrong{Third}, the documentation can be built using the batch/makefile using the
command:
\begin{itemize}
\item {} 
\sphinxAtStartPar
Windows: \sphinxstyleliteralstrong{\sphinxupquote{.make.bat <type>}}

\item {} 
\sphinxAtStartPar
Linux: \sphinxstyleliteralstrong{\sphinxupquote{make <type>}}

\end{itemize}

\sphinxAtStartPar
The \sphinxcode{\sphinxupquote{<type>}} should be replaced with the type of output desired,
suggestions below:
\begin{itemize}
\item {} 
\sphinxAtStartPar
\sphinxcode{\sphinxupquote{html}} A collection of webpages ordered and structured. This is the suggested method.

\item {} 
\sphinxAtStartPar
\sphinxcode{\sphinxupquote{singlehtml}} A single HTML page; useful when sending the documentation between devices.

\item {} 
\sphinxAtStartPar
\sphinxcode{\sphinxupquote{latexpdf}} A PDF compiled by built LaTeX files. Using this option invokes the entire toolchain for you.

\end{itemize}

\sphinxAtStartPar
The documentation can be built to other different forms, see the
\sphinxhref{https://www.sphinx-doc.org/en/master/man/sphinx-build.html}{sphinx\sphinxhyphen{}build documentation}%
\begin{footnote}[29]\sphinxAtStartFootnote
\sphinxnolinkurl{https://www.sphinx-doc.org/en/master/man/sphinx-build.html}
%
\end{footnote}.

\sphinxstepscope


\paragraph{Optional: Windows Compatibility}
\label{\detokenize{technical/installation/windows:optional-windows-compatibility}}\label{\detokenize{technical/installation/windows:technical-installation-windows-compatibility}}\label{\detokenize{technical/installation/windows::doc}}
\sphinxAtStartPar
Many of the parts of OpihiExarata are built for Linux\sphinxhyphen{}only operating systems.
However, Windows can use a Linux subsystem to call the compiled Linux parts,
allowing OpihiExarata to run on Windows, with a set of assumptions.

\sphinxAtStartPar
When a Linux call is needed, the software calls it via the Windows Subsystem
for Linux (WSL) using Powershell. Thus WSL must be installed and Powershell
must have script mode enabled.


\subparagraph{Powershell Execution Policy}
\label{\detokenize{technical/installation/windows:powershell-execution-policy}}
\sphinxAtStartPar
Powershell scripts are seen as the most reliable way to call WSL on Windows.
But, the default execution policy of most Windows machines disable scripts. To
enable scripts, do the following command in an administrator Powershell shell:

\begin{sphinxVerbatim}[commandchars=\\\{\}]
\PYG{n}{Set}\PYG{o}{\PYGZhy{}}\PYG{n}{ExecutionPolicy} \PYG{o}{\PYGZhy{}}\PYG{n}{ExecutionPolicy} \PYG{n}{RemoteSigned}
\end{sphinxVerbatim}

\sphinxAtStartPar
See Microsoft’s documentation for more information on
\sphinxhref{https://docs.microsoft.com/en-us/powershell/module/microsoft.powershell.security/set-executionpolicy}{setting Powershell execution policies}%
\begin{footnote}[30]\sphinxAtStartFootnote
\sphinxnolinkurl{https://docs.microsoft.com/en-us/powershell/module/microsoft.powershell.security/set-executionpolicy}
%
\end{footnote} and \sphinxhref{https://docs.microsoft.com/en-us/powershell/module/microsoft.powershell.core/about/about\_execution\_policies}{what policies are available}%
\begin{footnote}[31]\sphinxAtStartFootnote
\sphinxnolinkurl{https://docs.microsoft.com/en-us/powershell/module/microsoft.powershell.core/about/about\_execution\_policies}
%
\end{footnote}.


\subparagraph{Windows Subsystem for Linux}
\label{\detokenize{technical/installation/windows:windows-subsystem-for-linux}}
\sphinxAtStartPar
As the WSL system is slightly new, the steps to install it differ based on
what version of Windows is installed.

\sphinxAtStartPar
For Windows 10 version 2004 and higher (Build 19041 and higher) or Windows 11
and higher, see {\hyperref[\detokenize{technical/installation/windows:modern-install}]{\sphinxcrossref{Modern Install}}}.

\sphinxAtStartPar
For older versions of Windows 10 (version 1903 or higher, with build 18362 or
higher), see {\hyperref[\detokenize{technical/installation/windows:manual-install}]{\sphinxcrossref{Manual Install}}}.

\sphinxAtStartPar
OpihiExarata does not support or is designed to run on older versions of
Windows.


\subparagraph{Modern Install}
\label{\detokenize{technical/installation/windows:modern-install}}
\sphinxAtStartPar
In newer Windows versions (Windows 10 version 2004 and higher (Build 19041
and higher) or Windows 11), the installation of WSL is relatively simple.

\sphinxAtStartPar
In an administrator Powershell, run the following:

\begin{sphinxVerbatim}[commandchars=\\\{\}]
\PYG{n}{wsl} \PYG{o}{\PYGZhy{}}\PYG{o}{\PYGZhy{}}\PYG{n}{install}
\end{sphinxVerbatim}

\sphinxAtStartPar
This installs Ubuntu as the WSL distribution by default. See the
\sphinxhref{https://docs.microsoft.com/en-us/windows/wsl/install}{WSL install documentation}%
\begin{footnote}[32]\sphinxAtStartFootnote
\sphinxnolinkurl{https://docs.microsoft.com/en-us/windows/wsl/install}
%
\end{footnote} for more information.


\subparagraph{Manual Install}
\label{\detokenize{technical/installation/windows:manual-install}}
\sphinxAtStartPar
These installation instructions was taken from the \sphinxhref{https://docs.microsoft.com/en-us/windows/wsl/install-manual}{manual installation documentation}%
\begin{footnote}[33]\sphinxAtStartFootnote
\sphinxnolinkurl{https://docs.microsoft.com/en-us/windows/wsl/install-manual}
%
\end{footnote}
from Microsoft.

\sphinxAtStartPar
\sphinxstylestrong{Enable WSL}

\sphinxAtStartPar
In an administrator Powershell, enable the WSL system via:

\begin{sphinxVerbatim}[commandchars=\\\{\}]
\PYG{n}{dism}\PYG{o}{.}\PYG{n}{exe} \PYG{o}{/}\PYG{n}{online} \PYG{o}{/}\PYG{n}{enable}\PYG{o}{\PYGZhy{}}\PYG{n}{feature} \PYG{o}{/}\PYG{n}{featurename}\PYG{p}{:}\PYG{n}{Microsoft}\PYG{o}{\PYGZhy{}}\PYG{n}{Windows}\PYG{o}{\PYGZhy{}}\PYG{n}{Subsystem}\PYG{o}{\PYGZhy{}}\PYG{n}{Linux} \PYG{o}{/}\PYG{n+nb}{all} \PYG{o}{/}\PYG{n}{norestart}
\end{sphinxVerbatim}

\sphinxAtStartPar
\sphinxstylestrong{Enable Virtual Machine Feature}

\sphinxAtStartPar
In an administrator Powershell, enable the virtual machine platform:

\begin{sphinxVerbatim}[commandchars=\\\{\}]
\PYG{n}{dism}\PYG{o}{.}\PYG{n}{exe} \PYG{o}{/}\PYG{n}{online} \PYG{o}{/}\PYG{n}{enable}\PYG{o}{\PYGZhy{}}\PYG{n}{feature} \PYG{o}{/}\PYG{n}{featurename}\PYG{p}{:}\PYG{n}{VirtualMachinePlatform} \PYG{o}{/}\PYG{n+nb}{all} \PYG{o}{/}\PYG{n}{norestart}
\end{sphinxVerbatim}

\sphinxAtStartPar
\sphinxstylestrong{Download Linux Kernel Update Package}

\sphinxAtStartPar
You can download the latest kernel package from:
\sphinxhref{https://wslstorestorage.blob.core.windows.net/wslblob/wsl\_update\_x64.msi}{WSL2 Linux kernel update package for x64 machines}%
\begin{footnote}[34]\sphinxAtStartFootnote
\sphinxnolinkurl{https://wslstorestorage.blob.core.windows.net/wslblob/wsl\_update\_x64.msi}
%
\end{footnote}.

\sphinxAtStartPar
Run the downloaded package; you will likely be prompted for administrative
permissions.

\sphinxAtStartPar
\sphinxstylestrong{Set WSL 2 as Default}

\sphinxAtStartPar
In an administrator Powershell, use the following command to set WSL 2 as the
default version:

\begin{sphinxVerbatim}[commandchars=\\\{\}]
\PYG{n}{wsl} \PYG{o}{\PYGZhy{}}\PYG{o}{\PYGZhy{}}\PYG{n+nb}{set}\PYG{o}{\PYGZhy{}}\PYG{n}{default}\PYG{o}{\PYGZhy{}}\PYG{n}{version} \PYG{l+m+mi}{2}
\end{sphinxVerbatim}

\sphinxAtStartPar
This the more typical WSL installation, rather than WSL 1 which is a
compatibility layer.

\sphinxAtStartPar
\sphinxstylestrong{Download Linux Distribution}

\sphinxAtStartPar
Download your desired Linux distribution from the Windows store, we
suggest \sphinxhref{https://www.microsoft.com/store/apps/9n6svws3rx71}{WSL Ubuntu}%
\begin{footnote}[35]\sphinxAtStartFootnote
\sphinxnolinkurl{https://www.microsoft.com/store/apps/9n6svws3rx71}
%
\end{footnote}:


\subparagraph{Verify}
\label{\detokenize{technical/installation/windows:verify}}
\sphinxAtStartPar
To verify that that the Powershell execution policy is correct, check that
the Local Machine policy is at least RemoteSigned. You can check this by using
an administrative Powershell command:

\begin{sphinxVerbatim}[commandchars=\\\{\}]
\PYG{n}{Get}\PYG{o}{\PYGZhy{}}\PYG{n}{ExecutionPolicy} \PYG{o}{\PYGZhy{}}\PYG{n}{List}
\end{sphinxVerbatim}

\sphinxAtStartPar
To verify that you have installed and set up WSL properly, in an administrative Powershell run:

\begin{sphinxVerbatim}[commandchars=\\\{\}]
\PYG{n}{wsl}
\end{sphinxVerbatim}

\sphinxAtStartPar
If this is the first time you have installed the WSL system onto your computer,
you will have to setup a user account similar to a fresh install. Follow and
complete the prompt, and then check \sphinxstyleliteralstrong{\sphinxupquote{wsl}} again to ensure that it
enters a standard Linux shell without any interruptions.


\subparagraph{Final}
\label{\detokenize{technical/installation/windows:final}}
\sphinxAtStartPar
When installing other parts of the OpihiExarata software, follow the
instructions using the WSL OS installed where needed. The commands should be
run in your usual WSL shell in the proper relevant directory.

\sphinxstepscope


\paragraph{Install: Orbfit}
\label{\detokenize{technical/installation/orbfit:install-orbfit}}\label{\detokenize{technical/installation/orbfit:technical-installation-orbfit}}\label{\detokenize{technical/installation/orbfit::doc}}
\sphinxAtStartPar
Orbfit is one of the available asteroid orbit determiners.

\sphinxAtStartPar
The instructions of installing Orbfit is derived from the
\sphinxhref{http://adams.dm.unipi.it/~orbmaint/orbfit/OrbFit/doc/help.html\#install}{Orbfit installation documentation}%
\begin{footnote}[36]\sphinxAtStartFootnote
\sphinxnolinkurl{http://adams.dm.unipi.it/~orbmaint/orbfit/OrbFit/doc/help.html\#install}
%
\end{footnote}.


\subparagraph{Prerequisites}
\label{\detokenize{technical/installation/orbfit:prerequisites}}
\sphinxAtStartPar
There are a few prerequisites before the software can be installed.


\subparagraph{Directory}
\label{\detokenize{technical/installation/orbfit:directory}}
\sphinxAtStartPar
Enter into a directory which is accessible and usable by programs. In this
documentation, we will call this directory \sphinxcode{\sphinxupquote{Orbfit/}}. You should do the
commands of this section of the installation documentation within this
directory.


\subparagraph{Dependencies}
\label{\detokenize{technical/installation/orbfit:dependencies}}
\sphinxAtStartPar
The following dependencies should be installed. The installation of these are
dependent on the package manager and operating system used.

\sphinxAtStartPar
For APT\sphinxhyphen{}based systems:

\begin{sphinxVerbatim}[commandchars=\\\{\}]
\PYG{n}{sudo} \PYG{n}{apt} \PYG{n}{install} \PYG{n}{gcc} \PYG{n}{make} \PYG{n}{curl} \PYG{n}{time}
\end{sphinxVerbatim}

\sphinxAtStartPar
For YUM, DNF, or Zypper based systems:

\begin{sphinxVerbatim}[commandchars=\\\{\}]
\PYG{n}{sudo} \PYG{n}{yum} \PYG{n}{install} \PYG{n}{gcc} \PYG{n}{make} \PYG{n}{curl} \PYG{n}{time}
\end{sphinxVerbatim}


\subparagraph{Fortran Compiler}
\label{\detokenize{technical/installation/orbfit:fortran-compiler}}
\sphinxAtStartPar
A Fortran compiler is needed to compile the software. Although there are many
different compilers available, we document here how to use the Intel Fortran
compiler. The method of install is OS dependent, or more specifically, package
manager dependent. Other installation method exist, see
\sphinxhref{https://www.intel.com/content/www/us/en/develop/documentation/installation-guide-for-intel-oneapi-toolkits-linux/top/installation.html}{Intel Fortran compiler install}%
\begin{footnote}[37]\sphinxAtStartFootnote
\sphinxnolinkurl{https://www.intel.com/content/www/us/en/develop/documentation/installation-guide-for-intel-oneapi-toolkits-linux/top/installation.html}
%
\end{footnote} for more information (these instructions were
derived from it).


\subparagraph{Install}
\label{\detokenize{technical/installation/orbfit:install}}
\sphinxAtStartPar
Using package managers to install is the most convenient.


\subparagraph{APT Package Managers}
\label{\detokenize{technical/installation/orbfit:apt-package-managers}}
\sphinxAtStartPar
Download the Intel repository to the system keyring:

\begin{sphinxVerbatim}[commandchars=\\\{\}]
\PYG{c+c1}{\PYGZsh{} Downloading key...}
\PYG{n}{wget} \PYG{o}{\PYGZhy{}}\PYG{n}{O}\PYG{o}{\PYGZhy{}} \PYG{n}{https}\PYG{p}{:}\PYG{o}{/}\PYG{o}{/}\PYG{n}{apt}\PYG{o}{.}\PYG{n}{repos}\PYG{o}{.}\PYG{n}{intel}\PYG{o}{.}\PYG{n}{com}\PYG{o}{/}\PYG{n}{intel}\PYG{o}{\PYGZhy{}}\PYG{n}{gpg}\PYG{o}{\PYGZhy{}}\PYG{n}{keys}\PYG{o}{/}\PYG{n}{GPG}\PYG{o}{\PYGZhy{}}\PYG{n}{PUB}\PYG{o}{\PYGZhy{}}\PYG{n}{KEY}\PYG{o}{\PYGZhy{}}\PYG{n}{INTEL}\PYG{o}{\PYGZhy{}}\PYG{n}{SW}\PYG{o}{\PYGZhy{}}\PYG{n}{PRODUCTS}\PYG{o}{.}\PYG{n}{PUB} \PYGZbs{}
\PYG{o}{|} \PYG{n}{gpg} \PYG{o}{\PYGZhy{}}\PYG{o}{\PYGZhy{}}\PYG{n}{dearmor} \PYG{o}{|} \PYG{n}{sudo} \PYG{n}{tee} \PYG{o}{/}\PYG{n}{usr}\PYG{o}{/}\PYG{n}{share}\PYG{o}{/}\PYG{n}{keyrings}\PYG{o}{/}\PYG{n}{oneapi}\PYG{o}{\PYGZhy{}}\PYG{n}{archive}\PYG{o}{\PYGZhy{}}\PYG{n}{keyring}\PYG{o}{.}\PYG{n}{gpg} \PYG{o}{\PYGZgt{}} \PYG{o}{/}\PYG{n}{dev}\PYG{o}{/}\PYG{n}{null}
\PYG{c+c1}{\PYGZsh{} Adding signed entry...}
\PYG{n}{echo} \PYG{l+s+s2}{\PYGZdq{}}\PYG{l+s+s2}{deb [signed\PYGZhy{}by=/usr/share/keyrings/oneapi\PYGZhy{}archive\PYGZhy{}keyring.gpg] https://apt.repos.intel.com/oneapi all main}\PYG{l+s+s2}{\PYGZdq{}} \PYG{o}{|} \PYG{n}{sudo} \PYG{n}{tee} \PYG{o}{/}\PYG{n}{etc}\PYG{o}{/}\PYG{n}{apt}\PYG{o}{/}\PYG{n}{sources}\PYG{o}{.}\PYG{n}{list}\PYG{o}{.}\PYG{n}{d}\PYG{o}{/}\PYG{n}{oneAPI}\PYG{o}{.}\PYG{n}{list}
\PYG{c+c1}{\PYGZsh{} Update the repository list...}
\PYG{n}{sudo} \PYG{n}{apt} \PYG{n}{update}
\end{sphinxVerbatim}

\sphinxAtStartPar
Install the Intel Fortran compiler, the base toolkit is also needed as a
requirement:

\begin{sphinxVerbatim}[commandchars=\\\{\}]
\PYG{n}{sudo} \PYG{n}{apt} \PYG{n}{install} \PYG{n}{intel}\PYG{o}{\PYGZhy{}}\PYG{n}{basekit}
\PYG{n}{sudo} \PYG{n}{apt} \PYG{n}{install} \PYG{n}{intel}\PYG{o}{\PYGZhy{}}\PYG{n}{hpckit}
\end{sphinxVerbatim}

\sphinxAtStartPar
It is highly suggested to update or install the following toolchains:

\begin{sphinxVerbatim}[commandchars=\\\{\}]
\PYG{n}{sudo} \PYG{n}{apt} \PYG{n}{update}
\PYG{n}{sudo} \PYG{n}{apt} \PYG{n}{install} \PYG{n}{cmake} \PYG{n}{pkg}\PYG{o}{\PYGZhy{}}\PYG{n}{config} \PYG{n}{build}\PYG{o}{\PYGZhy{}}\PYG{n}{essential}
\end{sphinxVerbatim}


\subparagraph{YUM, DNF, Zypper Package Managers}
\label{\detokenize{technical/installation/orbfit:yum-dnf-zypper-package-managers}}
\sphinxAtStartPar
Download the Intel repository file and add it to the configuration directory:

\begin{sphinxVerbatim}[commandchars=\\\{\}]
\PYGZsh{} Creating repository file...
tee \PYGZgt{} /tmp/oneAPI.repo \PYGZlt{}\PYGZlt{} EOF
[oneAPI]
name=Intel® oneAPI repository
baseurl=https://yum.repos.intel.com/oneapi
enabled=1
gpgcheck=1
repo\PYGZus{}gpgcheck=1
gpgkey=https://yum.repos.intel.com/intel\PYGZhy{}gpg\PYGZhy{}keys/GPG\PYGZhy{}PUB\PYGZhy{}KEY\PYGZhy{}INTEL\PYGZhy{}SW\PYGZhy{}PRODUCTS.PUB
EOF
\PYGZsh{} Moving the file to the directory...
sudo mv /tmp/oneAPI.repo /etc/yum.repos.d
\PYGZsh{} Update the repository list...
sudo yum update
\end{sphinxVerbatim}

\sphinxAtStartPar
If you are using Zypper:

\begin{sphinxVerbatim}[commandchars=\\\{\}]
\PYG{c+c1}{\PYGZsh{} Creating repository file...}
\PYG{n}{sudo} \PYG{n}{zypper} \PYG{n}{addrepo} \PYG{n}{https}\PYG{p}{:}\PYG{o}{/}\PYG{o}{/}\PYG{n}{yum}\PYG{o}{.}\PYG{n}{repos}\PYG{o}{.}\PYG{n}{intel}\PYG{o}{.}\PYG{n}{com}\PYG{o}{/}\PYG{n}{oneapi} \PYG{n}{oneAPI}
\PYG{n}{rpm} \PYG{o}{\PYGZhy{}}\PYG{o}{\PYGZhy{}}\PYG{k+kn}{import} \PYG{n+nn}{https}\PYG{p}{:}\PYG{o}{/}\PYG{o}{/}\PYG{n}{yum}\PYG{o}{.}\PYG{n}{repos}\PYG{o}{.}\PYG{n}{intel}\PYG{o}{.}\PYG{n}{com}\PYG{o}{/}\PYG{n}{intel}\PYG{o}{\PYGZhy{}}\PYG{n}{gpg}\PYG{o}{\PYGZhy{}}\PYG{n}{keys}\PYG{o}{/}\PYG{n}{GPG}\PYG{o}{\PYGZhy{}}\PYG{n}{PUB}\PYG{o}{\PYGZhy{}}\PYG{n}{KEY}\PYG{o}{\PYGZhy{}}\PYG{n}{INTEL}\PYG{o}{\PYGZhy{}}\PYG{n}{SW}\PYG{o}{\PYGZhy{}}\PYG{n}{PRODUCTS}\PYG{o}{.}\PYG{n}{PUB}
\PYG{c+c1}{\PYGZsh{} Update the repository list...}
\PYG{n}{sudo} \PYG{n}{zypper} \PYG{n}{update}
\end{sphinxVerbatim}

\sphinxAtStartPar
Install the Intel Fortran compiler, the base toolkit is also needed as a
requirement:

\begin{sphinxVerbatim}[commandchars=\\\{\}]
\PYG{n}{sudo} \PYG{n}{yum} \PYG{n}{install} \PYG{n}{intel}\PYG{o}{\PYGZhy{}}\PYG{n}{basekit}
\PYG{n}{sudo} \PYG{n}{yum} \PYG{n}{install} \PYG{n}{intel}\PYG{o}{\PYGZhy{}}\PYG{n}{hpckit}
\end{sphinxVerbatim}

\sphinxAtStartPar
It is highly suggested to update or install the following toolchains:

\begin{sphinxVerbatim}[commandchars=\\\{\}]
\PYG{n}{sudo} \PYG{n}{yum} \PYG{n}{update}
\PYG{n}{sudo} \PYG{n}{yum} \PYG{n}{install} \PYG{n}{cmake} \PYG{n}{pkgconfig}
\PYG{n}{sudo} \PYG{n}{yum} \PYG{n}{groupinstall} \PYG{l+s+s2}{\PYGZdq{}}\PYG{l+s+s2}{Development Tools}\PYG{l+s+s2}{\PYGZdq{}}
\end{sphinxVerbatim}

\sphinxAtStartPar
For Zypper, both is done via:

\begin{sphinxVerbatim}[commandchars=\\\{\}]
\PYG{n}{sudo} \PYG{n}{zypper} \PYG{n}{install} \PYG{n}{intel}\PYG{o}{\PYGZhy{}}\PYG{n}{basekit}
\PYG{n}{sudo} \PYG{n}{zypper} \PYG{n}{install} \PYG{n}{intel}\PYG{o}{\PYGZhy{}}\PYG{n}{hpckit}
\PYG{n}{sudo} \PYG{n}{zypper} \PYG{n}{update}
\PYG{n}{sudo} \PYG{n}{zypper} \PYG{n}{install} \PYG{n}{cmake} \PYG{n}{pkg}\PYG{o}{\PYGZhy{}}\PYG{n}{config}
\PYG{n}{sudo} \PYG{n}{zypper} \PYG{n}{install} \PYG{n}{pattern} \PYG{n}{devel\PYGZus{}C\PYGZus{}C}\PYG{o}{+}\PYG{o}{+}
\end{sphinxVerbatim}


\subparagraph{Environment Variables}
\label{\detokenize{technical/installation/orbfit:environment-variables}}
\sphinxAtStartPar
Set the environment variables for command\sphinxhyphen{}line development. This command sets
the variables for the current session only. (For persistent environment
variable set up,consider adding it to your startup script.):

\begin{sphinxVerbatim}[commandchars=\\\{\}]
\PYG{o}{.} \PYG{o}{/}\PYG{n}{opt}\PYG{o}{/}\PYG{n}{intel}\PYG{o}{/}\PYG{n}{oneapi}\PYG{o}{/}\PYG{n}{setvars}\PYG{o}{.}\PYG{n}{sh}
\end{sphinxVerbatim}


\subparagraph{Download OrbFit}
\label{\detokenize{technical/installation/orbfit:download-orbfit}}
\sphinxAtStartPar
The software needs to be downloaded.

\sphinxAtStartPar
You can likely find the software package on the \sphinxhref{http://adams.dm.unipi.it/orbfit/}{OrbFit website}%
\begin{footnote}[38]\sphinxAtStartFootnote
\sphinxnolinkurl{http://adams.dm.unipi.it/orbfit/}
%
\end{footnote}. Otherwise,
a download command may work:

\begin{sphinxVerbatim}[commandchars=\\\{\}]
\PYG{n}{curl} \PYG{o}{\PYGZhy{}}\PYG{n}{O} \PYG{n}{http}\PYG{p}{:}\PYG{o}{/}\PYG{o}{/}\PYG{n}{adams}\PYG{o}{.}\PYG{n}{dm}\PYG{o}{.}\PYG{n}{unipi}\PYG{o}{.}\PYG{n}{it}\PYG{o}{/}\PYG{n}{orbfit}\PYG{o}{/}\PYG{n}{OrbFit5}\PYG{l+m+mf}{.0}\PYG{l+m+mf}{.7}\PYG{o}{.}\PYG{n}{tar}\PYG{o}{.}\PYG{n}{gz}
\end{sphinxVerbatim}

\sphinxAtStartPar
And it can thus be extracted:

\begin{sphinxVerbatim}[commandchars=\\\{\}]
\PYG{n}{tar} \PYG{o}{\PYGZhy{}}\PYG{n}{xvzf} \PYG{n}{OrbFit5}\PYG{l+m+mf}{.0}\PYG{l+m+mf}{.7}\PYG{o}{.}\PYG{n}{tar}\PYG{o}{.}\PYG{n}{gz}
\end{sphinxVerbatim}


\subparagraph{Compile}
\label{\detokenize{technical/installation/orbfit:compile}}
\sphinxAtStartPar
To configure the compilation flags, OrbFit comes with a set of files which
describe the flags. Initialize the proper compilation flags via the
\sphinxstyleliteralstrong{\sphinxupquote{config}} command, (flags for an optimized build using the Intel
compiler):

\begin{sphinxVerbatim}[commandchars=\\\{\}]
\PYG{o}{.}\PYG{o}{/}\PYG{n}{config} \PYG{o}{\PYGZhy{}}\PYG{n}{O} \PYG{n}{intel}
\end{sphinxVerbatim}

\sphinxAtStartPar
The generated compilation flags, however, needs to be changed. The generated
compilation flags can be found in the file \sphinxcode{\sphinxupquote{Orbfit/src/make.flags}}, as
generated by the \sphinxstyleliteralstrong{\sphinxupquote{config}} script. The options should be
(see \sphinxhref{https://www.intel.com/content/www/us/en/develop/documentation/fortran-compiler-oneapi-dev-guide-and-reference/top/compiler-reference/compiler-options/alphabetical-list-of-compiler-options.html}{Intel Fortran compiler options}%
\begin{footnote}[39]\sphinxAtStartFootnote
\sphinxnolinkurl{https://www.intel.com/content/www/us/en/develop/documentation/fortran-compiler-oneapi-dev-guide-and-reference/top/compiler-reference/compiler-options/alphabetical-list-of-compiler-options.html}
%
\end{footnote} for more information):

\begin{sphinxVerbatim}[commandchars=\\\{\}]
\PYG{n}{FFLAGS}\PYG{o}{=} \PYG{o}{\PYGZhy{}}\PYG{n}{warn} \PYG{n}{nousage} \PYG{o}{\PYGZhy{}}\PYG{n}{O} \PYG{o}{\PYGZhy{}}\PYG{n}{mp1} \PYG{o}{\PYGZhy{}}\PYG{n}{static}\PYG{o}{\PYGZhy{}}\PYG{n}{intel} \PYG{o}{\PYGZhy{}}\PYG{n}{save} \PYG{o}{\PYGZhy{}}\PYG{n}{assume} \PYG{n}{byterecl} \PYG{o}{\PYGZhy{}}\PYG{n}{I}\PYG{o}{.}\PYG{o}{.}\PYG{o}{/}\PYG{n}{include}
\end{sphinxVerbatim}

\begin{sphinxadmonition}{note}{Note:}
\sphinxAtStartPar
You make use the default compilation flags as well; the change to the
flags just allows the program to run with the Intel Fortran libraries
statically compiled with the program rather than linked in. This alleviates
the need for always needing to set up the environment variables whenever
OrbFit is run. The changing memory model does not seem to affect the orbit
determination part of the program.
\end{sphinxadmonition}

\sphinxAtStartPar
Finally the program can be compiled using the makefile:

\begin{sphinxVerbatim}[commandchars=\\\{\}]
\PYG{n}{make}
\end{sphinxVerbatim}

\sphinxAtStartPar
The compiled programs should exist in \sphinxcode{\sphinxupquote{Orbfit/bin/}}.


\subparagraph{Download JPL Ephemerides File}
\label{\detokenize{technical/installation/orbfit:download-jpl-ephemerides-file}}
\sphinxAtStartPar
The OrbFit package uses the JPL ephemerides file for its calculations, it
requires the binary ephemerides files. Current documentation suggests using
the 405 ephemerides set, but more updated sets (DE 407) also exist. For Linux,
precomputed binaries can be found at \sphinxhref{https://ssd.jpl.nasa.gov/ftp/eph/planets/Linux/}{JPL Ephemerides binary files for Linux}%
\begin{footnote}[40]\sphinxAtStartFootnote
\sphinxnolinkurl{https://ssd.jpl.nasa.gov/ftp/eph/planets/Linux/}
%
\end{footnote}
and the \sphinxhref{https://ssd.jpl.nasa.gov/planets/eph\_export.html}{JPL Ephemerides descriptions}%
\begin{footnote}[41]\sphinxAtStartFootnote
\sphinxnolinkurl{https://ssd.jpl.nasa.gov/planets/eph\_export.html}
%
\end{footnote}.

\sphinxAtStartPar
For DE405:

\begin{sphinxVerbatim}[commandchars=\\\{\}]
\PYG{n}{curl} \PYG{o}{\PYGZhy{}}\PYG{n}{O} \PYG{n}{https}\PYG{p}{:}\PYG{o}{/}\PYG{o}{/}\PYG{n}{ssd}\PYG{o}{.}\PYG{n}{jpl}\PYG{o}{.}\PYG{n}{nasa}\PYG{o}{.}\PYG{n}{gov}\PYG{o}{/}\PYG{n}{ftp}\PYG{o}{/}\PYG{n}{eph}\PYG{o}{/}\PYG{n}{planets}\PYG{o}{/}\PYG{n}{Linux}\PYG{o}{/}\PYG{n}{de405}\PYG{o}{/}\PYG{n}{lnxp1600p2200}\PYG{l+m+mf}{.405}
\end{sphinxVerbatim}

\sphinxAtStartPar
For DE440 (used in this documentation):

\begin{sphinxVerbatim}[commandchars=\\\{\}]
\PYG{n}{curl} \PYG{o}{\PYGZhy{}}\PYG{n}{O} \PYG{n}{https}\PYG{p}{:}\PYG{o}{/}\PYG{o}{/}\PYG{n}{ssd}\PYG{o}{.}\PYG{n}{jpl}\PYG{o}{.}\PYG{n}{nasa}\PYG{o}{.}\PYG{n}{gov}\PYG{o}{/}\PYG{n}{ftp}\PYG{o}{/}\PYG{n}{eph}\PYG{o}{/}\PYG{n}{planets}\PYG{o}{/}\PYG{n}{Linux}\PYG{o}{/}\PYG{n}{de440}\PYG{o}{/}\PYG{n}{linux\PYGZus{}p1550p2650}\PYG{l+m+mf}{.440}
\end{sphinxVerbatim}

\sphinxAtStartPar
The downloaded binary ephemerides file must be linked so that the OrbFit
program can properly utilize it; using a symbolic link in the
\sphinxcode{\sphinxupquote{Orbfit/lib/}} directory, and assuming the DE440 file was downloaded (the
command below should be changed to fit your file):

\begin{sphinxVerbatim}[commandchars=\\\{\}]
\PYG{n}{cd} \PYG{o}{.}\PYG{o}{/}\PYG{n}{lib}\PYG{o}{/}
\PYG{n}{ln} \PYG{o}{\PYGZhy{}}\PYG{n}{s} \PYG{o}{.}\PYG{o}{/}\PYG{o}{.}\PYG{o}{.}\PYG{o}{/}\PYG{n}{linux\PYGZus{}p1550p2650}\PYG{l+m+mf}{.440} \PYG{n}{jpleph}
\PYG{c+c1}{\PYGZsh{} Back to Orbfit/.}
\PYG{n}{cd} \PYG{o}{.}\PYG{o}{.}
\end{sphinxVerbatim}


\subparagraph{Testing Suite}
\label{\detokenize{technical/installation/orbfit:testing-suite}}
\sphinxAtStartPar
The software’s test suit can be executed from \sphinxcode{\sphinxupquote{Orbfit/}} via:

\begin{sphinxVerbatim}[commandchars=\\\{\}]
\PYG{n}{make} \PYG{n}{tests}
\end{sphinxVerbatim}

\sphinxAtStartPar
This ensures that the program has been installed correctly.


\subparagraph{Executable Path for Configuration File}
\label{\detokenize{technical/installation/orbfit:executable-path-for-configuration-file}}
\sphinxAtStartPar
For this program’s executables to be used, the path that they exist in must
be known to OpihiExarata. Copy the output of the working directory command
and add it to the configuration file’s entries as noted. The path should be
an absolute path; these commands should be run in the \sphinxcode{\sphinxupquote{Orbfit/}} directory:

\begin{sphinxVerbatim}[commandchars=\\\{\}]
\PYG{n}{cd} \PYG{o}{.}\PYG{p}{;} \PYG{n}{echo} \PYG{l+s+s2}{\PYGZdq{}}\PYG{l+s+s2}{ORBFIT\PYGZus{}DIRECTORY ==}\PYG{l+s+s2}{\PYGZdq{}}\PYG{p}{;} \PYG{n}{pwd}
\PYG{n}{cd} \PYG{o}{.}\PYG{o}{/}\PYG{n+nb}{bin}\PYG{p}{;} \PYG{n}{echo} \PYG{l+s+s2}{\PYGZdq{}}\PYG{l+s+s2}{ORBFIT\PYGZus{}BINARY\PYGZus{}EXECUTABLE\PYGZus{}DIRECTORY ==}\PYG{l+s+s2}{\PYGZdq{}}\PYG{p}{;} \PYG{n}{pwd}\PYG{p}{;} \PYG{n}{cd} \PYG{o}{.}\PYG{o}{.}
\end{sphinxVerbatim}

\sphinxAtStartPar
If your main operating system is Windows and you are installing this via WSL
Ubuntu, use the following instead:

\begin{sphinxVerbatim}[commandchars=\\\{\}]
cd .; echo \PYGZdq{}ORBFIT\PYGZus{}DIRECTORY ==\PYGZdq{}; echo \PYGZdq{}\PYGZbs{}\PYGZbs{}\PYGZbs{}\PYGZbs{}wsl\PYGZdl{}/Ubuntu\PYGZdq{}\PYGZdl{}(pwd)
cd ./bin; echo \PYGZdq{}ORBFIT\PYGZus{}BINARY\PYGZus{}EXECUTABLE\PYGZus{}DIRECTORY ==\PYGZdq{}; pwd; cd ..
\end{sphinxVerbatim}


\subparagraph{OpihiExarata Template Files}
\label{\detokenize{technical/installation/orbfit:opihiexarata-template-files}}
\sphinxAtStartPar
In order for OpihiExarata to properly use the OrbFit program, files which are
used by OrbFit must be collected for OpihiExarata to leverage.

\sphinxAtStartPar
Create and enter the directory (starting from \sphinxcode{\sphinxupquote{Orbfit/}}):

\begin{sphinxVerbatim}[commandchars=\\\{\}]
\PYG{n}{mkdir} \PYG{n}{exarata}\PYG{p}{;} \PYG{n}{cd} \PYG{n}{exarata}
\end{sphinxVerbatim}

\sphinxAtStartPar
Within the \sphinxcode{\sphinxupquote{Orbfit/exarata/}} directory, copy over the asteroid propagation
files. This contains ephemerides data for asteroids to better propagations.
Generally, these files can be found in \sphinxcode{\sphinxupquote{Orbfit/tests/bineph/testout}}. For
some reason, the file extensions on these files are not in a form which is
liked by Orbfit, so they are also changed. Namely, the commands below should
work:

\begin{sphinxVerbatim}[commandchars=\\\{\}]
\PYG{c+c1}{\PYGZsh{} Copying the files...}
\PYG{n}{cp} \PYG{o}{.}\PYG{o}{/}\PYG{o}{.}\PYG{o}{.}\PYG{o}{/}\PYG{n}{tests}\PYG{o}{/}\PYG{n}{bineph}\PYG{o}{/}\PYG{n}{testout}\PYG{o}{/}\PYG{n}{AST17}\PYG{o}{.}\PYG{o}{*} \PYG{o}{.}
\PYG{n}{cp} \PYG{o}{.}\PYG{o}{/}\PYG{o}{.}\PYG{o}{.}\PYG{o}{/}\PYG{n}{tests}\PYG{o}{/}\PYG{n}{bineph}\PYG{o}{/}\PYG{n}{testout}\PYG{o}{/}\PYG{n}{CPV}\PYG{o}{.}\PYG{o}{*} \PYG{o}{.}
\PYG{c+c1}{\PYGZsh{} Extension changes...}
\PYG{n}{mv} \PYG{n}{AST17}\PYG{o}{.}\PYG{n}{bai\PYGZus{}431\PYGZus{}fcct} \PYG{n}{AST17}\PYG{o}{.}\PYG{n}{bai}
\PYG{n}{mv} \PYG{n}{AST17}\PYG{o}{.}\PYG{n}{bep\PYGZus{}431\PYGZus{}fcct} \PYG{n}{AST17}\PYG{o}{.}\PYG{n}{bep}
\PYG{n}{mv} \PYG{n}{CPV}\PYG{o}{.}\PYG{n}{bai\PYGZus{}431\PYGZus{}fcct} \PYG{n}{CPV}\PYG{o}{.}\PYG{n}{bai}
\PYG{n}{mv} \PYG{n}{CPV}\PYG{o}{.}\PYG{n}{bep\PYGZus{}431\PYGZus{}fcct} \PYG{n}{CPV}\PYG{o}{.}\PYG{n}{bep}
\PYG{n}{mv} \PYG{n}{CPV\PYGZus{}iter}\PYG{o}{.}\PYG{n}{bop} \PYG{n}{CPV}\PYG{o}{.}\PYG{n}{bop}
\end{sphinxVerbatim}

\sphinxAtStartPar
Also, create the files which are to be used to input data into Orbfit from
OpihiExarata. The main file that needs to be created is \sphinxcode{\sphinxupquote{exarata.oop}}.
Create it with your favorite text editor and fill the file and save it with
the following:

\begin{sphinxVerbatim}[commandchars=\\\{\}]
! First object
object1.
    .name = exarata        ! Object name
    .obs\PYGZus{}dir = \PYGZsq{}.\PYGZsq{}         ! Observations directory

! Elements output
output.
    .epoch = CAL 2022/01/01  00:00:00 UTC  ! The epoch time of the orbit
    .elements = \PYGZsq{}KEP\PYGZsq{}                      ! Kepler output elements

! Operations: preliminary orbits, differential corrections, identification
operations.
    .init\PYGZus{}orbdet = 1    ! Initial orbit determination
                        ! (0 = no, 1 = yes)
    .diffcor = 1        ! Differential correction
                        ! (0 = no, 1 = yes)
    .ident = 0          ! Orbit identification
                        ! (0 = no, 1 = yes)
    .ephem = 0          ! Ephemerides
                        ! (0 = no, 1 = yes)

! Error model
error\PYGZus{}model.
    .name=\PYGZsq{}fcct14\PYGZsq{}      ! Error model

! Propagation
propag.

    .iast=17         ! 0=no asteroids with mass, n=no. of massive asteroids (def=0)
    .filbe=\PYGZsq{}AST17\PYGZsq{}   ! name of the asteroid ephemerides file (def=\PYGZsq{}CPV\PYGZsq{})
    .npoint=600      ! minimum number of data points for a deep close appr (def=100)
    .dmea=0.2d0      ! min. distance for ctrl. of close\PYGZhy{}app. to Earth only (def=0.1)
        .dter=0.05d0    ! min. distance for control of close\PYGZhy{}app.
                        ! to terrestrial planets (MVM)(def=0.1)

! Additional options
IERS.
    .extrapolation=.T.  ! extrapolation of Earth rotation

reject.
    .rejopp=.FALSE.     ! reject entire opposition
\end{sphinxVerbatim}

\sphinxAtStartPar
The other two files you can create with the following commands:

\begin{sphinxVerbatim}[commandchars=\\\{\}]
\PYG{n}{echo} \PYG{l+s+s2}{\PYGZdq{}}\PYG{l+s+s2}{exarata}\PYG{l+s+s2}{\PYGZdq{}} \PYG{o}{\PYGZgt{}} \PYG{n}{exarata}\PYG{o}{.}\PYG{n}{inp}
\PYG{n}{touch} \PYG{n}{exarata}\PYG{o}{.}\PYG{n}{obs}
\end{sphinxVerbatim}

\begin{sphinxadmonition}{note}{Note:}
\sphinxAtStartPar
OpihiExarata will check that these three files exist as a check to see
that these steps have been followed correctly. It is also a way to ensure
that the configuration paths provided in the configuration are valid and
point to the right location.
\end{sphinxadmonition}

\sphinxstepscope


\subsection{Architecture}
\label{\detokenize{technical/architecture/index:architecture}}\label{\detokenize{technical/architecture/index:technical-architecture}}\label{\detokenize{technical/architecture/index::doc}}
\sphinxAtStartPar
hello

\sphinxstepscope


\subsubsection{Services and Engines}
\label{\detokenize{technical/architecture/services_engines:services-and-engines}}\label{\detokenize{technical/architecture/services_engines:technical-architecture-services-engines}}\label{\detokenize{technical/architecture/services_engines::doc}}
\sphinxAtStartPar
The OpihiExarata software primarily outsources solving the main five problems
for solving an image to other internal or external modules, those being:
\begin{itemize}
\item {} 
\sphinxAtStartPar
The image astrometric solution, the pointing of the image and its WCS solution;

\item {} 
\sphinxAtStartPar
The photometric solution, determined from a photometric star catalog;

\item {} 
\sphinxAtStartPar
The preliminary orbit determination, calculated from a record of observations;

\item {} 
\sphinxAtStartPar
The ephemeris, calculated from a set of provided orbital elements;

\item {} 
\sphinxAtStartPar
The asteroid propagation, calculated from a set of asteroid observations.

\end{itemize}

\sphinxAtStartPar
There are many different types of services and programs which are available
that are able to solve the above problems. We decided that it is good to
allow the user (or the program in general) to be able to customize/swap which
service they use to process their data. This interchangeability also allows
for OpihiExarata to be more stable as if a given service fails, a different
one can be selected to continue instead of said inability breaking the
software and workflow.

\sphinxAtStartPar
Each service has different interfaces and protocols for communication between
it and external problems like OpihiExarata; some have nice APIs while others
are more complicated.

\sphinxAtStartPar
To deal with the fact that each service has unique interfaces, we implemented
a wrapper/abstraction layer for each and every supported service. This
abstraction layer allows for easier implementation of these services.
These abstraction layers are called Engines and are implemented in as
subclasses from {\hyperref[\detokenize{code/opihiexarata.library.engine:module-opihiexarata.library.engine}]{\sphinxcrossref{\sphinxcode{\sphinxupquote{opihiexarata.library.engine}}}}}. These engines are
classified under
{\hyperref[\detokenize{technical/architecture/services_engines:technical-architecture-services-engines-astrometryengines}]{\sphinxcrossref{\DUrole{std,std-ref}{AstrometryEngines}}}},
{\hyperref[\detokenize{technical/architecture/services_engines:technical-architecture-services-engines-photometryengines}]{\sphinxcrossref{\DUrole{std,std-ref}{PhotometryEngines}}}},
{\hyperref[\detokenize{technical/architecture/services_engines:technical-architecture-services-engines-orbitengines}]{\sphinxcrossref{\DUrole{std,std-ref}{OrbitEngines}}}},
{\hyperref[\detokenize{technical/architecture/services_engines:technical-architecture-services-engines-ephemerisengines}]{\sphinxcrossref{\DUrole{std,std-ref}{EphemerisEngines}}}},
{\hyperref[\detokenize{technical/architecture/services_engines:technical-architecture-services-engines-propagateengines}]{\sphinxcrossref{\DUrole{std,std-ref}{PropagateEngines}}}}
depending on the problem it solves.

\sphinxAtStartPar
We detail all of the available engines (which solve the five problems) here.
Note that the names of the engines themselves (when in the code or otherwise)
are case insensitive.

\sphinxAtStartPar
A lot of these engines use other external services, see {\hyperref[\detokenize{user/citations:user-citations}]{\sphinxcrossref{\DUrole{std,std-ref}{Citations}}}}
for our references.


\paragraph{AstrometryEngines}
\label{\detokenize{technical/architecture/services_engines:astrometryengines}}\label{\detokenize{technical/architecture/services_engines:technical-architecture-services-engines-astrometryengines}}
\sphinxAtStartPar
These engines solve for the astrometric plate solution of an image.


\subparagraph{Astrometry.net Nova}
\label{\detokenize{technical/architecture/services_engines:astrometry-net-nova}}
\sphinxAtStartPar
Implementation: {\hyperref[\detokenize{code/opihiexarata.astrometry.webclient:opihiexarata.astrometry.webclient.AstrometryNetWebAPIEngine}]{\sphinxcrossref{\sphinxcode{\sphinxupquote{opihiexarata.astrometry.webclient.AstrometryNetWebAPIEngine}}}}}

\sphinxAtStartPar
This allows for the leveraging of the
\sphinxhref{https://nova.astrometry.net/}{Astrometry.net Nova online service}%
\begin{footnote}[42]\sphinxAtStartFootnote
\sphinxnolinkurl{https://nova.astrometry.net/}
%
\end{footnote} for
astrometric plate solving. It is a stable system able to solve most images.
However, as it is a public system, there is a queue so solving a particular
image may take longer from your perspective because of the added wait time.
This implementation is based off of the
\sphinxhref{https://github.com/dstndstn/astrometry.net/blob/main/net/client/client.py}{official version}%
\begin{footnote}[43]\sphinxAtStartFootnote
\sphinxnolinkurl{https://github.com/dstndstn/astrometry.net/blob/main/net/client/client.py}
%
\end{footnote}.

\sphinxAtStartPar
The images taken are uploaded to the service to be solved, OpihiExarata
periodically requests the solution, parsing it when the image is astrometrically
solved successfully.


\paragraph{PhotometryEngines}
\label{\detokenize{technical/architecture/services_engines:photometryengines}}\label{\detokenize{technical/architecture/services_engines:technical-architecture-services-engines-photometryengines}}
\sphinxAtStartPar
These engines solve for the photometric calibration solution of an image.


\subparagraph{Pan\sphinxhyphen{}STARRS 3pi DR2 MAST}
\label{\detokenize{technical/architecture/services_engines:pan-starrs-3pi-dr2-mast}}
\sphinxAtStartPar
Implementation: {\hyperref[\detokenize{code/opihiexarata.photometry.panstarrs:opihiexarata.photometry.panstarrs.PanstarrsMastWebAPIEngine}]{\sphinxcrossref{\sphinxcode{\sphinxupquote{opihiexarata.photometry.panstarrs.PanstarrsMastWebAPIEngine}}}}}

\sphinxAtStartPar
This specifies that the photometric database to be used for photometric
calibration should be Pan\sphinxhyphen{}STARRS 3pi Data Release 2. We access it via a web
interface using the
\sphinxhref{https://catalogs.mast.stsci.edu/docs/panstarrs.html}{Pan\sphinxhyphen{}STARRS MAST API}%
\begin{footnote}[44]\sphinxAtStartFootnote
\sphinxnolinkurl{https://catalogs.mast.stsci.edu/docs/panstarrs.html}
%
\end{footnote}.

\sphinxAtStartPar
The Pan\sphinxhyphen{}STARRS 3pi Data Release 2 survey covers areas north of \sphinxhyphen{}30
degrees declination in only the g’, r’, i’, and z’ filters. OpihiExarata
queries the database around the field of view for photometric stars to use in
calculating relevant the photometric quantities of the image.


\paragraph{OrbitEngines}
\label{\detokenize{technical/architecture/services_engines:orbitengines}}\label{\detokenize{technical/architecture/services_engines:technical-architecture-services-engines-orbitengines}}
\sphinxAtStartPar
These engines solve for preliminary orbital elements from a list of
observations.


\subparagraph{OrbFit}
\label{\detokenize{technical/architecture/services_engines:orbfit}}
\sphinxAtStartPar
Implementation: {\hyperref[\detokenize{code/opihiexarata.orbit.orbfit:opihiexarata.orbit.orbfit.OrbfitOrbitDeterminerEngine}]{\sphinxcrossref{\sphinxcode{\sphinxupquote{opihiexarata.orbit.orbfit.OrbfitOrbitDeterminerEngine}}}}}

\sphinxAtStartPar
This utilizes the \sphinxhref{http://adams.dm.unipi.it/orbfit/}{OrbFit software system}%
\begin{footnote}[45]\sphinxAtStartFootnote
\sphinxnolinkurl{http://adams.dm.unipi.it/orbfit/}
%
\end{footnote}
in determining the preliminary orbital elements of a sun\sphinxhyphen{}orbiting object
(typically an asteroid) provided a list of on\sphinxhyphen{}sky coordinate observations
through time. This particular software system uses least\sphinxhyphen{}squares as its method
of orbit determination.


\subparagraph{Custom Orbit}
\label{\detokenize{technical/architecture/services_engines:custom-orbit}}
\sphinxAtStartPar
Implementation: {\hyperref[\detokenize{code/opihiexarata.orbit.custom:opihiexarata.orbit.custom.CustomOrbitEngine}]{\sphinxcrossref{\sphinxcode{\sphinxupquote{opihiexarata.orbit.custom.CustomOrbitEngine}}}}}

\sphinxAtStartPar
This does not utilize any actual service for orbital determination. The user
provides the orbital elements to use along with their respective epoch period.
No asteroid observational information is used. Typically orbital elements a
user provides through an interface (GUI) should be passed through the
vehicle arguments (see {\hyperref[\detokenize{technical/architecture/services_engines:technical-architecture-services-engines}]{\sphinxcrossref{\DUrole{std,std-ref}{Services and Engines}}}}).


\paragraph{EphemerisEngines}
\label{\detokenize{technical/architecture/services_engines:ephemerisengines}}\label{\detokenize{technical/architecture/services_engines:technical-architecture-services-engines-ephemerisengines}}
\sphinxAtStartPar
These engines solve for an asteroid’s on\sphinxhyphen{}sky track from a set of
Keplerian orbital elements.


\subparagraph{JPL Horizons}
\label{\detokenize{technical/architecture/services_engines:jpl-horizons}}
\sphinxAtStartPar
Implementation: {\hyperref[\detokenize{code/opihiexarata.ephemeris.jplhorizons:opihiexarata.ephemeris.jplhorizons.JPLHorizonsWebAPIEngine}]{\sphinxcrossref{\sphinxcode{\sphinxupquote{opihiexarata.ephemeris.jplhorizons.JPLHorizonsWebAPIEngine}}}}}

\sphinxAtStartPar
This utilizes the \sphinxhref{https://ssd.jpl.nasa.gov/horizons/}{JPL Horizons System}%
\begin{footnote}[46]\sphinxAtStartFootnote
\sphinxnolinkurl{https://ssd.jpl.nasa.gov/horizons/}
%
\end{footnote}
from the \sphinxhref{https://ssd.jpl.nasa.gov/}{JPL Solar Systems Dynamics}%
\begin{footnote}[47]\sphinxAtStartFootnote
\sphinxnolinkurl{https://ssd.jpl.nasa.gov/}
%
\end{footnote} group for
the determination of an ephemeris from a set of Keplerian orbital elements.
This software sends the orbital elements (and other observatory parameters)
via the \sphinxhref{https://ssd-api.jpl.nasa.gov/doc/horizons.html}{Horizons API}%
\begin{footnote}[48]\sphinxAtStartFootnote
\sphinxnolinkurl{https://ssd-api.jpl.nasa.gov/doc/horizons.html}
%
\end{footnote}
service and parses the returned ephemeris.


\paragraph{PropagateEngines}
\label{\detokenize{technical/architecture/services_engines:propagateengines}}\label{\detokenize{technical/architecture/services_engines:technical-architecture-services-engines-propagateengines}}
\sphinxAtStartPar
These engines solve for an asteroid’s (or other target’s) on\sphinxhyphen{}sky track from a
set of recent observations.


\subparagraph{Linear}
\label{\detokenize{technical/architecture/services_engines:linear}}
\sphinxAtStartPar
Implementation: {\hyperref[\detokenize{code/opihiexarata.propagate.polynomial:opihiexarata.propagate.polynomial.LinearPropagationEngine}]{\sphinxcrossref{\sphinxcode{\sphinxupquote{opihiexarata.propagate.polynomial.LinearPropagationEngine}}}}}

\sphinxAtStartPar
This takes the most recent (within a few hours) observations and fits a
first order (linear) polynomial function to both the RA and DEC as a function
of time. This method assumes a tangent plane projection and so is not suited
for propagations on long timescales. See
{\hyperref[\detokenize{technical/algorithms/polynomial_propagation:technical-algorithms-polynomial-propagation}]{\sphinxcrossref{\DUrole{std,std-ref}{Polynomial Propagation}}}} for more information on the
algorithm used.


\subparagraph{Quadratic}
\label{\detokenize{technical/architecture/services_engines:quadratic}}
\sphinxAtStartPar
Implementation: {\hyperref[\detokenize{code/opihiexarata.propagate.polynomial:opihiexarata.propagate.polynomial.QuadraticPropagationEngine}]{\sphinxcrossref{\sphinxcode{\sphinxupquote{opihiexarata.propagate.polynomial.QuadraticPropagationEngine}}}}}

\sphinxAtStartPar
This takes the most recent (within a few hours) observations and fits a
second order (quadratic) polynomial function to both the RA and DEC as a
function of time. This method assumes a tangent plane projection and so is not
suited for propagations on long timescales. See
{\hyperref[\detokenize{technical/algorithms/polynomial_propagation:technical-algorithms-polynomial-propagation}]{\sphinxcrossref{\DUrole{std,std-ref}{Polynomial Propagation}}}} for more information on the
algorithm used.

\sphinxstepscope


\subsubsection{Vehicles and Solutions}
\label{\detokenize{technical/architecture/vehicles_solutions:vehicles-and-solutions}}\label{\detokenize{technical/architecture/vehicles_solutions:technical-architecture-vehicles-solutions}}\label{\detokenize{technical/architecture/vehicles_solutions::doc}}
\sphinxstepscope


\subsubsection{Library}
\label{\detokenize{technical/architecture/library:library}}\label{\detokenize{technical/architecture/library:technical-architecture-library}}\label{\detokenize{technical/architecture/library::doc}}
\sphinxAtStartPar
Here we provide a brief summary of the available functionality provided by the
software library of OpihiExarata. The whole point of the library is to store
functions and subroutines which are useful across the entire software
package.

\sphinxAtStartPar
When developing and maintaining OpihiExarata itself, please utilize the library
before implementing something custom. If the library is incomplete, and the
missing functionality would likely be used again at least once, please add it
to the library and call it from there. The library does not typically
reimplement functionality already implemented by third\sphinxhyphen{}party package
dependencies; however, there may be some wrapper functions to streamline
the capabilities thereof to better fit the use cases for OpihiExarata.

\sphinxAtStartPar
The summaries provided here do not substitute a search through the
{\hyperref[\detokenize{index:home-code-manual}]{\sphinxcrossref{\DUrole{std,std-ref}{Code Manual}}}}, but hopefully they help in searching for library
functionality.


\paragraph{Configuration}
\label{\detokenize{technical/architecture/library:configuration}}
\sphinxAtStartPar
See {\hyperref[\detokenize{code/opihiexarata.library.config:module-opihiexarata.library.config}]{\sphinxcrossref{\sphinxcode{\sphinxupquote{opihiexarata.library.config}}}}}.

\sphinxAtStartPar
The implementation of configuration parameters is done via this module.

\sphinxAtStartPar
When a user specifies a configuration file to be applied to this software, the
file is loaded and its parameters and values are loaded into the the
namespace of this module. Therefore, the software internally can call these
configurations as variables in this module; an example,
\sphinxcode{\sphinxupquote{library.config.SQUARE}} would correspond to the \sphinxcode{\sphinxupquote{SQUARE}} configuration
in the configuration YAML file.

\sphinxAtStartPar
Both normal configuration parameters and secret parameters (detailed in
{\hyperref[\detokenize{user/configuration:user-configuration}]{\sphinxcrossref{\DUrole{std,std-ref}{Configuration}}}}) are taken from their respective files and placed
into this same namespace. Therefore, the parameters must be uniquely named.

\sphinxAtStartPar
The loading and applying of a configuration file (either secret or not),
provided by the user via regular methods, is done via
{\hyperref[\detokenize{code/opihiexarata.library.config:opihiexarata.library.config.load_then_apply_configuration}]{\sphinxcrossref{\sphinxcode{\sphinxupquote{opihiexarata.library.config.load\_then\_apply\_configuration()}}}}}. Note that
this will only apply the configuration to the current Python session.


\paragraph{Conversion}
\label{\detokenize{technical/architecture/library:conversion}}
\sphinxAtStartPar
See {\hyperref[\detokenize{code/opihiexarata.library.conversion:module-opihiexarata.library.conversion}]{\sphinxcrossref{\sphinxcode{\sphinxupquote{opihiexarata.library.conversion}}}}}.

\sphinxAtStartPar
All types of required conversions are implemented here. The OpihiExarata
software has specific conventions (see {\hyperref[\detokenize{technical/conventions:technical-conventions}]{\sphinxcrossref{\DUrole{std,std-ref}{Conventions}}}}) for units
so that data may be better easily exchanged. However, some of these values
needs to be converted for various reasons and so the conversions are
implemented here.

\sphinxAtStartPar
Functions for converting between Julian day (convention) to other various
formats are implemented.

\sphinxAtStartPar
Functions for formatting RA and DEC from degrees (convention) to sexagesimal
string formatting are implemented. Specifically formatted sexagesimal can also
be converted back to degrees.


\paragraph{Engines}
\label{\detokenize{technical/architecture/library:engines}}
\sphinxAtStartPar
See {\hyperref[\detokenize{code/opihiexarata.library.engine:module-opihiexarata.library.engine}]{\sphinxcrossref{\sphinxcode{\sphinxupquote{opihiexarata.library.engine}}}}}.

\sphinxAtStartPar
Base classes for different engines and solution implementations exist here.
They are typically subclassed for the actual implemented engines
({\hyperref[\detokenize{technical/architecture/services_engines:technical-architecture-services-engines}]{\sphinxcrossref{\DUrole{std,std-ref}{Services and Engines}}}}) and solutions
({\hyperref[\detokenize{technical/architecture/vehicles_solutions:technical-architecture-vehicles-solutions}]{\sphinxcrossref{\DUrole{std,std-ref}{Vehicles and Solutions}}}}). These are also useful for
type checking.


\paragraph{Error}
\label{\detokenize{technical/architecture/library:error}}
\sphinxAtStartPar
See {\hyperref[\detokenize{code/opihiexarata.library.error:module-opihiexarata.library.error}]{\sphinxcrossref{\sphinxcode{\sphinxupquote{opihiexarata.library.error}}}}}.

\sphinxAtStartPar
Error exceptions specific to OpihiExarata are created here. All errors that
come from OpihiExarata (either directly or indirectly) should be defined here.
Using built\sphinxhyphen{}in Python errors is not suggested as using an error here helps
specify that the issue comes from OpihiExarata.


\paragraph{FITS File Handing}
\label{\detokenize{technical/architecture/library:fits-file-handing}}
\sphinxAtStartPar
See {\hyperref[\detokenize{code/opihiexarata.library.fits:module-opihiexarata.library.fits}]{\sphinxcrossref{\sphinxcode{\sphinxupquote{opihiexarata.library.fits}}}}}.

\sphinxAtStartPar
This implements functions which assist in the reading and writing of image and
table FITS files. Astropy has a lot of functionality for this, and these
functions wrap around their implementation so that it is more specialized for
OpihiExarata and so the reading and writing of FITS files are uniformly applied.


\paragraph{Type Hint}
\label{\detokenize{technical/architecture/library:type-hint}}
\sphinxAtStartPar
See {\hyperref[\detokenize{code/opihiexarata.library.hint:module-opihiexarata.library.hint}]{\sphinxcrossref{\sphinxcode{\sphinxupquote{opihiexarata.library.hint}}}}}.

\sphinxAtStartPar
Python is a dynamically typed language. However it implements type hints
(see \index{Python Enhancement Proposals@\spxentry{Python Enhancement Proposals}!PEP 483@\spxentry{PEP 483}}\sphinxhref{https://peps.python.org/pep-0483/}{\sphinxstylestrong{PEP 483}}%
\begin{footnote}[49]\sphinxAtStartFootnote
\sphinxnolinkurl{https://peps.python.org/pep-0483/}
%
\end{footnote} and \index{Python Enhancement Proposals@\spxentry{Python Enhancement Proposals}!PEP 484@\spxentry{PEP 484}}\sphinxhref{https://peps.python.org/pep-0484/}{\sphinxstylestrong{PEP 484}}%
\begin{footnote}[50]\sphinxAtStartFootnote
\sphinxnolinkurl{https://peps.python.org/pep-0484/}
%
\end{footnote}) so that text editors and other development
tools and features are more accurate and detailed. OpihiExarata uses type hints
throughout and highly recommends them. However, to avoid extremely long
object calls and unnecessary importing, object types that otherwise need an
import are all imported in this one namespace to be used across the codebase.


\paragraph{HTTP Calls}
\label{\detokenize{technical/architecture/library:http-calls}}
\sphinxAtStartPar
See {\hyperref[\detokenize{code/opihiexarata.library.http:module-opihiexarata.library.http}]{\sphinxcrossref{\sphinxcode{\sphinxupquote{opihiexarata.library.http}}}}}.

\sphinxAtStartPar
Some of the functionality of OpihiExarata requires the use of HTTP APIs.
Although a lot of the HTTP web functionality is implemented outside of this
library where specifically needed (because of the unique nature of each
process), there are some functions common among them which are implemented
here.


\paragraph{Image Array Processing}
\label{\detokenize{technical/architecture/library:image-array-processing}}
\sphinxAtStartPar
See {\hyperref[\detokenize{code/opihiexarata.library.image:module-opihiexarata.library.image}]{\sphinxcrossref{\sphinxcode{\sphinxupquote{opihiexarata.library.image}}}}}.

\sphinxAtStartPar
Opihi is an imaging telescope and images are often represented as arrays.
However, there are some functionality that make sense in terms of images but
have more involved implementations when using arrays as images. Functions
here implement common manipulations of images.


\paragraph{JSON Parsing}
\label{\detokenize{technical/architecture/library:json-parsing}}
\sphinxAtStartPar
See {\hyperref[\detokenize{code/opihiexarata.library.json:module-opihiexarata.library.json}]{\sphinxcrossref{\sphinxcode{\sphinxupquote{opihiexarata.library.json}}}}}.

\sphinxAtStartPar
Although OpihiExarata prefers YAML formatting for configuration files and
other data serializations, JSON is another popular format which is used by
some of the services OpihiExarata relies on. Thus some JSON functionality
is implemented here as wrapper functions.


\paragraph{Minor Planet Center Records}
\label{\detokenize{technical/architecture/library:minor-planet-center-records}}
\sphinxAtStartPar
See {\hyperref[\detokenize{code/opihiexarata.library.mpcrecord:module-opihiexarata.library.mpcrecord}]{\sphinxcrossref{\sphinxcode{\sphinxupquote{opihiexarata.library.mpcrecord}}}}}.

\sphinxAtStartPar
One of the most ubiquitous ways of representing an observation of an asteroid
is using the
\sphinxhref{https://www.minorplanetcenter.net/iau/info/OpticalObs.html}{MPC 80\sphinxhyphen{}column record}%
\begin{footnote}[51]\sphinxAtStartFootnote
\sphinxnolinkurl{https://www.minorplanetcenter.net/iau/info/OpticalObs.html}
%
\end{footnote}.
However, it is not a very connivent format for Python to use and so
functions which convert between the 80\sphinxhyphen{}column format and an Astropy table
(see \sphinxcode{\sphinxupquote{astropy.table}}, or more specifically,
\sphinxcode{\sphinxupquote{astropy.table.Table}}). In general, the table format is better for
internal manipulation while the 80\sphinxhyphen{}column format is used primarily to record
and send asteroid observations to other services (including, obviously, the
Minor Planet Center.)


\paragraph{File and Directory Path Manipulations}
\label{\detokenize{technical/architecture/library:file-and-directory-path-manipulations}}
\sphinxAtStartPar
See {\hyperref[\detokenize{code/opihiexarata.library.path:module-opihiexarata.library.path}]{\sphinxcrossref{\sphinxcode{\sphinxupquote{opihiexarata.library.path}}}}}.

\sphinxAtStartPar
Path and filename manipulations are common across all aspects of OpihiExarata.
For uniform application and convenience, common path manipulations are
implemented here. This only has implementations for where the filepaths are
strings and not objects.


\paragraph{Photometric and Astrometric Data Handing Table}
\label{\detokenize{technical/architecture/library:photometric-and-astrometric-data-handing-table}}
\sphinxAtStartPar
See {\hyperref[\detokenize{code/opihiexarata.library.phototable:module-opihiexarata.library.phototable}]{\sphinxcrossref{\sphinxcode{\sphinxupquote{opihiexarata.library.phototable}}}}}.

\sphinxAtStartPar
The astrometric solution and the photometric solution
(see {\hyperref[\detokenize{technical/architecture/vehicles_solutions:technical-architecture-vehicles-solutions}]{\sphinxcrossref{\DUrole{std,std-ref}{Vehicles and Solutions}}}}) both have a lot of
similar information in tables. Older versions of this software had two
different tables which were very unwieldy as progress continued. As such,
this class implements a photometry table which is more coherent and
comprehensive. Feature expansion in this region is unlikely.


\paragraph{Temporary Directory}
\label{\detokenize{technical/architecture/library:temporary-directory}}
\sphinxAtStartPar
Sometimes the OpihiExarata software needs to save temporary files when
processing data and reading the results. In order for these files not to
mess up anything on the system this software is installed on, a temporary
directory is created where the files can be created and utilized. The exact
place where this directory is created is given by the configuration parameter
\sphinxcode{\sphinxupquote{TEMPORARY\_DIRECTORY}} (see {\hyperref[\detokenize{user/configuration:user-configuration}]{\sphinxcrossref{\DUrole{std,std-ref}{Configuration}}}}) Functions are
implemented here which help with the management of this temporary directory.

\sphinxstepscope


\subsubsection{Graphical User Interface}
\label{\detokenize{technical/architecture/graphical_user_interface:graphical-user-interface}}\label{\detokenize{technical/architecture/graphical_user_interface:technical-architecture-graphical-user-interface}}\label{\detokenize{technical/architecture/graphical_user_interface::doc}}
\sphinxAtStartPar
The graphical user interface (GUI) makes use of PySide6 which itself is a
Python implementation of the Qt 6 GUI framework.

\sphinxAtStartPar
We use PySide6 as opposed to PyQt 6 because of licensing preferences and
because PySide is developed by the Qt Company (the first party) and so it is
likely to be supported for longer.

\sphinxAtStartPar
(The PySide6 is LGPL licensed. Qt Designer is GPL licensed but we only use it
as a connivent tool and do not bundle or have our program require it. As far
at PySide6 is concerned, we are dynamically linking it and are bound only by
LGPL.)

\sphinxAtStartPar
The exact shape and form of the GUIs are a byproduct of the feature sets in
{\hyperref[\detokenize{user/manual_mode:user-manual-mode}]{\sphinxcrossref{\DUrole{std,std-ref}{Manual Mode}}}} and {\hyperref[\detokenize{user/automatic_mode:user-automatic-mode}]{\sphinxcrossref{\DUrole{std,std-ref}{Automatic Mode}}}}. The interface is
optimized to those use cases and does not otherwise follow a rigid theme.


\paragraph{Qt Designer}
\label{\detokenize{technical/architecture/graphical_user_interface:qt-designer}}
\sphinxAtStartPar
To design the GUIs, we use
\sphinxhref{https://doc.qt.io/qt-6/qtdesigner-manual.html}{Qt Designer}%
\begin{footnote}[52]\sphinxAtStartFootnote
\sphinxnolinkurl{https://doc.qt.io/qt-6/qtdesigner-manual.html}
%
\end{footnote} as it is an
easy interface. This program can be downloaded via the
\sphinxhref{https://doc.qt.io/qtdesignstudio/studio-installation.html}{Qt installer}%
\begin{footnote}[53]\sphinxAtStartFootnote
\sphinxnolinkurl{https://doc.qt.io/qtdesignstudio/studio-installation.html}
%
\end{footnote}.
Qt Designer does not need to be installed on the same machine as it just
generates the GUI design files.

\sphinxAtStartPar
When installing Pyside6, Qt Designer often is included and can be invoked by:

\begin{sphinxVerbatim}[commandchars=\\\{\}]
\PYG{n}{pyside6}\PYG{o}{\PYGZhy{}}\PYG{n}{designer}
\end{sphinxVerbatim}

\sphinxAtStartPar
The GUI design files are saved in the
\sphinxcode{\sphinxupquote{/OpihiExarata/src/opihiexarata/gui/qtui/}} directory typically as
something akin to \sphinxcode{\sphinxupquote{manual.ui}}. These files may be opened by Qt Designer
and modified as needed.


\paragraph{Building UI Files}
\label{\detokenize{technical/architecture/graphical_user_interface:building-ui-files}}\label{\detokenize{technical/architecture/graphical_user_interface:technical-architecture-graphical-user-interface-building-ui-files}}
\sphinxAtStartPar
It is part of development (not installation) to develop the Python versions of
the GUI files. The UI files created via Qt Designer can be converted to their
Python versions using \sphinxstyleliteralstrong{\sphinxupquote{pyside6\sphinxhyphen{}uic}}. (See
\sphinxhref{https://doc.qt.io/qtforpython/tutorials/basictutorial/uifiles.html\#using-ui-files-from-designer-or-qtcreator-with-quiloader-and-pyside6-uic}{Qt’s documentation}%
\begin{footnote}[54]\sphinxAtStartFootnote
\sphinxnolinkurl{https://doc.qt.io/qtforpython/tutorials/basictutorial/uifiles.html\#using-ui-files-from-designer-or-qtcreator-with-quiloader-and-pyside6-uic}
%
\end{footnote}
for more information.)

\sphinxAtStartPar
The UI files are stored in \sphinxcode{\sphinxupquote{/OpihiExarata/src/opihiexarata/gui/qtui/}}
and it is expected that the generated Python equivalent files will also be
written there as \sphinxcode{\sphinxupquote{qtui\_*.py}}, where the original UI filename fills the
wildcard.

\sphinxAtStartPar
For example, for a file \sphinxcode{\sphinxupquote{manual.ui}}, it can be converted to the proper
Python version using the command:

\begin{sphinxVerbatim}[commandchars=\\\{\}]
\PYG{n}{pyside6}\PYG{o}{\PYGZhy{}}\PYG{n}{uic} \PYG{n}{manual}\PYG{o}{.}\PYG{n}{ui} \PYG{o}{\PYGZgt{}} \PYG{n}{qtui\PYGZus{}manual}\PYG{o}{.}\PYG{n}{py}
\end{sphinxVerbatim}

\sphinxAtStartPar
Every UI file should be turned into their respective Python version. A
Powershell Core script has been written to automate this process. It can be
found in the same directory and is executed by:

\begin{sphinxVerbatim}[commandchars=\\\{\}]
\PYG{n}{pwsh} \PYG{n}{build\PYGZus{}qtui\PYGZus{}window}\PYG{o}{.}\PYG{n}{ps1}
\end{sphinxVerbatim}

\sphinxstepscope


\subsection{Conventions}
\label{\detokenize{technical/conventions:conventions}}\label{\detokenize{technical/conventions:technical-conventions}}\label{\detokenize{technical/conventions::doc}}
\sphinxstepscope


\subsection{Algorithms}
\label{\detokenize{technical/algorithms/index:algorithms}}\label{\detokenize{technical/algorithms/index::doc}}
\sphinxAtStartPar
Here we detail the methods behind a few of the algorithms that we have
implemented ourselves. The algorithms and implementations of other systems
(especially the engines and services of
{\hyperref[\detokenize{technical/architecture/services_engines:technical-architecture-services-engines}]{\sphinxcrossref{\DUrole{std,std-ref}{Services and Engines}}}}) are outside the scope of this
documentation. See {\hyperref[\detokenize{user/citations:user-citations}]{\sphinxcrossref{\DUrole{std,std-ref}{Citations}}}} for further references on algorithms
implemented by others that we use.

\sphinxAtStartPar
We detail the algorithms here so that its methods (and some particularities)
are documented and some of the choice made are not lost in future development
and upkeep of this software.

\sphinxAtStartPar
See the sidebar (for the web version of this documentation) for the details on
specific algorithms.

\sphinxstepscope


\subsubsection{Preprocessing}
\label{\detokenize{technical/algorithms/preprocessing:preprocessing}}\label{\detokenize{technical/algorithms/preprocessing:technical-algorithms-preprocessing}}\label{\detokenize{technical/algorithms/preprocessing::doc}}
\sphinxAtStartPar
The data that comes from the Opihi camera is considered raw data, it has many
systematic artifacts like hot pixels, dark current, and bias to name a few.

\sphinxstepscope


\subsubsection{Polynomial Propagation}
\label{\detokenize{technical/algorithms/polynomial_propagation:polynomial-propagation}}\label{\detokenize{technical/algorithms/polynomial_propagation:technical-algorithms-polynomial-propagation}}\label{\detokenize{technical/algorithms/polynomial_propagation::doc}}
\sphinxstepscope


\subsubsection{Spherical Kinematics}
\label{\detokenize{technical/algorithms/spherical_kinematics:spherical-kinematics}}\label{\detokenize{technical/algorithms/spherical_kinematics::doc}}
\sphinxAtStartPar
Here we detail the algorithm for computing the propagation of the position of
an asteroid given observations. This method uses the laws of kinematic motion
in spherical coordinates.

\sphinxAtStartPar
\DUrole{versionmodified,deprecated}{Deprecated since version 2022.2.1: }This method has not been implemented nor have the calculations below been
proven to be correct and work for the purposes of being a PropagationEngine.


\paragraph{Definitions}
\label{\detokenize{technical/algorithms/spherical_kinematics:definitions}}
\sphinxAtStartPar
We are using spherical coordinates, namely the coordinates
\((r, \theta, \phi)\) for the radial distance, polar angle, and azimuthal
angle respectively. In relation to astrometric coordinates RA and DEC,
\((\alpha, \delta)\):
\begin{equation*}
\begin{split}\alpha = \phi   \qquad   \delta = \frac{\pi}{2} - \theta\end{split}
\end{equation*}
\sphinxAtStartPar
Observations taken by the Opihi telescoped and processed by OpihiExarata are
represented as \((\alpha_n, \delta_n, t_n)\). These correspond to the
temporal spherical coordinates \((r=1, \phi_n, \delta_n, t_n)\). For
\(t\) is the absolute time of the observation; UNIX time or Julian date
time works best.

\sphinxAtStartPar
Moreover, in spherical coordinates, the position, velocity, and acceleration
vectors are given as: (see \sphinxhref{http://www.worldcat.org/oclc/1104053368}{Keplerian Ellipses Chapter 2 Reed 2019}%
\begin{footnote}[55]\sphinxAtStartFootnote
\sphinxnolinkurl{http://www.worldcat.org/oclc/1104053368}
%
\end{footnote})
\begin{equation*}
\begin{split}\mathbf{r} &= r \mathbf{\hat r} \\
\mathbf{v} &= \dot{r} \mathbf{\hat r} + r \dot\theta \hat{\boldsymbol\theta } + r \dot\phi \sin\theta \mathbf{\hat{\boldsymbol\phi}} \\
\mathbf{a} &= \left(\ddot{r} - r\dot\theta^2 - r\dot\phi^2\sin^2\theta \right)\mathbf{\hat r} \\
 &\quad + \left( r\ddot\theta + 2\dot{r}\dot\theta - r\dot\phi^2\sin\theta\cos\theta \right) \hat{\boldsymbol\theta } \\
 &\quad + \left( r\ddot\phi\sin\theta + 2\dot{r}\dot\phi\sin\theta + 2 r\dot\theta\dot\phi\cos\theta \right) \hat{\boldsymbol\phi}\end{split}
\end{equation*}
\sphinxAtStartPar
Where the basis vectors of the spherical coordinates are:
\begin{equation*}
\begin{split}\hat{\mathbf r} &= \sin\theta \cos\phi \hat{\mathbf x} + \sin\theta \sin\phi \hat{\mathbf y} + \cos\theta \hat{\mathbf z} \\
\hat{\boldsymbol\theta} &= \cos\theta \cos\phi \hat{\mathbf x} + \cos\theta \sin\phi \hat{\mathbf y} - \sin\theta \hat{\mathbf z} \\
\hat{\boldsymbol\phi} &= - \sin\phi \hat{\mathbf x} + \cos\phi \hat{\mathbf y} + 0 \hat{\mathbf z}\end{split}
\end{equation*}
\sphinxAtStartPar
However, for the purposes of finding by propagation, we only care about the
location of the asteroid on the sky as determined by the celestial coordinates.
The distance from the origin of the coordinate system does not change. Thus:
\begin{equation*}
\begin{split}r = 1 \qquad \dot{r} = \ddot{r} = 0\end{split}
\end{equation*}
\sphinxAtStartPar
And thus the kinematic vectors are:
\begin{equation*}
\begin{split}\mathbf{r} &= \mathbf{\hat r} \\
\mathbf{v} &=  \dot\theta \hat{\boldsymbol\theta } + \dot\phi \sin\theta \mathbf{\hat{\boldsymbol\phi}} \\
\mathbf{a} &= \left(-\dot\theta^2 - \dot\phi^2\sin^2\theta \right) \mathbf{\hat r} + \left(\ddot\theta - \dot\phi^2\sin\theta\cos\theta \right) \hat{\boldsymbol\theta } + \left(\ddot\phi\sin\theta  + 2 \dot\theta\dot\phi\cos\theta \right) \hat{\boldsymbol\phi}\end{split}
\end{equation*}
\sphinxAtStartPar
We can convert these vectors to Cartesian coordinates using the matrix
transformation. The spherical to Cartesian transformation matrix
\(\mathbf{R}\) can be derived from the defining angles of the spherical
coordinate system \(\theta\) and \(\phi\). Where
\(\mathbf{u}_\text{cart} = \mathbf{R} \mathbf{u}_\text{sph}\).:
\begin{equation*}
\begin{split}\mathbf{R} = \begin{bmatrix}
\sin\theta\cos\phi & \cos\theta\cos\phi & -\sin\phi \\
\sin\theta\sin\phi & \cos\theta\sin\phi &  \cos\phi \\
\cos\theta         & -\sin\theta        & 0
\end{bmatrix}\end{split}
\end{equation*}
\sphinxAtStartPar
Thus, in Cartesian coordinates:
\begin{equation*}
\begin{split}\mathbf{r} &= \mathbf{R} \begin{bmatrix} 1 \\ 0 \\ 0 \end{bmatrix} \\
\mathbf{v} &= \mathbf{R} \begin{bmatrix} 0 \\ \dot\theta \\ \dot\phi \sin\theta \end{bmatrix} \\
\mathbf{a} &= \mathbf{R} \begin{bmatrix} -\dot\theta^2 - \dot\phi^2\sin^2\theta \\ \ddot\theta - \dot\phi^2\sin\theta\cos\theta \\ \ddot\phi\sin\theta  + 2 \dot\theta\dot\phi\cos\theta \end{bmatrix}\end{split}
\end{equation*}
\sphinxAtStartPar
Generally, as kinematics are defined based on the time derivatives, the general
equation of motion for constant acceleration can be used to determine the
movement of the position vector representing the asteroid:
\begin{equation*}
\begin{split}\mathbf{r} \triangleq \mathbf{r} \qquad \mathbf{v} \triangleq \frac{\mathrm{d}\mathbf{r}}{\mathrm{d}t} \qquad \mathbf{a} \triangleq \frac{\mathrm{d}^2\mathbf{r}}{\mathrm{d}t^2} \qquad \mathbf{j} \triangleq \frac{\mathrm{d}^3\mathbf{r}}{\mathrm{d}t^3} = 0\end{split}
\end{equation*}\begin{equation*}
\begin{split}\mathbf{r} = \mathbf{r}_0 + \mathbf{v}_0 \tau + \frac{1}{2} \mathbf{a} \tau^2\end{split}
\end{equation*}
\sphinxAtStartPar
Where \(\tau\) is the time interval between the defining time of
observation to the current time.


\paragraph{Deriving Rates}
\label{\detokenize{technical/algorithms/spherical_kinematics:deriving-rates}}
\sphinxAtStartPar
Multiple observations from Opihi provides multiple sightings of an asteroid at
many different points in the sky, providing multiple RA and DEC coordinates,
\(\alpha_n\) and \(\delta_n\) at time \(t_n\). We have a total of
\(N\) RA DEC observations. (The propagation calculation will need to be redone for
a new observation set \(N' = N + 1\).)

\sphinxAtStartPar
We can convert this to spherical coordinates with \(\phi_n = \alpha_n\) and \(\theta_n = \frac{\pi}{2} - \delta_n\).

\sphinxAtStartPar
These multiple observations allows for the determination of the rates of
change of spherical coordinates for the asteroid, namely: (For the time
difference \(t_\Delta = t_{n+1} - t_n\).)
\begin{align*}\!\begin{aligned}
\dot\theta_p = \frac{\theta_{n+1} - \theta_{n}}{t_{n+1} - t_n}\\
\dot\phi_p = \frac{\phi_{n+1} - \phi_{n}}{t_{n+1} - t_n}\\
t'_p = \frac{1}{2} \left( t_{n+1} + t_n \right)\\
\end{aligned}\end{align*}
\sphinxAtStartPar
…and…
\begin{align*}\!\begin{aligned}
\ddot\theta_q = \frac{\dot\theta_{p+1} - \dot\theta_{p}}{t'_{p+1} - t'_p}\\
\ddot\phi_q = \frac{\dot\phi_{p+1} - \dot\phi_{p}}{t'_{p+1} - t'_p}\\
\end{aligned}\end{align*}
\sphinxAtStartPar
The first order rates changes over time. As such, it is required that two
observations be reserved as special observations which the first order rates
are calculated and to established the spherical coordinate system itself.
Although it does not need to be the first two observations, it is often
connivent to use them. As such, using the first two observations
\(n=0\) and \(n=1\), we have:
\begin{equation*}
\begin{split}\theta &= \theta_0 \\
\phi &= \phi_0 \\
\dot\theta &= \dot\theta_0 = \frac{\theta_1 - \theta_0}{t_1 - t_0} \\
\dot\phi &= \dot\phi_0 = \frac{\phi_1 - \phi_0}{t_1 - t_0} \\\end{split}
\end{equation*}
\sphinxAtStartPar
Because we assume constant acceleration (\(\mathbf{j} = 0\)), the second
differential values are assumed to be constant and thus an average is more
representational of the value. (A mean or median is valid.)
\begin{align*}\!\begin{aligned}
\ddot\theta = \frac{1}{Q} \sum_q^Q \ddot\theta_q \approx \mathrm{median} (\ddot\theta_q)\\
\ddot\phi = \frac{1}{Q} \sum_q^Q \ddot\phi_q \approx \mathrm{median} (\ddot\phi_q)\\
\end{aligned}\end{align*}
\sphinxAtStartPar
In the case for \(N=2\), then the total number of derived angular first
order rates is \(P=1\). As such the second order rates cannot be
calculated and \(Q=0\) (the cardinality of the arrays are zero). By
default, for this special case:
\begin{equation*}
\begin{split}\#(\ddot\theta_q) = \#(\ddot\phi_q) = 0 \implies Q = 0 \longrightarrow \ddot\theta = 0 \quad \ddot\phi = 0\end{split}
\end{equation*}

\paragraph{Spherical Motion}
\label{\detokenize{technical/algorithms/spherical_kinematics:spherical-motion}}
\sphinxAtStartPar
With the 0th, 1st, and 2nd order rates calculated from the set of \(N\)
observations, the kinematic vectors can be calculated. The special
observations defining the coordinate system and the velocities also define
the initial vectors from which kinematics shall be applied to. The
acceleration vector, being constant means \(\mathbf{a}_0 = \mathbf{a}\).
Namely, these vectors are, in Cartesian coordinates,
\begin{equation*}
\begin{split}\mathbf{r_0} &= \begin{bmatrix}
\sin\theta\cos\phi & \cos\theta\cos\phi & -\sin\phi \\
\sin\theta\sin\phi & \cos\theta\sin\phi &  \cos\phi \\
\cos\theta         & -\sin\theta        & 0
\end{bmatrix} \begin{bmatrix} 1 \\ 0 \\ 0 \end{bmatrix} \\
\mathbf{v_0} &= \begin{bmatrix}
\sin\theta\cos\phi & \cos\theta\cos\phi & -\sin\phi \\
\sin\theta\sin\phi & \cos\theta\sin\phi &  \cos\phi \\
\cos\theta         & -\sin\theta        & 0
\end{bmatrix} \begin{bmatrix} 0 \\ \dot\theta \\ \dot\phi \sin\theta \end{bmatrix} \\
\mathbf{a} &= \begin{bmatrix}
\sin\theta\cos\phi & \cos\theta\cos\phi & -\sin\phi \\
\sin\theta\sin\phi & \cos\theta\sin\phi &  \cos\phi \\
\cos\theta         & -\sin\theta        & 0
\end{bmatrix} \begin{bmatrix} -\dot\theta^2 - \dot\phi^2\sin^2\theta \\ \ddot\theta - \dot\phi^2\sin\theta\cos\theta \\ \ddot\phi\sin\theta  + 2 \dot\theta\dot\phi\cos\theta \end{bmatrix}\end{split}
\end{equation*}
\sphinxAtStartPar
All three of these vectors are constant in future time. The position at a
set of future observations at time(s) \(t^+_i\) can be calculated using
the kinematic equation; the time intervals \(\tau_i\) being
\(\tau_i = t^+_i - t_0\):
\begin{equation*}
\begin{split}\mathbf{r}^+_i = \mathbf{r}_0 + \mathbf{v}_0 \left(t^+_i - t_0\right) + \frac{1}{2} \mathbf{a} \left(t^+_i - t_0\right)^2\end{split}
\end{equation*}

\paragraph{Celestial Sphere}
\label{\detokenize{technical/algorithms/spherical_kinematics:celestial-sphere}}
\sphinxAtStartPar
These new future position vectors \(\mathbf{r}^+_i\) are in Cartesian
coordinates. The calculations should be done in Cartesian, provided the
conversion earlier.

\sphinxAtStartPar
Each position vector can be represented as:
\begin{equation*}
\begin{split}\mathbf{r}^+_i = X_i \mathbf{\hat x} + Y_i \mathbf{\hat y} + Z_i \mathbf{\hat z} = \begin{bmatrix} X_i \\ Y_i \\ Z_i \end{bmatrix}\end{split}
\end{equation*}
\sphinxAtStartPar
These Cartesian coordinate position vectors, centered on the origin, represents
where the asteroid is on the celestial sphere in the future at an observation
time of \(t^+_i\). From these Cartesian coordinates, we can extract their
location in spherical coordinates,
\begin{equation*}
\begin{split}r^+_i &= \sqrt{X_i^2 + Y_i^2 + Z_i^2} \\
\theta^+_i &= \arccos\left(\frac{Z_i}{r^+_i}\right) = \arccos\left(\frac{Z_i}{\sqrt{X_i^2 + Y_i^2 + Z_i^2}}\right) \\
\phi^+_i &= \arctan\!2(Y_i, X_i) \simeq \arctan\left(\frac{Y_i}{X_i}\right)\end{split}
\end{equation*}
\begin{sphinxadmonition}{note}{Note:}
\sphinxAtStartPar
In order to properly handle the quadrant issue, the 2\sphinxhyphen{}argument arctangent is
required. Moreover, if the 2\sphinxhyphen{}argument arctangent function returns in a range
\(-\pi \leq \angle \leq \pi\), it can be converted to the usual range of
\(0 \leq \phi \leq 2\pi\) with: \(\phi = \angle \mod 2\pi\)
or \(\phi = \angle \mod 360^\circ\)
\end{sphinxadmonition}

\sphinxAtStartPar
These spherical coordinate locations can then be converted into future RA and
DEC temporal coordinates \((\alpha^+_i, \delta^+_i, t^+_i)\):
\begin{equation*}
\begin{split}\alpha^+_i &= \phi^+_i \\
\delta^+_i &= \frac{\pi}{2} - \theta^+_i \\
t^+_i &= t^+_i\end{split}
\end{equation*}

\paragraph{Lemmas}
\label{\detokenize{technical/algorithms/spherical_kinematics:lemmas}}

\subparagraph{Derivation of Vector Equation of Motion}
\label{\detokenize{technical/algorithms/spherical_kinematics:derivation-of-vector-equation-of-motion}}
\sphinxAtStartPar
Newton’s second law and constant acceleration stipulates:
\begin{equation*}
\begin{split}\mathbf{F} = m \mathbf{a} = m \ddot{\mathbf{r}} \qquad \dot{\mathbf{F}} = 0\end{split}
\end{equation*}
\sphinxAtStartPar
This thus provides the differential equation of motion (For constant \(\mathbf{F}\).)
\begin{equation*}
\begin{split}\ddot{\mathbf{r}} = \frac{\mathrm{d}^2\mathbf{r}}{\mathrm{d}t^2} = \frac{\mathbf{F}}{m}\end{split}
\end{equation*}
\sphinxAtStartPar
We define based on the laws of integrations (and in essence the fundamental
theorem of calculus):
\begin{equation*}
\begin{split}\dot{\mathbf{f}} \triangleq \frac{\mathrm{d}\mathbf{f}}{\mathrm{d}t} \Longleftrightarrow \int \dot{\mathbf{f}} \mathrm{d} t = \mathbf{f} + \mathbf{C}\end{split}
\end{equation*}\begin{equation*}
\begin{split}\int \frac{\mathrm{d}\mathbf{f}}{\mathrm{d}t} \mathrm{d} t = \mathbf{f}\end{split}
\end{equation*}
\sphinxAtStartPar
We can solve the differential equation of motion:
\begin{equation*}
\begin{split}\ddot{\mathbf{r}} &= \frac{\mathbf{F}}{m} \\
\frac{\mathrm{d}}{\mathrm{d}t} \left( \frac{\mathrm{d}\mathbf{r}}{\mathrm{d}t} \right) &= \frac{\mathbf{F}}{m} \\
\int \frac{\mathrm{d}}{\mathrm{d}t} \left( \frac{\mathrm{d}\mathbf{r}}{\mathrm{d}t} \right) \mathrm{d}t &= \int \frac{\mathbf{F}}{m} \mathrm{d}t = \frac{\mathbf{F}}{m} \int 1 \mathrm{d}t = \frac{\mathbf{F}}{m} t + \mathbf{C_1} \\
\frac{\mathrm{d}\mathbf{r}}{\mathrm{d}t} &= \frac{\mathbf{F}}{m} t + \mathbf{C_1} \\
\int \frac{\mathrm{d}\mathbf{r}}{\mathrm{d}t} \mathrm{d}t &= \int \frac{\mathbf{F}}{m} t + \mathbf{C_1} \mathrm{d}t = \frac{\mathbf{F}}{m} \int t \mathrm{d}t + \int \mathbf{C_1} \mathrm{d}t = \frac{\mathbf{F}}{m} \frac{1}{2} t^2 + \mathbf{C_1} t + \mathbf{C_2} \\
\mathbf{r} &= \frac{\mathbf{F}}{m} \frac{1}{2} t^2 + \mathbf{C_1} t + \mathbf{C_2}\end{split}
\end{equation*}
\sphinxAtStartPar
For the initial conditions:
\begin{equation*}
\begin{split}t = 0 &\implies \mathbf{r} = \mathbf{C_2} = \mathbf{r_0} \\
t = 0 &\implies \frac{\mathrm{d}\mathbf{r}}{\mathrm{d}t} = \mathbf{C_1} = \mathbf{v_0} \\
\dot{\mathbf{F}} = 0 &\implies \frac{\mathrm{d}^2\mathbf{r}}{\mathrm{d}t^2} = \frac{\mathbf{F}}{m} = \mathbf{a_0} = \mathbf{a}\end{split}
\end{equation*}
\sphinxAtStartPar
Thus, the total valid solution is:
\begin{equation*}
\begin{split}\mathbf{r} = \mathbf{r_0} + \mathbf{v_0} t + \frac{1}{2} \mathbf{a} t^2\end{split}
\end{equation*}

\chapter{Code Manual}
\label{\detokenize{index:code-manual}}\label{\detokenize{index:home-code-manual}}
\sphinxAtStartPar
The code manual is primarily for software maintainers and other IRTF staff. It
details the software API documentation of OpihiExarata and its inner workings.
Unlike the {\hyperref[\detokenize{index:home-technical-manual}]{\sphinxcrossref{\DUrole{std,std-ref}{Technical Manual}}}}, this does not detail
any of the design principles of the software but instead is the function and
class documentation of the software (generated by Sphinx).

\sphinxAtStartPar
\sphinxhref{https://psmd-iberutaru.github.io/OpihiExarata/build/html/code/coverage/index.html}{Code coverage}%
\begin{footnote}[56]\sphinxAtStartFootnote
\sphinxnolinkurl{https://psmd-iberutaru.github.io/OpihiExarata/build/html/code/coverage/index.html}
%
\end{footnote} is available in html. The code manual (and overall
documentation) is generated via Sphinx and the code coverage is generated by
a different tool. As such, it is manually linked rather than integrated.

\sphinxstepscope


\section{opihiexarata}
\label{\detokenize{code/modules:opihiexarata}}\label{\detokenize{code/modules::doc}}
\sphinxstepscope


\subsection{opihiexarata package}
\label{\detokenize{code/opihiexarata:opihiexarata-package}}\label{\detokenize{code/opihiexarata::doc}}

\subsubsection{Subpackages}
\label{\detokenize{code/opihiexarata:subpackages}}
\sphinxstepscope


\paragraph{opihiexarata.astrometry package}
\label{\detokenize{code/opihiexarata.astrometry:opihiexarata-astrometry-package}}\label{\detokenize{code/opihiexarata.astrometry::doc}}

\subparagraph{Submodules}
\label{\detokenize{code/opihiexarata.astrometry:submodules}}
\sphinxstepscope


\subparagraph{opihiexarata.astrometry.solution module}
\label{\detokenize{code/opihiexarata.astrometry.solution:module-opihiexarata.astrometry.solution}}\label{\detokenize{code/opihiexarata.astrometry.solution:opihiexarata-astrometry-solution-module}}\label{\detokenize{code/opihiexarata.astrometry.solution::doc}}\index{module@\spxentry{module}!opihiexarata.astrometry.solution@\spxentry{opihiexarata.astrometry.solution}}\index{opihiexarata.astrometry.solution@\spxentry{opihiexarata.astrometry.solution}!module@\spxentry{module}}
\sphinxAtStartPar
The astrometric solution class.
\index{AstrometricSolution (class in opihiexarata.astrometry.solution)@\spxentry{AstrometricSolution}\spxextra{class in opihiexarata.astrometry.solution}}

\begin{savenotes}\begin{fulllineitems}
\phantomsection\label{\detokenize{code/opihiexarata.astrometry.solution:opihiexarata.astrometry.solution.AstrometricSolution}}
\pysigstartsignatures
\pysiglinewithargsret{\sphinxbfcode{\sphinxupquote{class\DUrole{w}{  }}}\sphinxcode{\sphinxupquote{opihiexarata.astrometry.solution.}}\sphinxbfcode{\sphinxupquote{AstrometricSolution}}}{\emph{\DUrole{n}{fits\_filename}\DUrole{p}{:}\DUrole{w}{  }\DUrole{n}{str}}, \emph{\DUrole{n}{solver\_engine}\DUrole{p}{:}\DUrole{w}{  }\DUrole{n}{{\hyperref[\detokenize{code/opihiexarata.library.engine:opihiexarata.library.engine.AstrometryEngine}]{\sphinxcrossref{AstrometryEngine}}}}}, \emph{\DUrole{n}{vehicle\_args}\DUrole{p}{:}\DUrole{w}{  }\DUrole{n}{dict}\DUrole{w}{  }\DUrole{o}{=}\DUrole{w}{  }\DUrole{default_value}{\{\}}}}{}
\pysigstopsignatures
\sphinxAtStartPar
Bases: {\hyperref[\detokenize{code/opihiexarata.library.engine:opihiexarata.library.engine.ExarataSolution}]{\sphinxcrossref{\sphinxcode{\sphinxupquote{ExarataSolution}}}}}

\sphinxAtStartPar
The primary class describing an astrometric solution, based on an image
provided.

\sphinxAtStartPar
This class is the middlewere class between the engines which solve the
astrometry, and the rest of the OpihiExarata code.
\index{\_original\_filename (opihiexarata.astrometry.solution.AstrometricSolution attribute)@\spxentry{\_original\_filename}\spxextra{opihiexarata.astrometry.solution.AstrometricSolution attribute}}

\begin{savenotes}\begin{fulllineitems}
\phantomsection\label{\detokenize{code/opihiexarata.astrometry.solution:opihiexarata.astrometry.solution.AstrometricSolution._original_filename}}
\pysigstartsignatures
\pysigline{\sphinxbfcode{\sphinxupquote{\_original\_filename}}}
\pysigstopsignatures
\sphinxAtStartPar
The original filename where the fits file is stored at, or copied to.
\begin{quote}\begin{description}
\sphinxlineitem{Type}
\sphinxAtStartPar
string

\end{description}\end{quote}

\end{fulllineitems}\end{savenotes}

\index{\_original\_header (opihiexarata.astrometry.solution.AstrometricSolution attribute)@\spxentry{\_original\_header}\spxextra{opihiexarata.astrometry.solution.AstrometricSolution attribute}}

\begin{savenotes}\begin{fulllineitems}
\phantomsection\label{\detokenize{code/opihiexarata.astrometry.solution:opihiexarata.astrometry.solution.AstrometricSolution._original_header}}
\pysigstartsignatures
\pysigline{\sphinxbfcode{\sphinxupquote{\_original\_header}}}
\pysigstopsignatures
\sphinxAtStartPar
The original header of the fits file that was pulled to solve for this
astrometric solution.
\begin{quote}\begin{description}
\sphinxlineitem{Type}
\sphinxAtStartPar
Header

\end{description}\end{quote}

\end{fulllineitems}\end{savenotes}

\index{\_original\_data (opihiexarata.astrometry.solution.AstrometricSolution attribute)@\spxentry{\_original\_data}\spxextra{opihiexarata.astrometry.solution.AstrometricSolution attribute}}

\begin{savenotes}\begin{fulllineitems}
\phantomsection\label{\detokenize{code/opihiexarata.astrometry.solution:opihiexarata.astrometry.solution.AstrometricSolution._original_data}}
\pysigstartsignatures
\pysigline{\sphinxbfcode{\sphinxupquote{\_original\_data}}}
\pysigstopsignatures
\sphinxAtStartPar
The original data of the fits file that was pulled to solve for this
astrometric solution.
\begin{quote}\begin{description}
\sphinxlineitem{Type}
\sphinxAtStartPar
array\sphinxhyphen{}like

\end{description}\end{quote}

\end{fulllineitems}\end{savenotes}

\index{skycoord (opihiexarata.astrometry.solution.AstrometricSolution attribute)@\spxentry{skycoord}\spxextra{opihiexarata.astrometry.solution.AstrometricSolution attribute}}

\begin{savenotes}\begin{fulllineitems}
\phantomsection\label{\detokenize{code/opihiexarata.astrometry.solution:opihiexarata.astrometry.solution.AstrometricSolution.skycoord}}
\pysigstartsignatures
\pysigline{\sphinxbfcode{\sphinxupquote{skycoord}}}
\pysigstopsignatures
\sphinxAtStartPar
The sky coordinate which describes the current astrometric solution.
\begin{quote}\begin{description}
\sphinxlineitem{Type}
\sphinxAtStartPar
SkyCoord

\end{description}\end{quote}

\end{fulllineitems}\end{savenotes}

\index{ra (opihiexarata.astrometry.solution.AstrometricSolution attribute)@\spxentry{ra}\spxextra{opihiexarata.astrometry.solution.AstrometricSolution attribute}}

\begin{savenotes}\begin{fulllineitems}
\phantomsection\label{\detokenize{code/opihiexarata.astrometry.solution:opihiexarata.astrometry.solution.AstrometricSolution.ra}}
\pysigstartsignatures
\pysigline{\sphinxbfcode{\sphinxupquote{ra}}}
\pysigstopsignatures
\sphinxAtStartPar
The right ascension of the center of the image, in decimal degrees.
\begin{quote}\begin{description}
\sphinxlineitem{Type}
\sphinxAtStartPar
float

\end{description}\end{quote}

\end{fulllineitems}\end{savenotes}

\index{dec (opihiexarata.astrometry.solution.AstrometricSolution attribute)@\spxentry{dec}\spxextra{opihiexarata.astrometry.solution.AstrometricSolution attribute}}

\begin{savenotes}\begin{fulllineitems}
\phantomsection\label{\detokenize{code/opihiexarata.astrometry.solution:opihiexarata.astrometry.solution.AstrometricSolution.dec}}
\pysigstartsignatures
\pysigline{\sphinxbfcode{\sphinxupquote{dec}}}
\pysigstopsignatures
\sphinxAtStartPar
The declination of the center of the image, in decimal degrees.
\begin{quote}\begin{description}
\sphinxlineitem{Type}
\sphinxAtStartPar
float

\end{description}\end{quote}

\end{fulllineitems}\end{savenotes}

\index{orientation (opihiexarata.astrometry.solution.AstrometricSolution attribute)@\spxentry{orientation}\spxextra{opihiexarata.astrometry.solution.AstrometricSolution attribute}}

\begin{savenotes}\begin{fulllineitems}
\phantomsection\label{\detokenize{code/opihiexarata.astrometry.solution:opihiexarata.astrometry.solution.AstrometricSolution.orientation}}
\pysigstartsignatures
\pysigline{\sphinxbfcode{\sphinxupquote{orientation}}}
\pysigstopsignatures
\sphinxAtStartPar
The angle of orientation that the image is at, in degrees.
\begin{quote}\begin{description}
\sphinxlineitem{Type}
\sphinxAtStartPar
float

\end{description}\end{quote}

\end{fulllineitems}\end{savenotes}

\index{radius (opihiexarata.astrometry.solution.AstrometricSolution attribute)@\spxentry{radius}\spxextra{opihiexarata.astrometry.solution.AstrometricSolution attribute}}

\begin{savenotes}\begin{fulllineitems}
\phantomsection\label{\detokenize{code/opihiexarata.astrometry.solution:opihiexarata.astrometry.solution.AstrometricSolution.radius}}
\pysigstartsignatures
\pysigline{\sphinxbfcode{\sphinxupquote{radius}}}
\pysigstopsignatures
\sphinxAtStartPar
The radius of the image, or more specifically, the approximate radius
that the image covers in the sky, in degrees.
\begin{quote}\begin{description}
\sphinxlineitem{Type}
\sphinxAtStartPar
float

\end{description}\end{quote}

\end{fulllineitems}\end{savenotes}

\index{pixel\_scale (opihiexarata.astrometry.solution.AstrometricSolution attribute)@\spxentry{pixel\_scale}\spxextra{opihiexarata.astrometry.solution.AstrometricSolution attribute}}

\begin{savenotes}\begin{fulllineitems}
\phantomsection\label{\detokenize{code/opihiexarata.astrometry.solution:opihiexarata.astrometry.solution.AstrometricSolution.pixel_scale}}
\pysigstartsignatures
\pysigline{\sphinxbfcode{\sphinxupquote{pixel\_scale}}}
\pysigstopsignatures
\sphinxAtStartPar
The pixel scale of the image, in arcseconds per pixel.
\begin{quote}\begin{description}
\sphinxlineitem{Type}
\sphinxAtStartPar
float

\end{description}\end{quote}

\end{fulllineitems}\end{savenotes}

\index{wcs (opihiexarata.astrometry.solution.AstrometricSolution attribute)@\spxentry{wcs}\spxextra{opihiexarata.astrometry.solution.AstrometricSolution attribute}}

\begin{savenotes}\begin{fulllineitems}
\phantomsection\label{\detokenize{code/opihiexarata.astrometry.solution:opihiexarata.astrometry.solution.AstrometricSolution.wcs}}
\pysigstartsignatures
\pysigline{\sphinxbfcode{\sphinxupquote{wcs}}}
\pysigstopsignatures
\sphinxAtStartPar
The world coordinate solution unified interface provided by Astropy
for interface to the world coordinate which allows conversion between
sky and pixel spaces.
\begin{quote}\begin{description}
\sphinxlineitem{Type}
\sphinxAtStartPar
Astropy WCS

\end{description}\end{quote}

\end{fulllineitems}\end{savenotes}

\index{star\_table (opihiexarata.astrometry.solution.AstrometricSolution attribute)@\spxentry{star\_table}\spxextra{opihiexarata.astrometry.solution.AstrometricSolution attribute}}

\begin{savenotes}\begin{fulllineitems}
\phantomsection\label{\detokenize{code/opihiexarata.astrometry.solution:opihiexarata.astrometry.solution.AstrometricSolution.star_table}}
\pysigstartsignatures
\pysigline{\sphinxbfcode{\sphinxupquote{star\_table}}}
\pysigstopsignatures
\sphinxAtStartPar
A table detailing the correlation of star locations in both pixel and
celestial space.
\begin{quote}\begin{description}
\sphinxlineitem{Type}
\sphinxAtStartPar
Table

\end{description}\end{quote}

\end{fulllineitems}\end{savenotes}

\index{\_\_init\_\_() (opihiexarata.astrometry.solution.AstrometricSolution method)@\spxentry{\_\_init\_\_()}\spxextra{opihiexarata.astrometry.solution.AstrometricSolution method}}

\begin{savenotes}\begin{fulllineitems}
\phantomsection\label{\detokenize{code/opihiexarata.astrometry.solution:opihiexarata.astrometry.solution.AstrometricSolution.__init__}}
\pysigstartsignatures
\pysiglinewithargsret{\sphinxbfcode{\sphinxupquote{\_\_init\_\_}}}{\emph{\DUrole{n}{fits\_filename}\DUrole{p}{:}\DUrole{w}{  }\DUrole{n}{str}}, \emph{\DUrole{n}{solver\_engine}\DUrole{p}{:}\DUrole{w}{  }\DUrole{n}{{\hyperref[\detokenize{code/opihiexarata.library.engine:opihiexarata.library.engine.AstrometryEngine}]{\sphinxcrossref{AstrometryEngine}}}}}, \emph{\DUrole{n}{vehicle\_args}\DUrole{p}{:}\DUrole{w}{  }\DUrole{n}{dict}\DUrole{w}{  }\DUrole{o}{=}\DUrole{w}{  }\DUrole{default_value}{\{\}}}}{{ $\rightarrow$ None}}
\pysigstopsignatures
\sphinxAtStartPar
Solving the astrometry via the image provided. The engine class must
also be provided.
\begin{quote}\begin{description}
\sphinxlineitem{Parameters}\begin{itemize}
\item {} 
\sphinxAtStartPar
\sphinxstyleliteralstrong{\sphinxupquote{fits\_filename}} (\sphinxstyleliteralemphasis{\sphinxupquote{string}}) – The path of the fits file that contains the data for the astrometric
solution.

\item {} 
\sphinxAtStartPar
\sphinxstyleliteralstrong{\sphinxupquote{solver\_engine}} ({\hyperref[\detokenize{code/opihiexarata.library.engine:opihiexarata.library.engine.AstrometryEngine}]{\sphinxcrossref{\sphinxstyleliteralemphasis{\sphinxupquote{AstrometryEngine}}}}}) – The astrometric solver engine class. This is what will act as the
“behind the scenes” and solve the field, using this middleware to
translate it into something that is easier.

\item {} 
\sphinxAtStartPar
\sphinxstyleliteralstrong{\sphinxupquote{vehicle\_args}} (\sphinxstyleliteralemphasis{\sphinxupquote{dictionary}}) – If the vehicle function for the provided solver engine needs
extra parameters not otherwise provided by the standard input,
they are given here.

\end{itemize}

\sphinxlineitem{Return type}
\sphinxAtStartPar
None

\end{description}\end{quote}

\end{fulllineitems}\end{savenotes}

\index{pixel\_to\_sky\_coordinates() (opihiexarata.astrometry.solution.AstrometricSolution method)@\spxentry{pixel\_to\_sky\_coordinates()}\spxextra{opihiexarata.astrometry.solution.AstrometricSolution method}}

\begin{savenotes}\begin{fulllineitems}
\phantomsection\label{\detokenize{code/opihiexarata.astrometry.solution:opihiexarata.astrometry.solution.AstrometricSolution.pixel_to_sky_coordinates}}
\pysigstartsignatures
\pysiglinewithargsret{\sphinxbfcode{\sphinxupquote{pixel\_to\_sky\_coordinates}}}{\emph{\DUrole{n}{x}\DUrole{p}{:}\DUrole{w}{  }\DUrole{n}{Union\DUrole{p}{{[}}float\DUrole{p}{,}\DUrole{w}{  }ndarray\DUrole{p}{{]}}}}, \emph{\DUrole{n}{y}\DUrole{p}{:}\DUrole{w}{  }\DUrole{n}{Union\DUrole{p}{{[}}float\DUrole{p}{,}\DUrole{w}{  }ndarray\DUrole{p}{{]}}}}}{{ $\rightarrow$ tuple\DUrole{p}{{[}}Union\DUrole{p}{{[}}float\DUrole{p}{,}\DUrole{w}{  }numpy.ndarray\DUrole{p}{{]}}\DUrole{p}{,}\DUrole{w}{  }Union\DUrole{p}{{[}}float\DUrole{p}{,}\DUrole{w}{  }numpy.ndarray\DUrole{p}{{]}}\DUrole{p}{{]}}}}
\pysigstopsignatures
\sphinxAtStartPar
Compute the RA and DEC sky coordinates of this image provided
the pixel coordinates. Floating point pixel values are supported.

\sphinxAtStartPar
This is a wrapper around the WCS\sphinxhyphen{}based method.
\begin{quote}\begin{description}
\sphinxlineitem{Parameters}\begin{itemize}
\item {} 
\sphinxAtStartPar
\sphinxstyleliteralstrong{\sphinxupquote{x}} (\sphinxstyleliteralemphasis{\sphinxupquote{float}}\sphinxstyleliteralemphasis{\sphinxupquote{ or }}\sphinxstyleliteralemphasis{\sphinxupquote{array\sphinxhyphen{}like}}) – The x pixel coordinate in the x\sphinxhyphen{}axis direction.

\item {} 
\sphinxAtStartPar
\sphinxstyleliteralstrong{\sphinxupquote{y}} (\sphinxstyleliteralemphasis{\sphinxupquote{float}}\sphinxstyleliteralemphasis{\sphinxupquote{ or }}\sphinxstyleliteralemphasis{\sphinxupquote{array\sphinxhyphen{}like}}) – The y pixel coordinate in the y\sphinxhyphen{}axis direction.

\end{itemize}

\sphinxlineitem{Returns}
\sphinxAtStartPar
\begin{itemize}
\item {} 
\sphinxAtStartPar
\sphinxstylestrong{ra} (\sphinxstyleemphasis{float or array\sphinxhyphen{}like}) – The right ascension of the pixel coordinate, in degrees.

\item {} 
\sphinxAtStartPar
\sphinxstylestrong{dec} (\sphinxstyleemphasis{float or array\sphinxhyphen{}like}) – The declination of the pixel coordinate, in degrees.

\end{itemize}


\end{description}\end{quote}

\end{fulllineitems}\end{savenotes}

\index{sky\_to\_pixel\_coordinates() (opihiexarata.astrometry.solution.AstrometricSolution method)@\spxentry{sky\_to\_pixel\_coordinates()}\spxextra{opihiexarata.astrometry.solution.AstrometricSolution method}}

\begin{savenotes}\begin{fulllineitems}
\phantomsection\label{\detokenize{code/opihiexarata.astrometry.solution:opihiexarata.astrometry.solution.AstrometricSolution.sky_to_pixel_coordinates}}
\pysigstartsignatures
\pysiglinewithargsret{\sphinxbfcode{\sphinxupquote{sky\_to\_pixel\_coordinates}}}{\emph{\DUrole{n}{ra}\DUrole{p}{:}\DUrole{w}{  }\DUrole{n}{Union\DUrole{p}{{[}}float\DUrole{p}{,}\DUrole{w}{  }ndarray\DUrole{p}{{]}}}}, \emph{\DUrole{n}{dec}\DUrole{p}{:}\DUrole{w}{  }\DUrole{n}{Union\DUrole{p}{{[}}float\DUrole{p}{,}\DUrole{w}{  }ndarray\DUrole{p}{{]}}}}}{{ $\rightarrow$ tuple\DUrole{p}{{[}}Union\DUrole{p}{{[}}float\DUrole{p}{,}\DUrole{w}{  }numpy.ndarray\DUrole{p}{{]}}\DUrole{p}{,}\DUrole{w}{  }Union\DUrole{p}{{[}}float\DUrole{p}{,}\DUrole{w}{  }numpy.ndarray\DUrole{p}{{]}}\DUrole{p}{{]}}}}
\pysigstopsignatures
\sphinxAtStartPar
Compute the x and y pixel coordinates of this image provided
the RA DEC sky coordinates.

\sphinxAtStartPar
This is a wrapper around the WCS\sphinxhyphen{}based method.
\begin{quote}\begin{description}
\sphinxlineitem{Parameters}\begin{itemize}
\item {} 
\sphinxAtStartPar
\sphinxstyleliteralstrong{\sphinxupquote{ra}} (\sphinxstyleliteralemphasis{\sphinxupquote{float}}\sphinxstyleliteralemphasis{\sphinxupquote{ or }}\sphinxstyleliteralemphasis{\sphinxupquote{array\sphinxhyphen{}like}}) – The right ascension of the pixel coordinate, in degrees.

\item {} 
\sphinxAtStartPar
\sphinxstyleliteralstrong{\sphinxupquote{dec}} (\sphinxstyleliteralemphasis{\sphinxupquote{float}}\sphinxstyleliteralemphasis{\sphinxupquote{ or }}\sphinxstyleliteralemphasis{\sphinxupquote{array\sphinxhyphen{}like}}) – The declination of the pixel coordinate, in degrees.

\end{itemize}

\sphinxlineitem{Returns}
\sphinxAtStartPar
\begin{itemize}
\item {} 
\sphinxAtStartPar
\sphinxstylestrong{x} (\sphinxstyleemphasis{float or array\sphinxhyphen{}like}) – The x pixel coordinate in the x\sphinxhyphen{}axis direction.

\item {} 
\sphinxAtStartPar
\sphinxstylestrong{y} (\sphinxstyleemphasis{float or array\sphinxhyphen{}like}) – The y pixel coordinate in the y\sphinxhyphen{}axis direction.

\end{itemize}


\end{description}\end{quote}

\end{fulllineitems}\end{savenotes}


\end{fulllineitems}\end{savenotes}

\index{\_vehicle\_astrometrynet\_web\_api() (in module opihiexarata.astrometry.solution)@\spxentry{\_vehicle\_astrometrynet\_web\_api()}\spxextra{in module opihiexarata.astrometry.solution}}

\begin{savenotes}\begin{fulllineitems}
\phantomsection\label{\detokenize{code/opihiexarata.astrometry.solution:opihiexarata.astrometry.solution._vehicle_astrometrynet_web_api}}
\pysigstartsignatures
\pysiglinewithargsret{\sphinxcode{\sphinxupquote{opihiexarata.astrometry.solution.}}\sphinxbfcode{\sphinxupquote{\_vehicle\_astrometrynet\_web\_api}}}{\emph{\DUrole{n}{fits\_filename}\DUrole{p}{:}\DUrole{w}{  }\DUrole{n}{str}}}{{ $\rightarrow$ dict}}
\pysigstopsignatures
\sphinxAtStartPar
A vehicle function for astrometric solutions. Solve the fits file
astrometry using the astrometry.net nova web API.
\begin{quote}\begin{description}
\sphinxlineitem{Parameters}
\sphinxAtStartPar
\sphinxstyleliteralstrong{\sphinxupquote{fits\_filename}} (\sphinxstyleliteralemphasis{\sphinxupquote{string}}) – The path of the fits file that contains the data for the astrometric
solution.

\sphinxlineitem{Returns}
\sphinxAtStartPar
\sphinxstylestrong{astrometry\_results} – A dictionary containing the results of the astrometric solution.

\sphinxlineitem{Return type}
\sphinxAtStartPar
dict

\end{description}\end{quote}

\end{fulllineitems}\end{savenotes}


\sphinxstepscope


\subparagraph{opihiexarata.astrometry.webclient module}
\label{\detokenize{code/opihiexarata.astrometry.webclient:module-opihiexarata.astrometry.webclient}}\label{\detokenize{code/opihiexarata.astrometry.webclient:opihiexarata-astrometry-webclient-module}}\label{\detokenize{code/opihiexarata.astrometry.webclient::doc}}\index{module@\spxentry{module}!opihiexarata.astrometry.webclient@\spxentry{opihiexarata.astrometry.webclient}}\index{opihiexarata.astrometry.webclient@\spxentry{opihiexarata.astrometry.webclient}!module@\spxentry{module}}\index{AstrometryNetWebAPIEngine (class in opihiexarata.astrometry.webclient)@\spxentry{AstrometryNetWebAPIEngine}\spxextra{class in opihiexarata.astrometry.webclient}}

\begin{savenotes}\begin{fulllineitems}
\phantomsection\label{\detokenize{code/opihiexarata.astrometry.webclient:opihiexarata.astrometry.webclient.AstrometryNetWebAPIEngine}}
\pysigstartsignatures
\pysiglinewithargsret{\sphinxbfcode{\sphinxupquote{class\DUrole{w}{  }}}\sphinxcode{\sphinxupquote{opihiexarata.astrometry.webclient.}}\sphinxbfcode{\sphinxupquote{AstrometryNetWebAPIEngine}}}{\emph{\DUrole{n}{url}\DUrole{o}{=}\DUrole{default_value}{None}}, \emph{\DUrole{n}{apikey}\DUrole{p}{:}\DUrole{w}{  }\DUrole{n}{Optional\DUrole{p}{{[}}str\DUrole{p}{{]}}}\DUrole{w}{  }\DUrole{o}{=}\DUrole{w}{  }\DUrole{default_value}{None}}, \emph{\DUrole{n}{silent}\DUrole{p}{:}\DUrole{w}{  }\DUrole{n}{bool}\DUrole{w}{  }\DUrole{o}{=}\DUrole{w}{  }\DUrole{default_value}{True}}}{}
\pysigstopsignatures
\sphinxAtStartPar
Bases: {\hyperref[\detokenize{code/opihiexarata.library.engine:opihiexarata.library.engine.AstrometryEngine}]{\sphinxcrossref{\sphinxcode{\sphinxupquote{AstrometryEngine}}}}}

\sphinxAtStartPar
A python\sphinxhyphen{}based wrapper around the web API for astrometry.net.

\sphinxAtStartPar
This API does not have the full functionality of the default Python client
seen at \sphinxurl{https://github.com/dstndstn/astrometry.net/blob/master/net/client/client.py}.
The point of this class is to be simple enough to be understood by others and
be specialized for OpihiExarata.
\index{\_ASTROMETRY\_NET\_API\_BASE\_URL (opihiexarata.astrometry.webclient.AstrometryNetWebAPIEngine attribute)@\spxentry{\_ASTROMETRY\_NET\_API\_BASE\_URL}\spxextra{opihiexarata.astrometry.webclient.AstrometryNetWebAPIEngine attribute}}

\begin{savenotes}\begin{fulllineitems}
\phantomsection\label{\detokenize{code/opihiexarata.astrometry.webclient:opihiexarata.astrometry.webclient.AstrometryNetWebAPIEngine._ASTROMETRY_NET_API_BASE_URL}}
\pysigstartsignatures
\pysigline{\sphinxbfcode{\sphinxupquote{\_ASTROMETRY\_NET\_API\_BASE\_URL}}}
\pysigstopsignatures
\sphinxAtStartPar
The base URL for the API which all other service URLs are derived from.
\begin{quote}\begin{description}
\sphinxlineitem{Type}
\sphinxAtStartPar
string

\end{description}\end{quote}

\end{fulllineitems}\end{savenotes}

\index{\_apikey (opihiexarata.astrometry.webclient.AstrometryNetWebAPIEngine attribute)@\spxentry{\_apikey}\spxextra{opihiexarata.astrometry.webclient.AstrometryNetWebAPIEngine attribute}}

\begin{savenotes}\begin{fulllineitems}
\phantomsection\label{\detokenize{code/opihiexarata.astrometry.webclient:opihiexarata.astrometry.webclient.AstrometryNetWebAPIEngine._apikey}}
\pysigstartsignatures
\pysigline{\sphinxbfcode{\sphinxupquote{\_apikey}}}
\pysigstopsignatures
\sphinxAtStartPar
The API key used to log in.
\begin{quote}\begin{description}
\sphinxlineitem{Type}
\sphinxAtStartPar
string

\end{description}\end{quote}

\end{fulllineitems}\end{savenotes}

\index{original\_upload\_filename (opihiexarata.astrometry.webclient.AstrometryNetWebAPIEngine attribute)@\spxentry{original\_upload\_filename}\spxextra{opihiexarata.astrometry.webclient.AstrometryNetWebAPIEngine attribute}}

\begin{savenotes}\begin{fulllineitems}
\phantomsection\label{\detokenize{code/opihiexarata.astrometry.webclient:opihiexarata.astrometry.webclient.AstrometryNetWebAPIEngine.original_upload_filename}}
\pysigstartsignatures
\pysigline{\sphinxbfcode{\sphinxupquote{original\_upload\_filename}}}
\pysigstopsignatures
\sphinxAtStartPar
The original filename that was used to upload the data.
\begin{quote}\begin{description}
\sphinxlineitem{Type}
\sphinxAtStartPar
string

\end{description}\end{quote}

\end{fulllineitems}\end{savenotes}

\index{session (opihiexarata.astrometry.webclient.AstrometryNetWebAPIEngine attribute)@\spxentry{session}\spxextra{opihiexarata.astrometry.webclient.AstrometryNetWebAPIEngine attribute}}

\begin{savenotes}\begin{fulllineitems}
\phantomsection\label{\detokenize{code/opihiexarata.astrometry.webclient:opihiexarata.astrometry.webclient.AstrometryNetWebAPIEngine.session}}
\pysigstartsignatures
\pysigline{\sphinxbfcode{\sphinxupquote{session}}}
\pysigstopsignatures
\sphinxAtStartPar
The session ID of this API connection to astrometry.net
\begin{quote}\begin{description}
\sphinxlineitem{Type}
\sphinxAtStartPar
string

\end{description}\end{quote}

\end{fulllineitems}\end{savenotes}

\index{\_\_del\_job\_id() (opihiexarata.astrometry.webclient.AstrometryNetWebAPIEngine method)@\spxentry{\_\_del\_job\_id()}\spxextra{opihiexarata.astrometry.webclient.AstrometryNetWebAPIEngine method}}

\begin{savenotes}\begin{fulllineitems}
\phantomsection\label{\detokenize{code/opihiexarata.astrometry.webclient:opihiexarata.astrometry.webclient.AstrometryNetWebAPIEngine.__del_job_id}}
\pysigstartsignatures
\pysiglinewithargsret{\sphinxbfcode{\sphinxupquote{\_\_del\_job\_id}}}{}{{ $\rightarrow$ None}}
\pysigstopsignatures
\sphinxAtStartPar
Remove the current job ID association.

\end{fulllineitems}\end{savenotes}

\index{\_\_del\_submission\_id() (opihiexarata.astrometry.webclient.AstrometryNetWebAPIEngine method)@\spxentry{\_\_del\_submission\_id()}\spxextra{opihiexarata.astrometry.webclient.AstrometryNetWebAPIEngine method}}

\begin{savenotes}\begin{fulllineitems}
\phantomsection\label{\detokenize{code/opihiexarata.astrometry.webclient:opihiexarata.astrometry.webclient.AstrometryNetWebAPIEngine.__del_submission_id}}
\pysigstartsignatures
\pysiglinewithargsret{\sphinxbfcode{\sphinxupquote{\_\_del\_submission\_id}}}{}{{ $\rightarrow$ None}}
\pysigstopsignatures
\sphinxAtStartPar
Remove the current submission ID association.

\end{fulllineitems}\end{savenotes}

\index{\_\_doc\_job\_id (opihiexarata.astrometry.webclient.AstrometryNetWebAPIEngine attribute)@\spxentry{\_\_doc\_job\_id}\spxextra{opihiexarata.astrometry.webclient.AstrometryNetWebAPIEngine attribute}}

\begin{savenotes}\begin{fulllineitems}
\phantomsection\label{\detokenize{code/opihiexarata.astrometry.webclient:opihiexarata.astrometry.webclient.AstrometryNetWebAPIEngine.__doc_job_id}}
\pysigstartsignatures
\pysigline{\sphinxbfcode{\sphinxupquote{\_\_doc\_job\_id}}\sphinxbfcode{\sphinxupquote{\DUrole{w}{  }\DUrole{p}{=}\DUrole{w}{  }'When file upload or table upload is sent to the API, the job ID of the submission is saved here.'}}}
\pysigstopsignatures
\end{fulllineitems}\end{savenotes}

\index{\_\_doc\_submission\_id (opihiexarata.astrometry.webclient.AstrometryNetWebAPIEngine attribute)@\spxentry{\_\_doc\_submission\_id}\spxextra{opihiexarata.astrometry.webclient.AstrometryNetWebAPIEngine attribute}}

\begin{savenotes}\begin{fulllineitems}
\phantomsection\label{\detokenize{code/opihiexarata.astrometry.webclient:opihiexarata.astrometry.webclient.AstrometryNetWebAPIEngine.__doc_submission_id}}
\pysigstartsignatures
\pysigline{\sphinxbfcode{\sphinxupquote{\_\_doc\_submission\_id}}\sphinxbfcode{\sphinxupquote{\DUrole{w}{  }\DUrole{p}{=}\DUrole{w}{  }'When file upload or table upload is sent to the API, the submission ID is saved here.'}}}
\pysigstopsignatures
\end{fulllineitems}\end{savenotes}

\index{\_\_get\_job\_id() (opihiexarata.astrometry.webclient.AstrometryNetWebAPIEngine method)@\spxentry{\_\_get\_job\_id()}\spxextra{opihiexarata.astrometry.webclient.AstrometryNetWebAPIEngine method}}

\begin{savenotes}\begin{fulllineitems}
\phantomsection\label{\detokenize{code/opihiexarata.astrometry.webclient:opihiexarata.astrometry.webclient.AstrometryNetWebAPIEngine.__get_job_id}}
\pysigstartsignatures
\pysiglinewithargsret{\sphinxbfcode{\sphinxupquote{\_\_get\_job\_id}}}{}{{ $\rightarrow$ str}}
\pysigstopsignatures
\sphinxAtStartPar
Extract the job ID from the image upload results. It may be the
case that there is not job yet associated with this submission.

\end{fulllineitems}\end{savenotes}

\index{\_\_get\_submission\_id() (opihiexarata.astrometry.webclient.AstrometryNetWebAPIEngine method)@\spxentry{\_\_get\_submission\_id()}\spxextra{opihiexarata.astrometry.webclient.AstrometryNetWebAPIEngine method}}

\begin{savenotes}\begin{fulllineitems}
\phantomsection\label{\detokenize{code/opihiexarata.astrometry.webclient:opihiexarata.astrometry.webclient.AstrometryNetWebAPIEngine.__get_submission_id}}
\pysigstartsignatures
\pysiglinewithargsret{\sphinxbfcode{\sphinxupquote{\_\_get\_submission\_id}}}{}{{ $\rightarrow$ str}}
\pysigstopsignatures
\sphinxAtStartPar
Extract the submission ID from the image upload results.

\end{fulllineitems}\end{savenotes}

\index{\_\_init\_\_() (opihiexarata.astrometry.webclient.AstrometryNetWebAPIEngine method)@\spxentry{\_\_init\_\_()}\spxextra{opihiexarata.astrometry.webclient.AstrometryNetWebAPIEngine method}}

\begin{savenotes}\begin{fulllineitems}
\phantomsection\label{\detokenize{code/opihiexarata.astrometry.webclient:opihiexarata.astrometry.webclient.AstrometryNetWebAPIEngine.__init__}}
\pysigstartsignatures
\pysiglinewithargsret{\sphinxbfcode{\sphinxupquote{\_\_init\_\_}}}{\emph{\DUrole{n}{url}\DUrole{o}{=}\DUrole{default_value}{None}}, \emph{\DUrole{n}{apikey}\DUrole{p}{:}\DUrole{w}{  }\DUrole{n}{Optional\DUrole{p}{{[}}str\DUrole{p}{{]}}}\DUrole{w}{  }\DUrole{o}{=}\DUrole{w}{  }\DUrole{default_value}{None}}, \emph{\DUrole{n}{silent}\DUrole{p}{:}\DUrole{w}{  }\DUrole{n}{bool}\DUrole{w}{  }\DUrole{o}{=}\DUrole{w}{  }\DUrole{default_value}{True}}}{{ $\rightarrow$ None}}
\pysigstopsignatures
\sphinxAtStartPar
The instantiation, connecting to the web API using the API key.
\begin{quote}\begin{description}
\sphinxlineitem{Parameters}\begin{itemize}
\item {} 
\sphinxAtStartPar
\sphinxstyleliteralstrong{\sphinxupquote{url}} (\sphinxstyleliteralemphasis{\sphinxupquote{string}}\sphinxstyleliteralemphasis{\sphinxupquote{, }}\sphinxstyleliteralemphasis{\sphinxupquote{default = None}}) – The base url which all other API URL links are derived from. This
should be used if the API is a self\sphinxhyphen{}hosted install or has a
different web source than nova.astrometry.net. Defaults to the
nova.astrometry.net api service.

\item {} 
\sphinxAtStartPar
\sphinxstyleliteralstrong{\sphinxupquote{apikey}} (\sphinxstyleliteralemphasis{\sphinxupquote{string}}) – The API key of the user.

\item {} 
\sphinxAtStartPar
\sphinxstyleliteralstrong{\sphinxupquote{silent}} (\sphinxstyleliteralemphasis{\sphinxupquote{bool}}\sphinxstyleliteralemphasis{\sphinxupquote{, }}\sphinxstyleliteralemphasis{\sphinxupquote{default = True}}) – Should there be printed messages as the processes are executed.
This is helpful for debugging or similar processes.

\end{itemize}

\sphinxlineitem{Return type}
\sphinxAtStartPar
None

\end{description}\end{quote}

\end{fulllineitems}\end{savenotes}

\index{\_\_job\_id (opihiexarata.astrometry.webclient.AstrometryNetWebAPIEngine attribute)@\spxentry{\_\_job\_id}\spxextra{opihiexarata.astrometry.webclient.AstrometryNetWebAPIEngine attribute}}

\begin{savenotes}\begin{fulllineitems}
\phantomsection\label{\detokenize{code/opihiexarata.astrometry.webclient:opihiexarata.astrometry.webclient.AstrometryNetWebAPIEngine.__job_id}}
\pysigstartsignatures
\pysigline{\sphinxbfcode{\sphinxupquote{\_\_job\_id}}\sphinxbfcode{\sphinxupquote{\DUrole{w}{  }\DUrole{p}{=}\DUrole{w}{  }None}}}
\pysigstopsignatures
\end{fulllineitems}\end{savenotes}

\index{\_\_login() (opihiexarata.astrometry.webclient.AstrometryNetWebAPIEngine method)@\spxentry{\_\_login()}\spxextra{opihiexarata.astrometry.webclient.AstrometryNetWebAPIEngine method}}

\begin{savenotes}\begin{fulllineitems}
\phantomsection\label{\detokenize{code/opihiexarata.astrometry.webclient:opihiexarata.astrometry.webclient.AstrometryNetWebAPIEngine.__login}}
\pysigstartsignatures
\pysiglinewithargsret{\sphinxbfcode{\sphinxupquote{\_\_login}}}{\emph{\DUrole{n}{apikey}\DUrole{p}{:}\DUrole{w}{  }\DUrole{n}{str}}}{{ $\rightarrow$ str}}
\pysigstopsignatures
\sphinxAtStartPar
The method to log into the API system.
\begin{quote}\begin{description}
\sphinxlineitem{Parameters}
\sphinxAtStartPar
\sphinxstyleliteralstrong{\sphinxupquote{apikey}} (\sphinxstyleliteralemphasis{\sphinxupquote{string}}) – The API key for the web API service.

\sphinxlineitem{Returns}
\sphinxAtStartPar
\sphinxstylestrong{session\_key} – The session key for this login session.

\sphinxlineitem{Return type}
\sphinxAtStartPar
string

\end{description}\end{quote}

\end{fulllineitems}\end{savenotes}

\index{\_\_set\_job\_id() (opihiexarata.astrometry.webclient.AstrometryNetWebAPIEngine method)@\spxentry{\_\_set\_job\_id()}\spxextra{opihiexarata.astrometry.webclient.AstrometryNetWebAPIEngine method}}

\begin{savenotes}\begin{fulllineitems}
\phantomsection\label{\detokenize{code/opihiexarata.astrometry.webclient:opihiexarata.astrometry.webclient.AstrometryNetWebAPIEngine.__set_job_id}}
\pysigstartsignatures
\pysiglinewithargsret{\sphinxbfcode{\sphinxupquote{\_\_set\_job\_id}}}{\emph{\DUrole{n}{job\_id}}}{{ $\rightarrow$ None}}
\pysigstopsignatures
\sphinxAtStartPar
Assign the job ID, it should only be done once when the
image is obtained.

\end{fulllineitems}\end{savenotes}

\index{\_\_set\_submission\_id() (opihiexarata.astrometry.webclient.AstrometryNetWebAPIEngine method)@\spxentry{\_\_set\_submission\_id()}\spxextra{opihiexarata.astrometry.webclient.AstrometryNetWebAPIEngine method}}

\begin{savenotes}\begin{fulllineitems}
\phantomsection\label{\detokenize{code/opihiexarata.astrometry.webclient:opihiexarata.astrometry.webclient.AstrometryNetWebAPIEngine.__set_submission_id}}
\pysigstartsignatures
\pysiglinewithargsret{\sphinxbfcode{\sphinxupquote{\_\_set\_submission\_id}}}{\emph{\DUrole{n}{sub\_id}}}{{ $\rightarrow$ None}}
\pysigstopsignatures
\sphinxAtStartPar
Assign the submission ID, it should only be done once when the
image is obtained.

\end{fulllineitems}\end{savenotes}

\index{\_\_submission\_id (opihiexarata.astrometry.webclient.AstrometryNetWebAPIEngine attribute)@\spxentry{\_\_submission\_id}\spxextra{opihiexarata.astrometry.webclient.AstrometryNetWebAPIEngine attribute}}

\begin{savenotes}\begin{fulllineitems}
\phantomsection\label{\detokenize{code/opihiexarata.astrometry.webclient:opihiexarata.astrometry.webclient.AstrometryNetWebAPIEngine.__submission_id}}
\pysigstartsignatures
\pysigline{\sphinxbfcode{\sphinxupquote{\_\_submission\_id}}\sphinxbfcode{\sphinxupquote{\DUrole{w}{  }\DUrole{p}{=}\DUrole{w}{  }None}}}
\pysigstopsignatures
\end{fulllineitems}\end{savenotes}

\index{\_generate\_service\_url() (opihiexarata.astrometry.webclient.AstrometryNetWebAPIEngine method)@\spxentry{\_generate\_service\_url()}\spxextra{opihiexarata.astrometry.webclient.AstrometryNetWebAPIEngine method}}

\begin{savenotes}\begin{fulllineitems}
\phantomsection\label{\detokenize{code/opihiexarata.astrometry.webclient:opihiexarata.astrometry.webclient.AstrometryNetWebAPIEngine._generate_service_url}}
\pysigstartsignatures
\pysiglinewithargsret{\sphinxbfcode{\sphinxupquote{\_generate\_service\_url}}}{\emph{\DUrole{n}{service}\DUrole{p}{:}\DUrole{w}{  }\DUrole{n}{str}}}{{ $\rightarrow$ str}}
\pysigstopsignatures
\sphinxAtStartPar
Generate the correct URL for the desired service. Because astrometry.net
uses a convention, we can follow it to obtain the desired service URL.
\begin{quote}\begin{description}
\sphinxlineitem{Parameters}
\sphinxAtStartPar
\sphinxstyleliteralstrong{\sphinxupquote{service}} (\sphinxstyleliteralemphasis{\sphinxupquote{str}}) – The service which the API URL for should be generated from.

\sphinxlineitem{Returns}
\sphinxAtStartPar
\sphinxstylestrong{url} – The URL for the service.

\sphinxlineitem{Return type}
\sphinxAtStartPar
str

\end{description}\end{quote}

\end{fulllineitems}\end{savenotes}

\index{\_generate\_upload\_args() (opihiexarata.astrometry.webclient.AstrometryNetWebAPIEngine method)@\spxentry{\_generate\_upload\_args()}\spxextra{opihiexarata.astrometry.webclient.AstrometryNetWebAPIEngine method}}

\begin{savenotes}\begin{fulllineitems}
\phantomsection\label{\detokenize{code/opihiexarata.astrometry.webclient:opihiexarata.astrometry.webclient.AstrometryNetWebAPIEngine._generate_upload_args}}
\pysigstartsignatures
\pysiglinewithargsret{\sphinxbfcode{\sphinxupquote{\_generate\_upload\_args}}}{\emph{\DUrole{o}{**}\DUrole{n}{kwargs}}}{{ $\rightarrow$ dict}}
\pysigstopsignatures
\sphinxAtStartPar
Generate the arguments for sending a request. This constructs the
needed arguments, replacing the defaults with user provided arguments
where desired.
\begin{quote}\begin{description}
\sphinxlineitem{Parameters}
\sphinxAtStartPar
\sphinxstyleliteralstrong{\sphinxupquote{**kwargs}} (\sphinxstyleliteralemphasis{\sphinxupquote{dict}}) – Arguments which would override the defaults.

\sphinxlineitem{Returns}
\sphinxAtStartPar
\sphinxstylestrong{args} – The arguments which can be used to send the request.

\sphinxlineitem{Return type}
\sphinxAtStartPar
dict

\end{description}\end{quote}

\end{fulllineitems}\end{savenotes}

\index{\_send\_web\_request() (opihiexarata.astrometry.webclient.AstrometryNetWebAPIEngine method)@\spxentry{\_send\_web\_request()}\spxextra{opihiexarata.astrometry.webclient.AstrometryNetWebAPIEngine method}}

\begin{savenotes}\begin{fulllineitems}
\phantomsection\label{\detokenize{code/opihiexarata.astrometry.webclient:opihiexarata.astrometry.webclient.AstrometryNetWebAPIEngine._send_web_request}}
\pysigstartsignatures
\pysiglinewithargsret{\sphinxbfcode{\sphinxupquote{\_send\_web\_request}}}{\emph{\DUrole{n}{service}\DUrole{p}{:}\DUrole{w}{  }\DUrole{n}{str}}, \emph{\DUrole{n}{args}\DUrole{p}{:}\DUrole{w}{  }\DUrole{n}{dict}\DUrole{w}{  }\DUrole{o}{=}\DUrole{w}{  }\DUrole{default_value}{\{\}}}, \emph{\DUrole{n}{file\_args}\DUrole{p}{:}\DUrole{w}{  }\DUrole{n}{Optional\DUrole{p}{{[}}dict\DUrole{p}{{]}}}\DUrole{w}{  }\DUrole{o}{=}\DUrole{w}{  }\DUrole{default_value}{None}}}{{ $\rightarrow$ dict}}
\pysigstopsignatures
\sphinxAtStartPar
A wrapper function for sending a webrequest to the astrometry.net API
service. Returns the results as well.
\begin{quote}\begin{description}
\sphinxlineitem{Parameters}\begin{itemize}
\item {} 
\sphinxAtStartPar
\sphinxstyleliteralstrong{\sphinxupquote{service}} (\sphinxstyleliteralemphasis{\sphinxupquote{string}}) – The service which is being requested. The web URL is constructed
from this string.

\item {} 
\sphinxAtStartPar
\sphinxstyleliteralstrong{\sphinxupquote{args}} (\sphinxstyleliteralemphasis{\sphinxupquote{dictionary}}\sphinxstyleliteralemphasis{\sphinxupquote{, }}\sphinxstyleliteralemphasis{\sphinxupquote{default = \{\}}}) – The arguments being sent over the web request.

\item {} 
\sphinxAtStartPar
\sphinxstyleliteralstrong{\sphinxupquote{file\_args}} (\sphinxstyleliteralemphasis{\sphinxupquote{dictionary}}\sphinxstyleliteralemphasis{\sphinxupquote{, }}\sphinxstyleliteralemphasis{\sphinxupquote{default = None}}) – If a file is being uploaded instead, special care must be taken to
sure it matches the upload specifications.

\end{itemize}

\sphinxlineitem{Returns}
\sphinxAtStartPar
\sphinxstylestrong{results} – The results of the web request if it did not fail.

\sphinxlineitem{Return type}
\sphinxAtStartPar
dictionary

\end{description}\end{quote}

\end{fulllineitems}\end{savenotes}

\index{download\_result\_file() (opihiexarata.astrometry.webclient.AstrometryNetWebAPIEngine method)@\spxentry{download\_result\_file()}\spxextra{opihiexarata.astrometry.webclient.AstrometryNetWebAPIEngine method}}

\begin{savenotes}\begin{fulllineitems}
\phantomsection\label{\detokenize{code/opihiexarata.astrometry.webclient:opihiexarata.astrometry.webclient.AstrometryNetWebAPIEngine.download_result_file}}
\pysigstartsignatures
\pysiglinewithargsret{\sphinxbfcode{\sphinxupquote{download\_result\_file}}}{\emph{\DUrole{n}{filename}\DUrole{p}{:}\DUrole{w}{  }\DUrole{n}{str}}, \emph{\DUrole{n}{file\_type}\DUrole{p}{:}\DUrole{w}{  }\DUrole{n}{str}}, \emph{\DUrole{n}{job\_id}\DUrole{p}{:}\DUrole{w}{  }\DUrole{n}{Optional\DUrole{p}{{[}}str\DUrole{p}{{]}}}\DUrole{w}{  }\DUrole{o}{=}\DUrole{w}{  }\DUrole{default_value}{None}}}{{ $\rightarrow$ None}}
\pysigstopsignatures
\sphinxAtStartPar
Downloads fits data table files which correspond to the job id.
\begin{quote}\begin{description}
\sphinxlineitem{Parameters}\begin{itemize}
\item {} 
\sphinxAtStartPar
\sphinxstyleliteralstrong{\sphinxupquote{filename}} (\sphinxstyleliteralemphasis{\sphinxupquote{str}}) – The filename of the file when it is downloaded and saved to disk.

\item {} 
\sphinxAtStartPar
\sphinxstyleliteralstrong{\sphinxupquote{file\_type}} (\sphinxstyleliteralemphasis{\sphinxupquote{str}}) – 
\sphinxAtStartPar
The type of file to be downloaded from astrometry.net. It should
one of the following:
\begin{itemize}
\item {} 
\sphinxAtStartPar
\sphinxtitleref{wcs}: The world corrdinate data table file.

\item {} 
\sphinxAtStartPar
\sphinxtitleref{new\_fits}, \sphinxtitleref{new\_image}: A new fits file, containing the
original image, annotations, and WCS header information.

\item {} 
\sphinxAtStartPar
\sphinxtitleref{rdls}: A table of reference stars nearby.

\item {} 
\sphinxAtStartPar
\sphinxtitleref{axy}: A table in of the location of stars detected in the
provided image.

\item {} 
\sphinxAtStartPar
\sphinxtitleref{corr}: A table of the correspondences between reference
stars location in the sky and in pixel space.

\end{itemize}


\item {} 
\sphinxAtStartPar
\sphinxstyleliteralstrong{\sphinxupquote{job\_id}} (\sphinxstyleliteralemphasis{\sphinxupquote{str}}\sphinxstyleliteralemphasis{\sphinxupquote{, }}\sphinxstyleliteralemphasis{\sphinxupquote{default = None}}) – The ID of the job that the results should be obtained from. If not
provided, the ID determined by the file upload is used.

\end{itemize}

\sphinxlineitem{Return type}
\sphinxAtStartPar
None

\end{description}\end{quote}

\end{fulllineitems}\end{savenotes}

\index{get\_job\_results() (opihiexarata.astrometry.webclient.AstrometryNetWebAPIEngine method)@\spxentry{get\_job\_results()}\spxextra{opihiexarata.astrometry.webclient.AstrometryNetWebAPIEngine method}}

\begin{savenotes}\begin{fulllineitems}
\phantomsection\label{\detokenize{code/opihiexarata.astrometry.webclient:opihiexarata.astrometry.webclient.AstrometryNetWebAPIEngine.get_job_results}}
\pysigstartsignatures
\pysiglinewithargsret{\sphinxbfcode{\sphinxupquote{get\_job\_results}}}{\emph{\DUrole{n}{job\_id}\DUrole{p}{:}\DUrole{w}{  }\DUrole{n}{Optional\DUrole{p}{{[}}str\DUrole{p}{{]}}}\DUrole{w}{  }\DUrole{o}{=}\DUrole{w}{  }\DUrole{default_value}{None}}}{{ $\rightarrow$ dict}}
\pysigstopsignatures
\sphinxAtStartPar
Get the results of a job sent to the API service.
\begin{quote}\begin{description}
\sphinxlineitem{Parameters}
\sphinxAtStartPar
\sphinxstyleliteralstrong{\sphinxupquote{job\_id}} (\sphinxstyleliteralemphasis{\sphinxupquote{str}}\sphinxstyleliteralemphasis{\sphinxupquote{, }}\sphinxstyleliteralemphasis{\sphinxupquote{default = None}}) – The ID of the job that the results should be obtained from. If not
provided, the ID determined by the file upload is used.

\sphinxlineitem{Returns}
\sphinxAtStartPar

\sphinxAtStartPar
\sphinxstylestrong{results} – The results of the astrometry.net job. They are, in general: (If
the job has not finished yet, None is returned.)
\begin{itemize}
\item {} 
\sphinxAtStartPar
Status : The status of the job.

\item {} 
\sphinxAtStartPar
Calibration : Calibration of the image uploaded.

\item {} 
\sphinxAtStartPar
Tags : Known tagged objects in the image, people inputted.

\item {} 
\sphinxAtStartPar
Machine Tags : Ditto for tags, but only via machine inputs.

\item {} 
\sphinxAtStartPar
Objects in field : Known objects in the image field.

\item {} 
\sphinxAtStartPar
Annotations : Known objects in the field, with annotations.

\item {} 
\sphinxAtStartPar
Info : A collection of most everything above.

\end{itemize}


\sphinxlineitem{Return type}
\sphinxAtStartPar
dict

\end{description}\end{quote}

\end{fulllineitems}\end{savenotes}

\index{get\_job\_status() (opihiexarata.astrometry.webclient.AstrometryNetWebAPIEngine method)@\spxentry{get\_job\_status()}\spxextra{opihiexarata.astrometry.webclient.AstrometryNetWebAPIEngine method}}

\begin{savenotes}\begin{fulllineitems}
\phantomsection\label{\detokenize{code/opihiexarata.astrometry.webclient:opihiexarata.astrometry.webclient.AstrometryNetWebAPIEngine.get_job_status}}
\pysigstartsignatures
\pysiglinewithargsret{\sphinxbfcode{\sphinxupquote{get\_job\_status}}}{\emph{\DUrole{n}{job\_id}\DUrole{p}{:}\DUrole{w}{  }\DUrole{n}{Optional\DUrole{p}{{[}}str\DUrole{p}{{]}}}\DUrole{w}{  }\DUrole{o}{=}\DUrole{w}{  }\DUrole{default_value}{None}}}{{ $\rightarrow$ str}}
\pysigstopsignatures
\sphinxAtStartPar
Get the status of a job specified by its ID.
\begin{quote}\begin{description}
\sphinxlineitem{Parameters}
\sphinxAtStartPar
\sphinxstyleliteralstrong{\sphinxupquote{job\_id}} (\sphinxstyleliteralemphasis{\sphinxupquote{str}}\sphinxstyleliteralemphasis{\sphinxupquote{, }}\sphinxstyleliteralemphasis{\sphinxupquote{default = None}}) – The ID of the job that the results should be obtained from. If not
provided, the ID determined by the file upload is used.

\sphinxlineitem{Returns}
\sphinxAtStartPar
\sphinxstylestrong{status} – The status of the submission. If the job has not run yet, None is
returned instead.

\sphinxlineitem{Return type}
\sphinxAtStartPar
string

\end{description}\end{quote}

\end{fulllineitems}\end{savenotes}

\index{get\_reference\_star\_pixel\_correlation() (opihiexarata.astrometry.webclient.AstrometryNetWebAPIEngine method)@\spxentry{get\_reference\_star\_pixel\_correlation()}\spxextra{opihiexarata.astrometry.webclient.AstrometryNetWebAPIEngine method}}

\begin{savenotes}\begin{fulllineitems}
\phantomsection\label{\detokenize{code/opihiexarata.astrometry.webclient:opihiexarata.astrometry.webclient.AstrometryNetWebAPIEngine.get_reference_star_pixel_correlation}}
\pysigstartsignatures
\pysiglinewithargsret{\sphinxbfcode{\sphinxupquote{get\_reference\_star\_pixel\_correlation}}}{\emph{\DUrole{n}{job\_id}\DUrole{p}{:}\DUrole{w}{  }\DUrole{n}{Optional\DUrole{p}{{[}}str\DUrole{p}{{]}}}\DUrole{w}{  }\DUrole{o}{=}\DUrole{w}{  }\DUrole{default_value}{None}}, \emph{\DUrole{n}{temp\_filename}\DUrole{p}{:}\DUrole{w}{  }\DUrole{n}{Optional\DUrole{p}{{[}}str\DUrole{p}{{]}}}\DUrole{w}{  }\DUrole{o}{=}\DUrole{w}{  }\DUrole{default_value}{None}}, \emph{\DUrole{n}{delete\_after}\DUrole{p}{:}\DUrole{w}{  }\DUrole{n}{bool}\DUrole{w}{  }\DUrole{o}{=}\DUrole{w}{  }\DUrole{default_value}{True}}}{{ $\rightarrow$ Table}}
\pysigstopsignatures
\sphinxAtStartPar
This obtains the table that correlates the location of reference
stars and their pixel locations. It is obtained from the fits corr file
that is downloaded into a temporary directory.
\begin{quote}\begin{description}
\sphinxlineitem{Parameters}\begin{itemize}
\item {} 
\sphinxAtStartPar
\sphinxstyleliteralstrong{\sphinxupquote{job\_id}} (\sphinxstyleliteralemphasis{\sphinxupquote{string}}\sphinxstyleliteralemphasis{\sphinxupquote{, }}\sphinxstyleliteralemphasis{\sphinxupquote{default = None}}) – The ID of the job that the results should be obtained from. If not
provided, the ID determined by the file upload is used.

\item {} 
\sphinxAtStartPar
\sphinxstyleliteralstrong{\sphinxupquote{temp\_filename}} (\sphinxstyleliteralemphasis{\sphinxupquote{string}}\sphinxstyleliteralemphasis{\sphinxupquote{, }}\sphinxstyleliteralemphasis{\sphinxupquote{default = None}}) – The filename that the downloaded correlation file will be
downloaded as. The path is going to still be in the temporary
directory.

\item {} 
\sphinxAtStartPar
\sphinxstyleliteralstrong{\sphinxupquote{delete\_after}} (\sphinxstyleliteralemphasis{\sphinxupquote{bool}}\sphinxstyleliteralemphasis{\sphinxupquote{, }}\sphinxstyleliteralemphasis{\sphinxupquote{default = True}}) – Delete the file after downloading it to extract its information.

\end{itemize}

\sphinxlineitem{Returns}
\sphinxAtStartPar
\sphinxstylestrong{correlation\_table} – The table which details the correlation between the coordinates of
the stars and their pixel locations.

\sphinxlineitem{Return type}
\sphinxAtStartPar
Table

\end{description}\end{quote}

\end{fulllineitems}\end{savenotes}

\index{get\_submission\_results() (opihiexarata.astrometry.webclient.AstrometryNetWebAPIEngine method)@\spxentry{get\_submission\_results()}\spxextra{opihiexarata.astrometry.webclient.AstrometryNetWebAPIEngine method}}

\begin{savenotes}\begin{fulllineitems}
\phantomsection\label{\detokenize{code/opihiexarata.astrometry.webclient:opihiexarata.astrometry.webclient.AstrometryNetWebAPIEngine.get_submission_results}}
\pysigstartsignatures
\pysiglinewithargsret{\sphinxbfcode{\sphinxupquote{get\_submission\_results}}}{\emph{\DUrole{n}{submission\_id}\DUrole{p}{:}\DUrole{w}{  }\DUrole{n}{Optional\DUrole{p}{{[}}str\DUrole{p}{{]}}}\DUrole{w}{  }\DUrole{o}{=}\DUrole{w}{  }\DUrole{default_value}{None}}}{{ $\rightarrow$ dict}}
\pysigstopsignatures
\sphinxAtStartPar
Get the results of a submission specified by its ID.
\begin{quote}\begin{description}
\sphinxlineitem{Parameters}
\sphinxAtStartPar
\sphinxstyleliteralstrong{\sphinxupquote{submission\_id}} (\sphinxstyleliteralemphasis{\sphinxupquote{str}}) – The ID of the submission. If it is not passed, the ID determined
by the file upload is used.

\sphinxlineitem{Returns}
\sphinxAtStartPar
\sphinxstylestrong{result} – The result of the submission.

\sphinxlineitem{Return type}
\sphinxAtStartPar
dict

\end{description}\end{quote}

\end{fulllineitems}\end{savenotes}

\index{get\_submission\_status() (opihiexarata.astrometry.webclient.AstrometryNetWebAPIEngine method)@\spxentry{get\_submission\_status()}\spxextra{opihiexarata.astrometry.webclient.AstrometryNetWebAPIEngine method}}

\begin{savenotes}\begin{fulllineitems}
\phantomsection\label{\detokenize{code/opihiexarata.astrometry.webclient:opihiexarata.astrometry.webclient.AstrometryNetWebAPIEngine.get_submission_status}}
\pysigstartsignatures
\pysiglinewithargsret{\sphinxbfcode{\sphinxupquote{get\_submission\_status}}}{\emph{\DUrole{n}{submission\_id}\DUrole{p}{:}\DUrole{w}{  }\DUrole{n}{Optional\DUrole{p}{{[}}str\DUrole{p}{{]}}}\DUrole{w}{  }\DUrole{o}{=}\DUrole{w}{  }\DUrole{default_value}{None}}}{{ $\rightarrow$ str}}
\pysigstopsignatures
\sphinxAtStartPar
Get the status of a submission specified by its ID.
\begin{quote}\begin{description}
\sphinxlineitem{Parameters}
\sphinxAtStartPar
\sphinxstyleliteralstrong{\sphinxupquote{submission\_id}} (\sphinxstyleliteralemphasis{\sphinxupquote{str}}\sphinxstyleliteralemphasis{\sphinxupquote{, }}\sphinxstyleliteralemphasis{\sphinxupquote{default = None}}) – The ID of the submission. If it is not passed, the ID determined
by the file upload is used.

\sphinxlineitem{Returns}
\sphinxAtStartPar
\sphinxstylestrong{status} – The status of the submission.

\sphinxlineitem{Return type}
\sphinxAtStartPar
string

\end{description}\end{quote}

\end{fulllineitems}\end{savenotes}

\index{get\_wcs() (opihiexarata.astrometry.webclient.AstrometryNetWebAPIEngine method)@\spxentry{get\_wcs()}\spxextra{opihiexarata.astrometry.webclient.AstrometryNetWebAPIEngine method}}

\begin{savenotes}\begin{fulllineitems}
\phantomsection\label{\detokenize{code/opihiexarata.astrometry.webclient:opihiexarata.astrometry.webclient.AstrometryNetWebAPIEngine.get_wcs}}
\pysigstartsignatures
\pysiglinewithargsret{\sphinxbfcode{\sphinxupquote{get\_wcs}}}{\emph{\DUrole{n}{job\_id}\DUrole{p}{:}\DUrole{w}{  }\DUrole{n}{Optional\DUrole{p}{{[}}str\DUrole{p}{{]}}}\DUrole{w}{  }\DUrole{o}{=}\DUrole{w}{  }\DUrole{default_value}{None}}, \emph{\DUrole{n}{temp\_filename}\DUrole{p}{:}\DUrole{w}{  }\DUrole{n}{Optional\DUrole{p}{{[}}str\DUrole{p}{{]}}}\DUrole{w}{  }\DUrole{o}{=}\DUrole{w}{  }\DUrole{default_value}{None}}, \emph{\DUrole{n}{delete\_after}\DUrole{p}{:}\DUrole{w}{  }\DUrole{n}{bool}\DUrole{w}{  }\DUrole{o}{=}\DUrole{w}{  }\DUrole{default_value}{True}}}{{ $\rightarrow$ WCS}}
\pysigstopsignatures
\sphinxAtStartPar
This obtains the wcs header file and then computes World Coordinate
System solution from it. Because astrometry.net computes it for us,
we just extract it from the header file using Astropy.
\begin{quote}\begin{description}
\sphinxlineitem{Parameters}\begin{itemize}
\item {} 
\sphinxAtStartPar
\sphinxstyleliteralstrong{\sphinxupquote{job\_id}} (\sphinxstyleliteralemphasis{\sphinxupquote{string}}\sphinxstyleliteralemphasis{\sphinxupquote{, }}\sphinxstyleliteralemphasis{\sphinxupquote{default = None}}) – The ID of the job that the results should be obtained from. If not
provided, the ID determined by the file upload is used.

\item {} 
\sphinxAtStartPar
\sphinxstyleliteralstrong{\sphinxupquote{temp\_filename}} (\sphinxstyleliteralemphasis{\sphinxupquote{string}}\sphinxstyleliteralemphasis{\sphinxupquote{, }}\sphinxstyleliteralemphasis{\sphinxupquote{default = None}}) – The filename that the downloaded wcs file will be downloaded as.
The path is going to still be in the temporary directory.

\item {} 
\sphinxAtStartPar
\sphinxstyleliteralstrong{\sphinxupquote{delete\_after}} (\sphinxstyleliteralemphasis{\sphinxupquote{bool}}\sphinxstyleliteralemphasis{\sphinxupquote{, }}\sphinxstyleliteralemphasis{\sphinxupquote{default = True}}) – Delete the file after downloading it to extract its information.

\end{itemize}

\sphinxlineitem{Returns}
\sphinxAtStartPar
\sphinxstylestrong{wcs} – The world coordinate solution class for the image provided.

\sphinxlineitem{Return type}
\sphinxAtStartPar
Astropy WCS

\end{description}\end{quote}

\end{fulllineitems}\end{savenotes}

\index{job\_id (opihiexarata.astrometry.webclient.AstrometryNetWebAPIEngine property)@\spxentry{job\_id}\spxextra{opihiexarata.astrometry.webclient.AstrometryNetWebAPIEngine property}}

\begin{savenotes}\begin{fulllineitems}
\phantomsection\label{\detokenize{code/opihiexarata.astrometry.webclient:opihiexarata.astrometry.webclient.AstrometryNetWebAPIEngine.job_id}}
\pysigstartsignatures
\pysigline{\sphinxbfcode{\sphinxupquote{property\DUrole{w}{  }}}\sphinxbfcode{\sphinxupquote{job\_id}}\sphinxbfcode{\sphinxupquote{\DUrole{p}{:}\DUrole{w}{  }str}}}
\pysigstopsignatures
\sphinxAtStartPar
When file upload or table upload is sent to the API, the job ID of the submission is saved here.

\end{fulllineitems}\end{savenotes}

\index{submission\_id (opihiexarata.astrometry.webclient.AstrometryNetWebAPIEngine property)@\spxentry{submission\_id}\spxextra{opihiexarata.astrometry.webclient.AstrometryNetWebAPIEngine property}}

\begin{savenotes}\begin{fulllineitems}
\phantomsection\label{\detokenize{code/opihiexarata.astrometry.webclient:opihiexarata.astrometry.webclient.AstrometryNetWebAPIEngine.submission_id}}
\pysigstartsignatures
\pysigline{\sphinxbfcode{\sphinxupquote{property\DUrole{w}{  }}}\sphinxbfcode{\sphinxupquote{submission\_id}}\sphinxbfcode{\sphinxupquote{\DUrole{p}{:}\DUrole{w}{  }str}}}
\pysigstopsignatures
\sphinxAtStartPar
When file upload or table upload is sent to the API, the submission ID is saved here.

\end{fulllineitems}\end{savenotes}

\index{upload\_file() (opihiexarata.astrometry.webclient.AstrometryNetWebAPIEngine method)@\spxentry{upload\_file()}\spxextra{opihiexarata.astrometry.webclient.AstrometryNetWebAPIEngine method}}

\begin{savenotes}\begin{fulllineitems}
\phantomsection\label{\detokenize{code/opihiexarata.astrometry.webclient:opihiexarata.astrometry.webclient.AstrometryNetWebAPIEngine.upload_file}}
\pysigstartsignatures
\pysiglinewithargsret{\sphinxbfcode{\sphinxupquote{upload\_file}}}{\emph{\DUrole{n}{pathname}\DUrole{p}{:}\DUrole{w}{  }\DUrole{n}{str}}, \emph{\DUrole{o}{**}\DUrole{n}{kwargs}}}{{ $\rightarrow$ dict}}
\pysigstopsignatures
\sphinxAtStartPar
A wrapper to allow for the uploading of files or images to the API.

\sphinxAtStartPar
This also determines the submission ID and the job ID for the uploaded
image and saves it.
\begin{quote}\begin{description}
\sphinxlineitem{Parameters}
\sphinxAtStartPar
\sphinxstyleliteralstrong{\sphinxupquote{pathname}} (\sphinxstyleliteralemphasis{\sphinxupquote{str}}) – The pathname of the file to open. The filename is extracted and
used as well.

\sphinxlineitem{Returns}
\sphinxAtStartPar
\sphinxstylestrong{results} – The results of the API call to upload the image.

\sphinxlineitem{Return type}
\sphinxAtStartPar
dict

\end{description}\end{quote}

\end{fulllineitems}\end{savenotes}


\end{fulllineitems}\end{savenotes}



\subparagraph{Module contents}
\label{\detokenize{code/opihiexarata.astrometry:module-opihiexarata.astrometry}}\label{\detokenize{code/opihiexarata.astrometry:module-contents}}\index{module@\spxentry{module}!opihiexarata.astrometry@\spxentry{opihiexarata.astrometry}}\index{opihiexarata.astrometry@\spxentry{opihiexarata.astrometry}!module@\spxentry{module}}
\sphinxstepscope


\paragraph{opihiexarata.ephemeris package}
\label{\detokenize{code/opihiexarata.ephemeris:opihiexarata-ephemeris-package}}\label{\detokenize{code/opihiexarata.ephemeris::doc}}

\subparagraph{Submodules}
\label{\detokenize{code/opihiexarata.ephemeris:submodules}}
\sphinxstepscope


\subparagraph{opihiexarata.ephemeris.jplhorizons module}
\label{\detokenize{code/opihiexarata.ephemeris.jplhorizons:module-opihiexarata.ephemeris.jplhorizons}}\label{\detokenize{code/opihiexarata.ephemeris.jplhorizons:opihiexarata-ephemeris-jplhorizons-module}}\label{\detokenize{code/opihiexarata.ephemeris.jplhorizons::doc}}\index{module@\spxentry{module}!opihiexarata.ephemeris.jplhorizons@\spxentry{opihiexarata.ephemeris.jplhorizons}}\index{opihiexarata.ephemeris.jplhorizons@\spxentry{opihiexarata.ephemeris.jplhorizons}!module@\spxentry{module}}
\sphinxAtStartPar
The different ephemeris engines which use JPL horizons as its backend.
\index{JPLHorizonsWebAPIEngine (class in opihiexarata.ephemeris.jplhorizons)@\spxentry{JPLHorizonsWebAPIEngine}\spxextra{class in opihiexarata.ephemeris.jplhorizons}}

\begin{savenotes}\begin{fulllineitems}
\phantomsection\label{\detokenize{code/opihiexarata.ephemeris.jplhorizons:opihiexarata.ephemeris.jplhorizons.JPLHorizonsWebAPIEngine}}
\pysigstartsignatures
\pysiglinewithargsret{\sphinxbfcode{\sphinxupquote{class\DUrole{w}{  }}}\sphinxcode{\sphinxupquote{opihiexarata.ephemeris.jplhorizons.}}\sphinxbfcode{\sphinxupquote{JPLHorizonsWebAPIEngine}}}{\emph{\DUrole{n}{semimajor\_axis}\DUrole{p}{:}\DUrole{w}{  }\DUrole{n}{float}}, \emph{\DUrole{n}{eccentricity}\DUrole{p}{:}\DUrole{w}{  }\DUrole{n}{float}}, \emph{\DUrole{n}{inclination}\DUrole{p}{:}\DUrole{w}{  }\DUrole{n}{float}}, \emph{\DUrole{n}{longitude\_ascending\_node}\DUrole{p}{:}\DUrole{w}{  }\DUrole{n}{float}}, \emph{\DUrole{n}{argument\_perihelion}\DUrole{p}{:}\DUrole{w}{  }\DUrole{n}{float}}, \emph{\DUrole{n}{mean\_anomaly}\DUrole{p}{:}\DUrole{w}{  }\DUrole{n}{float}}, \emph{\DUrole{n}{epoch}\DUrole{p}{:}\DUrole{w}{  }\DUrole{n}{float}}}{}
\pysigstopsignatures
\sphinxAtStartPar
Bases: {\hyperref[\detokenize{code/opihiexarata.library.engine:opihiexarata.library.engine.EphemerisEngine}]{\sphinxcrossref{\sphinxcode{\sphinxupquote{EphemerisEngine}}}}}

\sphinxAtStartPar
This obtains the ephemeris of an asteroid using JPL horizons provided
the Keplerian orbital elements of the asteroid as determined by orbital
solutions.
\index{semimajor\_axis (opihiexarata.ephemeris.jplhorizons.JPLHorizonsWebAPIEngine attribute)@\spxentry{semimajor\_axis}\spxextra{opihiexarata.ephemeris.jplhorizons.JPLHorizonsWebAPIEngine attribute}}

\begin{savenotes}\begin{fulllineitems}
\phantomsection\label{\detokenize{code/opihiexarata.ephemeris.jplhorizons:opihiexarata.ephemeris.jplhorizons.JPLHorizonsWebAPIEngine.semimajor_axis}}
\pysigstartsignatures
\pysigline{\sphinxbfcode{\sphinxupquote{semimajor\_axis}}}
\pysigstopsignatures
\sphinxAtStartPar
The semi\sphinxhyphen{}major axis of the orbit, in AU.
\begin{quote}\begin{description}
\sphinxlineitem{Type}
\sphinxAtStartPar
float

\end{description}\end{quote}

\end{fulllineitems}\end{savenotes}

\index{eccentricity (opihiexarata.ephemeris.jplhorizons.JPLHorizonsWebAPIEngine attribute)@\spxentry{eccentricity}\spxextra{opihiexarata.ephemeris.jplhorizons.JPLHorizonsWebAPIEngine attribute}}

\begin{savenotes}\begin{fulllineitems}
\phantomsection\label{\detokenize{code/opihiexarata.ephemeris.jplhorizons:opihiexarata.ephemeris.jplhorizons.JPLHorizonsWebAPIEngine.eccentricity}}
\pysigstartsignatures
\pysigline{\sphinxbfcode{\sphinxupquote{eccentricity}}}
\pysigstopsignatures
\sphinxAtStartPar
The eccentricity of the orbit.
\begin{quote}\begin{description}
\sphinxlineitem{Type}
\sphinxAtStartPar
float

\end{description}\end{quote}

\end{fulllineitems}\end{savenotes}

\index{inclination (opihiexarata.ephemeris.jplhorizons.JPLHorizonsWebAPIEngine attribute)@\spxentry{inclination}\spxextra{opihiexarata.ephemeris.jplhorizons.JPLHorizonsWebAPIEngine attribute}}

\begin{savenotes}\begin{fulllineitems}
\phantomsection\label{\detokenize{code/opihiexarata.ephemeris.jplhorizons:opihiexarata.ephemeris.jplhorizons.JPLHorizonsWebAPIEngine.inclination}}
\pysigstartsignatures
\pysigline{\sphinxbfcode{\sphinxupquote{inclination}}}
\pysigstopsignatures
\sphinxAtStartPar
The angle of inclination of the orbit, in degrees.
\begin{quote}\begin{description}
\sphinxlineitem{Type}
\sphinxAtStartPar
float

\end{description}\end{quote}

\end{fulllineitems}\end{savenotes}

\index{longitude\_ascending\_node (opihiexarata.ephemeris.jplhorizons.JPLHorizonsWebAPIEngine attribute)@\spxentry{longitude\_ascending\_node}\spxextra{opihiexarata.ephemeris.jplhorizons.JPLHorizonsWebAPIEngine attribute}}

\begin{savenotes}\begin{fulllineitems}
\phantomsection\label{\detokenize{code/opihiexarata.ephemeris.jplhorizons:opihiexarata.ephemeris.jplhorizons.JPLHorizonsWebAPIEngine.longitude_ascending_node}}
\pysigstartsignatures
\pysigline{\sphinxbfcode{\sphinxupquote{longitude\_ascending\_node}}}
\pysigstopsignatures
\sphinxAtStartPar
The longitude of the ascending node of the orbit, in degrees.
\begin{quote}\begin{description}
\sphinxlineitem{Type}
\sphinxAtStartPar
float

\end{description}\end{quote}

\end{fulllineitems}\end{savenotes}

\index{argument\_perihelion (opihiexarata.ephemeris.jplhorizons.JPLHorizonsWebAPIEngine attribute)@\spxentry{argument\_perihelion}\spxextra{opihiexarata.ephemeris.jplhorizons.JPLHorizonsWebAPIEngine attribute}}

\begin{savenotes}\begin{fulllineitems}
\phantomsection\label{\detokenize{code/opihiexarata.ephemeris.jplhorizons:opihiexarata.ephemeris.jplhorizons.JPLHorizonsWebAPIEngine.argument_perihelion}}
\pysigstartsignatures
\pysigline{\sphinxbfcode{\sphinxupquote{argument\_perihelion}}}
\pysigstopsignatures
\sphinxAtStartPar
The argument of perihelion of the orbit, in degrees.
\begin{quote}\begin{description}
\sphinxlineitem{Type}
\sphinxAtStartPar
float

\end{description}\end{quote}

\end{fulllineitems}\end{savenotes}

\index{mean\_anomaly (opihiexarata.ephemeris.jplhorizons.JPLHorizonsWebAPIEngine attribute)@\spxentry{mean\_anomaly}\spxextra{opihiexarata.ephemeris.jplhorizons.JPLHorizonsWebAPIEngine attribute}}

\begin{savenotes}\begin{fulllineitems}
\phantomsection\label{\detokenize{code/opihiexarata.ephemeris.jplhorizons:opihiexarata.ephemeris.jplhorizons.JPLHorizonsWebAPIEngine.mean_anomaly}}
\pysigstartsignatures
\pysigline{\sphinxbfcode{\sphinxupquote{mean\_anomaly}}}
\pysigstopsignatures
\sphinxAtStartPar
The mean anomaly of the orbit, in degrees.
\begin{quote}\begin{description}
\sphinxlineitem{Type}
\sphinxAtStartPar
float

\end{description}\end{quote}

\end{fulllineitems}\end{savenotes}

\index{epoch (opihiexarata.ephemeris.jplhorizons.JPLHorizonsWebAPIEngine attribute)@\spxentry{epoch}\spxextra{opihiexarata.ephemeris.jplhorizons.JPLHorizonsWebAPIEngine attribute}}

\begin{savenotes}\begin{fulllineitems}
\phantomsection\label{\detokenize{code/opihiexarata.ephemeris.jplhorizons:opihiexarata.ephemeris.jplhorizons.JPLHorizonsWebAPIEngine.epoch}}
\pysigstartsignatures
\pysigline{\sphinxbfcode{\sphinxupquote{epoch}}}
\pysigstopsignatures
\sphinxAtStartPar
The modified Julian date epoch of the osculating orbital elements.
\begin{quote}\begin{description}
\sphinxlineitem{Type}
\sphinxAtStartPar
float

\end{description}\end{quote}

\end{fulllineitems}\end{savenotes}

\index{ra\_velocity (opihiexarata.ephemeris.jplhorizons.JPLHorizonsWebAPIEngine attribute)@\spxentry{ra\_velocity}\spxextra{opihiexarata.ephemeris.jplhorizons.JPLHorizonsWebAPIEngine attribute}}

\begin{savenotes}\begin{fulllineitems}
\phantomsection\label{\detokenize{code/opihiexarata.ephemeris.jplhorizons:opihiexarata.ephemeris.jplhorizons.JPLHorizonsWebAPIEngine.ra_velocity}}
\pysigstartsignatures
\pysigline{\sphinxbfcode{\sphinxupquote{ra\_velocity}}}
\pysigstopsignatures
\sphinxAtStartPar
The right ascension angular velocity of the target, in degrees per
second.
\begin{quote}\begin{description}
\sphinxlineitem{Type}
\sphinxAtStartPar
float

\end{description}\end{quote}

\end{fulllineitems}\end{savenotes}

\index{dec\_velocity (opihiexarata.ephemeris.jplhorizons.JPLHorizonsWebAPIEngine attribute)@\spxentry{dec\_velocity}\spxextra{opihiexarata.ephemeris.jplhorizons.JPLHorizonsWebAPIEngine attribute}}

\begin{savenotes}\begin{fulllineitems}
\phantomsection\label{\detokenize{code/opihiexarata.ephemeris.jplhorizons:opihiexarata.ephemeris.jplhorizons.JPLHorizonsWebAPIEngine.dec_velocity}}
\pysigstartsignatures
\pysigline{\sphinxbfcode{\sphinxupquote{dec\_velocity}}}
\pysigstopsignatures
\sphinxAtStartPar
The declination angular velocity of the target, in degrees per
second.
\begin{quote}\begin{description}
\sphinxlineitem{Type}
\sphinxAtStartPar
float

\end{description}\end{quote}

\end{fulllineitems}\end{savenotes}

\index{ra\_acceleration (opihiexarata.ephemeris.jplhorizons.JPLHorizonsWebAPIEngine attribute)@\spxentry{ra\_acceleration}\spxextra{opihiexarata.ephemeris.jplhorizons.JPLHorizonsWebAPIEngine attribute}}

\begin{savenotes}\begin{fulllineitems}
\phantomsection\label{\detokenize{code/opihiexarata.ephemeris.jplhorizons:opihiexarata.ephemeris.jplhorizons.JPLHorizonsWebAPIEngine.ra_acceleration}}
\pysigstartsignatures
\pysigline{\sphinxbfcode{\sphinxupquote{ra\_acceleration}}}
\pysigstopsignatures
\sphinxAtStartPar
The right ascension angular acceleration of the target, in degrees per
second squared.
\begin{quote}\begin{description}
\sphinxlineitem{Type}
\sphinxAtStartPar
float

\end{description}\end{quote}

\end{fulllineitems}\end{savenotes}

\index{dec\_acceleration (opihiexarata.ephemeris.jplhorizons.JPLHorizonsWebAPIEngine attribute)@\spxentry{dec\_acceleration}\spxextra{opihiexarata.ephemeris.jplhorizons.JPLHorizonsWebAPIEngine attribute}}

\begin{savenotes}\begin{fulllineitems}
\phantomsection\label{\detokenize{code/opihiexarata.ephemeris.jplhorizons:opihiexarata.ephemeris.jplhorizons.JPLHorizonsWebAPIEngine.dec_acceleration}}
\pysigstartsignatures
\pysigline{\sphinxbfcode{\sphinxupquote{dec\_acceleration}}}
\pysigstopsignatures
\sphinxAtStartPar
The declination angular acceleration of the target, in degrees per
second squared.
\begin{quote}\begin{description}
\sphinxlineitem{Type}
\sphinxAtStartPar
float

\end{description}\end{quote}

\end{fulllineitems}\end{savenotes}

\index{\_\_init\_\_() (opihiexarata.ephemeris.jplhorizons.JPLHorizonsWebAPIEngine method)@\spxentry{\_\_init\_\_()}\spxextra{opihiexarata.ephemeris.jplhorizons.JPLHorizonsWebAPIEngine method}}

\begin{savenotes}\begin{fulllineitems}
\phantomsection\label{\detokenize{code/opihiexarata.ephemeris.jplhorizons:opihiexarata.ephemeris.jplhorizons.JPLHorizonsWebAPIEngine.__init__}}
\pysigstartsignatures
\pysiglinewithargsret{\sphinxbfcode{\sphinxupquote{\_\_init\_\_}}}{\emph{\DUrole{n}{semimajor\_axis}\DUrole{p}{:}\DUrole{w}{  }\DUrole{n}{float}}, \emph{\DUrole{n}{eccentricity}\DUrole{p}{:}\DUrole{w}{  }\DUrole{n}{float}}, \emph{\DUrole{n}{inclination}\DUrole{p}{:}\DUrole{w}{  }\DUrole{n}{float}}, \emph{\DUrole{n}{longitude\_ascending\_node}\DUrole{p}{:}\DUrole{w}{  }\DUrole{n}{float}}, \emph{\DUrole{n}{argument\_perihelion}\DUrole{p}{:}\DUrole{w}{  }\DUrole{n}{float}}, \emph{\DUrole{n}{mean\_anomaly}\DUrole{p}{:}\DUrole{w}{  }\DUrole{n}{float}}, \emph{\DUrole{n}{epoch}\DUrole{p}{:}\DUrole{w}{  }\DUrole{n}{float}}}{{ $\rightarrow$ None}}
\pysigstopsignatures
\sphinxAtStartPar
Creating the engine.
\begin{quote}\begin{description}
\sphinxlineitem{Parameters}\begin{itemize}
\item {} 
\sphinxAtStartPar
\sphinxstyleliteralstrong{\sphinxupquote{semimajor\_axis}} (\sphinxstyleliteralemphasis{\sphinxupquote{float}}) – The semi\sphinxhyphen{}major axis of the orbit, in AU.

\item {} 
\sphinxAtStartPar
\sphinxstyleliteralstrong{\sphinxupquote{eccentricity}} (\sphinxstyleliteralemphasis{\sphinxupquote{float}}) – The eccentricity of the orbit.

\item {} 
\sphinxAtStartPar
\sphinxstyleliteralstrong{\sphinxupquote{inclination}} (\sphinxstyleliteralemphasis{\sphinxupquote{float}}) – The angle of inclination of the orbit, in degrees.

\item {} 
\sphinxAtStartPar
\sphinxstyleliteralstrong{\sphinxupquote{longitude\_ascending\_node}} (\sphinxstyleliteralemphasis{\sphinxupquote{float}}) – The longitude of the ascending node of the orbit, in degrees.

\item {} 
\sphinxAtStartPar
\sphinxstyleliteralstrong{\sphinxupquote{argument\_perihelion}} (\sphinxstyleliteralemphasis{\sphinxupquote{float}}) – The argument of perihelion of the orbit, in degrees.

\item {} 
\sphinxAtStartPar
\sphinxstyleliteralstrong{\sphinxupquote{mean\_anomaly}} (\sphinxstyleliteralemphasis{\sphinxupquote{float}}) – The mean anomaly of the orbit, in degrees.

\item {} 
\sphinxAtStartPar
\sphinxstyleliteralstrong{\sphinxupquote{epoch}} (\sphinxstyleliteralemphasis{\sphinxupquote{float}}) – The full Julian date epoch of these osculating orbital elements.

\end{itemize}

\sphinxlineitem{Return type}
\sphinxAtStartPar
None

\end{description}\end{quote}

\end{fulllineitems}\end{savenotes}

\index{\_\_parse\_jpl\_horizons\_output() (opihiexarata.ephemeris.jplhorizons.JPLHorizonsWebAPIEngine method)@\spxentry{\_\_parse\_jpl\_horizons\_output()}\spxextra{opihiexarata.ephemeris.jplhorizons.JPLHorizonsWebAPIEngine method}}

\begin{savenotes}\begin{fulllineitems}
\phantomsection\label{\detokenize{code/opihiexarata.ephemeris.jplhorizons:opihiexarata.ephemeris.jplhorizons.JPLHorizonsWebAPIEngine.__parse_jpl_horizons_output}}
\pysigstartsignatures
\pysiglinewithargsret{\sphinxbfcode{\sphinxupquote{\_\_parse\_jpl\_horizons\_output}}}{}{{ $\rightarrow$ Table}}
\pysigstopsignatures
\sphinxAtStartPar
This function serves to parse the output from the JPL horizons. It
is a text output that is human readable but some parsing is needed.
We do that here, assuming the quantities in the original request.
\begin{quote}\begin{description}
\sphinxlineitem{Parameters}
\sphinxAtStartPar
\sphinxstyleliteralstrong{\sphinxupquote{response\_text}} (\sphinxstyleliteralemphasis{\sphinxupquote{str}}) – The raw response from the JPL horizons web API service.

\end{description}\end{quote}

\end{fulllineitems}\end{savenotes}

\index{\_query\_jpl\_horizons() (opihiexarata.ephemeris.jplhorizons.JPLHorizonsWebAPIEngine method)@\spxentry{\_query\_jpl\_horizons()}\spxextra{opihiexarata.ephemeris.jplhorizons.JPLHorizonsWebAPIEngine method}}

\begin{savenotes}\begin{fulllineitems}
\phantomsection\label{\detokenize{code/opihiexarata.ephemeris.jplhorizons:opihiexarata.ephemeris.jplhorizons.JPLHorizonsWebAPIEngine._query_jpl_horizons}}
\pysigstartsignatures
\pysiglinewithargsret{\sphinxbfcode{\sphinxupquote{\_query\_jpl\_horizons}}}{\emph{\DUrole{n}{start\_time}\DUrole{p}{:}\DUrole{w}{  }\DUrole{n}{float}}, \emph{\DUrole{n}{stop\_time}\DUrole{p}{:}\DUrole{w}{  }\DUrole{n}{float}}, \emph{\DUrole{n}{time\_step}\DUrole{p}{:}\DUrole{w}{  }\DUrole{n}{float}}}{{ $\rightarrow$ Table}}
\pysigstopsignatures
\sphinxAtStartPar
This function queries JPL horizons. Using the current orbital
elements, and provided a minimum time, maximum time, and time step,
we can fetch the new table of ephemeris measurements.
\begin{quote}\begin{description}
\sphinxlineitem{Parameters}\begin{itemize}
\item {} 
\sphinxAtStartPar
\sphinxstyleliteralstrong{\sphinxupquote{start\_time}} (\sphinxstyleliteralemphasis{\sphinxupquote{float}}) – The time that the ephemeris should start at, in Julian days.

\item {} 
\sphinxAtStartPar
\sphinxstyleliteralstrong{\sphinxupquote{stop\_time}} (\sphinxstyleliteralemphasis{\sphinxupquote{float}}) – The time that the ephemeris should stop at, in Julian days.

\item {} 
\sphinxAtStartPar
\sphinxstyleliteralstrong{\sphinxupquote{time\_step}} (\sphinxstyleliteralemphasis{\sphinxupquote{float}}) – The time step of the entries of the ephemeris calculation, in
seconds. (Note, JPL does not accept anything less than a minute.)

\end{itemize}

\sphinxlineitem{Returns}
\sphinxAtStartPar
\sphinxstylestrong{epidermis\_table} – The table of computed sky locations over time from JPL horizons.

\sphinxlineitem{Return type}
\sphinxAtStartPar
Table

\end{description}\end{quote}

\end{fulllineitems}\end{savenotes}

\index{\_refresh\_ephemeris() (opihiexarata.ephemeris.jplhorizons.JPLHorizonsWebAPIEngine method)@\spxentry{\_refresh\_ephemeris()}\spxextra{opihiexarata.ephemeris.jplhorizons.JPLHorizonsWebAPIEngine method}}

\begin{savenotes}\begin{fulllineitems}
\phantomsection\label{\detokenize{code/opihiexarata.ephemeris.jplhorizons:opihiexarata.ephemeris.jplhorizons.JPLHorizonsWebAPIEngine._refresh_ephemeris}}
\pysigstartsignatures
\pysiglinewithargsret{\sphinxbfcode{\sphinxupquote{\_refresh\_ephemeris}}}{\emph{\DUrole{n}{start\_time}\DUrole{p}{:}\DUrole{w}{  }\DUrole{n}{Optional\DUrole{p}{{[}}float\DUrole{p}{{]}}}\DUrole{w}{  }\DUrole{o}{=}\DUrole{w}{  }\DUrole{default_value}{None}}, \emph{\DUrole{n}{stop\_time}\DUrole{p}{:}\DUrole{w}{  }\DUrole{n}{Optional\DUrole{p}{{[}}float\DUrole{p}{{]}}}\DUrole{w}{  }\DUrole{o}{=}\DUrole{w}{  }\DUrole{default_value}{None}}, \emph{\DUrole{n}{time\_step}\DUrole{p}{:}\DUrole{w}{  }\DUrole{n}{Optional\DUrole{p}{{[}}float\DUrole{p}{{]}}}\DUrole{w}{  }\DUrole{o}{=}\DUrole{w}{  }\DUrole{default_value}{None}}}{{ $\rightarrow$ None}}
\pysigstopsignatures
\sphinxAtStartPar
This function refreshes the calculations rederiving the solution
table and the ephemeris function.
\begin{quote}\begin{description}
\sphinxlineitem{Parameters}\begin{itemize}
\item {} 
\sphinxAtStartPar
\sphinxstyleliteralstrong{\sphinxupquote{start\_time}} (\sphinxstyleliteralemphasis{\sphinxupquote{float}}\sphinxstyleliteralemphasis{\sphinxupquote{, }}\sphinxstyleliteralemphasis{\sphinxupquote{default = None}}) – The time that the ephemeris should start at, in Julian days. If
not provided, then the old start time will be used instead.

\item {} 
\sphinxAtStartPar
\sphinxstyleliteralstrong{\sphinxupquote{stop\_time}} (\sphinxstyleliteralemphasis{\sphinxupquote{float}}\sphinxstyleliteralemphasis{\sphinxupquote{, }}\sphinxstyleliteralemphasis{\sphinxupquote{default = None}}) – The time that the ephemeris should stop at, in Julian days. If
not provided, then the old stop time will be used instead.

\item {} 
\sphinxAtStartPar
\sphinxstyleliteralstrong{\sphinxupquote{time\_step}} (\sphinxstyleliteralemphasis{\sphinxupquote{float}}\sphinxstyleliteralemphasis{\sphinxupquote{, }}\sphinxstyleliteralemphasis{\sphinxupquote{default = None}}) – The time step of the entries of the ephemeris calculation, in
seconds. (Note, JPL does not accept anything less than a minute.)
If not provided, then the old time step will be used instead.

\end{itemize}

\sphinxlineitem{Return type}
\sphinxAtStartPar
None

\end{description}\end{quote}

\end{fulllineitems}\end{savenotes}

\index{forward\_ephemeris() (opihiexarata.ephemeris.jplhorizons.JPLHorizonsWebAPIEngine method)@\spxentry{forward\_ephemeris()}\spxextra{opihiexarata.ephemeris.jplhorizons.JPLHorizonsWebAPIEngine method}}

\begin{savenotes}\begin{fulllineitems}
\phantomsection\label{\detokenize{code/opihiexarata.ephemeris.jplhorizons:opihiexarata.ephemeris.jplhorizons.JPLHorizonsWebAPIEngine.forward_ephemeris}}
\pysigstartsignatures
\pysiglinewithargsret{\sphinxbfcode{\sphinxupquote{forward\_ephemeris}}}{\emph{\DUrole{n}{future\_time}\DUrole{p}{:}\DUrole{w}{  }\DUrole{n}{float}}}{{ $\rightarrow$ tuple\DUrole{p}{{[}}float\DUrole{p}{,}\DUrole{w}{  }float\DUrole{p}{{]}}}}
\pysigstopsignatures
\sphinxAtStartPar
This allows the computation of future positions at a future time
using the derived ephemeris.
\begin{quote}\begin{description}
\sphinxlineitem{Parameters}
\sphinxAtStartPar
\sphinxstyleliteralstrong{\sphinxupquote{future\_time}} (\sphinxstyleliteralemphasis{\sphinxupquote{array\sphinxhyphen{}like}}) – The set of future times which to derive new RA and DEC coordinates.
The time must be in Julian day time.

\sphinxlineitem{Returns}
\sphinxAtStartPar
\begin{itemize}
\item {} 
\sphinxAtStartPar
\sphinxstylestrong{future\_ra} (\sphinxstyleemphasis{ndarray}) – The set of right ascensions that corresponds to the future times,
in degrees.

\item {} 
\sphinxAtStartPar
\sphinxstylestrong{future\_dec} (\sphinxstyleemphasis{ndarray}) – The set of declinations that corresponds to the future times, in
degrees.

\end{itemize}


\end{description}\end{quote}

\end{fulllineitems}\end{savenotes}


\end{fulllineitems}\end{savenotes}


\sphinxstepscope


\subparagraph{opihiexarata.ephemeris.solution module}
\label{\detokenize{code/opihiexarata.ephemeris.solution:module-opihiexarata.ephemeris.solution}}\label{\detokenize{code/opihiexarata.ephemeris.solution:opihiexarata-ephemeris-solution-module}}\label{\detokenize{code/opihiexarata.ephemeris.solution::doc}}\index{module@\spxentry{module}!opihiexarata.ephemeris.solution@\spxentry{opihiexarata.ephemeris.solution}}\index{opihiexarata.ephemeris.solution@\spxentry{opihiexarata.ephemeris.solution}!module@\spxentry{module}}
\sphinxAtStartPar
The ephemeris solution class.
\index{EphemeriticSolution (class in opihiexarata.ephemeris.solution)@\spxentry{EphemeriticSolution}\spxextra{class in opihiexarata.ephemeris.solution}}

\begin{savenotes}\begin{fulllineitems}
\phantomsection\label{\detokenize{code/opihiexarata.ephemeris.solution:opihiexarata.ephemeris.solution.EphemeriticSolution}}
\pysigstartsignatures
\pysiglinewithargsret{\sphinxbfcode{\sphinxupquote{class\DUrole{w}{  }}}\sphinxcode{\sphinxupquote{opihiexarata.ephemeris.solution.}}\sphinxbfcode{\sphinxupquote{EphemeriticSolution}}}{\emph{\DUrole{n}{orbitals}\DUrole{p}{:}\DUrole{w}{  }\DUrole{n}{{\hyperref[\detokenize{code/opihiexarata.orbit.solution:opihiexarata.orbit.solution.OrbitalSolution}]{\sphinxcrossref{OrbitalSolution}}}}}, \emph{\DUrole{n}{solver\_engine}\DUrole{p}{:}\DUrole{w}{  }\DUrole{n}{{\hyperref[\detokenize{code/opihiexarata.library.engine:opihiexarata.library.engine.EphemerisEngine}]{\sphinxcrossref{EphemerisEngine}}}}}, \emph{\DUrole{n}{vehicle\_args}\DUrole{p}{:}\DUrole{w}{  }\DUrole{n}{dict}\DUrole{w}{  }\DUrole{o}{=}\DUrole{w}{  }\DUrole{default_value}{\{\}}}}{}
\pysigstopsignatures
\sphinxAtStartPar
Bases: {\hyperref[\detokenize{code/opihiexarata.library.engine:opihiexarata.library.engine.ExarataSolution}]{\sphinxcrossref{\sphinxcode{\sphinxupquote{ExarataSolution}}}}}

\sphinxAtStartPar
This obtains the ephemeris of an asteroid using an ephemeris engine
provided the Keplerian orbital elements of the asteroid as determined
by orbital solutions.
\index{orbitals (opihiexarata.ephemeris.solution.EphemeriticSolution attribute)@\spxentry{orbitals}\spxextra{opihiexarata.ephemeris.solution.EphemeriticSolution attribute}}

\begin{savenotes}\begin{fulllineitems}
\phantomsection\label{\detokenize{code/opihiexarata.ephemeris.solution:opihiexarata.ephemeris.solution.EphemeriticSolution.orbitals}}
\pysigstartsignatures
\pysigline{\sphinxbfcode{\sphinxupquote{orbitals}}}
\pysigstopsignatures
\sphinxAtStartPar
The orbital solution from which the orbital elements will be taken from
to determine the orbit of the target.
\begin{quote}\begin{description}
\sphinxlineitem{Type}
\sphinxAtStartPar
{\hyperref[\detokenize{code/opihiexarata.orbit.solution:opihiexarata.orbit.solution.OrbitalSolution}]{\sphinxcrossref{OrbitalSolution}}}

\end{description}\end{quote}

\end{fulllineitems}\end{savenotes}

\index{ra\_velocity (opihiexarata.ephemeris.solution.EphemeriticSolution attribute)@\spxentry{ra\_velocity}\spxextra{opihiexarata.ephemeris.solution.EphemeriticSolution attribute}}

\begin{savenotes}\begin{fulllineitems}
\phantomsection\label{\detokenize{code/opihiexarata.ephemeris.solution:opihiexarata.ephemeris.solution.EphemeriticSolution.ra_velocity}}
\pysigstartsignatures
\pysigline{\sphinxbfcode{\sphinxupquote{ra\_velocity}}}
\pysigstopsignatures
\sphinxAtStartPar
The right ascension angular velocity of the target, in degrees per
second.
\begin{quote}\begin{description}
\sphinxlineitem{Type}
\sphinxAtStartPar
float

\end{description}\end{quote}

\end{fulllineitems}\end{savenotes}

\index{dec\_velocity (opihiexarata.ephemeris.solution.EphemeriticSolution attribute)@\spxentry{dec\_velocity}\spxextra{opihiexarata.ephemeris.solution.EphemeriticSolution attribute}}

\begin{savenotes}\begin{fulllineitems}
\phantomsection\label{\detokenize{code/opihiexarata.ephemeris.solution:opihiexarata.ephemeris.solution.EphemeriticSolution.dec_velocity}}
\pysigstartsignatures
\pysigline{\sphinxbfcode{\sphinxupquote{dec\_velocity}}}
\pysigstopsignatures
\sphinxAtStartPar
The declination angular velocity of the target, in degrees per
second.
\begin{quote}\begin{description}
\sphinxlineitem{Type}
\sphinxAtStartPar
float

\end{description}\end{quote}

\end{fulllineitems}\end{savenotes}

\index{ra\_acceleration (opihiexarata.ephemeris.solution.EphemeriticSolution attribute)@\spxentry{ra\_acceleration}\spxextra{opihiexarata.ephemeris.solution.EphemeriticSolution attribute}}

\begin{savenotes}\begin{fulllineitems}
\phantomsection\label{\detokenize{code/opihiexarata.ephemeris.solution:opihiexarata.ephemeris.solution.EphemeriticSolution.ra_acceleration}}
\pysigstartsignatures
\pysigline{\sphinxbfcode{\sphinxupquote{ra\_acceleration}}}
\pysigstopsignatures
\sphinxAtStartPar
The right ascension angular acceleration of the target, in degrees per
second squared.
\begin{quote}\begin{description}
\sphinxlineitem{Type}
\sphinxAtStartPar
float

\end{description}\end{quote}

\end{fulllineitems}\end{savenotes}

\index{dec\_acceleration (opihiexarata.ephemeris.solution.EphemeriticSolution attribute)@\spxentry{dec\_acceleration}\spxextra{opihiexarata.ephemeris.solution.EphemeriticSolution attribute}}

\begin{savenotes}\begin{fulllineitems}
\phantomsection\label{\detokenize{code/opihiexarata.ephemeris.solution:opihiexarata.ephemeris.solution.EphemeriticSolution.dec_acceleration}}
\pysigstartsignatures
\pysigline{\sphinxbfcode{\sphinxupquote{dec\_acceleration}}}
\pysigstopsignatures
\sphinxAtStartPar
The declination angular acceleration of the target, in degrees per
second squared.
\begin{quote}\begin{description}
\sphinxlineitem{Type}
\sphinxAtStartPar
float

\end{description}\end{quote}

\end{fulllineitems}\end{savenotes}

\index{\_\_init\_\_() (opihiexarata.ephemeris.solution.EphemeriticSolution method)@\spxentry{\_\_init\_\_()}\spxextra{opihiexarata.ephemeris.solution.EphemeriticSolution method}}

\begin{savenotes}\begin{fulllineitems}
\phantomsection\label{\detokenize{code/opihiexarata.ephemeris.solution:opihiexarata.ephemeris.solution.EphemeriticSolution.__init__}}
\pysigstartsignatures
\pysiglinewithargsret{\sphinxbfcode{\sphinxupquote{\_\_init\_\_}}}{\emph{\DUrole{n}{orbitals}\DUrole{p}{:}\DUrole{w}{  }\DUrole{n}{{\hyperref[\detokenize{code/opihiexarata.orbit.solution:opihiexarata.orbit.solution.OrbitalSolution}]{\sphinxcrossref{OrbitalSolution}}}}}, \emph{\DUrole{n}{solver\_engine}\DUrole{p}{:}\DUrole{w}{  }\DUrole{n}{{\hyperref[\detokenize{code/opihiexarata.library.engine:opihiexarata.library.engine.EphemerisEngine}]{\sphinxcrossref{EphemerisEngine}}}}}, \emph{\DUrole{n}{vehicle\_args}\DUrole{p}{:}\DUrole{w}{  }\DUrole{n}{dict}\DUrole{w}{  }\DUrole{o}{=}\DUrole{w}{  }\DUrole{default_value}{\{\}}}}{{ $\rightarrow$ None}}
\pysigstopsignatures
\sphinxAtStartPar
Instantiating the solution class.
\begin{quote}\begin{description}
\sphinxlineitem{Parameters}\begin{itemize}
\item {} 
\sphinxAtStartPar
\sphinxstyleliteralstrong{\sphinxupquote{orbitals}} ({\hyperref[\detokenize{code/opihiexarata.orbit.solution:opihiexarata.orbit.solution.OrbitalSolution}]{\sphinxcrossref{\sphinxstyleliteralemphasis{\sphinxupquote{OrbitalSolution}}}}}) – The orbital solution from which the orbital elements will be taken
from to determine the orbit of the target.

\item {} 
\sphinxAtStartPar
\sphinxstyleliteralstrong{\sphinxupquote{solver\_engine}} ({\hyperref[\detokenize{code/opihiexarata.library.engine:opihiexarata.library.engine.EphemerisEngine}]{\sphinxcrossref{\sphinxstyleliteralemphasis{\sphinxupquote{EphemerisEngine}}}}}) – The ephemeris solver engine class. This is what will act as the
“behind the scenes” and solve the orbit, using this middleware to
translate it into something that is easier to read.

\item {} 
\sphinxAtStartPar
\sphinxstyleliteralstrong{\sphinxupquote{vehicle\_args}} (\sphinxstyleliteralemphasis{\sphinxupquote{dictionary}}) – If the vehicle function for the provided solver engine needs
extra parameters not otherwise provided by the standard input,
they are given here.

\end{itemize}

\sphinxlineitem{Return type}
\sphinxAtStartPar
None

\end{description}\end{quote}

\end{fulllineitems}\end{savenotes}

\index{forward\_ephemeris() (opihiexarata.ephemeris.solution.EphemeriticSolution method)@\spxentry{forward\_ephemeris()}\spxextra{opihiexarata.ephemeris.solution.EphemeriticSolution method}}

\begin{savenotes}\begin{fulllineitems}
\phantomsection\label{\detokenize{code/opihiexarata.ephemeris.solution:opihiexarata.ephemeris.solution.EphemeriticSolution.forward_ephemeris}}
\pysigstartsignatures
\pysiglinewithargsret{\sphinxbfcode{\sphinxupquote{forward\_ephemeris}}}{\emph{\DUrole{n}{future\_time}\DUrole{p}{:}\DUrole{w}{  }\DUrole{n}{ndarray}}}{{ $\rightarrow$ tuple\DUrole{p}{{[}}numpy.ndarray\DUrole{p}{,}\DUrole{w}{  }numpy.ndarray\DUrole{p}{{]}}}}
\pysigstopsignatures
\sphinxAtStartPar
A wrapper call around the engine’s ephemeris function. This
allows the computation of future positions at a future time using
the ephemeris derived from the orbital elements.
\begin{quote}\begin{description}
\sphinxlineitem{Parameters}
\sphinxAtStartPar
\sphinxstyleliteralstrong{\sphinxupquote{future\_time}} (\sphinxstyleliteralemphasis{\sphinxupquote{array\sphinxhyphen{}like}}) – The set of future times which to derive new RA and DEC coordinates.
The time must be in Julian days.

\sphinxlineitem{Returns}
\sphinxAtStartPar
\begin{itemize}
\item {} 
\sphinxAtStartPar
\sphinxstylestrong{future\_ra} (\sphinxstyleemphasis{ndarray}) – The set of right ascensions that corresponds to the future times,
in degrees.

\item {} 
\sphinxAtStartPar
\sphinxstylestrong{future\_dec} (\sphinxstyleemphasis{ndarray}) – The set of declinations that corresponds to the future times, in
degrees.

\end{itemize}


\end{description}\end{quote}

\end{fulllineitems}\end{savenotes}


\end{fulllineitems}\end{savenotes}

\index{\_vehicle\_jpl\_horizons\_web\_api() (in module opihiexarata.ephemeris.solution)@\spxentry{\_vehicle\_jpl\_horizons\_web\_api()}\spxextra{in module opihiexarata.ephemeris.solution}}

\begin{savenotes}\begin{fulllineitems}
\phantomsection\label{\detokenize{code/opihiexarata.ephemeris.solution:opihiexarata.ephemeris.solution._vehicle_jpl_horizons_web_api}}
\pysigstartsignatures
\pysiglinewithargsret{\sphinxcode{\sphinxupquote{opihiexarata.ephemeris.solution.}}\sphinxbfcode{\sphinxupquote{\_vehicle\_jpl\_horizons\_web\_api}}}{\emph{\DUrole{n}{orbitals}\DUrole{p}{:}\DUrole{w}{  }\DUrole{n}{{\hyperref[\detokenize{code/opihiexarata.orbit.solution:opihiexarata.orbit.solution.OrbitalSolution}]{\sphinxcrossref{OrbitalSolution}}}}}}{}
\pysigstopsignatures
\sphinxAtStartPar
This uses the JPL Horizons web URL API service to derive the ephemeris.
\begin{quote}\begin{description}
\sphinxlineitem{Parameters}
\sphinxAtStartPar
\sphinxstyleliteralstrong{\sphinxupquote{orbitals}} ({\hyperref[\detokenize{code/opihiexarata.orbit.solution:opihiexarata.orbit.solution.OrbitalSolution}]{\sphinxcrossref{\sphinxstyleliteralemphasis{\sphinxupquote{OrbitalSolution}}}}}) – The orbital solution to use to get the orbital elements to send off to
the JPL Horizons API.

\sphinxlineitem{Returns}
\sphinxAtStartPar
\sphinxstylestrong{ephemeris\_results} – The results of the ephemeris engine which then gets integrated into
the solution.

\sphinxlineitem{Return type}
\sphinxAtStartPar
dictionary

\end{description}\end{quote}

\end{fulllineitems}\end{savenotes}



\subparagraph{Module contents}
\label{\detokenize{code/opihiexarata.ephemeris:module-opihiexarata.ephemeris}}\label{\detokenize{code/opihiexarata.ephemeris:module-contents}}\index{module@\spxentry{module}!opihiexarata.ephemeris@\spxentry{opihiexarata.ephemeris}}\index{opihiexarata.ephemeris@\spxentry{opihiexarata.ephemeris}!module@\spxentry{module}}
\sphinxAtStartPar
This is the module for all of the ephemeris calculations, provided
orbital elements.

\sphinxstepscope


\paragraph{opihiexarata.gui package}
\label{\detokenize{code/opihiexarata.gui:opihiexarata-gui-package}}\label{\detokenize{code/opihiexarata.gui::doc}}

\subparagraph{Subpackages}
\label{\detokenize{code/opihiexarata.gui:subpackages}}
\sphinxstepscope


\subparagraph{opihiexarata.gui.qtui package}
\label{\detokenize{code/opihiexarata.gui.qtui:opihiexarata-gui-qtui-package}}\label{\detokenize{code/opihiexarata.gui.qtui::doc}}

\subparagraph{Submodules}
\label{\detokenize{code/opihiexarata.gui.qtui:submodules}}
\sphinxstepscope


\subparagraph{opihiexarata.gui.qtui.qtui\_automatic module}
\label{\detokenize{code/opihiexarata.gui.qtui.qtui_automatic:module-opihiexarata.gui.qtui.qtui_automatic}}\label{\detokenize{code/opihiexarata.gui.qtui.qtui_automatic:opihiexarata-gui-qtui-qtui-automatic-module}}\label{\detokenize{code/opihiexarata.gui.qtui.qtui_automatic::doc}}\index{module@\spxentry{module}!opihiexarata.gui.qtui.qtui\_automatic@\spxentry{opihiexarata.gui.qtui.qtui\_automatic}}\index{opihiexarata.gui.qtui.qtui\_automatic@\spxentry{opihiexarata.gui.qtui.qtui\_automatic}!module@\spxentry{module}}\index{Ui\_AutomaticWindow (class in opihiexarata.gui.qtui.qtui\_automatic)@\spxentry{Ui\_AutomaticWindow}\spxextra{class in opihiexarata.gui.qtui.qtui\_automatic}}

\begin{savenotes}\begin{fulllineitems}
\phantomsection\label{\detokenize{code/opihiexarata.gui.qtui.qtui_automatic:opihiexarata.gui.qtui.qtui_automatic.Ui_AutomaticWindow}}
\pysigstartsignatures
\pysigline{\sphinxbfcode{\sphinxupquote{class\DUrole{w}{  }}}\sphinxcode{\sphinxupquote{opihiexarata.gui.qtui.qtui\_automatic.}}\sphinxbfcode{\sphinxupquote{Ui\_AutomaticWindow}}}
\pysigstopsignatures
\sphinxAtStartPar
Bases: \sphinxcode{\sphinxupquote{object}}
\index{retranslateUi() (opihiexarata.gui.qtui.qtui\_automatic.Ui\_AutomaticWindow method)@\spxentry{retranslateUi()}\spxextra{opihiexarata.gui.qtui.qtui\_automatic.Ui\_AutomaticWindow method}}

\begin{savenotes}\begin{fulllineitems}
\phantomsection\label{\detokenize{code/opihiexarata.gui.qtui.qtui_automatic:opihiexarata.gui.qtui.qtui_automatic.Ui_AutomaticWindow.retranslateUi}}
\pysigstartsignatures
\pysiglinewithargsret{\sphinxbfcode{\sphinxupquote{retranslateUi}}}{\emph{\DUrole{n}{AutomaticWindow}}}{}
\pysigstopsignatures
\end{fulllineitems}\end{savenotes}

\index{setupUi() (opihiexarata.gui.qtui.qtui\_automatic.Ui\_AutomaticWindow method)@\spxentry{setupUi()}\spxextra{opihiexarata.gui.qtui.qtui\_automatic.Ui\_AutomaticWindow method}}

\begin{savenotes}\begin{fulllineitems}
\phantomsection\label{\detokenize{code/opihiexarata.gui.qtui.qtui_automatic:opihiexarata.gui.qtui.qtui_automatic.Ui_AutomaticWindow.setupUi}}
\pysigstartsignatures
\pysiglinewithargsret{\sphinxbfcode{\sphinxupquote{setupUi}}}{\emph{\DUrole{n}{AutomaticWindow}}}{}
\pysigstopsignatures
\end{fulllineitems}\end{savenotes}


\end{fulllineitems}\end{savenotes}


\sphinxstepscope


\subparagraph{opihiexarata.gui.qtui.qtui\_manual module}
\label{\detokenize{code/opihiexarata.gui.qtui.qtui_manual:module-opihiexarata.gui.qtui.qtui_manual}}\label{\detokenize{code/opihiexarata.gui.qtui.qtui_manual:opihiexarata-gui-qtui-qtui-manual-module}}\label{\detokenize{code/opihiexarata.gui.qtui.qtui_manual::doc}}\index{module@\spxentry{module}!opihiexarata.gui.qtui.qtui\_manual@\spxentry{opihiexarata.gui.qtui.qtui\_manual}}\index{opihiexarata.gui.qtui.qtui\_manual@\spxentry{opihiexarata.gui.qtui.qtui\_manual}!module@\spxentry{module}}\index{Ui\_ManualWindow (class in opihiexarata.gui.qtui.qtui\_manual)@\spxentry{Ui\_ManualWindow}\spxextra{class in opihiexarata.gui.qtui.qtui\_manual}}

\begin{savenotes}\begin{fulllineitems}
\phantomsection\label{\detokenize{code/opihiexarata.gui.qtui.qtui_manual:opihiexarata.gui.qtui.qtui_manual.Ui_ManualWindow}}
\pysigstartsignatures
\pysigline{\sphinxbfcode{\sphinxupquote{class\DUrole{w}{  }}}\sphinxcode{\sphinxupquote{opihiexarata.gui.qtui.qtui\_manual.}}\sphinxbfcode{\sphinxupquote{Ui\_ManualWindow}}}
\pysigstopsignatures
\sphinxAtStartPar
Bases: \sphinxcode{\sphinxupquote{object}}
\index{retranslateUi() (opihiexarata.gui.qtui.qtui\_manual.Ui\_ManualWindow method)@\spxentry{retranslateUi()}\spxextra{opihiexarata.gui.qtui.qtui\_manual.Ui\_ManualWindow method}}

\begin{savenotes}\begin{fulllineitems}
\phantomsection\label{\detokenize{code/opihiexarata.gui.qtui.qtui_manual:opihiexarata.gui.qtui.qtui_manual.Ui_ManualWindow.retranslateUi}}
\pysigstartsignatures
\pysiglinewithargsret{\sphinxbfcode{\sphinxupquote{retranslateUi}}}{\emph{\DUrole{n}{ManualWindow}}}{}
\pysigstopsignatures
\end{fulllineitems}\end{savenotes}

\index{setupUi() (opihiexarata.gui.qtui.qtui\_manual.Ui\_ManualWindow method)@\spxentry{setupUi()}\spxextra{opihiexarata.gui.qtui.qtui\_manual.Ui\_ManualWindow method}}

\begin{savenotes}\begin{fulllineitems}
\phantomsection\label{\detokenize{code/opihiexarata.gui.qtui.qtui_manual:opihiexarata.gui.qtui.qtui_manual.Ui_ManualWindow.setupUi}}
\pysigstartsignatures
\pysiglinewithargsret{\sphinxbfcode{\sphinxupquote{setupUi}}}{\emph{\DUrole{n}{ManualWindow}}}{}
\pysigstopsignatures
\end{fulllineitems}\end{savenotes}


\end{fulllineitems}\end{savenotes}


\sphinxstepscope


\subparagraph{opihiexarata.gui.qtui.qtui\_selector module}
\label{\detokenize{code/opihiexarata.gui.qtui.qtui_selector:module-opihiexarata.gui.qtui.qtui_selector}}\label{\detokenize{code/opihiexarata.gui.qtui.qtui_selector:opihiexarata-gui-qtui-qtui-selector-module}}\label{\detokenize{code/opihiexarata.gui.qtui.qtui_selector::doc}}\index{module@\spxentry{module}!opihiexarata.gui.qtui.qtui\_selector@\spxentry{opihiexarata.gui.qtui.qtui\_selector}}\index{opihiexarata.gui.qtui.qtui\_selector@\spxentry{opihiexarata.gui.qtui.qtui\_selector}!module@\spxentry{module}}\index{Ui\_SelectorWindow (class in opihiexarata.gui.qtui.qtui\_selector)@\spxentry{Ui\_SelectorWindow}\spxextra{class in opihiexarata.gui.qtui.qtui\_selector}}

\begin{savenotes}\begin{fulllineitems}
\phantomsection\label{\detokenize{code/opihiexarata.gui.qtui.qtui_selector:opihiexarata.gui.qtui.qtui_selector.Ui_SelectorWindow}}
\pysigstartsignatures
\pysigline{\sphinxbfcode{\sphinxupquote{class\DUrole{w}{  }}}\sphinxcode{\sphinxupquote{opihiexarata.gui.qtui.qtui\_selector.}}\sphinxbfcode{\sphinxupquote{Ui\_SelectorWindow}}}
\pysigstopsignatures
\sphinxAtStartPar
Bases: \sphinxcode{\sphinxupquote{object}}
\index{retranslateUi() (opihiexarata.gui.qtui.qtui\_selector.Ui\_SelectorWindow method)@\spxentry{retranslateUi()}\spxextra{opihiexarata.gui.qtui.qtui\_selector.Ui\_SelectorWindow method}}

\begin{savenotes}\begin{fulllineitems}
\phantomsection\label{\detokenize{code/opihiexarata.gui.qtui.qtui_selector:opihiexarata.gui.qtui.qtui_selector.Ui_SelectorWindow.retranslateUi}}
\pysigstartsignatures
\pysiglinewithargsret{\sphinxbfcode{\sphinxupquote{retranslateUi}}}{\emph{\DUrole{n}{SelectorWindow}}}{}
\pysigstopsignatures
\end{fulllineitems}\end{savenotes}

\index{setupUi() (opihiexarata.gui.qtui.qtui\_selector.Ui\_SelectorWindow method)@\spxentry{setupUi()}\spxextra{opihiexarata.gui.qtui.qtui\_selector.Ui\_SelectorWindow method}}

\begin{savenotes}\begin{fulllineitems}
\phantomsection\label{\detokenize{code/opihiexarata.gui.qtui.qtui_selector:opihiexarata.gui.qtui.qtui_selector.Ui_SelectorWindow.setupUi}}
\pysigstartsignatures
\pysiglinewithargsret{\sphinxbfcode{\sphinxupquote{setupUi}}}{\emph{\DUrole{n}{SelectorWindow}}}{}
\pysigstopsignatures
\end{fulllineitems}\end{savenotes}


\end{fulllineitems}\end{savenotes}



\subparagraph{Module contents}
\label{\detokenize{code/opihiexarata.gui.qtui:module-opihiexarata.gui.qtui}}\label{\detokenize{code/opihiexarata.gui.qtui:module-contents}}\index{module@\spxentry{module}!opihiexarata.gui.qtui@\spxentry{opihiexarata.gui.qtui}}\index{opihiexarata.gui.qtui@\spxentry{opihiexarata.gui.qtui}!module@\spxentry{module}}
\sphinxAtStartPar
All of the QtUI frameworks.


\subparagraph{Submodules}
\label{\detokenize{code/opihiexarata.gui:submodules}}
\sphinxstepscope


\subparagraph{opihiexarata.gui.automatic module}
\label{\detokenize{code/opihiexarata.gui.automatic:module-opihiexarata.gui.automatic}}\label{\detokenize{code/opihiexarata.gui.automatic:opihiexarata-gui-automatic-module}}\label{\detokenize{code/opihiexarata.gui.automatic::doc}}\index{module@\spxentry{module}!opihiexarata.gui.automatic@\spxentry{opihiexarata.gui.automatic}}\index{opihiexarata.gui.automatic@\spxentry{opihiexarata.gui.automatic}!module@\spxentry{module}}
\sphinxAtStartPar
This is where the automatic mode window is implemented.
\index{OpihiAutomaticWindow (class in opihiexarata.gui.automatic)@\spxentry{OpihiAutomaticWindow}\spxextra{class in opihiexarata.gui.automatic}}

\begin{savenotes}\begin{fulllineitems}
\phantomsection\label{\detokenize{code/opihiexarata.gui.automatic:opihiexarata.gui.automatic.OpihiAutomaticWindow}}
\pysigstartsignatures
\pysigline{\sphinxbfcode{\sphinxupquote{class\DUrole{w}{  }}}\sphinxcode{\sphinxupquote{opihiexarata.gui.automatic.}}\sphinxbfcode{\sphinxupquote{OpihiAutomaticWindow}}}
\pysigstopsignatures
\sphinxAtStartPar
Bases: \sphinxcode{\sphinxupquote{QMainWindow}}
\index{\_\_connect\_push\_button\_change\_directory() (opihiexarata.gui.automatic.OpihiAutomaticWindow method)@\spxentry{\_\_connect\_push\_button\_change\_directory()}\spxextra{opihiexarata.gui.automatic.OpihiAutomaticWindow method}}

\begin{savenotes}\begin{fulllineitems}
\phantomsection\label{\detokenize{code/opihiexarata.gui.automatic:opihiexarata.gui.automatic.OpihiAutomaticWindow.__connect_push_button_change_directory}}
\pysigstartsignatures
\pysiglinewithargsret{\sphinxbfcode{\sphinxupquote{\_\_connect\_push\_button\_change\_directory}}}{}{{ $\rightarrow$ None}}
\pysigstopsignatures
\sphinxAtStartPar
The connection for the button to change the automatic fetch
directory.
\begin{quote}\begin{description}
\sphinxlineitem{Parameters}
\sphinxAtStartPar
\sphinxstyleliteralstrong{\sphinxupquote{None}} – 

\sphinxlineitem{Return type}
\sphinxAtStartPar
None

\end{description}\end{quote}

\end{fulllineitems}\end{savenotes}

\index{\_\_connect\_push\_button\_start() (opihiexarata.gui.automatic.OpihiAutomaticWindow method)@\spxentry{\_\_connect\_push\_button\_start()}\spxextra{opihiexarata.gui.automatic.OpihiAutomaticWindow method}}

\begin{savenotes}\begin{fulllineitems}
\phantomsection\label{\detokenize{code/opihiexarata.gui.automatic:opihiexarata.gui.automatic.OpihiAutomaticWindow.__connect_push_button_start}}
\pysigstartsignatures
\pysiglinewithargsret{\sphinxbfcode{\sphinxupquote{\_\_connect\_push\_button\_start}}}{}{{ $\rightarrow$ None}}
\pysigstopsignatures
\sphinxAtStartPar
This enables the automatic active mode by changing the flag and
starting the process.
\begin{quote}\begin{description}
\sphinxlineitem{Parameters}
\sphinxAtStartPar
\sphinxstyleliteralstrong{\sphinxupquote{None}} – 

\sphinxlineitem{Return type}
\sphinxAtStartPar
None

\end{description}\end{quote}

\end{fulllineitems}\end{savenotes}

\index{\_\_connect\_push\_button\_stop() (opihiexarata.gui.automatic.OpihiAutomaticWindow method)@\spxentry{\_\_connect\_push\_button\_stop()}\spxextra{opihiexarata.gui.automatic.OpihiAutomaticWindow method}}

\begin{savenotes}\begin{fulllineitems}
\phantomsection\label{\detokenize{code/opihiexarata.gui.automatic:opihiexarata.gui.automatic.OpihiAutomaticWindow.__connect_push_button_stop}}
\pysigstartsignatures
\pysiglinewithargsret{\sphinxbfcode{\sphinxupquote{\_\_connect\_push\_button\_stop}}}{}{{ $\rightarrow$ None}}
\pysigstopsignatures
\sphinxAtStartPar
This disables the automatic active mode by changing the flag. The
loop itself should detect that the flag has changed. It finishes the
current process but does not fetch any more.
\begin{quote}\begin{description}
\sphinxlineitem{Parameters}
\sphinxAtStartPar
\sphinxstyleliteralstrong{\sphinxupquote{None}} – 

\sphinxlineitem{Return type}
\sphinxAtStartPar
None

\end{description}\end{quote}

\end{fulllineitems}\end{savenotes}

\index{\_\_connect\_push\_button\_trigger() (opihiexarata.gui.automatic.OpihiAutomaticWindow method)@\spxentry{\_\_connect\_push\_button\_trigger()}\spxextra{opihiexarata.gui.automatic.OpihiAutomaticWindow method}}

\begin{savenotes}\begin{fulllineitems}
\phantomsection\label{\detokenize{code/opihiexarata.gui.automatic:opihiexarata.gui.automatic.OpihiAutomaticWindow.__connect_push_button_trigger}}
\pysigstartsignatures
\pysiglinewithargsret{\sphinxbfcode{\sphinxupquote{\_\_connect\_push\_button\_trigger}}}{}{{ $\rightarrow$ None}}
\pysigstopsignatures
\sphinxAtStartPar
This does one process, fetching a single image and processing it as
normal. However, it does not trigger the automatic mode loop as it is
built for a single image only.
\begin{quote}\begin{description}
\sphinxlineitem{Parameters}
\sphinxAtStartPar
\sphinxstyleliteralstrong{\sphinxupquote{None}} – 

\sphinxlineitem{Return type}
\sphinxAtStartPar
None

\end{description}\end{quote}

\end{fulllineitems}\end{savenotes}

\index{\_\_init\_\_() (opihiexarata.gui.automatic.OpihiAutomaticWindow method)@\spxentry{\_\_init\_\_()}\spxextra{opihiexarata.gui.automatic.OpihiAutomaticWindow method}}

\begin{savenotes}\begin{fulllineitems}
\phantomsection\label{\detokenize{code/opihiexarata.gui.automatic:opihiexarata.gui.automatic.OpihiAutomaticWindow.__init__}}
\pysigstartsignatures
\pysiglinewithargsret{\sphinxbfcode{\sphinxupquote{\_\_init\_\_}}}{}{{ $\rightarrow$ None}}
\pysigstopsignatures
\sphinxAtStartPar
The automatic GUI window for OpihiExarata. This interacts with
the user with regards to the automatic solving mode of Opihi.
\begin{quote}\begin{description}
\sphinxlineitem{Parameters}
\sphinxAtStartPar
\sphinxstyleliteralstrong{\sphinxupquote{None}} – 

\sphinxlineitem{Return type}
\sphinxAtStartPar
None

\end{description}\end{quote}

\end{fulllineitems}\end{savenotes}

\index{\_\_init\_gui\_connections() (opihiexarata.gui.automatic.OpihiAutomaticWindow method)@\spxentry{\_\_init\_gui\_connections()}\spxextra{opihiexarata.gui.automatic.OpihiAutomaticWindow method}}

\begin{savenotes}\begin{fulllineitems}
\phantomsection\label{\detokenize{code/opihiexarata.gui.automatic:opihiexarata.gui.automatic.OpihiAutomaticWindow.__init_gui_connections}}
\pysigstartsignatures
\pysiglinewithargsret{\sphinxbfcode{\sphinxupquote{\_\_init\_gui\_connections}}}{}{{ $\rightarrow$ None}}
\pysigstopsignatures
\sphinxAtStartPar
Creating the function connections for the GUI interface.
\begin{quote}\begin{description}
\sphinxlineitem{Parameters}
\sphinxAtStartPar
\sphinxstyleliteralstrong{\sphinxupquote{None}} – 

\sphinxlineitem{Return type}
\sphinxAtStartPar
None

\end{description}\end{quote}

\end{fulllineitems}\end{savenotes}

\index{\_\_refresh\_dynamic\_label\_text() (opihiexarata.gui.automatic.OpihiAutomaticWindow method)@\spxentry{\_\_refresh\_dynamic\_label\_text()}\spxextra{opihiexarata.gui.automatic.OpihiAutomaticWindow method}}

\begin{savenotes}\begin{fulllineitems}
\phantomsection\label{\detokenize{code/opihiexarata.gui.automatic:opihiexarata.gui.automatic.OpihiAutomaticWindow.__refresh_dynamic_label_text}}
\pysigstartsignatures
\pysiglinewithargsret{\sphinxbfcode{\sphinxupquote{\_\_refresh\_dynamic\_label\_text}}}{}{{ $\rightarrow$ None}}
\pysigstopsignatures
\sphinxAtStartPar
Refreshes the GUI window’s dynamic text.
\begin{quote}\begin{description}
\sphinxlineitem{Parameters}
\sphinxAtStartPar
\sphinxstyleliteralstrong{\sphinxupquote{None}} – 

\sphinxlineitem{Return type}
\sphinxAtStartPar
None

\end{description}\end{quote}

\end{fulllineitems}\end{savenotes}

\index{\_automatic\_triggering\_check\_stops() (opihiexarata.gui.automatic.OpihiAutomaticWindow method)@\spxentry{\_automatic\_triggering\_check\_stops()}\spxextra{opihiexarata.gui.automatic.OpihiAutomaticWindow method}}

\begin{savenotes}\begin{fulllineitems}
\phantomsection\label{\detokenize{code/opihiexarata.gui.automatic:opihiexarata.gui.automatic.OpihiAutomaticWindow._automatic_triggering_check_stops}}
\pysigstartsignatures
\pysiglinewithargsret{\sphinxbfcode{\sphinxupquote{\_automatic\_triggering\_check\_stops}}}{}{{ $\rightarrow$ bool}}
\pysigstopsignatures
\sphinxAtStartPar
This function checks for the stops to stop the automatic triggering
of the next image.
\begin{quote}\begin{description}
\sphinxlineitem{Parameters}
\sphinxAtStartPar
\sphinxstyleliteralstrong{\sphinxupquote{None}} – 

\sphinxlineitem{Returns}
\sphinxAtStartPar
\sphinxstylestrong{stop\_check} – This is the flag which signifies if the triggering should stop or
not. If True, the triggering should stop.

\sphinxlineitem{Return type}
\sphinxAtStartPar
bool

\end{description}\end{quote}

\end{fulllineitems}\end{savenotes}

\index{\_automatic\_triggering\_infinite\_loop() (opihiexarata.gui.automatic.OpihiAutomaticWindow method)@\spxentry{\_automatic\_triggering\_infinite\_loop()}\spxextra{opihiexarata.gui.automatic.OpihiAutomaticWindow method}}

\begin{savenotes}\begin{fulllineitems}
\phantomsection\label{\detokenize{code/opihiexarata.gui.automatic:opihiexarata.gui.automatic.OpihiAutomaticWindow._automatic_triggering_infinite_loop}}
\pysigstartsignatures
\pysiglinewithargsret{\sphinxbfcode{\sphinxupquote{\_automatic\_triggering\_infinite\_loop}}}{}{{ $\rightarrow$ None}}
\pysigstopsignatures
\sphinxAtStartPar
This is where the actual infinite loop is done.

\sphinxAtStartPar
All of the stops are checked before a new trigger is executed.
\begin{quote}\begin{description}
\sphinxlineitem{Parameters}
\sphinxAtStartPar
\sphinxstyleliteralstrong{\sphinxupquote{None}} – 

\sphinxlineitem{Return type}
\sphinxAtStartPar
None

\end{description}\end{quote}

\end{fulllineitems}\end{savenotes}

\index{automatic\_triggering() (opihiexarata.gui.automatic.OpihiAutomaticWindow method)@\spxentry{automatic\_triggering()}\spxextra{opihiexarata.gui.automatic.OpihiAutomaticWindow method}}

\begin{savenotes}\begin{fulllineitems}
\phantomsection\label{\detokenize{code/opihiexarata.gui.automatic:opihiexarata.gui.automatic.OpihiAutomaticWindow.automatic_triggering}}
\pysigstartsignatures
\pysiglinewithargsret{\sphinxbfcode{\sphinxupquote{automatic\_triggering}}}{}{{ $\rightarrow$ None}}
\pysigstopsignatures
\sphinxAtStartPar
This function executes the continuous running automatic mode loop.

\sphinxAtStartPar
All of the stops are checked before a new trigger is executed.
\begin{quote}\begin{description}
\sphinxlineitem{Parameters}
\sphinxAtStartPar
\sphinxstyleliteralstrong{\sphinxupquote{None}} – 

\sphinxlineitem{Return type}
\sphinxAtStartPar
None

\end{description}\end{quote}

\end{fulllineitems}\end{savenotes}

\index{fetch\_new\_filename() (opihiexarata.gui.automatic.OpihiAutomaticWindow method)@\spxentry{fetch\_new\_filename()}\spxextra{opihiexarata.gui.automatic.OpihiAutomaticWindow method}}

\begin{savenotes}\begin{fulllineitems}
\phantomsection\label{\detokenize{code/opihiexarata.gui.automatic:opihiexarata.gui.automatic.OpihiAutomaticWindow.fetch_new_filename}}
\pysigstartsignatures
\pysiglinewithargsret{\sphinxbfcode{\sphinxupquote{fetch\_new\_filename}}}{}{{ $\rightarrow$ str}}
\pysigstopsignatures
\sphinxAtStartPar
This function fetches a new fits filename based on the most recent
filename within the automatic fetching directory.
\begin{quote}\begin{description}
\sphinxlineitem{Parameters}
\sphinxAtStartPar
\sphinxstyleliteralstrong{\sphinxupquote{None}} – 

\sphinxlineitem{Returns}
\sphinxAtStartPar
\sphinxstylestrong{fetched\_filename} – The filename that was fetched. It is the most recent file added to
the automatic fetching directory.

\sphinxlineitem{Return type}
\sphinxAtStartPar
string

\end{description}\end{quote}

\end{fulllineitems}\end{savenotes}

\index{refresh\_window() (opihiexarata.gui.automatic.OpihiAutomaticWindow method)@\spxentry{refresh\_window()}\spxextra{opihiexarata.gui.automatic.OpihiAutomaticWindow method}}

\begin{savenotes}\begin{fulllineitems}
\phantomsection\label{\detokenize{code/opihiexarata.gui.automatic:opihiexarata.gui.automatic.OpihiAutomaticWindow.refresh_window}}
\pysigstartsignatures
\pysiglinewithargsret{\sphinxbfcode{\sphinxupquote{refresh\_window}}}{}{{ $\rightarrow$ None}}
\pysigstopsignatures
\sphinxAtStartPar
Refreshes the GUI window with new information where available.
\begin{quote}\begin{description}
\sphinxlineitem{Parameters}
\sphinxAtStartPar
\sphinxstyleliteralstrong{\sphinxupquote{None}} – 

\sphinxlineitem{Return type}
\sphinxAtStartPar
None

\end{description}\end{quote}

\end{fulllineitems}\end{savenotes}

\index{reset\_window() (opihiexarata.gui.automatic.OpihiAutomaticWindow method)@\spxentry{reset\_window()}\spxextra{opihiexarata.gui.automatic.OpihiAutomaticWindow method}}

\begin{savenotes}\begin{fulllineitems}
\phantomsection\label{\detokenize{code/opihiexarata.gui.automatic:opihiexarata.gui.automatic.OpihiAutomaticWindow.reset_window}}
\pysigstartsignatures
\pysiglinewithargsret{\sphinxbfcode{\sphinxupquote{reset\_window}}}{}{{ $\rightarrow$ None}}
\pysigstopsignatures
\sphinxAtStartPar
This function resets the window to the default values or
parameters
\begin{quote}\begin{description}
\sphinxlineitem{Parameters}
\sphinxAtStartPar
\sphinxstyleliteralstrong{\sphinxupquote{None}} – 

\sphinxlineitem{Return type}
\sphinxAtStartPar
None

\end{description}\end{quote}

\end{fulllineitems}\end{savenotes}

\index{solve\_astrometry\_photometry\_single\_image() (opihiexarata.gui.automatic.OpihiAutomaticWindow method)@\spxentry{solve\_astrometry\_photometry\_single\_image()}\spxextra{opihiexarata.gui.automatic.OpihiAutomaticWindow method}}

\begin{savenotes}\begin{fulllineitems}
\phantomsection\label{\detokenize{code/opihiexarata.gui.automatic:opihiexarata.gui.automatic.OpihiAutomaticWindow.solve_astrometry_photometry_single_image}}
\pysigstartsignatures
\pysiglinewithargsret{\sphinxbfcode{\sphinxupquote{solve\_astrometry\_photometry\_single\_image}}}{\emph{\DUrole{n}{filename}\DUrole{p}{:}\DUrole{w}{  }\DUrole{n}{str}}}{{ $\rightarrow$ {\hyperref[\detokenize{code/opihiexarata.opihi.solution:opihiexarata.opihi.solution.OpihiSolution}]{\sphinxcrossref{OpihiSolution}}}}}
\pysigstopsignatures
\sphinxAtStartPar
This solves for the astrometric and photometric solutions of a
provided file. The engines are provided based on the dropdown menus.

\sphinxAtStartPar
Note this calculation does not affect the \sphinxtitleref{opihi\_solution} instance of
the class. That is a job for a different function.
\begin{quote}\begin{description}
\sphinxlineitem{Parameters}
\sphinxAtStartPar
\sphinxstyleliteralstrong{\sphinxupquote{filename}} (\sphinxstyleliteralemphasis{\sphinxupquote{string}}) – The filename of the fits file to be solved.

\sphinxlineitem{Returns}
\sphinxAtStartPar
\sphinxstylestrong{opihi\_solution} – The solution with the astrometry and photometry engines solved.

\sphinxlineitem{Return type}
\sphinxAtStartPar
{\hyperref[\detokenize{code/opihiexarata.opihi.solution:opihiexarata.opihi.solution.OpihiSolution}]{\sphinxcrossref{OpihiSolution}}}

\end{description}\end{quote}

\end{fulllineitems}\end{savenotes}

\index{staticMetaObject (opihiexarata.gui.automatic.OpihiAutomaticWindow attribute)@\spxentry{staticMetaObject}\spxextra{opihiexarata.gui.automatic.OpihiAutomaticWindow attribute}}

\begin{savenotes}\begin{fulllineitems}
\phantomsection\label{\detokenize{code/opihiexarata.gui.automatic:opihiexarata.gui.automatic.OpihiAutomaticWindow.staticMetaObject}}
\pysigstartsignatures
\pysigline{\sphinxbfcode{\sphinxupquote{staticMetaObject}}\sphinxbfcode{\sphinxupquote{\DUrole{w}{  }\DUrole{p}{=}\DUrole{w}{  }PySide6.QtCore.QMetaObject("OpihiAutomaticWindow" inherits "QMainWindow": )}}}
\pysigstopsignatures
\end{fulllineitems}\end{savenotes}

\index{trigger\_next\_image\_solve() (opihiexarata.gui.automatic.OpihiAutomaticWindow method)@\spxentry{trigger\_next\_image\_solve()}\spxextra{opihiexarata.gui.automatic.OpihiAutomaticWindow method}}

\begin{savenotes}\begin{fulllineitems}
\phantomsection\label{\detokenize{code/opihiexarata.gui.automatic:opihiexarata.gui.automatic.OpihiAutomaticWindow.trigger_next_image_solve}}
\pysigstartsignatures
\pysiglinewithargsret{\sphinxbfcode{\sphinxupquote{trigger\_next\_image\_solve}}}{}{{ $\rightarrow$ None}}
\pysigstopsignatures
\sphinxAtStartPar
This function triggers the next iteration of the automatic solving
loop.
\begin{quote}\begin{description}
\sphinxlineitem{Parameters}
\sphinxAtStartPar
\sphinxstyleliteralstrong{\sphinxupquote{None}} – 

\sphinxlineitem{Return type}
\sphinxAtStartPar
None

\end{description}\end{quote}

\end{fulllineitems}\end{savenotes}


\end{fulllineitems}\end{savenotes}

\index{start\_automatic\_window() (in module opihiexarata.gui.automatic)@\spxentry{start\_automatic\_window()}\spxextra{in module opihiexarata.gui.automatic}}

\begin{savenotes}\begin{fulllineitems}
\phantomsection\label{\detokenize{code/opihiexarata.gui.automatic:opihiexarata.gui.automatic.start_automatic_window}}
\pysigstartsignatures
\pysiglinewithargsret{\sphinxcode{\sphinxupquote{opihiexarata.gui.automatic.}}\sphinxbfcode{\sphinxupquote{start\_automatic\_window}}}{}{{ $\rightarrow$ None}}
\pysigstopsignatures
\sphinxAtStartPar
This is the function to create the automatic window for usage.
\begin{quote}\begin{description}
\sphinxlineitem{Parameters}
\sphinxAtStartPar
\sphinxstyleliteralstrong{\sphinxupquote{None}} – 

\sphinxlineitem{Return type}
\sphinxAtStartPar
None

\end{description}\end{quote}

\end{fulllineitems}\end{savenotes}


\sphinxstepscope


\subparagraph{opihiexarata.gui.functions module}
\label{\detokenize{code/opihiexarata.gui.functions:module-opihiexarata.gui.functions}}\label{\detokenize{code/opihiexarata.gui.functions:opihiexarata-gui-functions-module}}\label{\detokenize{code/opihiexarata.gui.functions::doc}}\index{module@\spxentry{module}!opihiexarata.gui.functions@\spxentry{opihiexarata.gui.functions}}\index{opihiexarata.gui.functions@\spxentry{opihiexarata.gui.functions}!module@\spxentry{module}}
\sphinxAtStartPar
Where helpful functions which otherwise do not belong in the library,
for the GUIs, exist.
\index{pick\_engine\_class\_from\_name() (in module opihiexarata.gui.functions)@\spxentry{pick\_engine\_class\_from\_name()}\spxextra{in module opihiexarata.gui.functions}}

\begin{savenotes}\begin{fulllineitems}
\phantomsection\label{\detokenize{code/opihiexarata.gui.functions:opihiexarata.gui.functions.pick_engine_class_from_name}}
\pysigstartsignatures
\pysiglinewithargsret{\sphinxcode{\sphinxupquote{opihiexarata.gui.functions.}}\sphinxbfcode{\sphinxupquote{pick\_engine\_class\_from\_name}}}{\emph{\DUrole{n}{engine\_name: str}}, \emph{\DUrole{n}{engine\_type: \textasciitilde{}opihiexarata.library.engine.ExarataEngine = <class 'opihiexarata.library.engine.ExarataEngine'>}}}{{ $\rightarrow$ {\hyperref[\detokenize{code/opihiexarata.library.engine:opihiexarata.library.engine.ExarataEngine}]{\sphinxcrossref{ExarataEngine}}}}}
\pysigstopsignatures
\sphinxAtStartPar
This returns a specific engine class provided its user friendly name.
This is a convince function for both development and implementation.

\sphinxAtStartPar
If an engine name provided is not present, this raises.
\begin{quote}\begin{description}
\sphinxlineitem{Parameters}\begin{itemize}
\item {} 
\sphinxAtStartPar
\sphinxstyleliteralstrong{\sphinxupquote{engine\_name}} (\sphinxstyleliteralemphasis{\sphinxupquote{str}}) – The engine name, the user friendly version. It is case insensitive.

\item {} 
\sphinxAtStartPar
\sphinxstyleliteralstrong{\sphinxupquote{engine\_type}} ({\hyperref[\detokenize{code/opihiexarata.library.engine:opihiexarata.library.engine.ExarataEngine}]{\sphinxcrossref{\sphinxstyleliteralemphasis{\sphinxupquote{ExarataEngine}}}}}) – The engine subtype, if not provided, then it searches through all
available implemented engines.

\end{itemize}

\sphinxlineitem{Returns}
\sphinxAtStartPar
\sphinxstylestrong{engine\_class} – The more specific engine class based on the engine name.

\sphinxlineitem{Return type}
\sphinxAtStartPar
{\hyperref[\detokenize{code/opihiexarata.library.engine:opihiexarata.library.engine.ExarataEngine}]{\sphinxcrossref{ExarataEngine}}}

\end{description}\end{quote}

\end{fulllineitems}\end{savenotes}


\sphinxstepscope


\subparagraph{opihiexarata.gui.manual module}
\label{\detokenize{code/opihiexarata.gui.manual:module-opihiexarata.gui.manual}}\label{\detokenize{code/opihiexarata.gui.manual:opihiexarata-gui-manual-module}}\label{\detokenize{code/opihiexarata.gui.manual::doc}}\index{module@\spxentry{module}!opihiexarata.gui.manual@\spxentry{opihiexarata.gui.manual}}\index{opihiexarata.gui.manual@\spxentry{opihiexarata.gui.manual}!module@\spxentry{module}}
\sphinxAtStartPar
The manual GUI window.
\index{OpihiManualWindow (class in opihiexarata.gui.manual)@\spxentry{OpihiManualWindow}\spxextra{class in opihiexarata.gui.manual}}

\begin{savenotes}\begin{fulllineitems}
\phantomsection\label{\detokenize{code/opihiexarata.gui.manual:opihiexarata.gui.manual.OpihiManualWindow}}
\pysigstartsignatures
\pysigline{\sphinxbfcode{\sphinxupquote{class\DUrole{w}{  }}}\sphinxcode{\sphinxupquote{opihiexarata.gui.manual.}}\sphinxbfcode{\sphinxupquote{OpihiManualWindow}}}
\pysigstopsignatures
\sphinxAtStartPar
Bases: \sphinxcode{\sphinxupquote{QMainWindow}}
\index{\_\_connect\_push\_button\_astrometry\_custom\_solve() (opihiexarata.gui.manual.OpihiManualWindow method)@\spxentry{\_\_connect\_push\_button\_astrometry\_custom\_solve()}\spxextra{opihiexarata.gui.manual.OpihiManualWindow method}}

\begin{savenotes}\begin{fulllineitems}
\phantomsection\label{\detokenize{code/opihiexarata.gui.manual:opihiexarata.gui.manual.OpihiManualWindow.__connect_push_button_astrometry_custom_solve}}
\pysigstartsignatures
\pysiglinewithargsret{\sphinxbfcode{\sphinxupquote{\_\_connect\_push\_button\_astrometry\_custom\_solve}}}{}{{ $\rightarrow$ None}}
\pysigstopsignatures
\sphinxAtStartPar
“The button which uses an astrometric solution to solve for a
custom pixel location or RA DEC location depending on entry.

\sphinxAtStartPar
This prioritizes solving RA DEC from pixel location.
\begin{quote}\begin{description}
\sphinxlineitem{Parameters}
\sphinxAtStartPar
\sphinxstyleliteralstrong{\sphinxupquote{None}} – 

\sphinxlineitem{Return type}
\sphinxAtStartPar
None

\end{description}\end{quote}

\end{fulllineitems}\end{savenotes}

\index{\_\_connect\_push\_button\_astrometry\_solve\_astrometry() (opihiexarata.gui.manual.OpihiManualWindow method)@\spxentry{\_\_connect\_push\_button\_astrometry\_solve\_astrometry()}\spxextra{opihiexarata.gui.manual.OpihiManualWindow method}}

\begin{savenotes}\begin{fulllineitems}
\phantomsection\label{\detokenize{code/opihiexarata.gui.manual:opihiexarata.gui.manual.OpihiManualWindow.__connect_push_button_astrometry_solve_astrometry}}
\pysigstartsignatures
\pysiglinewithargsret{\sphinxbfcode{\sphinxupquote{\_\_connect\_push\_button\_astrometry\_solve\_astrometry}}}{}{{ $\rightarrow$ None}}
\pysigstopsignatures
\sphinxAtStartPar
The button to instruct on the solving of the astrometric solution.
\begin{quote}\begin{description}
\sphinxlineitem{Parameters}
\sphinxAtStartPar
\sphinxstyleliteralstrong{\sphinxupquote{None}} – 

\sphinxlineitem{Return type}
\sphinxAtStartPar
None

\end{description}\end{quote}

\end{fulllineitems}\end{savenotes}

\index{\_\_connect\_push\_button\_new\_image\_automatic() (opihiexarata.gui.manual.OpihiManualWindow method)@\spxentry{\_\_connect\_push\_button\_new\_image\_automatic()}\spxextra{opihiexarata.gui.manual.OpihiManualWindow method}}

\begin{savenotes}\begin{fulllineitems}
\phantomsection\label{\detokenize{code/opihiexarata.gui.manual:opihiexarata.gui.manual.OpihiManualWindow.__connect_push_button_new_image_automatic}}
\pysigstartsignatures
\pysiglinewithargsret{\sphinxbfcode{\sphinxupquote{\_\_connect\_push\_button\_new\_image\_automatic}}}{}{}
\pysigstopsignatures
\sphinxAtStartPar
The automatic method relying on earliest fits file available in
the expected directory. This function is a connected function action to
a button in the GUI.
\begin{quote}\begin{description}
\sphinxlineitem{Parameters}
\sphinxAtStartPar
\sphinxstyleliteralstrong{\sphinxupquote{None}} – 

\sphinxlineitem{Return type}
\sphinxAtStartPar
None

\end{description}\end{quote}

\end{fulllineitems}\end{savenotes}

\index{\_\_connect\_push\_button\_new\_image\_manual() (opihiexarata.gui.manual.OpihiManualWindow method)@\spxentry{\_\_connect\_push\_button\_new\_image\_manual()}\spxextra{opihiexarata.gui.manual.OpihiManualWindow method}}

\begin{savenotes}\begin{fulllineitems}
\phantomsection\label{\detokenize{code/opihiexarata.gui.manual:opihiexarata.gui.manual.OpihiManualWindow.__connect_push_button_new_image_manual}}
\pysigstartsignatures
\pysiglinewithargsret{\sphinxbfcode{\sphinxupquote{\_\_connect\_push\_button\_new\_image\_manual}}}{}{}
\pysigstopsignatures
\sphinxAtStartPar
The manual method relying on earliest fits file available in
the expected directory. This function is a connected function action to
a button in the GUI.
\begin{quote}\begin{description}
\sphinxlineitem{Parameters}
\sphinxAtStartPar
\sphinxstyleliteralstrong{\sphinxupquote{None}} – 

\sphinxlineitem{Return type}
\sphinxAtStartPar
None

\end{description}\end{quote}

\end{fulllineitems}\end{savenotes}

\index{\_\_connect\_push\_button\_new\_target() (opihiexarata.gui.manual.OpihiManualWindow method)@\spxentry{\_\_connect\_push\_button\_new\_target()}\spxextra{opihiexarata.gui.manual.OpihiManualWindow method}}

\begin{savenotes}\begin{fulllineitems}
\phantomsection\label{\detokenize{code/opihiexarata.gui.manual:opihiexarata.gui.manual.OpihiManualWindow.__connect_push_button_new_target}}
\pysigstartsignatures
\pysiglinewithargsret{\sphinxbfcode{\sphinxupquote{\_\_connect\_push\_button\_new\_target}}}{}{{ $\rightarrow$ None}}
\pysigstopsignatures
\sphinxAtStartPar
The function serving to set the software to be on a new target.
A name is prompted from the user.
\begin{quote}\begin{description}
\sphinxlineitem{Parameters}
\sphinxAtStartPar
\sphinxstyleliteralstrong{\sphinxupquote{None}} – 

\sphinxlineitem{Return type}
\sphinxAtStartPar
None

\end{description}\end{quote}

\end{fulllineitems}\end{savenotes}

\index{\_\_connect\_push\_button\_orbit\_solve\_ephemeris() (opihiexarata.gui.manual.OpihiManualWindow method)@\spxentry{\_\_connect\_push\_button\_orbit\_solve\_ephemeris()}\spxextra{opihiexarata.gui.manual.OpihiManualWindow method}}

\begin{savenotes}\begin{fulllineitems}
\phantomsection\label{\detokenize{code/opihiexarata.gui.manual:opihiexarata.gui.manual.OpihiManualWindow.__connect_push_button_orbit_solve_ephemeris}}
\pysigstartsignatures
\pysiglinewithargsret{\sphinxbfcode{\sphinxupquote{\_\_connect\_push\_button\_orbit\_solve\_ephemeris}}}{}{{ $\rightarrow$ None}}
\pysigstopsignatures
\sphinxAtStartPar
A routine to use the current observation and historical observations
to derive the orbit solution.
\begin{quote}\begin{description}
\sphinxlineitem{Parameters}
\sphinxAtStartPar
\sphinxstyleliteralstrong{\sphinxupquote{None}} – 

\sphinxlineitem{Return type}
\sphinxAtStartPar
None

\end{description}\end{quote}

\end{fulllineitems}\end{savenotes}

\index{\_\_connect\_push\_button\_orbit\_solve\_orbit() (opihiexarata.gui.manual.OpihiManualWindow method)@\spxentry{\_\_connect\_push\_button\_orbit\_solve\_orbit()}\spxextra{opihiexarata.gui.manual.OpihiManualWindow method}}

\begin{savenotes}\begin{fulllineitems}
\phantomsection\label{\detokenize{code/opihiexarata.gui.manual:opihiexarata.gui.manual.OpihiManualWindow.__connect_push_button_orbit_solve_orbit}}
\pysigstartsignatures
\pysiglinewithargsret{\sphinxbfcode{\sphinxupquote{\_\_connect\_push\_button\_orbit\_solve\_orbit}}}{}{{ $\rightarrow$ None}}
\pysigstopsignatures
\sphinxAtStartPar
A routine to use the current observation and historical observations
to derive the orbit solution.
\begin{quote}\begin{description}
\sphinxlineitem{Parameters}
\sphinxAtStartPar
\sphinxstyleliteralstrong{\sphinxupquote{None}} – 

\sphinxlineitem{Return type}
\sphinxAtStartPar
None

\end{description}\end{quote}

\end{fulllineitems}\end{savenotes}

\index{\_\_connect\_push\_button\_photometry\_solve\_photometry() (opihiexarata.gui.manual.OpihiManualWindow method)@\spxentry{\_\_connect\_push\_button\_photometry\_solve\_photometry()}\spxextra{opihiexarata.gui.manual.OpihiManualWindow method}}

\begin{savenotes}\begin{fulllineitems}
\phantomsection\label{\detokenize{code/opihiexarata.gui.manual:opihiexarata.gui.manual.OpihiManualWindow.__connect_push_button_photometry_solve_photometry}}
\pysigstartsignatures
\pysiglinewithargsret{\sphinxbfcode{\sphinxupquote{\_\_connect\_push\_button\_photometry\_solve\_photometry}}}{}{{ $\rightarrow$ None}}
\pysigstopsignatures
\sphinxAtStartPar
A routine to use the current observation and historical observations
to derive the propagation solution.
\begin{quote}\begin{description}
\sphinxlineitem{Parameters}
\sphinxAtStartPar
\sphinxstyleliteralstrong{\sphinxupquote{None}} – 

\sphinxlineitem{Return type}
\sphinxAtStartPar
None

\end{description}\end{quote}

\end{fulllineitems}\end{savenotes}

\index{\_\_connect\_push\_button\_propagate\_custom\_solve() (opihiexarata.gui.manual.OpihiManualWindow method)@\spxentry{\_\_connect\_push\_button\_propagate\_custom\_solve()}\spxextra{opihiexarata.gui.manual.OpihiManualWindow method}}

\begin{savenotes}\begin{fulllineitems}
\phantomsection\label{\detokenize{code/opihiexarata.gui.manual:opihiexarata.gui.manual.OpihiManualWindow.__connect_push_button_propagate_custom_solve}}
\pysigstartsignatures
\pysiglinewithargsret{\sphinxbfcode{\sphinxupquote{\_\_connect\_push\_button\_propagate\_custom\_solve}}}{}{{ $\rightarrow$ None}}
\pysigstopsignatures
\sphinxAtStartPar
Solving for the location of the target through propagation based on
the time and date provided by the user.
\begin{quote}\begin{description}
\sphinxlineitem{Parameters}
\sphinxAtStartPar
\sphinxstyleliteralstrong{\sphinxupquote{None}} – 

\sphinxlineitem{Return type}
\sphinxAtStartPar
None

\end{description}\end{quote}

\end{fulllineitems}\end{savenotes}

\index{\_\_connect\_push\_button\_propagate\_solve\_propagation() (opihiexarata.gui.manual.OpihiManualWindow method)@\spxentry{\_\_connect\_push\_button\_propagate\_solve\_propagation()}\spxextra{opihiexarata.gui.manual.OpihiManualWindow method}}

\begin{savenotes}\begin{fulllineitems}
\phantomsection\label{\detokenize{code/opihiexarata.gui.manual:opihiexarata.gui.manual.OpihiManualWindow.__connect_push_button_propagate_solve_propagation}}
\pysigstartsignatures
\pysiglinewithargsret{\sphinxbfcode{\sphinxupquote{\_\_connect\_push\_button\_propagate\_solve\_propagation}}}{}{{ $\rightarrow$ None}}
\pysigstopsignatures
\sphinxAtStartPar
A routine to use the current observation and historical observations
to derive the propagation solution.
\begin{quote}\begin{description}
\sphinxlineitem{Parameters}
\sphinxAtStartPar
\sphinxstyleliteralstrong{\sphinxupquote{None}} – 

\sphinxlineitem{Return type}
\sphinxAtStartPar
None

\end{description}\end{quote}

\end{fulllineitems}\end{savenotes}

\index{\_\_connect\_push\_button\_refresh\_window() (opihiexarata.gui.manual.OpihiManualWindow method)@\spxentry{\_\_connect\_push\_button\_refresh\_window()}\spxextra{opihiexarata.gui.manual.OpihiManualWindow method}}

\begin{savenotes}\begin{fulllineitems}
\phantomsection\label{\detokenize{code/opihiexarata.gui.manual:opihiexarata.gui.manual.OpihiManualWindow.__connect_push_button_refresh_window}}
\pysigstartsignatures
\pysiglinewithargsret{\sphinxbfcode{\sphinxupquote{\_\_connect\_push\_button\_refresh\_window}}}{}{{ $\rightarrow$ None}}
\pysigstopsignatures
\sphinxAtStartPar
The function serving to refresh the window and redrawing the plot.
\begin{quote}\begin{description}
\sphinxlineitem{Parameters}
\sphinxAtStartPar
\sphinxstyleliteralstrong{\sphinxupquote{None}} – 

\sphinxlineitem{Return type}
\sphinxAtStartPar
None

\end{description}\end{quote}

\end{fulllineitems}\end{savenotes}

\index{\_\_get\_mpc\_record\_filename() (opihiexarata.gui.manual.OpihiManualWindow method)@\spxentry{\_\_get\_mpc\_record\_filename()}\spxextra{opihiexarata.gui.manual.OpihiManualWindow method}}

\begin{savenotes}\begin{fulllineitems}
\phantomsection\label{\detokenize{code/opihiexarata.gui.manual:opihiexarata.gui.manual.OpihiManualWindow.__get_mpc_record_filename}}
\pysigstartsignatures
\pysiglinewithargsret{\sphinxbfcode{\sphinxupquote{\_\_get\_mpc\_record\_filename}}}{}{{ $\rightarrow$ str}}
\pysigstopsignatures
\sphinxAtStartPar
This is a function which derives the MPC record filename from
naming conventions and the current fits file name.
\begin{quote}\begin{description}
\sphinxlineitem{Parameters}
\sphinxAtStartPar
\sphinxstyleliteralstrong{\sphinxupquote{None}} – 

\sphinxlineitem{Returns}
\sphinxAtStartPar
\sphinxstylestrong{mpc\_record\_filename} – The filename of the MPC record for this object/image.

\sphinxlineitem{Return type}
\sphinxAtStartPar
str

\end{description}\end{quote}

\end{fulllineitems}\end{savenotes}

\index{\_\_init\_\_() (opihiexarata.gui.manual.OpihiManualWindow method)@\spxentry{\_\_init\_\_()}\spxextra{opihiexarata.gui.manual.OpihiManualWindow method}}

\begin{savenotes}\begin{fulllineitems}
\phantomsection\label{\detokenize{code/opihiexarata.gui.manual:opihiexarata.gui.manual.OpihiManualWindow.__init__}}
\pysigstartsignatures
\pysiglinewithargsret{\sphinxbfcode{\sphinxupquote{\_\_init\_\_}}}{}{{ $\rightarrow$ None}}
\pysigstopsignatures
\sphinxAtStartPar
The manual GUI window for OpihiExarata. This interacts directly
with the total solution object of Opihi.
\begin{quote}\begin{description}
\sphinxlineitem{Parameters}
\sphinxAtStartPar
\sphinxstyleliteralstrong{\sphinxupquote{None}} – 

\sphinxlineitem{Return type}
\sphinxAtStartPar
None

\end{description}\end{quote}

\end{fulllineitems}\end{savenotes}

\index{\_\_init\_gui\_connections() (opihiexarata.gui.manual.OpihiManualWindow method)@\spxentry{\_\_init\_gui\_connections()}\spxextra{opihiexarata.gui.manual.OpihiManualWindow method}}

\begin{savenotes}\begin{fulllineitems}
\phantomsection\label{\detokenize{code/opihiexarata.gui.manual:opihiexarata.gui.manual.OpihiManualWindow.__init_gui_connections}}
\pysigstartsignatures
\pysiglinewithargsret{\sphinxbfcode{\sphinxupquote{\_\_init\_gui\_connections}}}{}{{ $\rightarrow$ None}}
\pysigstopsignatures
\sphinxAtStartPar
Assign the action bindings for the buttons which get new
file(names).
\begin{quote}\begin{description}
\sphinxlineitem{Parameters}
\sphinxAtStartPar
\sphinxstyleliteralstrong{\sphinxupquote{None}} – 

\sphinxlineitem{Return type}
\sphinxAtStartPar
None

\end{description}\end{quote}

\end{fulllineitems}\end{savenotes}

\index{\_\_init\_opihi\_image() (opihiexarata.gui.manual.OpihiManualWindow method)@\spxentry{\_\_init\_opihi\_image()}\spxextra{opihiexarata.gui.manual.OpihiManualWindow method}}

\begin{savenotes}\begin{fulllineitems}
\phantomsection\label{\detokenize{code/opihiexarata.gui.manual:opihiexarata.gui.manual.OpihiManualWindow.__init_opihi_image}}
\pysigstartsignatures
\pysiglinewithargsret{\sphinxbfcode{\sphinxupquote{\_\_init\_opihi\_image}}}{}{{ $\rightarrow$ None}}
\pysigstopsignatures
\sphinxAtStartPar
Create the image area which will display what Opihi took from the
sky. This takes advantage of a reserved image vertical layout in the
design of the window.
\begin{quote}\begin{description}
\sphinxlineitem{Parameters}
\sphinxAtStartPar
\sphinxstyleliteralstrong{\sphinxupquote{None}} – 

\sphinxlineitem{Return type}
\sphinxAtStartPar
None

\end{description}\end{quote}

\end{fulllineitems}\end{savenotes}

\index{\_\_init\_preprocess\_solution() (opihiexarata.gui.manual.OpihiManualWindow method)@\spxentry{\_\_init\_preprocess\_solution()}\spxextra{opihiexarata.gui.manual.OpihiManualWindow method}}

\begin{savenotes}\begin{fulllineitems}
\phantomsection\label{\detokenize{code/opihiexarata.gui.manual:opihiexarata.gui.manual.OpihiManualWindow.__init_preprocess_solution}}
\pysigstartsignatures
\pysiglinewithargsret{\sphinxbfcode{\sphinxupquote{\_\_init\_preprocess\_solution}}}{}{}
\pysigstopsignatures
\sphinxAtStartPar
Initialize the preprocessing solution. The preprocessing files
should be specified in the configuration file.
\begin{quote}\begin{description}
\sphinxlineitem{Parameters}
\sphinxAtStartPar
\sphinxstyleliteralstrong{\sphinxupquote{None}} – 

\sphinxlineitem{Return type}
\sphinxAtStartPar
None

\end{description}\end{quote}

\end{fulllineitems}\end{savenotes}

\index{\_\_refresh\_dynamic\_label\_text\_astrometry() (opihiexarata.gui.manual.OpihiManualWindow method)@\spxentry{\_\_refresh\_dynamic\_label\_text\_astrometry()}\spxextra{opihiexarata.gui.manual.OpihiManualWindow method}}

\begin{savenotes}\begin{fulllineitems}
\phantomsection\label{\detokenize{code/opihiexarata.gui.manual:opihiexarata.gui.manual.OpihiManualWindow.__refresh_dynamic_label_text_astrometry}}
\pysigstartsignatures
\pysiglinewithargsret{\sphinxbfcode{\sphinxupquote{\_\_refresh\_dynamic\_label\_text\_astrometry}}}{}{{ $\rightarrow$ None}}
\pysigstopsignatures
\sphinxAtStartPar
Refresh all of the dynamic label text for astrometry.
This fills out the information based on the current solutions
available and solved.

\sphinxAtStartPar
An astrometric solution must exist.
\begin{quote}\begin{description}
\sphinxlineitem{Parameters}
\sphinxAtStartPar
\sphinxstyleliteralstrong{\sphinxupquote{None}} – 

\sphinxlineitem{Return type}
\sphinxAtStartPar
None

\end{description}\end{quote}

\end{fulllineitems}\end{savenotes}

\index{\_\_refresh\_dynamic\_label\_text\_ephemeris() (opihiexarata.gui.manual.OpihiManualWindow method)@\spxentry{\_\_refresh\_dynamic\_label\_text\_ephemeris()}\spxextra{opihiexarata.gui.manual.OpihiManualWindow method}}

\begin{savenotes}\begin{fulllineitems}
\phantomsection\label{\detokenize{code/opihiexarata.gui.manual:opihiexarata.gui.manual.OpihiManualWindow.__refresh_dynamic_label_text_ephemeris}}
\pysigstartsignatures
\pysiglinewithargsret{\sphinxbfcode{\sphinxupquote{\_\_refresh\_dynamic\_label\_text\_ephemeris}}}{}{{ $\rightarrow$ None}}
\pysigstopsignatures
\sphinxAtStartPar
Refresh all of the dynamic label text for ephemerides.
This fills out the information based on the current solutions
available and solved.

\sphinxAtStartPar
An ephemeritic solution must exist.
\begin{quote}\begin{description}
\sphinxlineitem{Parameters}
\sphinxAtStartPar
\sphinxstyleliteralstrong{\sphinxupquote{None}} – 

\sphinxlineitem{Return type}
\sphinxAtStartPar
None

\end{description}\end{quote}

\end{fulllineitems}\end{savenotes}

\index{\_\_refresh\_dynamic\_label\_text\_orbit() (opihiexarata.gui.manual.OpihiManualWindow method)@\spxentry{\_\_refresh\_dynamic\_label\_text\_orbit()}\spxextra{opihiexarata.gui.manual.OpihiManualWindow method}}

\begin{savenotes}\begin{fulllineitems}
\phantomsection\label{\detokenize{code/opihiexarata.gui.manual:opihiexarata.gui.manual.OpihiManualWindow.__refresh_dynamic_label_text_orbit}}
\pysigstartsignatures
\pysiglinewithargsret{\sphinxbfcode{\sphinxupquote{\_\_refresh\_dynamic\_label\_text\_orbit}}}{}{{ $\rightarrow$ None}}
\pysigstopsignatures
\sphinxAtStartPar
Refresh all of the dynamic label text for orbit.
This fills out the information based on the current solutions
available and solved.

\sphinxAtStartPar
An orbital solution must exist.
\begin{quote}\begin{description}
\sphinxlineitem{Parameters}
\sphinxAtStartPar
\sphinxstyleliteralstrong{\sphinxupquote{None}} – 

\sphinxlineitem{Return type}
\sphinxAtStartPar
None

\end{description}\end{quote}

\end{fulllineitems}\end{savenotes}

\index{\_\_refresh\_dynamic\_label\_text\_photometry() (opihiexarata.gui.manual.OpihiManualWindow method)@\spxentry{\_\_refresh\_dynamic\_label\_text\_photometry()}\spxextra{opihiexarata.gui.manual.OpihiManualWindow method}}

\begin{savenotes}\begin{fulllineitems}
\phantomsection\label{\detokenize{code/opihiexarata.gui.manual:opihiexarata.gui.manual.OpihiManualWindow.__refresh_dynamic_label_text_photometry}}
\pysigstartsignatures
\pysiglinewithargsret{\sphinxbfcode{\sphinxupquote{\_\_refresh\_dynamic\_label\_text\_photometry}}}{}{{ $\rightarrow$ None}}
\pysigstopsignatures
\sphinxAtStartPar
Refresh all of the dynamic label text for photometry.
This fills out the information based on the current solutions
available and solved.

\sphinxAtStartPar
A photometric solution must exist.
\begin{quote}\begin{description}
\sphinxlineitem{Parameters}
\sphinxAtStartPar
\sphinxstyleliteralstrong{\sphinxupquote{None}} – 

\sphinxlineitem{Return type}
\sphinxAtStartPar
None

\end{description}\end{quote}

\end{fulllineitems}\end{savenotes}

\index{\_\_refresh\_dynamic\_label\_text\_propagate() (opihiexarata.gui.manual.OpihiManualWindow method)@\spxentry{\_\_refresh\_dynamic\_label\_text\_propagate()}\spxextra{opihiexarata.gui.manual.OpihiManualWindow method}}

\begin{savenotes}\begin{fulllineitems}
\phantomsection\label{\detokenize{code/opihiexarata.gui.manual:opihiexarata.gui.manual.OpihiManualWindow.__refresh_dynamic_label_text_propagate}}
\pysigstartsignatures
\pysiglinewithargsret{\sphinxbfcode{\sphinxupquote{\_\_refresh\_dynamic\_label\_text\_propagate}}}{}{{ $\rightarrow$ None}}
\pysigstopsignatures
\sphinxAtStartPar
Refresh all of the dynamic label text for propagate.
This fills out the information based on the current solutions
available and solved.

\sphinxAtStartPar
A propagative solution must exist.
\begin{quote}\begin{description}
\sphinxlineitem{Parameters}
\sphinxAtStartPar
\sphinxstyleliteralstrong{\sphinxupquote{None}} – 

\sphinxlineitem{Return type}
\sphinxAtStartPar
None

\end{description}\end{quote}

\end{fulllineitems}\end{savenotes}

\index{\_load\_fits\_file() (opihiexarata.gui.manual.OpihiManualWindow method)@\spxentry{\_load\_fits\_file()}\spxextra{opihiexarata.gui.manual.OpihiManualWindow method}}

\begin{savenotes}\begin{fulllineitems}
\phantomsection\label{\detokenize{code/opihiexarata.gui.manual:opihiexarata.gui.manual.OpihiManualWindow._load_fits_file}}
\pysigstartsignatures
\pysiglinewithargsret{\sphinxbfcode{\sphinxupquote{\_load\_fits\_file}}}{\emph{\DUrole{n}{fits\_filename}\DUrole{p}{:}\DUrole{w}{  }\DUrole{n}{str}}}{{ $\rightarrow$ None}}
\pysigstopsignatures
\sphinxAtStartPar
Load a fits file into the GUI to derive all of the solutions needed.

\sphinxAtStartPar
This loads the fits file into an OpihiSolution class for which this
GUI interacts with and derives the information from.
\begin{quote}\begin{description}
\sphinxlineitem{Parameters}
\sphinxAtStartPar
\sphinxstyleliteralstrong{\sphinxupquote{fits\_filename}} (\sphinxstyleliteralemphasis{\sphinxupquote{str}}) – The fits filename which will be loaded.

\sphinxlineitem{Return type}
\sphinxAtStartPar
None

\end{description}\end{quote}

\end{fulllineitems}\end{savenotes}

\index{\_parse\_custom\_orbital\_elements() (opihiexarata.gui.manual.OpihiManualWindow method)@\spxentry{\_parse\_custom\_orbital\_elements()}\spxextra{opihiexarata.gui.manual.OpihiManualWindow method}}

\begin{savenotes}\begin{fulllineitems}
\phantomsection\label{\detokenize{code/opihiexarata.gui.manual:opihiexarata.gui.manual.OpihiManualWindow._parse_custom_orbital_elements}}
\pysigstartsignatures
\pysiglinewithargsret{\sphinxbfcode{\sphinxupquote{\_parse\_custom\_orbital\_elements}}}{}{{ $\rightarrow$ dict}}
\pysigstopsignatures
\sphinxAtStartPar
This function takes the textual form of the orbital elements as
entered and tries to parse it into a set of orbital elements and errors.
\begin{quote}\begin{description}
\sphinxlineitem{Parameters}
\sphinxAtStartPar
\sphinxstyleliteralstrong{\sphinxupquote{None}} – 

\sphinxlineitem{Returns}
\sphinxAtStartPar
\sphinxstylestrong{orbital\_elements} – A dictionary of the orbital elements and their errors, if they
exist.

\sphinxlineitem{Return type}
\sphinxAtStartPar
dictionary

\end{description}\end{quote}

\end{fulllineitems}\end{savenotes}

\index{\_preprocess\_fits\_file() (opihiexarata.gui.manual.OpihiManualWindow method)@\spxentry{\_preprocess\_fits\_file()}\spxextra{opihiexarata.gui.manual.OpihiManualWindow method}}

\begin{savenotes}\begin{fulllineitems}
\phantomsection\label{\detokenize{code/opihiexarata.gui.manual:opihiexarata.gui.manual.OpihiManualWindow._preprocess_fits_file}}
\pysigstartsignatures
\pysiglinewithargsret{\sphinxbfcode{\sphinxupquote{\_preprocess\_fits\_file}}}{\emph{\DUrole{n}{raw\_filename}\DUrole{p}{:}\DUrole{w}{  }\DUrole{n}{str}}, \emph{\DUrole{n}{process\_filename}\DUrole{p}{:}\DUrole{w}{  }\DUrole{n}{Optional\DUrole{p}{{[}}str\DUrole{p}{{]}}}\DUrole{w}{  }\DUrole{o}{=}\DUrole{w}{  }\DUrole{default_value}{None}}}{{ $\rightarrow$ str}}
\pysigstopsignatures
\sphinxAtStartPar
Using the provided filename, preprocess the fits file listed.

\sphinxAtStartPar
This is used primarily during the loading of a new fits file.
\begin{quote}\begin{description}
\sphinxlineitem{Parameters}\begin{itemize}
\item {} 
\sphinxAtStartPar
\sphinxstyleliteralstrong{\sphinxupquote{raw\_filename}} (\sphinxstyleliteralemphasis{\sphinxupquote{string}}) – The filename of the raw fits file to be preprocessed.

\item {} 
\sphinxAtStartPar
\sphinxstyleliteralstrong{\sphinxupquote{process\_filename}} (\sphinxstyleliteralemphasis{\sphinxupquote{string}}\sphinxstyleliteralemphasis{\sphinxupquote{, }}\sphinxstyleliteralemphasis{\sphinxupquote{default = None}}) – The filename where the processed fits file will be saved to. If
None, it defaults to a sensible naming convention to distinguish
the two.

\end{itemize}

\sphinxlineitem{Returns}
\sphinxAtStartPar
\sphinxstylestrong{process\_filename} – The filename of the preprocessed file is returned, this is just
in case the value was derived from the naming convention. This is
an absolute path.

\sphinxlineitem{Return type}
\sphinxAtStartPar
string

\end{description}\end{quote}

\end{fulllineitems}\end{savenotes}

\index{clear\_dynamic\_label\_text() (opihiexarata.gui.manual.OpihiManualWindow method)@\spxentry{clear\_dynamic\_label\_text()}\spxextra{opihiexarata.gui.manual.OpihiManualWindow method}}

\begin{savenotes}\begin{fulllineitems}
\phantomsection\label{\detokenize{code/opihiexarata.gui.manual:opihiexarata.gui.manual.OpihiManualWindow.clear_dynamic_label_text}}
\pysigstartsignatures
\pysiglinewithargsret{\sphinxbfcode{\sphinxupquote{clear\_dynamic\_label\_text}}}{}{{ $\rightarrow$ None}}
\pysigstopsignatures
\sphinxAtStartPar
Clear all of the dynamic label text and other related fields,
this is traditionally done just before a new image is going to be
introduced.

\sphinxAtStartPar
This resets the text back to their defaults as per the GUI builder.

\sphinxAtStartPar
The selections on the engines do not need to be reset as they are
just a selection criteria and do not actually provide any information.
\begin{quote}\begin{description}
\sphinxlineitem{Parameters}
\sphinxAtStartPar
\sphinxstyleliteralstrong{\sphinxupquote{None}} – 

\sphinxlineitem{Return type}
\sphinxAtStartPar
None

\end{description}\end{quote}

\end{fulllineitems}\end{savenotes}

\index{redraw\_opihi\_image() (opihiexarata.gui.manual.OpihiManualWindow method)@\spxentry{redraw\_opihi\_image()}\spxextra{opihiexarata.gui.manual.OpihiManualWindow method}}

\begin{savenotes}\begin{fulllineitems}
\phantomsection\label{\detokenize{code/opihiexarata.gui.manual:opihiexarata.gui.manual.OpihiManualWindow.redraw_opihi_image}}
\pysigstartsignatures
\pysiglinewithargsret{\sphinxbfcode{\sphinxupquote{redraw\_opihi\_image}}}{}{{ $\rightarrow$ None}}
\pysigstopsignatures
\sphinxAtStartPar
Redraw the Opihi image given that new results may have been added
because some solutions were completed. This modifies the GUI in\sphinxhyphen{}place.
\begin{quote}\begin{description}
\sphinxlineitem{Parameters}
\sphinxAtStartPar
\sphinxstyleliteralstrong{\sphinxupquote{None}} – 

\sphinxlineitem{Return type}
\sphinxAtStartPar
None

\end{description}\end{quote}

\end{fulllineitems}\end{savenotes}

\index{refresh\_dynamic\_label\_text() (opihiexarata.gui.manual.OpihiManualWindow method)@\spxentry{refresh\_dynamic\_label\_text()}\spxextra{opihiexarata.gui.manual.OpihiManualWindow method}}

\begin{savenotes}\begin{fulllineitems}
\phantomsection\label{\detokenize{code/opihiexarata.gui.manual:opihiexarata.gui.manual.OpihiManualWindow.refresh_dynamic_label_text}}
\pysigstartsignatures
\pysiglinewithargsret{\sphinxbfcode{\sphinxupquote{refresh\_dynamic\_label\_text}}}{}{{ $\rightarrow$ None}}
\pysigstopsignatures
\sphinxAtStartPar
Refresh all of the dynamic label text, this fills out the
information based on the current solutions available and solved.
\begin{quote}\begin{description}
\sphinxlineitem{Parameters}
\sphinxAtStartPar
\sphinxstyleliteralstrong{\sphinxupquote{None}} – 

\sphinxlineitem{Return type}
\sphinxAtStartPar
None

\end{description}\end{quote}

\end{fulllineitems}\end{savenotes}

\index{save\_results() (opihiexarata.gui.manual.OpihiManualWindow method)@\spxentry{save\_results()}\spxextra{opihiexarata.gui.manual.OpihiManualWindow method}}

\begin{savenotes}\begin{fulllineitems}
\phantomsection\label{\detokenize{code/opihiexarata.gui.manual:opihiexarata.gui.manual.OpihiManualWindow.save_results}}
\pysigstartsignatures
\pysiglinewithargsret{\sphinxbfcode{\sphinxupquote{save\_results}}}{}{{ $\rightarrow$ None}}
\pysigstopsignatures
\sphinxAtStartPar
Save all of the results of the solutions to date. This is especially
done upon selecting a new image, the previous image results are
saved.

\sphinxAtStartPar
If there is no solution class, then there is no results to save either
and this function does nothing.
\begin{quote}\begin{description}
\sphinxlineitem{Parameters}
\sphinxAtStartPar
\sphinxstyleliteralstrong{\sphinxupquote{None}} – 

\sphinxlineitem{Return type}
\sphinxAtStartPar
None

\end{description}\end{quote}

\end{fulllineitems}\end{savenotes}

\index{staticMetaObject (opihiexarata.gui.manual.OpihiManualWindow attribute)@\spxentry{staticMetaObject}\spxextra{opihiexarata.gui.manual.OpihiManualWindow attribute}}

\begin{savenotes}\begin{fulllineitems}
\phantomsection\label{\detokenize{code/opihiexarata.gui.manual:opihiexarata.gui.manual.OpihiManualWindow.staticMetaObject}}
\pysigstartsignatures
\pysigline{\sphinxbfcode{\sphinxupquote{staticMetaObject}}\sphinxbfcode{\sphinxupquote{\DUrole{w}{  }\DUrole{p}{=}\DUrole{w}{  }PySide6.QtCore.QMetaObject("OpihiManualWindow" inherits "QMainWindow": )}}}
\pysigstopsignatures
\end{fulllineitems}\end{savenotes}


\end{fulllineitems}\end{savenotes}

\index{start\_manual\_window() (in module opihiexarata.gui.manual)@\spxentry{start\_manual\_window()}\spxextra{in module opihiexarata.gui.manual}}

\begin{savenotes}\begin{fulllineitems}
\phantomsection\label{\detokenize{code/opihiexarata.gui.manual:opihiexarata.gui.manual.start_manual_window}}
\pysigstartsignatures
\pysiglinewithargsret{\sphinxcode{\sphinxupquote{opihiexarata.gui.manual.}}\sphinxbfcode{\sphinxupquote{start\_manual\_window}}}{}{{ $\rightarrow$ None}}
\pysigstopsignatures
\sphinxAtStartPar
This is the function to create the manual window for usage.
\begin{quote}\begin{description}
\sphinxlineitem{Parameters}
\sphinxAtStartPar
\sphinxstyleliteralstrong{\sphinxupquote{None}} – 

\sphinxlineitem{Return type}
\sphinxAtStartPar
None

\end{description}\end{quote}

\end{fulllineitems}\end{savenotes}


\sphinxstepscope


\subparagraph{opihiexarata.gui.name module}
\label{\detokenize{code/opihiexarata.gui.name:module-opihiexarata.gui.name}}\label{\detokenize{code/opihiexarata.gui.name:opihiexarata-gui-name-module}}\label{\detokenize{code/opihiexarata.gui.name::doc}}\index{module@\spxentry{module}!opihiexarata.gui.name@\spxentry{opihiexarata.gui.name}}\index{opihiexarata.gui.name@\spxentry{opihiexarata.gui.name}!module@\spxentry{module}}
\sphinxAtStartPar
This window is for allowing the user to fill out the name of the object
which they are observing, used when doing new targets.

\sphinxAtStartPar
This is just a simple input dialog.
\index{TargetNameWindow (class in opihiexarata.gui.name)@\spxentry{TargetNameWindow}\spxextra{class in opihiexarata.gui.name}}

\begin{savenotes}\begin{fulllineitems}
\phantomsection\label{\detokenize{code/opihiexarata.gui.name:opihiexarata.gui.name.TargetNameWindow}}
\pysigstartsignatures
\pysigline{\sphinxbfcode{\sphinxupquote{class\DUrole{w}{  }}}\sphinxcode{\sphinxupquote{opihiexarata.gui.name.}}\sphinxbfcode{\sphinxupquote{TargetNameWindow}}}
\pysigstopsignatures
\sphinxAtStartPar
Bases: \sphinxcode{\sphinxupquote{QWidget}}
\index{\_\_init\_\_() (opihiexarata.gui.name.TargetNameWindow method)@\spxentry{\_\_init\_\_()}\spxextra{opihiexarata.gui.name.TargetNameWindow method}}

\begin{savenotes}\begin{fulllineitems}
\phantomsection\label{\detokenize{code/opihiexarata.gui.name:opihiexarata.gui.name.TargetNameWindow.__init__}}
\pysigstartsignatures
\pysiglinewithargsret{\sphinxbfcode{\sphinxupquote{\_\_init\_\_}}}{}{{ $\rightarrow$ None}}
\pysigstopsignatures
\sphinxAtStartPar
Setting up the window for the user prompt.
\begin{quote}\begin{description}
\sphinxlineitem{Parameters}
\sphinxAtStartPar
\sphinxstyleliteralstrong{\sphinxupquote{None}} – 

\sphinxlineitem{Return type}
\sphinxAtStartPar
None

\end{description}\end{quote}

\end{fulllineitems}\end{savenotes}

\index{\_sanitize\_input() (opihiexarata.gui.name.TargetNameWindow static method)@\spxentry{\_sanitize\_input()}\spxextra{opihiexarata.gui.name.TargetNameWindow static method}}

\begin{savenotes}\begin{fulllineitems}
\phantomsection\label{\detokenize{code/opihiexarata.gui.name:opihiexarata.gui.name.TargetNameWindow._sanitize_input}}
\pysigstartsignatures
\pysiglinewithargsret{\sphinxbfcode{\sphinxupquote{static\DUrole{w}{  }}}\sphinxbfcode{\sphinxupquote{\_sanitize\_input}}}{\emph{\DUrole{n}{raw\_input}\DUrole{p}{:}\DUrole{w}{  }\DUrole{n}{Optional\DUrole{p}{{[}}str\DUrole{p}{{]}}}\DUrole{w}{  }\DUrole{o}{=}\DUrole{w}{  }\DUrole{default_value}{None}}}{{ $\rightarrow$ str}}
\pysigstopsignatures
\sphinxAtStartPar
Sanitization of the raw input to the proper input.
\begin{quote}\begin{description}
\sphinxlineitem{Parameters}
\sphinxAtStartPar
\sphinxstyleliteralstrong{\sphinxupquote{raw\_input}} (\sphinxstyleliteralemphasis{\sphinxupquote{string}}) – The raw input to be sanitized.

\sphinxlineitem{Returns}
\sphinxAtStartPar
\sphinxstylestrong{sanitized\_input} – The sanitized input.

\sphinxlineitem{Return type}
\sphinxAtStartPar
string

\end{description}\end{quote}

\end{fulllineitems}\end{savenotes}

\index{staticMetaObject (opihiexarata.gui.name.TargetNameWindow attribute)@\spxentry{staticMetaObject}\spxextra{opihiexarata.gui.name.TargetNameWindow attribute}}

\begin{savenotes}\begin{fulllineitems}
\phantomsection\label{\detokenize{code/opihiexarata.gui.name:opihiexarata.gui.name.TargetNameWindow.staticMetaObject}}
\pysigstartsignatures
\pysigline{\sphinxbfcode{\sphinxupquote{staticMetaObject}}\sphinxbfcode{\sphinxupquote{\DUrole{w}{  }\DUrole{p}{=}\DUrole{w}{  }PySide6.QtCore.QMetaObject("TargetNameWindow" inherits "QWidget": )}}}
\pysigstopsignatures
\end{fulllineitems}\end{savenotes}


\end{fulllineitems}\end{savenotes}

\index{ask\_user\_target\_name\_window() (in module opihiexarata.gui.name)@\spxentry{ask\_user\_target\_name\_window()}\spxextra{in module opihiexarata.gui.name}}

\begin{savenotes}\begin{fulllineitems}
\phantomsection\label{\detokenize{code/opihiexarata.gui.name:opihiexarata.gui.name.ask_user_target_name_window}}
\pysigstartsignatures
\pysiglinewithargsret{\sphinxcode{\sphinxupquote{opihiexarata.gui.name.}}\sphinxbfcode{\sphinxupquote{ask\_user\_target\_name\_window}}}{\emph{\DUrole{n}{default}\DUrole{p}{:}\DUrole{w}{  }\DUrole{n}{Optional\DUrole{p}{{[}}str\DUrole{p}{{]}}}\DUrole{w}{  }\DUrole{o}{=}\DUrole{w}{  }\DUrole{default_value}{None}}}{{ $\rightarrow$ str}}
\pysigstopsignatures
\sphinxAtStartPar
Use the target name window to prompt the user for the name of the
object that they are studying.
\begin{quote}\begin{description}
\sphinxlineitem{Parameters}
\sphinxAtStartPar
\sphinxstyleliteralstrong{\sphinxupquote{default}} (\sphinxstyleliteralemphasis{\sphinxupquote{string}}) – The default name to provide should the user interact with the dialog
box to submit a name. (That is, they cancel or close it.)

\sphinxlineitem{Returns}
\sphinxAtStartPar
\sphinxstylestrong{target\_name} – The target name as specified (or the default if something went amiss.)

\sphinxlineitem{Return type}
\sphinxAtStartPar
string

\end{description}\end{quote}

\end{fulllineitems}\end{savenotes}

\index{main() (in module opihiexarata.gui.name)@\spxentry{main()}\spxextra{in module opihiexarata.gui.name}}

\begin{savenotes}\begin{fulllineitems}
\phantomsection\label{\detokenize{code/opihiexarata.gui.name:opihiexarata.gui.name.main}}
\pysigstartsignatures
\pysiglinewithargsret{\sphinxcode{\sphinxupquote{opihiexarata.gui.name.}}\sphinxbfcode{\sphinxupquote{main}}}{}{{ $\rightarrow$ None}}
\pysigstopsignatures
\end{fulllineitems}\end{savenotes}


\sphinxstepscope


\subparagraph{opihiexarata.gui.selector module}
\label{\detokenize{code/opihiexarata.gui.selector:module-opihiexarata.gui.selector}}\label{\detokenize{code/opihiexarata.gui.selector:opihiexarata-gui-selector-module}}\label{\detokenize{code/opihiexarata.gui.selector::doc}}\index{module@\spxentry{module}!opihiexarata.gui.selector@\spxentry{opihiexarata.gui.selector}}\index{opihiexarata.gui.selector@\spxentry{opihiexarata.gui.selector}!module@\spxentry{module}}\index{TargetSelectorWindow (class in opihiexarata.gui.selector)@\spxentry{TargetSelectorWindow}\spxextra{class in opihiexarata.gui.selector}}

\begin{savenotes}\begin{fulllineitems}
\phantomsection\label{\detokenize{code/opihiexarata.gui.selector:opihiexarata.gui.selector.TargetSelectorWindow}}
\pysigstartsignatures
\pysiglinewithargsret{\sphinxbfcode{\sphinxupquote{class\DUrole{w}{  }}}\sphinxcode{\sphinxupquote{opihiexarata.gui.selector.}}\sphinxbfcode{\sphinxupquote{TargetSelectorWindow}}}{\emph{\DUrole{n}{current\_fits\_filename}\DUrole{p}{:}\DUrole{w}{  }\DUrole{n}{str}}, \emph{\DUrole{n}{reference\_fits\_filename}\DUrole{p}{:}\DUrole{w}{  }\DUrole{n}{Optional\DUrole{p}{{[}}str\DUrole{p}{{]}}}\DUrole{w}{  }\DUrole{o}{=}\DUrole{w}{  }\DUrole{default_value}{None}}}{}
\pysigstopsignatures
\sphinxAtStartPar
Bases: \sphinxcode{\sphinxupquote{QWidget}}
\index{\_\_connect\_check\_box\_autoscale\_1\_99() (opihiexarata.gui.selector.TargetSelectorWindow method)@\spxentry{\_\_connect\_check\_box\_autoscale\_1\_99()}\spxextra{opihiexarata.gui.selector.TargetSelectorWindow method}}

\begin{savenotes}\begin{fulllineitems}
\phantomsection\label{\detokenize{code/opihiexarata.gui.selector:opihiexarata.gui.selector.TargetSelectorWindow.__connect_check_box_autoscale_1_99}}
\pysigstartsignatures
\pysiglinewithargsret{\sphinxbfcode{\sphinxupquote{\_\_connect\_check\_box\_autoscale\_1\_99}}}{}{{ $\rightarrow$ None}}
\pysigstopsignatures
\sphinxAtStartPar
This check box allows the user to force the autoscaling of images
when the subtraction method changes.
\begin{quote}\begin{description}
\sphinxlineitem{Parameters}
\sphinxAtStartPar
\sphinxstyleliteralstrong{\sphinxupquote{None}} – 

\sphinxlineitem{Return type}
\sphinxAtStartPar
None

\end{description}\end{quote}

\end{fulllineitems}\end{savenotes}

\index{\_\_connect\_line\_edit\_dynamic\_scale\_high() (opihiexarata.gui.selector.TargetSelectorWindow method)@\spxentry{\_\_connect\_line\_edit\_dynamic\_scale\_high()}\spxextra{opihiexarata.gui.selector.TargetSelectorWindow method}}

\begin{savenotes}\begin{fulllineitems}
\phantomsection\label{\detokenize{code/opihiexarata.gui.selector:opihiexarata.gui.selector.TargetSelectorWindow.__connect_line_edit_dynamic_scale_high}}
\pysigstartsignatures
\pysiglinewithargsret{\sphinxbfcode{\sphinxupquote{\_\_connect\_line\_edit\_dynamic\_scale\_high}}}{}{{ $\rightarrow$ None}}
\pysigstopsignatures
\sphinxAtStartPar
A function to operate on the change of the text of the high scale.
\begin{quote}\begin{description}
\sphinxlineitem{Parameters}
\sphinxAtStartPar
\sphinxstyleliteralstrong{\sphinxupquote{None}} – 

\sphinxlineitem{Return type}
\sphinxAtStartPar
None

\end{description}\end{quote}

\end{fulllineitems}\end{savenotes}

\index{\_\_connect\_line\_edit\_dynamic\_scale\_low() (opihiexarata.gui.selector.TargetSelectorWindow method)@\spxentry{\_\_connect\_line\_edit\_dynamic\_scale\_low()}\spxextra{opihiexarata.gui.selector.TargetSelectorWindow method}}

\begin{savenotes}\begin{fulllineitems}
\phantomsection\label{\detokenize{code/opihiexarata.gui.selector:opihiexarata.gui.selector.TargetSelectorWindow.__connect_line_edit_dynamic_scale_low}}
\pysigstartsignatures
\pysiglinewithargsret{\sphinxbfcode{\sphinxupquote{\_\_connect\_line\_edit\_dynamic\_scale\_low}}}{}{{ $\rightarrow$ None}}
\pysigstopsignatures
\sphinxAtStartPar
A function to operate on the change of the text of the low scale.
\begin{quote}\begin{description}
\sphinxlineitem{Parameters}
\sphinxAtStartPar
\sphinxstyleliteralstrong{\sphinxupquote{None}} – 

\sphinxlineitem{Return type}
\sphinxAtStartPar
None

\end{description}\end{quote}

\end{fulllineitems}\end{savenotes}

\index{\_\_connect\_matplotlib\_mouse\_press\_event() (opihiexarata.gui.selector.TargetSelectorWindow method)@\spxentry{\_\_connect\_matplotlib\_mouse\_press\_event()}\spxextra{opihiexarata.gui.selector.TargetSelectorWindow method}}

\begin{savenotes}\begin{fulllineitems}
\phantomsection\label{\detokenize{code/opihiexarata.gui.selector:opihiexarata.gui.selector.TargetSelectorWindow.__connect_matplotlib_mouse_press_event}}
\pysigstartsignatures
\pysiglinewithargsret{\sphinxbfcode{\sphinxupquote{\_\_connect\_matplotlib\_mouse\_press\_event}}}{\emph{\DUrole{n}{event}\DUrole{p}{:}\DUrole{w}{  }\DUrole{n}{MouseEvent}}}{{ $\rightarrow$ None}}
\pysigstopsignatures
\sphinxAtStartPar
A function to describe what would happen when a mouse press is
done on the Matplotlib image.

\sphinxAtStartPar
This function defaults to the toolbar functionality when the toolbar
is considered active.
\begin{quote}\begin{description}
\sphinxlineitem{Parameters}
\sphinxAtStartPar
\sphinxstyleliteralstrong{\sphinxupquote{event}} (\sphinxstyleliteralemphasis{\sphinxupquote{MouseEvent}}) – The event of the click itself.

\sphinxlineitem{Return type}
\sphinxAtStartPar
None

\end{description}\end{quote}

\end{fulllineitems}\end{savenotes}

\index{\_\_connect\_matplotlib\_mouse\_release\_event() (opihiexarata.gui.selector.TargetSelectorWindow method)@\spxentry{\_\_connect\_matplotlib\_mouse\_release\_event()}\spxextra{opihiexarata.gui.selector.TargetSelectorWindow method}}

\begin{savenotes}\begin{fulllineitems}
\phantomsection\label{\detokenize{code/opihiexarata.gui.selector:opihiexarata.gui.selector.TargetSelectorWindow.__connect_matplotlib_mouse_release_event}}
\pysigstartsignatures
\pysiglinewithargsret{\sphinxbfcode{\sphinxupquote{\_\_connect\_matplotlib\_mouse\_release\_event}}}{\emph{\DUrole{n}{event}\DUrole{p}{:}\DUrole{w}{  }\DUrole{n}{MouseEvent}}}{{ $\rightarrow$ None}}
\pysigstopsignatures
\sphinxAtStartPar
A function to describe what would happen when a mouse press is
released done on the Matplotlib image.
\begin{quote}\begin{description}
\sphinxlineitem{Parameters}
\sphinxAtStartPar
\sphinxstyleliteralstrong{\sphinxupquote{event}} (\sphinxstyleliteralemphasis{\sphinxupquote{MouseEvent}}) – The event of the click itself.

\sphinxlineitem{Return type}
\sphinxAtStartPar
None

\end{description}\end{quote}

\end{fulllineitems}\end{savenotes}

\index{\_\_connect\_push\_button\_change\_current\_filename() (opihiexarata.gui.selector.TargetSelectorWindow method)@\spxentry{\_\_connect\_push\_button\_change\_current\_filename()}\spxextra{opihiexarata.gui.selector.TargetSelectorWindow method}}

\begin{savenotes}\begin{fulllineitems}
\phantomsection\label{\detokenize{code/opihiexarata.gui.selector:opihiexarata.gui.selector.TargetSelectorWindow.__connect_push_button_change_current_filename}}
\pysigstartsignatures
\pysiglinewithargsret{\sphinxbfcode{\sphinxupquote{\_\_connect\_push\_button\_change\_current\_filename}}}{}{{ $\rightarrow$ None}}
\pysigstopsignatures
\sphinxAtStartPar
This function provides a popup dialog to prompt the user to change
the current fits filename.
\begin{quote}\begin{description}
\sphinxlineitem{Parameters}
\sphinxAtStartPar
\sphinxstyleliteralstrong{\sphinxupquote{None}} – 

\sphinxlineitem{Return type}
\sphinxAtStartPar
None

\end{description}\end{quote}

\end{fulllineitems}\end{savenotes}

\index{\_\_connect\_push\_button\_change\_reference\_filename() (opihiexarata.gui.selector.TargetSelectorWindow method)@\spxentry{\_\_connect\_push\_button\_change\_reference\_filename()}\spxextra{opihiexarata.gui.selector.TargetSelectorWindow method}}

\begin{savenotes}\begin{fulllineitems}
\phantomsection\label{\detokenize{code/opihiexarata.gui.selector:opihiexarata.gui.selector.TargetSelectorWindow.__connect_push_button_change_reference_filename}}
\pysigstartsignatures
\pysiglinewithargsret{\sphinxbfcode{\sphinxupquote{\_\_connect\_push\_button\_change\_reference\_filename}}}{}{{ $\rightarrow$ None}}
\pysigstopsignatures
\sphinxAtStartPar
This function provides a popup dialog to prompt the user to change
the reference fits filename.
\begin{quote}\begin{description}
\sphinxlineitem{Parameters}
\sphinxAtStartPar
\sphinxstyleliteralstrong{\sphinxupquote{None}} – 

\sphinxlineitem{Return type}
\sphinxAtStartPar
None

\end{description}\end{quote}

\end{fulllineitems}\end{savenotes}

\index{\_\_connect\_push\_button\_mode\_non\_sidereal() (opihiexarata.gui.selector.TargetSelectorWindow method)@\spxentry{\_\_connect\_push\_button\_mode\_non\_sidereal()}\spxextra{opihiexarata.gui.selector.TargetSelectorWindow method}}

\begin{savenotes}\begin{fulllineitems}
\phantomsection\label{\detokenize{code/opihiexarata.gui.selector:opihiexarata.gui.selector.TargetSelectorWindow.__connect_push_button_mode_non_sidereal}}
\pysigstartsignatures
\pysiglinewithargsret{\sphinxbfcode{\sphinxupquote{\_\_connect\_push\_button\_mode\_non\_sidereal}}}{}{{ $\rightarrow$ None}}
\pysigstopsignatures
\sphinxAtStartPar
This function sets the subtraction method to non\sphinxhyphen{}sidereal, for
comparing the current image from the reference image.

\sphinxAtStartPar
This method assumes the approximation that the target itself did not
move at all compared to both images, but the stars do as they are
moving siderally.
\begin{quote}\begin{description}
\sphinxlineitem{Parameters}
\sphinxAtStartPar
\sphinxstyleliteralstrong{\sphinxupquote{None}} – 

\sphinxlineitem{Return type}
\sphinxAtStartPar
None

\end{description}\end{quote}

\end{fulllineitems}\end{savenotes}

\index{\_\_connect\_push\_button\_mode\_none() (opihiexarata.gui.selector.TargetSelectorWindow method)@\spxentry{\_\_connect\_push\_button\_mode\_none()}\spxextra{opihiexarata.gui.selector.TargetSelectorWindow method}}

\begin{savenotes}\begin{fulllineitems}
\phantomsection\label{\detokenize{code/opihiexarata.gui.selector:opihiexarata.gui.selector.TargetSelectorWindow.__connect_push_button_mode_none}}
\pysigstartsignatures
\pysiglinewithargsret{\sphinxbfcode{\sphinxupquote{\_\_connect\_push\_button\_mode\_none}}}{}{{ $\rightarrow$ None}}
\pysigstopsignatures
\sphinxAtStartPar
This function sets the subtraction method to None, for comparing
the current image from the reference image.

\sphinxAtStartPar
Both None the type and the string is valid as no subtraction. The
type just means that it has not been formally specified using the GUI.

\sphinxAtStartPar
This method has no subtraction and thus no comparison to the reference
image.
\begin{quote}\begin{description}
\sphinxlineitem{Parameters}
\sphinxAtStartPar
\sphinxstyleliteralstrong{\sphinxupquote{None}} – 

\sphinxlineitem{Return type}
\sphinxAtStartPar
None

\end{description}\end{quote}

\end{fulllineitems}\end{savenotes}

\index{\_\_connect\_push\_button\_mode\_reference() (opihiexarata.gui.selector.TargetSelectorWindow method)@\spxentry{\_\_connect\_push\_button\_mode\_reference()}\spxextra{opihiexarata.gui.selector.TargetSelectorWindow method}}

\begin{savenotes}\begin{fulllineitems}
\phantomsection\label{\detokenize{code/opihiexarata.gui.selector:opihiexarata.gui.selector.TargetSelectorWindow.__connect_push_button_mode_reference}}
\pysigstartsignatures
\pysiglinewithargsret{\sphinxbfcode{\sphinxupquote{\_\_connect\_push\_button\_mode\_reference}}}{}{{ $\rightarrow$ None}}
\pysigstopsignatures
\sphinxAtStartPar
This function sets the subtraction method to Reference, plotting
the reference image instead of the current image.

\sphinxAtStartPar
This method has no subtraction and thus no comparison to the current
image.
\begin{quote}\begin{description}
\sphinxlineitem{Parameters}
\sphinxAtStartPar
\sphinxstyleliteralstrong{\sphinxupquote{None}} – 

\sphinxlineitem{Return type}
\sphinxAtStartPar
None

\end{description}\end{quote}

\end{fulllineitems}\end{savenotes}

\index{\_\_connect\_push\_button\_mode\_sidereal() (opihiexarata.gui.selector.TargetSelectorWindow method)@\spxentry{\_\_connect\_push\_button\_mode\_sidereal()}\spxextra{opihiexarata.gui.selector.TargetSelectorWindow method}}

\begin{savenotes}\begin{fulllineitems}
\phantomsection\label{\detokenize{code/opihiexarata.gui.selector:opihiexarata.gui.selector.TargetSelectorWindow.__connect_push_button_mode_sidereal}}
\pysigstartsignatures
\pysiglinewithargsret{\sphinxbfcode{\sphinxupquote{\_\_connect\_push\_button\_mode\_sidereal}}}{}{{ $\rightarrow$ None}}
\pysigstopsignatures
\sphinxAtStartPar
This function sets the subtraction method to sidereal, for comparing
the current image from the reference image.

\sphinxAtStartPar
This method assumes the approximation that both the current and
reference images are pointing to the same point in the sky.
\begin{quote}\begin{description}
\sphinxlineitem{Parameters}
\sphinxAtStartPar
\sphinxstyleliteralstrong{\sphinxupquote{None}} – 

\sphinxlineitem{Return type}
\sphinxAtStartPar
None

\end{description}\end{quote}

\end{fulllineitems}\end{savenotes}

\index{\_\_connect\_push\_button\_scale\_1\_99() (opihiexarata.gui.selector.TargetSelectorWindow method)@\spxentry{\_\_connect\_push\_button\_scale\_1\_99()}\spxextra{opihiexarata.gui.selector.TargetSelectorWindow method}}

\begin{savenotes}\begin{fulllineitems}
\phantomsection\label{\detokenize{code/opihiexarata.gui.selector:opihiexarata.gui.selector.TargetSelectorWindow.__connect_push_button_scale_1_99}}
\pysigstartsignatures
\pysiglinewithargsret{\sphinxbfcode{\sphinxupquote{\_\_connect\_push\_button\_scale\_1\_99}}}{}{{ $\rightarrow$ None}}
\pysigstopsignatures
\sphinxAtStartPar
A function set the scale automatically to 1\sphinxhyphen{}99 within the
region currently displayed on the screen.
\begin{quote}\begin{description}
\sphinxlineitem{Parameters}
\sphinxAtStartPar
\sphinxstyleliteralstrong{\sphinxupquote{None}} – 

\sphinxlineitem{Return type}
\sphinxAtStartPar
None

\end{description}\end{quote}

\end{fulllineitems}\end{savenotes}

\index{\_\_connect\_push\_button\_submit\_target() (opihiexarata.gui.selector.TargetSelectorWindow method)@\spxentry{\_\_connect\_push\_button\_submit\_target()}\spxextra{opihiexarata.gui.selector.TargetSelectorWindow method}}

\begin{savenotes}\begin{fulllineitems}
\phantomsection\label{\detokenize{code/opihiexarata.gui.selector:opihiexarata.gui.selector.TargetSelectorWindow.__connect_push_button_submit_target}}
\pysigstartsignatures
\pysiglinewithargsret{\sphinxbfcode{\sphinxupquote{\_\_connect\_push\_button\_submit\_target}}}{}{{ $\rightarrow$ None}}
\pysigstopsignatures
\sphinxAtStartPar
This button submits the current location of the target and closes
the window. (The target information is saved within the class
instance.)

\sphinxAtStartPar
If the text within the line edits differ than what the box selection
has selected, then this prioritizes the values as manually defined.
Although this should be rare as any time a box is drawn, the values
and text boxes should be updated.

\sphinxAtStartPar
If no entry is properly convertible, we default to center of the image.
\begin{quote}\begin{description}
\sphinxlineitem{Parameters}
\sphinxAtStartPar
\sphinxstyleliteralstrong{\sphinxupquote{None}} – 

\sphinxlineitem{Return type}
\sphinxAtStartPar
None

\end{description}\end{quote}

\end{fulllineitems}\end{savenotes}

\index{\_\_init\_\_() (opihiexarata.gui.selector.TargetSelectorWindow method)@\spxentry{\_\_init\_\_()}\spxextra{opihiexarata.gui.selector.TargetSelectorWindow method}}

\begin{savenotes}\begin{fulllineitems}
\phantomsection\label{\detokenize{code/opihiexarata.gui.selector:opihiexarata.gui.selector.TargetSelectorWindow.__init__}}
\pysigstartsignatures
\pysiglinewithargsret{\sphinxbfcode{\sphinxupquote{\_\_init\_\_}}}{\emph{\DUrole{n}{current\_fits\_filename}\DUrole{p}{:}\DUrole{w}{  }\DUrole{n}{str}}, \emph{\DUrole{n}{reference\_fits\_filename}\DUrole{p}{:}\DUrole{w}{  }\DUrole{n}{Optional\DUrole{p}{{[}}str\DUrole{p}{{]}}}\DUrole{w}{  }\DUrole{o}{=}\DUrole{w}{  }\DUrole{default_value}{None}}}{{ $\rightarrow$ None}}
\pysigstopsignatures
\sphinxAtStartPar
Create the target selector window. Though often used for asteroids,
there is no reason why is should specific to them; so we use a general
name.
\begin{quote}\begin{description}
\sphinxlineitem{Parameters}\begin{itemize}
\item {} 
\sphinxAtStartPar
\sphinxstyleliteralstrong{\sphinxupquote{current\_fits\_filename}} (\sphinxstyleliteralemphasis{\sphinxupquote{string}}) – The current fits filename which will be used to determine where the
location of the target is.

\item {} 
\sphinxAtStartPar
\sphinxstyleliteralstrong{\sphinxupquote{reference\_fits\_filename}} (\sphinxstyleliteralemphasis{\sphinxupquote{string}}\sphinxstyleliteralemphasis{\sphinxupquote{, }}\sphinxstyleliteralemphasis{\sphinxupquote{default = None}}) – The reference fits filename which will be used to compare against the
current fits filename to determine where the location of the target
is. If None, then no image will be loaded until manually specified.

\end{itemize}

\sphinxlineitem{Return type}
\sphinxAtStartPar
None

\end{description}\end{quote}

\end{fulllineitems}\end{savenotes}

\index{\_\_init\_gui\_connections() (opihiexarata.gui.selector.TargetSelectorWindow method)@\spxentry{\_\_init\_gui\_connections()}\spxextra{opihiexarata.gui.selector.TargetSelectorWindow method}}

\begin{savenotes}\begin{fulllineitems}
\phantomsection\label{\detokenize{code/opihiexarata.gui.selector:opihiexarata.gui.selector.TargetSelectorWindow.__init_gui_connections}}
\pysigstartsignatures
\pysiglinewithargsret{\sphinxbfcode{\sphinxupquote{\_\_init\_gui\_connections}}}{}{}
\pysigstopsignatures
\sphinxAtStartPar
A initiation set of functions that attach to the buttons on the
GUI.
\begin{quote}\begin{description}
\sphinxlineitem{Parameters}
\sphinxAtStartPar
\sphinxstyleliteralstrong{\sphinxupquote{None}} – 

\sphinxlineitem{Return type}
\sphinxAtStartPar
None

\end{description}\end{quote}

\end{fulllineitems}\end{savenotes}

\index{\_\_init\_opihi\_image() (opihiexarata.gui.selector.TargetSelectorWindow method)@\spxentry{\_\_init\_opihi\_image()}\spxextra{opihiexarata.gui.selector.TargetSelectorWindow method}}

\begin{savenotes}\begin{fulllineitems}
\phantomsection\label{\detokenize{code/opihiexarata.gui.selector:opihiexarata.gui.selector.TargetSelectorWindow.__init_opihi_image}}
\pysigstartsignatures
\pysiglinewithargsret{\sphinxbfcode{\sphinxupquote{\_\_init\_opihi\_image}}}{}{{ $\rightarrow$ None}}
\pysigstopsignatures
\sphinxAtStartPar
Create the image area which will display what Opihi took from the
sky. This takes advantage of a reserved image vertical layout in the
design of the window.
\begin{quote}\begin{description}
\sphinxlineitem{Parameters}
\sphinxAtStartPar
\sphinxstyleliteralstrong{\sphinxupquote{None}} – 

\sphinxlineitem{Return type}
\sphinxAtStartPar
None

\end{description}\end{quote}

\end{fulllineitems}\end{savenotes}

\index{\_recompute\_colorbar\_autoscale() (opihiexarata.gui.selector.TargetSelectorWindow method)@\spxentry{\_recompute\_colorbar\_autoscale()}\spxextra{opihiexarata.gui.selector.TargetSelectorWindow method}}

\begin{savenotes}\begin{fulllineitems}
\phantomsection\label{\detokenize{code/opihiexarata.gui.selector:opihiexarata.gui.selector.TargetSelectorWindow._recompute_colorbar_autoscale}}
\pysigstartsignatures
\pysiglinewithargsret{\sphinxbfcode{\sphinxupquote{\_recompute\_colorbar\_autoscale}}}{\emph{\DUrole{n}{lower\_percentile}\DUrole{p}{:}\DUrole{w}{  }\DUrole{n}{float}\DUrole{w}{  }\DUrole{o}{=}\DUrole{w}{  }\DUrole{default_value}{1}}, \emph{\DUrole{n}{higher\_percentile}\DUrole{p}{:}\DUrole{w}{  }\DUrole{n}{float}\DUrole{w}{  }\DUrole{o}{=}\DUrole{w}{  }\DUrole{default_value}{99}}}{{ $\rightarrow$ None}}
\pysigstopsignatures
\sphinxAtStartPar
This is a function to recompute the autoscaling of the colorbar.

\sphinxAtStartPar
This function needs to be split from the connection buttons otherwise
an infinite loop occurs because of their inherent and expected calls
to refresh the window.
\begin{quote}\begin{description}
\sphinxlineitem{Parameters}\begin{itemize}
\item {} 
\sphinxAtStartPar
\sphinxstyleliteralstrong{\sphinxupquote{lower\_percentile}} (\sphinxstyleliteralemphasis{\sphinxupquote{float}}\sphinxstyleliteralemphasis{\sphinxupquote{, }}\sphinxstyleliteralemphasis{\sphinxupquote{default = 1}}) – The lower percentile value which will be defined at the zero point
for the colorbar.

\item {} 
\sphinxAtStartPar
\sphinxstyleliteralstrong{\sphinxupquote{higher\_percentile}} (\sphinxstyleliteralemphasis{\sphinxupquote{float}}\sphinxstyleliteralemphasis{\sphinxupquote{, }}\sphinxstyleliteralemphasis{\sphinxupquote{default = 99}}) – The higher (upper) percentile value which will be defined as the
one point for the colorbar.

\end{itemize}

\sphinxlineitem{Return type}
\sphinxAtStartPar
None

\end{description}\end{quote}

\end{fulllineitems}\end{savenotes}

\index{\_recompute\_subtraction\_arrays() (opihiexarata.gui.selector.TargetSelectorWindow method)@\spxentry{\_recompute\_subtraction\_arrays()}\spxextra{opihiexarata.gui.selector.TargetSelectorWindow method}}

\begin{savenotes}\begin{fulllineitems}
\phantomsection\label{\detokenize{code/opihiexarata.gui.selector:opihiexarata.gui.selector.TargetSelectorWindow._recompute_subtraction_arrays}}
\pysigstartsignatures
\pysiglinewithargsret{\sphinxbfcode{\sphinxupquote{\_recompute\_subtraction\_arrays}}}{}{{ $\rightarrow$ None}}
\pysigstopsignatures
\sphinxAtStartPar
This computes the subtracted arrays for both none, sidereal, and
non\sphinxhyphen{}sidereal subtractions. This is done mostly for speed considerations
as the values can be computed and stored during image loading.
\begin{quote}\begin{description}
\sphinxlineitem{Parameters}
\sphinxAtStartPar
\sphinxstyleliteralstrong{\sphinxupquote{None}} – 

\sphinxlineitem{Return type}
\sphinxAtStartPar
None

\end{description}\end{quote}

\end{fulllineitems}\end{savenotes}

\index{\_refresh\_image() (opihiexarata.gui.selector.TargetSelectorWindow method)@\spxentry{\_refresh\_image()}\spxextra{opihiexarata.gui.selector.TargetSelectorWindow method}}

\begin{savenotes}\begin{fulllineitems}
\phantomsection\label{\detokenize{code/opihiexarata.gui.selector:opihiexarata.gui.selector.TargetSelectorWindow._refresh_image}}
\pysigstartsignatures
\pysiglinewithargsret{\sphinxbfcode{\sphinxupquote{\_refresh\_image}}}{}{{ $\rightarrow$ None}}
\pysigstopsignatures
\sphinxAtStartPar
Redraw and refresh the image, this is mostly used to allow for the
program to update where the user selected.
\begin{quote}\begin{description}
\sphinxlineitem{Parameters}
\sphinxAtStartPar
\sphinxstyleliteralstrong{\sphinxupquote{None}} – 

\sphinxlineitem{Return type}
\sphinxAtStartPar
None

\end{description}\end{quote}

\end{fulllineitems}\end{savenotes}

\index{\_refresh\_text() (opihiexarata.gui.selector.TargetSelectorWindow method)@\spxentry{\_refresh\_text()}\spxextra{opihiexarata.gui.selector.TargetSelectorWindow method}}

\begin{savenotes}\begin{fulllineitems}
\phantomsection\label{\detokenize{code/opihiexarata.gui.selector:opihiexarata.gui.selector.TargetSelectorWindow._refresh_text}}
\pysigstartsignatures
\pysiglinewithargsret{\sphinxbfcode{\sphinxupquote{\_refresh\_text}}}{}{{ $\rightarrow$ None}}
\pysigstopsignatures
\sphinxAtStartPar
This function just refreshes the GUI text based on the current
actual values.
\begin{quote}\begin{description}
\sphinxlineitem{Parameters}
\sphinxAtStartPar
\sphinxstyleliteralstrong{\sphinxupquote{None}} – 

\sphinxlineitem{Return type}
\sphinxAtStartPar
None

\end{description}\end{quote}

\end{fulllineitems}\end{savenotes}

\index{close\_window() (opihiexarata.gui.selector.TargetSelectorWindow method)@\spxentry{close\_window()}\spxextra{opihiexarata.gui.selector.TargetSelectorWindow method}}

\begin{savenotes}\begin{fulllineitems}
\phantomsection\label{\detokenize{code/opihiexarata.gui.selector:opihiexarata.gui.selector.TargetSelectorWindow.close_window}}
\pysigstartsignatures
\pysiglinewithargsret{\sphinxbfcode{\sphinxupquote{close\_window}}}{}{{ $\rightarrow$ None}}
\pysigstopsignatures
\sphinxAtStartPar
Closes the window. Generally called when it is all done.
\begin{quote}\begin{description}
\sphinxlineitem{Parameters}
\sphinxAtStartPar
\sphinxstyleliteralstrong{\sphinxupquote{None}} – 

\sphinxlineitem{Return type}
\sphinxAtStartPar
None

\end{description}\end{quote}

\end{fulllineitems}\end{savenotes}

\index{find\_target\_location() (opihiexarata.gui.selector.TargetSelectorWindow method)@\spxentry{find\_target\_location()}\spxextra{opihiexarata.gui.selector.TargetSelectorWindow method}}

\begin{savenotes}\begin{fulllineitems}
\phantomsection\label{\detokenize{code/opihiexarata.gui.selector:opihiexarata.gui.selector.TargetSelectorWindow.find_target_location}}
\pysigstartsignatures
\pysiglinewithargsret{\sphinxbfcode{\sphinxupquote{find\_target\_location}}}{\emph{\DUrole{n}{x0}\DUrole{p}{:}\DUrole{w}{  }\DUrole{n}{float}}, \emph{\DUrole{n}{x1}\DUrole{p}{:}\DUrole{w}{  }\DUrole{n}{float}}, \emph{\DUrole{n}{y0}\DUrole{p}{:}\DUrole{w}{  }\DUrole{n}{float}}, \emph{\DUrole{n}{y1}\DUrole{p}{:}\DUrole{w}{  }\DUrole{n}{float}}}{{ $\rightarrow$ tuple\DUrole{p}{{[}}float\DUrole{p}{,}\DUrole{w}{  }float\DUrole{p}{{]}}}}
\pysigstopsignatures
\sphinxAtStartPar
Find the location of a target by using a guessed location.
The bounds of the search is specified by the rectangle.
\begin{quote}\begin{description}
\sphinxlineitem{Parameters}\begin{itemize}
\item {} 
\sphinxAtStartPar
\sphinxstyleliteralstrong{\sphinxupquote{x0}} (\sphinxstyleliteralemphasis{\sphinxupquote{float}}) – The lower x axis bound of the search area. These values are cast
into integers upon indexing the search area.

\item {} 
\sphinxAtStartPar
\sphinxstyleliteralstrong{\sphinxupquote{x1}} (\sphinxstyleliteralemphasis{\sphinxupquote{float}}) – The upper x axis bound of the search area. These values are cast
into integers upon indexing the search area.

\item {} 
\sphinxAtStartPar
\sphinxstyleliteralstrong{\sphinxupquote{y0}} (\sphinxstyleliteralemphasis{\sphinxupquote{float}}) – The lower y axis bound of the search area. These values are cast
into integers upon indexing the search area.

\item {} 
\sphinxAtStartPar
\sphinxstyleliteralstrong{\sphinxupquote{y1}} (\sphinxstyleliteralemphasis{\sphinxupquote{float}}) – The upper y axis bound of the search area. These values are cast
into integers upon indexing the search area.

\end{itemize}

\sphinxlineitem{Returns}
\sphinxAtStartPar
\begin{itemize}
\item {} 
\sphinxAtStartPar
\sphinxstylestrong{target\_x} (\sphinxstyleemphasis{float}) – The location of the target, based on the guess.

\item {} 
\sphinxAtStartPar
\sphinxstylestrong{target\_y} (\sphinxstyleemphasis{float}) – The location of the target, based on the guess.

\end{itemize}


\end{description}\end{quote}

\end{fulllineitems}\end{savenotes}

\index{refresh\_window() (opihiexarata.gui.selector.TargetSelectorWindow method)@\spxentry{refresh\_window()}\spxextra{opihiexarata.gui.selector.TargetSelectorWindow method}}

\begin{savenotes}\begin{fulllineitems}
\phantomsection\label{\detokenize{code/opihiexarata.gui.selector:opihiexarata.gui.selector.TargetSelectorWindow.refresh_window}}
\pysigstartsignatures
\pysiglinewithargsret{\sphinxbfcode{\sphinxupquote{refresh\_window}}}{}{{ $\rightarrow$ None}}
\pysigstopsignatures
\sphinxAtStartPar
Refresh the text content of the window given new information.
This refreshes both the dynamic label text and redraws the image.
\begin{quote}\begin{description}
\sphinxlineitem{Parameters}
\sphinxAtStartPar
\sphinxstyleliteralstrong{\sphinxupquote{None}} – 

\sphinxlineitem{Return type}
\sphinxAtStartPar
None

\end{description}\end{quote}

\end{fulllineitems}\end{savenotes}

\index{staticMetaObject (opihiexarata.gui.selector.TargetSelectorWindow attribute)@\spxentry{staticMetaObject}\spxextra{opihiexarata.gui.selector.TargetSelectorWindow attribute}}

\begin{savenotes}\begin{fulllineitems}
\phantomsection\label{\detokenize{code/opihiexarata.gui.selector:opihiexarata.gui.selector.TargetSelectorWindow.staticMetaObject}}
\pysigstartsignatures
\pysigline{\sphinxbfcode{\sphinxupquote{staticMetaObject}}\sphinxbfcode{\sphinxupquote{\DUrole{w}{  }\DUrole{p}{=}\DUrole{w}{  }PySide6.QtCore.QMetaObject("TargetSelectorWindow" inherits "QWidget": )}}}
\pysigstopsignatures
\end{fulllineitems}\end{savenotes}


\end{fulllineitems}\end{savenotes}

\index{ask\_user\_target\_selector\_window() (in module opihiexarata.gui.selector)@\spxentry{ask\_user\_target\_selector\_window()}\spxextra{in module opihiexarata.gui.selector}}

\begin{savenotes}\begin{fulllineitems}
\phantomsection\label{\detokenize{code/opihiexarata.gui.selector:opihiexarata.gui.selector.ask_user_target_selector_window}}
\pysigstartsignatures
\pysiglinewithargsret{\sphinxcode{\sphinxupquote{opihiexarata.gui.selector.}}\sphinxbfcode{\sphinxupquote{ask\_user\_target\_selector\_window}}}{\emph{\DUrole{n}{current\_fits\_filename}}, \emph{\DUrole{n}{reference\_fits\_filename}\DUrole{p}{:}\DUrole{w}{  }\DUrole{n}{Optional\DUrole{p}{{[}}str\DUrole{p}{{]}}}\DUrole{w}{  }\DUrole{o}{=}\DUrole{w}{  }\DUrole{default_value}{None}}}{{ $\rightarrow$ tuple\DUrole{p}{{[}}float\DUrole{p}{,}\DUrole{w}{  }float\DUrole{p}{{]}}}}
\pysigstopsignatures
\sphinxAtStartPar
Use the target selector window to have the user provide the
information needed to determine the location of the target.
\begin{quote}\begin{description}
\sphinxlineitem{Parameters}\begin{itemize}
\item {} 
\sphinxAtStartPar
\sphinxstyleliteralstrong{\sphinxupquote{current\_fits\_filename}} (\sphinxstyleliteralemphasis{\sphinxupquote{string}}) – The current fits filename which will be used to determine where the
location of the target is.

\item {} 
\sphinxAtStartPar
\sphinxstyleliteralstrong{\sphinxupquote{reference\_fits\_filename}} (\sphinxstyleliteralemphasis{\sphinxupquote{string}}\sphinxstyleliteralemphasis{\sphinxupquote{, }}\sphinxstyleliteralemphasis{\sphinxupquote{default = None}}) – The reference fits filename which will be used to compare against the
current fits filename to determine where the location of the target
is. If None, then no image will be loaded until manually specified.

\end{itemize}

\sphinxlineitem{Returns}
\sphinxAtStartPar
\begin{itemize}
\item {} 
\sphinxAtStartPar
\sphinxstylestrong{target\_x} (\sphinxstyleemphasis{float}) – The location of the target in the x axis direction.

\item {} 
\sphinxAtStartPar
\sphinxstylestrong{target\_y} (\sphinxstyleemphasis{float}) – The location of the target in the y axis direction.

\end{itemize}


\end{description}\end{quote}

\end{fulllineitems}\end{savenotes}

\index{main() (in module opihiexarata.gui.selector)@\spxentry{main()}\spxextra{in module opihiexarata.gui.selector}}

\begin{savenotes}\begin{fulllineitems}
\phantomsection\label{\detokenize{code/opihiexarata.gui.selector:opihiexarata.gui.selector.main}}
\pysigstartsignatures
\pysiglinewithargsret{\sphinxcode{\sphinxupquote{opihiexarata.gui.selector.}}\sphinxbfcode{\sphinxupquote{main}}}{}{}
\pysigstopsignatures
\end{fulllineitems}\end{savenotes}



\subparagraph{Module contents}
\label{\detokenize{code/opihiexarata.gui:module-opihiexarata.gui}}\label{\detokenize{code/opihiexarata.gui:module-contents}}\index{module@\spxentry{module}!opihiexarata.gui@\spxentry{opihiexarata.gui}}\index{opihiexarata.gui@\spxentry{opihiexarata.gui}!module@\spxentry{module}}
\sphinxAtStartPar
Parts of the Exarata GUI.

\sphinxstepscope


\paragraph{opihiexarata.library package}
\label{\detokenize{code/opihiexarata.library:opihiexarata-library-package}}\label{\detokenize{code/opihiexarata.library::doc}}

\subparagraph{Submodules}
\label{\detokenize{code/opihiexarata.library:submodules}}
\sphinxstepscope


\subparagraph{opihiexarata.library.config module}
\label{\detokenize{code/opihiexarata.library.config:module-opihiexarata.library.config}}\label{\detokenize{code/opihiexarata.library.config:opihiexarata-library-config-module}}\label{\detokenize{code/opihiexarata.library.config::doc}}\index{module@\spxentry{module}!opihiexarata.library.config@\spxentry{opihiexarata.library.config}}\index{opihiexarata.library.config@\spxentry{opihiexarata.library.config}!module@\spxentry{module}}
\sphinxAtStartPar
Controls the inputting of configuration files. This also serves to bring all
of the configuration parameters into a more accessable space which other parts
of Exarata can use.

\sphinxAtStartPar
Note these configuration constant parameters are all accessed using capital
letters regardless of the configuration file’s labels. Moreover, there are
constant parameters which are stored here which are not otherwise changeable
by the configuration file.
\index{generate\_configuration\_file\_copy() (in module opihiexarata.library.config)@\spxentry{generate\_configuration\_file\_copy()}\spxextra{in module opihiexarata.library.config}}

\begin{savenotes}\begin{fulllineitems}
\phantomsection\label{\detokenize{code/opihiexarata.library.config:opihiexarata.library.config.generate_configuration_file_copy}}
\pysigstartsignatures
\pysiglinewithargsret{\sphinxcode{\sphinxupquote{opihiexarata.library.config.}}\sphinxbfcode{\sphinxupquote{generate\_configuration\_file\_copy}}}{\emph{\DUrole{n}{filename}\DUrole{p}{:}\DUrole{w}{  }\DUrole{n}{str}}, \emph{\DUrole{n}{overwrite}\DUrole{o}{=}\DUrole{default_value}{False}}}{{ $\rightarrow$ None}}
\pysigstopsignatures
\sphinxAtStartPar
This generates a copy of the default configuration file to the given
location.
\begin{quote}\begin{description}
\sphinxlineitem{Parameters}\begin{itemize}
\item {} 
\sphinxAtStartPar
\sphinxstyleliteralstrong{\sphinxupquote{filename}} (\sphinxstyleliteralemphasis{\sphinxupquote{string}}) – The pathname or filename where the configuration file should be put
to. If it does not have the proper yaml extension, it will be added.

\item {} 
\sphinxAtStartPar
\sphinxstyleliteralstrong{\sphinxupquote{overwrite}} (\sphinxstyleliteralemphasis{\sphinxupquote{bool}}\sphinxstyleliteralemphasis{\sphinxupquote{, }}\sphinxstyleliteralemphasis{\sphinxupquote{default = False}}) – If the file already exists, overwrite it. If False, it would raise
an error instead.

\end{itemize}

\sphinxlineitem{Return type}
\sphinxAtStartPar
None

\end{description}\end{quote}

\end{fulllineitems}\end{savenotes}

\index{generate\_secrets\_file\_copy() (in module opihiexarata.library.config)@\spxentry{generate\_secrets\_file\_copy()}\spxextra{in module opihiexarata.library.config}}

\begin{savenotes}\begin{fulllineitems}
\phantomsection\label{\detokenize{code/opihiexarata.library.config:opihiexarata.library.config.generate_secrets_file_copy}}
\pysigstartsignatures
\pysiglinewithargsret{\sphinxcode{\sphinxupquote{opihiexarata.library.config.}}\sphinxbfcode{\sphinxupquote{generate\_secrets\_file\_copy}}}{\emph{\DUrole{n}{filename}\DUrole{p}{:}\DUrole{w}{  }\DUrole{n}{str}}, \emph{\DUrole{n}{overwrite}\DUrole{o}{=}\DUrole{default_value}{False}}}{{ $\rightarrow$ None}}
\pysigstopsignatures
\sphinxAtStartPar
This generates a copy of the secrets configuration file to the given
location.
\begin{quote}\begin{description}
\sphinxlineitem{Parameters}\begin{itemize}
\item {} 
\sphinxAtStartPar
\sphinxstyleliteralstrong{\sphinxupquote{filename}} (\sphinxstyleliteralemphasis{\sphinxupquote{string}}) – The pathname or filename where the configuration file should be put
to. If it does not have the proper yaml extension, it will be added.

\item {} 
\sphinxAtStartPar
\sphinxstyleliteralstrong{\sphinxupquote{overwrite}} (\sphinxstyleliteralemphasis{\sphinxupquote{bool}}\sphinxstyleliteralemphasis{\sphinxupquote{, }}\sphinxstyleliteralemphasis{\sphinxupquote{default = False}}) – If the file already exists, overwrite it. If False, it would raise
an error instead.

\end{itemize}

\sphinxlineitem{Return type}
\sphinxAtStartPar
None

\end{description}\end{quote}

\end{fulllineitems}\end{savenotes}

\index{load\_configuration\_file() (in module opihiexarata.library.config)@\spxentry{load\_configuration\_file()}\spxextra{in module opihiexarata.library.config}}

\begin{savenotes}\begin{fulllineitems}
\phantomsection\label{\detokenize{code/opihiexarata.library.config:opihiexarata.library.config.load_configuration_file}}
\pysigstartsignatures
\pysiglinewithargsret{\sphinxcode{\sphinxupquote{opihiexarata.library.config.}}\sphinxbfcode{\sphinxupquote{load\_configuration\_file}}}{\emph{\DUrole{n}{filename}\DUrole{p}{:}\DUrole{w}{  }\DUrole{n}{str}}}{{ $\rightarrow$ dict}}
\pysigstopsignatures
\sphinxAtStartPar
Loads a configuration file and outputs a dictionary of parameters.

\sphinxAtStartPar
Note configuration files should be flat, there should be no nested
configuration parameters.
\begin{quote}\begin{description}
\sphinxlineitem{Parameters}
\sphinxAtStartPar
\sphinxstyleliteralstrong{\sphinxupquote{filename}} (\sphinxstyleliteralemphasis{\sphinxupquote{string}}) – The filename of the configuration file, with the extension. Will raise
if the filename is not the correct extension, just as a quick check.

\sphinxlineitem{Returns}
\sphinxAtStartPar
\sphinxstylestrong{configuration\_dict} – The dictionary which contains all of the configuration parameters
within it.

\sphinxlineitem{Return type}
\sphinxAtStartPar
dictionary

\end{description}\end{quote}

\end{fulllineitems}\end{savenotes}

\index{load\_then\_apply\_configuration() (in module opihiexarata.library.config)@\spxentry{load\_then\_apply\_configuration()}\spxextra{in module opihiexarata.library.config}}

\begin{savenotes}\begin{fulllineitems}
\phantomsection\label{\detokenize{code/opihiexarata.library.config:opihiexarata.library.config.load_then_apply_configuration}}
\pysigstartsignatures
\pysiglinewithargsret{\sphinxcode{\sphinxupquote{opihiexarata.library.config.}}\sphinxbfcode{\sphinxupquote{load\_then\_apply\_configuration}}}{\emph{\DUrole{n}{filename}\DUrole{p}{:}\DUrole{w}{  }\DUrole{n}{str}}}{{ $\rightarrow$ None}}
\pysigstopsignatures
\sphinxAtStartPar
Loads a configuration file, then applies it to the entire Exarata system.

\sphinxAtStartPar
Loads a configuration file and overwrites any overlapping
configurations. It writes the configuration to the configuration module
for usage throughout the entire program.

\sphinxAtStartPar
Note configuration files should be flat, there should be no nested
configuration parameters.
\begin{quote}\begin{description}
\sphinxlineitem{Parameters}
\sphinxAtStartPar
\sphinxstyleliteralstrong{\sphinxupquote{filename}} (\sphinxstyleliteralemphasis{\sphinxupquote{string}}) – The filename of the configuration file, with the extension. Will raise
if the filename is not the correct extension, just as a quick check.

\sphinxlineitem{Return type}
\sphinxAtStartPar
None

\end{description}\end{quote}

\end{fulllineitems}\end{savenotes}


\sphinxstepscope


\subparagraph{opihiexarata.library.conversion module}
\label{\detokenize{code/opihiexarata.library.conversion:module-opihiexarata.library.conversion}}\label{\detokenize{code/opihiexarata.library.conversion:opihiexarata-library-conversion-module}}\label{\detokenize{code/opihiexarata.library.conversion::doc}}\index{module@\spxentry{module}!opihiexarata.library.conversion@\spxentry{opihiexarata.library.conversion}}\index{opihiexarata.library.conversion@\spxentry{opihiexarata.library.conversion}!module@\spxentry{module}}
\sphinxAtStartPar
For miscellaneous conversions.
\index{current\_utc\_to\_julian\_day() (in module opihiexarata.library.conversion)@\spxentry{current\_utc\_to\_julian\_day()}\spxextra{in module opihiexarata.library.conversion}}

\begin{savenotes}\begin{fulllineitems}
\phantomsection\label{\detokenize{code/opihiexarata.library.conversion:opihiexarata.library.conversion.current_utc_to_julian_day}}
\pysigstartsignatures
\pysiglinewithargsret{\sphinxcode{\sphinxupquote{opihiexarata.library.conversion.}}\sphinxbfcode{\sphinxupquote{current\_utc\_to\_julian\_day}}}{}{{ $\rightarrow$ float}}
\pysigstopsignatures
\sphinxAtStartPar
Return the current UTC time when this function is called as a Julian day
time.
\begin{quote}\begin{description}
\sphinxlineitem{Parameters}
\sphinxAtStartPar
\sphinxstyleliteralstrong{\sphinxupquote{None}} – 

\sphinxlineitem{Returns}
\sphinxAtStartPar
\sphinxstylestrong{current\_jd} – The current time in Julian date format.

\sphinxlineitem{Return type}
\sphinxAtStartPar
float

\end{description}\end{quote}

\end{fulllineitems}\end{savenotes}

\index{decimal\_day\_to\_julian\_day() (in module opihiexarata.library.conversion)@\spxentry{decimal\_day\_to\_julian\_day()}\spxextra{in module opihiexarata.library.conversion}}

\begin{savenotes}\begin{fulllineitems}
\phantomsection\label{\detokenize{code/opihiexarata.library.conversion:opihiexarata.library.conversion.decimal_day_to_julian_day}}
\pysigstartsignatures
\pysiglinewithargsret{\sphinxcode{\sphinxupquote{opihiexarata.library.conversion.}}\sphinxbfcode{\sphinxupquote{decimal\_day\_to\_julian\_day}}}{\emph{\DUrole{n}{year}\DUrole{p}{:}\DUrole{w}{  }\DUrole{n}{int}}, \emph{\DUrole{n}{month}\DUrole{p}{:}\DUrole{w}{  }\DUrole{n}{int}}, \emph{\DUrole{n}{day}\DUrole{p}{:}\DUrole{w}{  }\DUrole{n}{float}}}{}
\pysigstopsignatures
\sphinxAtStartPar
A function to convert decimal day time formats to the Julian day.
\begin{quote}\begin{description}
\sphinxlineitem{Parameters}\begin{itemize}
\item {} 
\sphinxAtStartPar
\sphinxstyleliteralstrong{\sphinxupquote{year}} (\sphinxstyleliteralemphasis{\sphinxupquote{int}}) – The year of the time stamp to be converted.

\item {} 
\sphinxAtStartPar
\sphinxstyleliteralstrong{\sphinxupquote{month}} (\sphinxstyleliteralemphasis{\sphinxupquote{int}}) – The month of the time stamp to be converted.

\item {} 
\sphinxAtStartPar
\sphinxstyleliteralstrong{\sphinxupquote{day}} (\sphinxstyleliteralemphasis{\sphinxupquote{float}}) – The day of the time stamp to be converted, may be decimal, the
values are just passed down.

\end{itemize}

\sphinxlineitem{Returns}
\sphinxAtStartPar
\sphinxstylestrong{julian\_day} – The Julian day of the date provided.

\sphinxlineitem{Return type}
\sphinxAtStartPar
float

\end{description}\end{quote}

\end{fulllineitems}\end{savenotes}

\index{degrees\_to\_sexagesimal\_ra\_dec() (in module opihiexarata.library.conversion)@\spxentry{degrees\_to\_sexagesimal\_ra\_dec()}\spxextra{in module opihiexarata.library.conversion}}

\begin{savenotes}\begin{fulllineitems}
\phantomsection\label{\detokenize{code/opihiexarata.library.conversion:opihiexarata.library.conversion.degrees_to_sexagesimal_ra_dec}}
\pysigstartsignatures
\pysiglinewithargsret{\sphinxcode{\sphinxupquote{opihiexarata.library.conversion.}}\sphinxbfcode{\sphinxupquote{degrees\_to\_sexagesimal\_ra\_dec}}}{\emph{\DUrole{n}{ra\_deg}\DUrole{p}{:}\DUrole{w}{  }\DUrole{n}{float}}, \emph{\DUrole{n}{dec\_deg}\DUrole{p}{:}\DUrole{w}{  }\DUrole{n}{float}}, \emph{\DUrole{n}{precision}\DUrole{p}{:}\DUrole{w}{  }\DUrole{n}{int}\DUrole{w}{  }\DUrole{o}{=}\DUrole{w}{  }\DUrole{default_value}{2}}}{{ $\rightarrow$ tuple\DUrole{p}{{[}}str\DUrole{p}{,}\DUrole{w}{  }str\DUrole{p}{{]}}}}
\pysigstopsignatures
\sphinxAtStartPar
Convert RA and DEC degree measurements to the more familiar HMSDMS
sexagesimal format.
\begin{quote}\begin{description}
\sphinxlineitem{Parameters}\begin{itemize}
\item {} 
\sphinxAtStartPar
\sphinxstyleliteralstrong{\sphinxupquote{ra\_deg}} (\sphinxstyleliteralemphasis{\sphinxupquote{float}}) – The right ascension in degrees.

\item {} 
\sphinxAtStartPar
\sphinxstyleliteralstrong{\sphinxupquote{dec\_deg}} (\sphinxstyleliteralemphasis{\sphinxupquote{float}}) – The declination in degrees.

\item {} 
\sphinxAtStartPar
\sphinxstyleliteralstrong{\sphinxupquote{precision}} (\sphinxstyleliteralemphasis{\sphinxupquote{int}}) – The precision of the conversion’s seconds term, i.e. how many numbers
are used.

\end{itemize}

\sphinxlineitem{Returns}
\sphinxAtStartPar
\begin{itemize}
\item {} 
\sphinxAtStartPar
\sphinxstylestrong{ra\_sex} (\sphinxstyleemphasis{str}) – The right ascension in hour:minute:second sexagesimal.

\item {} 
\sphinxAtStartPar
\sphinxstylestrong{dec\_sex} (\sphinxstyleemphasis{str}) – The declination in degree:minute:second sexagesimal.

\end{itemize}


\end{description}\end{quote}

\end{fulllineitems}\end{savenotes}

\index{full\_date\_to\_julian\_day() (in module opihiexarata.library.conversion)@\spxentry{full\_date\_to\_julian\_day()}\spxextra{in module opihiexarata.library.conversion}}

\begin{savenotes}\begin{fulllineitems}
\phantomsection\label{\detokenize{code/opihiexarata.library.conversion:opihiexarata.library.conversion.full_date_to_julian_day}}
\pysigstartsignatures
\pysiglinewithargsret{\sphinxcode{\sphinxupquote{opihiexarata.library.conversion.}}\sphinxbfcode{\sphinxupquote{full\_date\_to\_julian\_day}}}{\emph{\DUrole{n}{year}\DUrole{p}{:}\DUrole{w}{  }\DUrole{n}{int}}, \emph{\DUrole{n}{month}\DUrole{p}{:}\DUrole{w}{  }\DUrole{n}{int}}, \emph{\DUrole{n}{day}\DUrole{p}{:}\DUrole{w}{  }\DUrole{n}{int}}, \emph{\DUrole{n}{hour}\DUrole{p}{:}\DUrole{w}{  }\DUrole{n}{int}}, \emph{\DUrole{n}{minute}\DUrole{p}{:}\DUrole{w}{  }\DUrole{n}{int}}, \emph{\DUrole{n}{second}\DUrole{p}{:}\DUrole{w}{  }\DUrole{n}{float}}}{{ $\rightarrow$ float}}
\pysigstopsignatures
\sphinxAtStartPar
A function to convert the a whole date format into the Julian day time.
\begin{quote}\begin{description}
\sphinxlineitem{Parameters}\begin{itemize}
\item {} 
\sphinxAtStartPar
\sphinxstyleliteralstrong{\sphinxupquote{year}} (\sphinxstyleliteralemphasis{\sphinxupquote{int}}) – The year of the time stamp to be converted.

\item {} 
\sphinxAtStartPar
\sphinxstyleliteralstrong{\sphinxupquote{month}} (\sphinxstyleliteralemphasis{\sphinxupquote{int}}) – The month of the time stamp to be converted.

\item {} 
\sphinxAtStartPar
\sphinxstyleliteralstrong{\sphinxupquote{day}} (\sphinxstyleliteralemphasis{\sphinxupquote{int}}) – The day of the time stamp to be converted.

\item {} 
\sphinxAtStartPar
\sphinxstyleliteralstrong{\sphinxupquote{hour}} (\sphinxstyleliteralemphasis{\sphinxupquote{int}}) – The hour of the time stamp to be converted.

\item {} 
\sphinxAtStartPar
\sphinxstyleliteralstrong{\sphinxupquote{minute}} (\sphinxstyleliteralemphasis{\sphinxupquote{int}}) – The minute of the time stamp to be converted.

\item {} 
\sphinxAtStartPar
\sphinxstyleliteralstrong{\sphinxupquote{second}} (\sphinxstyleliteralemphasis{\sphinxupquote{float}}) – The second of the time stamp to be converted.

\end{itemize}

\sphinxlineitem{Returns}
\sphinxAtStartPar
\sphinxstylestrong{julian\_day} – The time input converted into the Julian day.

\sphinxlineitem{Return type}
\sphinxAtStartPar
float

\end{description}\end{quote}

\end{fulllineitems}\end{savenotes}

\index{julian\_day\_to\_decimal\_day() (in module opihiexarata.library.conversion)@\spxentry{julian\_day\_to\_decimal\_day()}\spxextra{in module opihiexarata.library.conversion}}

\begin{savenotes}\begin{fulllineitems}
\phantomsection\label{\detokenize{code/opihiexarata.library.conversion:opihiexarata.library.conversion.julian_day_to_decimal_day}}
\pysigstartsignatures
\pysiglinewithargsret{\sphinxcode{\sphinxupquote{opihiexarata.library.conversion.}}\sphinxbfcode{\sphinxupquote{julian\_day\_to\_decimal\_day}}}{\emph{\DUrole{n}{jd}\DUrole{p}{:}\DUrole{w}{  }\DUrole{n}{float}}}{{ $\rightarrow$ tuple}}
\pysigstopsignatures
\sphinxAtStartPar
A function to convert the Julian day time to the decimal day time.
\begin{quote}\begin{description}
\sphinxlineitem{Parameters}
\sphinxAtStartPar
\sphinxstyleliteralstrong{\sphinxupquote{jd}} (\sphinxstyleliteralemphasis{\sphinxupquote{float}}) – The Julian day time that is going to be converted.

\sphinxlineitem{Returns}
\sphinxAtStartPar
\begin{itemize}
\item {} 
\sphinxAtStartPar
\sphinxstylestrong{year} (\sphinxstyleemphasis{int}) – The year of the Julian day time.

\item {} 
\sphinxAtStartPar
\sphinxstylestrong{month} (\sphinxstyleemphasis{int}) – The month of the Julian day time.

\item {} 
\sphinxAtStartPar
\sphinxstylestrong{day} (\sphinxstyleemphasis{float}) – The day of the the Julian day time, the hours, minute, and seconds are all
contained as a decimal.

\end{itemize}


\end{description}\end{quote}

\end{fulllineitems}\end{savenotes}

\index{julian\_day\_to\_full\_date() (in module opihiexarata.library.conversion)@\spxentry{julian\_day\_to\_full\_date()}\spxextra{in module opihiexarata.library.conversion}}

\begin{savenotes}\begin{fulllineitems}
\phantomsection\label{\detokenize{code/opihiexarata.library.conversion:opihiexarata.library.conversion.julian_day_to_full_date}}
\pysigstartsignatures
\pysiglinewithargsret{\sphinxcode{\sphinxupquote{opihiexarata.library.conversion.}}\sphinxbfcode{\sphinxupquote{julian\_day\_to\_full\_date}}}{\emph{\DUrole{n}{jd}\DUrole{p}{:}\DUrole{w}{  }\DUrole{n}{float}}}{{ $\rightarrow$ tuple}}
\pysigstopsignatures
\sphinxAtStartPar
A function to convert the Julian day to a full date time.
\begin{quote}\begin{description}
\sphinxlineitem{Parameters}
\sphinxAtStartPar
\sphinxstyleliteralstrong{\sphinxupquote{jd}} (\sphinxstyleliteralemphasis{\sphinxupquote{float}}) – The Julian day time that is going to be converted.

\sphinxlineitem{Returns}
\sphinxAtStartPar
\begin{itemize}
\item {} 
\sphinxAtStartPar
\sphinxstylestrong{year} (\sphinxstyleemphasis{int}) – The year of the Julian day provided.

\item {} 
\sphinxAtStartPar
\sphinxstylestrong{month} (\sphinxstyleemphasis{int}) – The month of the Julian day provided.

\item {} 
\sphinxAtStartPar
\sphinxstylestrong{day} (\sphinxstyleemphasis{int}) – The day of the Julian day provided.

\item {} 
\sphinxAtStartPar
\sphinxstylestrong{hour} (\sphinxstyleemphasis{int}) – The hour of the Julian day provided.

\item {} 
\sphinxAtStartPar
\sphinxstylestrong{minute} (\sphinxstyleemphasis{int}) – The minute of the Julian day provided.

\item {} 
\sphinxAtStartPar
\sphinxstylestrong{second} (\sphinxstyleemphasis{float}) – The second of the Julian day provided.

\end{itemize}


\end{description}\end{quote}

\end{fulllineitems}\end{savenotes}

\index{julian\_day\_to\_modified\_julian\_day() (in module opihiexarata.library.conversion)@\spxentry{julian\_day\_to\_modified\_julian\_day()}\spxextra{in module opihiexarata.library.conversion}}

\begin{savenotes}\begin{fulllineitems}
\phantomsection\label{\detokenize{code/opihiexarata.library.conversion:opihiexarata.library.conversion.julian_day_to_modified_julian_day}}
\pysigstartsignatures
\pysiglinewithargsret{\sphinxcode{\sphinxupquote{opihiexarata.library.conversion.}}\sphinxbfcode{\sphinxupquote{julian\_day\_to\_modified\_julian\_day}}}{\emph{\DUrole{n}{jd}\DUrole{p}{:}\DUrole{w}{  }\DUrole{n}{float}}}{{ $\rightarrow$ float}}
\pysigstopsignatures
\sphinxAtStartPar
A function to convert Julian days to  modified Julian days.
\begin{quote}\begin{description}
\sphinxlineitem{Parameters}
\sphinxAtStartPar
\sphinxstyleliteralstrong{\sphinxupquote{jd}} (\sphinxstyleliteralemphasis{\sphinxupquote{float}}) – The Julian day to be converted to a modified Julian day.

\sphinxlineitem{Returns}
\sphinxAtStartPar
\sphinxstylestrong{mjd} – The modified Julian day value after conversion.

\sphinxlineitem{Return type}
\sphinxAtStartPar
float

\end{description}\end{quote}

\end{fulllineitems}\end{savenotes}

\index{julian\_day\_to\_unix\_time() (in module opihiexarata.library.conversion)@\spxentry{julian\_day\_to\_unix\_time()}\spxextra{in module opihiexarata.library.conversion}}

\begin{savenotes}\begin{fulllineitems}
\phantomsection\label{\detokenize{code/opihiexarata.library.conversion:opihiexarata.library.conversion.julian_day_to_unix_time}}
\pysigstartsignatures
\pysiglinewithargsret{\sphinxcode{\sphinxupquote{opihiexarata.library.conversion.}}\sphinxbfcode{\sphinxupquote{julian\_day\_to\_unix\_time}}}{\emph{\DUrole{n}{jd}\DUrole{p}{:}\DUrole{w}{  }\DUrole{n}{float}}}{{ $\rightarrow$ float}}
\pysigstopsignatures
\sphinxAtStartPar
A function to convert between Julian days to UNIX time.
\begin{quote}\begin{description}
\sphinxlineitem{Parameters}
\sphinxAtStartPar
\sphinxstyleliteralstrong{\sphinxupquote{jd}} (\sphinxstyleliteralemphasis{\sphinxupquote{float}}) – The Julian day to be converted.

\sphinxlineitem{Returns}
\sphinxAtStartPar
\sphinxstylestrong{unix\_time} – The time converted to UNIX time.

\sphinxlineitem{Return type}
\sphinxAtStartPar
float

\end{description}\end{quote}

\end{fulllineitems}\end{savenotes}

\index{modified\_julian\_day\_to\_julian\_day() (in module opihiexarata.library.conversion)@\spxentry{modified\_julian\_day\_to\_julian\_day()}\spxextra{in module opihiexarata.library.conversion}}

\begin{savenotes}\begin{fulllineitems}
\phantomsection\label{\detokenize{code/opihiexarata.library.conversion:opihiexarata.library.conversion.modified_julian_day_to_julian_day}}
\pysigstartsignatures
\pysiglinewithargsret{\sphinxcode{\sphinxupquote{opihiexarata.library.conversion.}}\sphinxbfcode{\sphinxupquote{modified\_julian\_day\_to\_julian\_day}}}{\emph{\DUrole{n}{mjd}\DUrole{p}{:}\DUrole{w}{  }\DUrole{n}{float}}}{{ $\rightarrow$ float}}
\pysigstopsignatures
\sphinxAtStartPar
A function to convert modified Julian days to Julian days.
\begin{quote}\begin{description}
\sphinxlineitem{Parameters}
\sphinxAtStartPar
\sphinxstyleliteralstrong{\sphinxupquote{mjd}} (\sphinxstyleliteralemphasis{\sphinxupquote{float}}) – The modified Julian day to be converted to a Julian day.

\sphinxlineitem{Returns}
\sphinxAtStartPar
\sphinxstylestrong{jd} – The Julian day value after conversion.

\sphinxlineitem{Return type}
\sphinxAtStartPar
float

\end{description}\end{quote}

\end{fulllineitems}\end{savenotes}

\index{sexagesimal\_ra\_dec\_to\_degrees() (in module opihiexarata.library.conversion)@\spxentry{sexagesimal\_ra\_dec\_to\_degrees()}\spxextra{in module opihiexarata.library.conversion}}

\begin{savenotes}\begin{fulllineitems}
\phantomsection\label{\detokenize{code/opihiexarata.library.conversion:opihiexarata.library.conversion.sexagesimal_ra_dec_to_degrees}}
\pysigstartsignatures
\pysiglinewithargsret{\sphinxcode{\sphinxupquote{opihiexarata.library.conversion.}}\sphinxbfcode{\sphinxupquote{sexagesimal\_ra\_dec\_to\_degrees}}}{\emph{\DUrole{n}{ra\_sex}\DUrole{p}{:}\DUrole{w}{  }\DUrole{n}{str}}, \emph{\DUrole{n}{dec\_sex}\DUrole{p}{:}\DUrole{w}{  }\DUrole{n}{str}}}{{ $\rightarrow$ tuple\DUrole{p}{{[}}float\DUrole{p}{,}\DUrole{w}{  }float\DUrole{p}{{]}}}}
\pysigstopsignatures
\sphinxAtStartPar
Convert RA and DEC measurements from the more familiar HMSDMS
sexagesimal format to degrees.
\begin{quote}\begin{description}
\sphinxlineitem{Parameters}\begin{itemize}
\item {} 
\sphinxAtStartPar
\sphinxstyleliteralstrong{\sphinxupquote{ra\_sex}} (\sphinxstyleliteralemphasis{\sphinxupquote{str}}) – The right ascension in hour:minute:second sexagesimal.

\item {} 
\sphinxAtStartPar
\sphinxstyleliteralstrong{\sphinxupquote{dec\_sex}} (\sphinxstyleliteralemphasis{\sphinxupquote{str}}) – The declination in degree:minute:second sexagesimal.

\end{itemize}

\sphinxlineitem{Returns}
\sphinxAtStartPar
\begin{itemize}
\item {} 
\sphinxAtStartPar
\sphinxstylestrong{ra\_deg} (\sphinxstyleemphasis{float}) – The right ascension in degrees.

\item {} 
\sphinxAtStartPar
\sphinxstylestrong{dec\_deg} (\sphinxstyleemphasis{float}) – The declination in degrees.

\end{itemize}


\end{description}\end{quote}

\end{fulllineitems}\end{savenotes}

\index{string\_month\_to\_number() (in module opihiexarata.library.conversion)@\spxentry{string\_month\_to\_number()}\spxextra{in module opihiexarata.library.conversion}}

\begin{savenotes}\begin{fulllineitems}
\phantomsection\label{\detokenize{code/opihiexarata.library.conversion:opihiexarata.library.conversion.string_month_to_number}}
\pysigstartsignatures
\pysiglinewithargsret{\sphinxcode{\sphinxupquote{opihiexarata.library.conversion.}}\sphinxbfcode{\sphinxupquote{string\_month\_to\_number}}}{\emph{\DUrole{n}{month\_str}\DUrole{p}{:}\DUrole{w}{  }\DUrole{n}{str}}}{{ $\rightarrow$ int}}
\pysigstopsignatures
\sphinxAtStartPar
A function to convert from the name of a month to the month number.
This is just because it is easy to have here and to add a package import
for something like this is silly.
\begin{quote}\begin{description}
\sphinxlineitem{Parameters}
\sphinxAtStartPar
\sphinxstyleliteralstrong{\sphinxupquote{month\_str}} (\sphinxstyleliteralemphasis{\sphinxupquote{string}}) – The month name.

\sphinxlineitem{Returns}
\sphinxAtStartPar
\sphinxstylestrong{month\_int} – The month number integer.

\sphinxlineitem{Return type}
\sphinxAtStartPar
int

\end{description}\end{quote}

\end{fulllineitems}\end{savenotes}

\index{unix\_time\_to\_julian\_day() (in module opihiexarata.library.conversion)@\spxentry{unix\_time\_to\_julian\_day()}\spxextra{in module opihiexarata.library.conversion}}

\begin{savenotes}\begin{fulllineitems}
\phantomsection\label{\detokenize{code/opihiexarata.library.conversion:opihiexarata.library.conversion.unix_time_to_julian_day}}
\pysigstartsignatures
\pysiglinewithargsret{\sphinxcode{\sphinxupquote{opihiexarata.library.conversion.}}\sphinxbfcode{\sphinxupquote{unix\_time\_to\_julian\_day}}}{\emph{\DUrole{n}{unix\_time}\DUrole{p}{:}\DUrole{w}{  }\DUrole{n}{float}}}{{ $\rightarrow$ float}}
\pysigstopsignatures
\sphinxAtStartPar
A function to convert between julian days to Unix time.
\begin{quote}\begin{description}
\sphinxlineitem{Parameters}
\sphinxAtStartPar
\sphinxstyleliteralstrong{\sphinxupquote{unix\_time}} (\sphinxstyleliteralemphasis{\sphinxupquote{float}}) – The UNIX time to be converted to a Julian day.

\sphinxlineitem{Returns}
\sphinxAtStartPar
\sphinxstylestrong{jd} – The Julian day value as converted.

\sphinxlineitem{Return type}
\sphinxAtStartPar
float

\end{description}\end{quote}

\end{fulllineitems}\end{savenotes}


\sphinxstepscope


\subparagraph{opihiexarata.library.engine module}
\label{\detokenize{code/opihiexarata.library.engine:module-opihiexarata.library.engine}}\label{\detokenize{code/opihiexarata.library.engine:opihiexarata-library-engine-module}}\label{\detokenize{code/opihiexarata.library.engine::doc}}\index{module@\spxentry{module}!opihiexarata.library.engine@\spxentry{opihiexarata.library.engine}}\index{opihiexarata.library.engine@\spxentry{opihiexarata.library.engine}!module@\spxentry{module}}
\sphinxAtStartPar
Where the base classes of the solvers and engines lie.
\index{AstrometryEngine (class in opihiexarata.library.engine)@\spxentry{AstrometryEngine}\spxextra{class in opihiexarata.library.engine}}

\begin{savenotes}\begin{fulllineitems}
\phantomsection\label{\detokenize{code/opihiexarata.library.engine:opihiexarata.library.engine.AstrometryEngine}}
\pysigstartsignatures
\pysigline{\sphinxbfcode{\sphinxupquote{class\DUrole{w}{  }}}\sphinxcode{\sphinxupquote{opihiexarata.library.engine.}}\sphinxbfcode{\sphinxupquote{AstrometryEngine}}}
\pysigstopsignatures
\sphinxAtStartPar
Bases: {\hyperref[\detokenize{code/opihiexarata.library.engine:opihiexarata.library.engine.ExarataEngine}]{\sphinxcrossref{\sphinxcode{\sphinxupquote{ExarataEngine}}}}}

\sphinxAtStartPar
The base class where the Astrometry engines are derived from. Should
not be used other than for type hinting and subclassing.

\end{fulllineitems}\end{savenotes}

\index{EphemerisEngine (class in opihiexarata.library.engine)@\spxentry{EphemerisEngine}\spxextra{class in opihiexarata.library.engine}}

\begin{savenotes}\begin{fulllineitems}
\phantomsection\label{\detokenize{code/opihiexarata.library.engine:opihiexarata.library.engine.EphemerisEngine}}
\pysigstartsignatures
\pysigline{\sphinxbfcode{\sphinxupquote{class\DUrole{w}{  }}}\sphinxcode{\sphinxupquote{opihiexarata.library.engine.}}\sphinxbfcode{\sphinxupquote{EphemerisEngine}}}
\pysigstopsignatures
\sphinxAtStartPar
Bases: {\hyperref[\detokenize{code/opihiexarata.library.engine:opihiexarata.library.engine.ExarataEngine}]{\sphinxcrossref{\sphinxcode{\sphinxupquote{ExarataEngine}}}}}

\sphinxAtStartPar
The base class for all of the Ephemeris determination engines. Should not
be used other than for type hinting and subclassing.

\end{fulllineitems}\end{savenotes}

\index{ExarataEngine (class in opihiexarata.library.engine)@\spxentry{ExarataEngine}\spxextra{class in opihiexarata.library.engine}}

\begin{savenotes}\begin{fulllineitems}
\phantomsection\label{\detokenize{code/opihiexarata.library.engine:opihiexarata.library.engine.ExarataEngine}}
\pysigstartsignatures
\pysigline{\sphinxbfcode{\sphinxupquote{class\DUrole{w}{  }}}\sphinxcode{\sphinxupquote{opihiexarata.library.engine.}}\sphinxbfcode{\sphinxupquote{ExarataEngine}}}
\pysigstopsignatures
\sphinxAtStartPar
Bases: \sphinxcode{\sphinxupquote{object}}

\sphinxAtStartPar
The base class for all of the engines which power the interfaces for
the OpihiExarata system. This should not used for anything other than
type hinting and subclassing.

\end{fulllineitems}\end{savenotes}

\index{ExarataSolution (class in opihiexarata.library.engine)@\spxentry{ExarataSolution}\spxextra{class in opihiexarata.library.engine}}

\begin{savenotes}\begin{fulllineitems}
\phantomsection\label{\detokenize{code/opihiexarata.library.engine:opihiexarata.library.engine.ExarataSolution}}
\pysigstartsignatures
\pysigline{\sphinxbfcode{\sphinxupquote{class\DUrole{w}{  }}}\sphinxcode{\sphinxupquote{opihiexarata.library.engine.}}\sphinxbfcode{\sphinxupquote{ExarataSolution}}}
\pysigstopsignatures
\sphinxAtStartPar
Bases: \sphinxcode{\sphinxupquote{object}}

\sphinxAtStartPar
The base class for all of the solution classes which use engines to
solve particular problems.

\end{fulllineitems}\end{savenotes}

\index{OrbitEngine (class in opihiexarata.library.engine)@\spxentry{OrbitEngine}\spxextra{class in opihiexarata.library.engine}}

\begin{savenotes}\begin{fulllineitems}
\phantomsection\label{\detokenize{code/opihiexarata.library.engine:opihiexarata.library.engine.OrbitEngine}}
\pysigstartsignatures
\pysigline{\sphinxbfcode{\sphinxupquote{class\DUrole{w}{  }}}\sphinxcode{\sphinxupquote{opihiexarata.library.engine.}}\sphinxbfcode{\sphinxupquote{OrbitEngine}}}
\pysigstopsignatures
\sphinxAtStartPar
Bases: {\hyperref[\detokenize{code/opihiexarata.library.engine:opihiexarata.library.engine.ExarataEngine}]{\sphinxcrossref{\sphinxcode{\sphinxupquote{ExarataEngine}}}}}

\sphinxAtStartPar
The base class for all of the Orbit determination engines. Should not
be used other than for type hinting and subclassing.

\end{fulllineitems}\end{savenotes}

\index{PhotometryEngine (class in opihiexarata.library.engine)@\spxentry{PhotometryEngine}\spxextra{class in opihiexarata.library.engine}}

\begin{savenotes}\begin{fulllineitems}
\phantomsection\label{\detokenize{code/opihiexarata.library.engine:opihiexarata.library.engine.PhotometryEngine}}
\pysigstartsignatures
\pysigline{\sphinxbfcode{\sphinxupquote{class\DUrole{w}{  }}}\sphinxcode{\sphinxupquote{opihiexarata.library.engine.}}\sphinxbfcode{\sphinxupquote{PhotometryEngine}}}
\pysigstopsignatures
\sphinxAtStartPar
Bases: {\hyperref[\detokenize{code/opihiexarata.library.engine:opihiexarata.library.engine.ExarataEngine}]{\sphinxcrossref{\sphinxcode{\sphinxupquote{ExarataEngine}}}}}

\sphinxAtStartPar
The base class where the Photometry engines are derived from. Should
not be used other than for type hinting and subclassing.

\end{fulllineitems}\end{savenotes}

\index{PropagationEngine (class in opihiexarata.library.engine)@\spxentry{PropagationEngine}\spxextra{class in opihiexarata.library.engine}}

\begin{savenotes}\begin{fulllineitems}
\phantomsection\label{\detokenize{code/opihiexarata.library.engine:opihiexarata.library.engine.PropagationEngine}}
\pysigstartsignatures
\pysigline{\sphinxbfcode{\sphinxupquote{class\DUrole{w}{  }}}\sphinxcode{\sphinxupquote{opihiexarata.library.engine.}}\sphinxbfcode{\sphinxupquote{PropagationEngine}}}
\pysigstopsignatures
\sphinxAtStartPar
Bases: {\hyperref[\detokenize{code/opihiexarata.library.engine:opihiexarata.library.engine.ExarataEngine}]{\sphinxcrossref{\sphinxcode{\sphinxupquote{ExarataEngine}}}}}

\sphinxAtStartPar
The base class where the Propagation engines are derived from. Should
not be used other than for type hinting and subclassing.

\end{fulllineitems}\end{savenotes}


\sphinxstepscope


\subparagraph{opihiexarata.library.error module}
\label{\detokenize{code/opihiexarata.library.error:module-opihiexarata.library.error}}\label{\detokenize{code/opihiexarata.library.error:opihiexarata-library-error-module}}\label{\detokenize{code/opihiexarata.library.error::doc}}\index{module@\spxentry{module}!opihiexarata.library.error@\spxentry{opihiexarata.library.error}}\index{opihiexarata.library.error@\spxentry{opihiexarata.library.error}!module@\spxentry{module}}
\sphinxAtStartPar
Error, warning, and logging functionality pertinent to the function of Exarata.
\index{CommandLineError@\spxentry{CommandLineError}}

\begin{savenotes}\begin{fulllineitems}
\phantomsection\label{\detokenize{code/opihiexarata.library.error:opihiexarata.library.error.CommandLineError}}
\pysigstartsignatures
\pysiglinewithargsret{\sphinxbfcode{\sphinxupquote{exception\DUrole{w}{  }}}\sphinxcode{\sphinxupquote{opihiexarata.library.error.}}\sphinxbfcode{\sphinxupquote{CommandLineError}}}{\emph{\DUrole{n}{message}\DUrole{p}{:}\DUrole{w}{  }\DUrole{n}{Optional\DUrole{p}{{[}}str\DUrole{p}{{]}}}\DUrole{w}{  }\DUrole{o}{=}\DUrole{w}{  }\DUrole{default_value}{None}}}{}
\pysigstopsignatures
\sphinxAtStartPar
Bases: {\hyperref[\detokenize{code/opihiexarata.library.error:opihiexarata.library.error.ExarataException}]{\sphinxcrossref{\sphinxcode{\sphinxupquote{ExarataException}}}}}

\sphinxAtStartPar
An error to be used where the parameters or arguments entered in the
command line were not correct.

\end{fulllineitems}\end{savenotes}

\index{ConfigurationError@\spxentry{ConfigurationError}}

\begin{savenotes}\begin{fulllineitems}
\phantomsection\label{\detokenize{code/opihiexarata.library.error:opihiexarata.library.error.ConfigurationError}}
\pysigstartsignatures
\pysiglinewithargsret{\sphinxbfcode{\sphinxupquote{exception\DUrole{w}{  }}}\sphinxcode{\sphinxupquote{opihiexarata.library.error.}}\sphinxbfcode{\sphinxupquote{ConfigurationError}}}{\emph{\DUrole{n}{message}\DUrole{p}{:}\DUrole{w}{  }\DUrole{n}{Optional\DUrole{p}{{[}}str\DUrole{p}{{]}}}\DUrole{w}{  }\DUrole{o}{=}\DUrole{w}{  }\DUrole{default_value}{None}}}{}
\pysigstopsignatures
\sphinxAtStartPar
Bases: {\hyperref[\detokenize{code/opihiexarata.library.error:opihiexarata.library.error.ExarataException}]{\sphinxcrossref{\sphinxcode{\sphinxupquote{ExarataException}}}}}

\sphinxAtStartPar
An error to be used where the expectation of how configuration files
and configuration parameters are structures are violated.

\end{fulllineitems}\end{savenotes}

\index{DevelopmentError@\spxentry{DevelopmentError}}

\begin{savenotes}\begin{fulllineitems}
\phantomsection\label{\detokenize{code/opihiexarata.library.error:opihiexarata.library.error.DevelopmentError}}
\pysigstartsignatures
\pysiglinewithargsret{\sphinxbfcode{\sphinxupquote{exception\DUrole{w}{  }}}\sphinxcode{\sphinxupquote{opihiexarata.library.error.}}\sphinxbfcode{\sphinxupquote{DevelopmentError}}}{\emph{\DUrole{n}{message}\DUrole{p}{:}\DUrole{w}{  }\DUrole{n}{Optional\DUrole{p}{{[}}str\DUrole{p}{{]}}}\DUrole{w}{  }\DUrole{o}{=}\DUrole{w}{  }\DUrole{default_value}{None}}}{}
\pysigstopsignatures
\sphinxAtStartPar
Bases: {\hyperref[\detokenize{code/opihiexarata.library.error:opihiexarata.library.error.ExarataBaseException}]{\sphinxcrossref{\sphinxcode{\sphinxupquote{ExarataBaseException}}}}}

\sphinxAtStartPar
This is an error where the development of OpihiExarata is not correct and
something is not coded based on the expectations of the software itself.
This is not the fault of the user.

\end{fulllineitems}\end{savenotes}

\index{DirectoryError@\spxentry{DirectoryError}}

\begin{savenotes}\begin{fulllineitems}
\phantomsection\label{\detokenize{code/opihiexarata.library.error:opihiexarata.library.error.DirectoryError}}
\pysigstartsignatures
\pysiglinewithargsret{\sphinxbfcode{\sphinxupquote{exception\DUrole{w}{  }}}\sphinxcode{\sphinxupquote{opihiexarata.library.error.}}\sphinxbfcode{\sphinxupquote{DirectoryError}}}{\emph{\DUrole{n}{message}\DUrole{p}{:}\DUrole{w}{  }\DUrole{n}{Optional\DUrole{p}{{[}}str\DUrole{p}{{]}}}\DUrole{w}{  }\DUrole{o}{=}\DUrole{w}{  }\DUrole{default_value}{None}}}{}
\pysigstopsignatures
\sphinxAtStartPar
Bases: {\hyperref[\detokenize{code/opihiexarata.library.error:opihiexarata.library.error.ExarataException}]{\sphinxcrossref{\sphinxcode{\sphinxupquote{ExarataException}}}}}

\sphinxAtStartPar
An error to be used when there are issues specifically with directories
and not just files.

\end{fulllineitems}\end{savenotes}

\index{EngineError@\spxentry{EngineError}}

\begin{savenotes}\begin{fulllineitems}
\phantomsection\label{\detokenize{code/opihiexarata.library.error:opihiexarata.library.error.EngineError}}
\pysigstartsignatures
\pysiglinewithargsret{\sphinxbfcode{\sphinxupquote{exception\DUrole{w}{  }}}\sphinxcode{\sphinxupquote{opihiexarata.library.error.}}\sphinxbfcode{\sphinxupquote{EngineError}}}{\emph{\DUrole{n}{message}\DUrole{p}{:}\DUrole{w}{  }\DUrole{n}{Optional\DUrole{p}{{[}}str\DUrole{p}{{]}}}\DUrole{w}{  }\DUrole{o}{=}\DUrole{w}{  }\DUrole{default_value}{None}}}{}
\pysigstopsignatures
\sphinxAtStartPar
Bases: {\hyperref[\detokenize{code/opihiexarata.library.error:opihiexarata.library.error.ExarataException}]{\sphinxcrossref{\sphinxcode{\sphinxupquote{ExarataException}}}}}

\sphinxAtStartPar
This error is for when the astrometric, photometric, or asteroid\sphinxhyphen{}metric
solver engines provided are not valid or expected. This error can also be
used when an engine fails to solve or otherwise does not work as
intended.

\end{fulllineitems}\end{savenotes}

\index{ExarataBaseException@\spxentry{ExarataBaseException}}

\begin{savenotes}\begin{fulllineitems}
\phantomsection\label{\detokenize{code/opihiexarata.library.error:opihiexarata.library.error.ExarataBaseException}}
\pysigstartsignatures
\pysiglinewithargsret{\sphinxbfcode{\sphinxupquote{exception\DUrole{w}{  }}}\sphinxcode{\sphinxupquote{opihiexarata.library.error.}}\sphinxbfcode{\sphinxupquote{ExarataBaseException}}}{\emph{\DUrole{n}{message}\DUrole{p}{:}\DUrole{w}{  }\DUrole{n}{Optional\DUrole{p}{{[}}str\DUrole{p}{{]}}}\DUrole{w}{  }\DUrole{o}{=}\DUrole{w}{  }\DUrole{default_value}{None}}}{}
\pysigstopsignatures
\sphinxAtStartPar
Bases: \sphinxcode{\sphinxupquote{BaseException}}

\sphinxAtStartPar
The base exception class. This is for exceptions that should never be
caught and should bring everything to a halt.
\index{\_\_init\_\_() (opihiexarata.library.error.ExarataBaseException method)@\spxentry{\_\_init\_\_()}\spxextra{opihiexarata.library.error.ExarataBaseException method}}

\begin{savenotes}\begin{fulllineitems}
\phantomsection\label{\detokenize{code/opihiexarata.library.error:opihiexarata.library.error.ExarataBaseException.__init__}}
\pysigstartsignatures
\pysiglinewithargsret{\sphinxbfcode{\sphinxupquote{\_\_init\_\_}}}{\emph{\DUrole{n}{message}\DUrole{p}{:}\DUrole{w}{  }\DUrole{n}{Optional\DUrole{p}{{[}}str\DUrole{p}{{]}}}\DUrole{w}{  }\DUrole{o}{=}\DUrole{w}{  }\DUrole{default_value}{None}}}{{ $\rightarrow$ None}}
\pysigstopsignatures
\sphinxAtStartPar
The initialization of a base exception for OpihiExarata.
\begin{quote}\begin{description}
\sphinxlineitem{Parameters}
\sphinxAtStartPar
\sphinxstyleliteralstrong{\sphinxupquote{message}} (\sphinxstyleliteralemphasis{\sphinxupquote{string}}) – The message of the error message.

\sphinxlineitem{Return type}
\sphinxAtStartPar
None

\end{description}\end{quote}

\end{fulllineitems}\end{savenotes}


\end{fulllineitems}\end{savenotes}

\index{ExarataException@\spxentry{ExarataException}}

\begin{savenotes}\begin{fulllineitems}
\phantomsection\label{\detokenize{code/opihiexarata.library.error:opihiexarata.library.error.ExarataException}}
\pysigstartsignatures
\pysiglinewithargsret{\sphinxbfcode{\sphinxupquote{exception\DUrole{w}{  }}}\sphinxcode{\sphinxupquote{opihiexarata.library.error.}}\sphinxbfcode{\sphinxupquote{ExarataException}}}{\emph{\DUrole{n}{message}\DUrole{p}{:}\DUrole{w}{  }\DUrole{n}{Optional\DUrole{p}{{[}}str\DUrole{p}{{]}}}\DUrole{w}{  }\DUrole{o}{=}\DUrole{w}{  }\DUrole{default_value}{None}}}{}
\pysigstopsignatures
\sphinxAtStartPar
Bases: \sphinxcode{\sphinxupquote{Exception}}

\sphinxAtStartPar
The main inheriting class which all exceptions use as their base. This
is done for ease of error handling and is something that can and should be
managed.
\index{\_\_init\_\_() (opihiexarata.library.error.ExarataException method)@\spxentry{\_\_init\_\_()}\spxextra{opihiexarata.library.error.ExarataException method}}

\begin{savenotes}\begin{fulllineitems}
\phantomsection\label{\detokenize{code/opihiexarata.library.error:opihiexarata.library.error.ExarataException.__init__}}
\pysigstartsignatures
\pysiglinewithargsret{\sphinxbfcode{\sphinxupquote{\_\_init\_\_}}}{\emph{\DUrole{n}{message}\DUrole{p}{:}\DUrole{w}{  }\DUrole{n}{Optional\DUrole{p}{{[}}str\DUrole{p}{{]}}}\DUrole{w}{  }\DUrole{o}{=}\DUrole{w}{  }\DUrole{default_value}{None}}}{{ $\rightarrow$ None}}
\pysigstopsignatures
\sphinxAtStartPar
The initialization of a normal exception.
\begin{quote}\begin{description}
\sphinxlineitem{Parameters}
\sphinxAtStartPar
\sphinxstyleliteralstrong{\sphinxupquote{message}} (\sphinxstyleliteralemphasis{\sphinxupquote{string}}) – The message of the error message.

\sphinxlineitem{Return type}
\sphinxAtStartPar
None

\end{description}\end{quote}

\end{fulllineitems}\end{savenotes}


\end{fulllineitems}\end{savenotes}

\index{ExarataWarning@\spxentry{ExarataWarning}}

\begin{savenotes}\begin{fulllineitems}
\phantomsection\label{\detokenize{code/opihiexarata.library.error:opihiexarata.library.error.ExarataWarning}}
\pysigstartsignatures
\pysigline{\sphinxbfcode{\sphinxupquote{exception\DUrole{w}{  }}}\sphinxcode{\sphinxupquote{opihiexarata.library.error.}}\sphinxbfcode{\sphinxupquote{ExarataWarning}}}
\pysigstopsignatures
\sphinxAtStartPar
Bases: \sphinxcode{\sphinxupquote{UserWarning}}

\sphinxAtStartPar
The base warning class which all of the other OpihiExarata warnings
are derived from.

\end{fulllineitems}\end{savenotes}

\index{FileError@\spxentry{FileError}}

\begin{savenotes}\begin{fulllineitems}
\phantomsection\label{\detokenize{code/opihiexarata.library.error:opihiexarata.library.error.FileError}}
\pysigstartsignatures
\pysiglinewithargsret{\sphinxbfcode{\sphinxupquote{exception\DUrole{w}{  }}}\sphinxcode{\sphinxupquote{opihiexarata.library.error.}}\sphinxbfcode{\sphinxupquote{FileError}}}{\emph{\DUrole{n}{message}\DUrole{p}{:}\DUrole{w}{  }\DUrole{n}{Optional\DUrole{p}{{[}}str\DUrole{p}{{]}}}\DUrole{w}{  }\DUrole{o}{=}\DUrole{w}{  }\DUrole{default_value}{None}}}{}
\pysigstopsignatures
\sphinxAtStartPar
Bases: {\hyperref[\detokenize{code/opihiexarata.library.error:opihiexarata.library.error.ExarataException}]{\sphinxcrossref{\sphinxcode{\sphinxupquote{ExarataException}}}}}

\sphinxAtStartPar
An error to be used when obtaining data files or configuration files
and something fails.

\end{fulllineitems}\end{savenotes}

\index{InputError@\spxentry{InputError}}

\begin{savenotes}\begin{fulllineitems}
\phantomsection\label{\detokenize{code/opihiexarata.library.error:opihiexarata.library.error.InputError}}
\pysigstartsignatures
\pysiglinewithargsret{\sphinxbfcode{\sphinxupquote{exception\DUrole{w}{  }}}\sphinxcode{\sphinxupquote{opihiexarata.library.error.}}\sphinxbfcode{\sphinxupquote{InputError}}}{\emph{\DUrole{n}{message}\DUrole{p}{:}\DUrole{w}{  }\DUrole{n}{Optional\DUrole{p}{{[}}str\DUrole{p}{{]}}}\DUrole{w}{  }\DUrole{o}{=}\DUrole{w}{  }\DUrole{default_value}{None}}}{}
\pysigstopsignatures
\sphinxAtStartPar
Bases: {\hyperref[\detokenize{code/opihiexarata.library.error:opihiexarata.library.error.ExarataException}]{\sphinxcrossref{\sphinxcode{\sphinxupquote{ExarataException}}}}}

\sphinxAtStartPar
An error to be used when the inputs to a function are not valid and do
not match the expectations of that function.

\end{fulllineitems}\end{savenotes}

\index{InstallError@\spxentry{InstallError}}

\begin{savenotes}\begin{fulllineitems}
\phantomsection\label{\detokenize{code/opihiexarata.library.error:opihiexarata.library.error.InstallError}}
\pysigstartsignatures
\pysiglinewithargsret{\sphinxbfcode{\sphinxupquote{exception\DUrole{w}{  }}}\sphinxcode{\sphinxupquote{opihiexarata.library.error.}}\sphinxbfcode{\sphinxupquote{InstallError}}}{\emph{\DUrole{n}{message}\DUrole{p}{:}\DUrole{w}{  }\DUrole{n}{Optional\DUrole{p}{{[}}str\DUrole{p}{{]}}}\DUrole{w}{  }\DUrole{o}{=}\DUrole{w}{  }\DUrole{default_value}{None}}}{}
\pysigstopsignatures
\sphinxAtStartPar
Bases: {\hyperref[\detokenize{code/opihiexarata.library.error:opihiexarata.library.error.ExarataException}]{\sphinxcrossref{\sphinxcode{\sphinxupquote{ExarataException}}}}}

\sphinxAtStartPar
An error to be used when informing the user or the program that the
installation was not done properly and lack some of the features and
assumptions which are a consequence of it.

\end{fulllineitems}\end{savenotes}

\index{IntentionalError@\spxentry{IntentionalError}}

\begin{savenotes}\begin{fulllineitems}
\phantomsection\label{\detokenize{code/opihiexarata.library.error:opihiexarata.library.error.IntentionalError}}
\pysigstartsignatures
\pysiglinewithargsret{\sphinxbfcode{\sphinxupquote{exception\DUrole{w}{  }}}\sphinxcode{\sphinxupquote{opihiexarata.library.error.}}\sphinxbfcode{\sphinxupquote{IntentionalError}}}{\emph{\DUrole{n}{message}\DUrole{p}{:}\DUrole{w}{  }\DUrole{n}{Optional\DUrole{p}{{[}}str\DUrole{p}{{]}}}\DUrole{w}{  }\DUrole{o}{=}\DUrole{w}{  }\DUrole{default_value}{None}}}{}
\pysigstopsignatures
\sphinxAtStartPar
Bases: {\hyperref[\detokenize{code/opihiexarata.library.error:opihiexarata.library.error.ExarataException}]{\sphinxcrossref{\sphinxcode{\sphinxupquote{ExarataException}}}}}

\sphinxAtStartPar
An error to be used where error catching is helpful. This error
generally should always be caught by the code in context.

\end{fulllineitems}\end{savenotes}

\index{LogicFlowError@\spxentry{LogicFlowError}}

\begin{savenotes}\begin{fulllineitems}
\phantomsection\label{\detokenize{code/opihiexarata.library.error:opihiexarata.library.error.LogicFlowError}}
\pysigstartsignatures
\pysiglinewithargsret{\sphinxbfcode{\sphinxupquote{exception\DUrole{w}{  }}}\sphinxcode{\sphinxupquote{opihiexarata.library.error.}}\sphinxbfcode{\sphinxupquote{LogicFlowError}}}{\emph{\DUrole{n}{message}\DUrole{p}{:}\DUrole{w}{  }\DUrole{n}{Optional\DUrole{p}{{[}}str\DUrole{p}{{]}}}\DUrole{w}{  }\DUrole{o}{=}\DUrole{w}{  }\DUrole{default_value}{None}}}{}
\pysigstopsignatures
\sphinxAtStartPar
Bases: {\hyperref[\detokenize{code/opihiexarata.library.error:opihiexarata.library.error.ExarataBaseException}]{\sphinxcrossref{\sphinxcode{\sphinxupquote{ExarataBaseException}}}}}

\sphinxAtStartPar
This is an error to ensure that the logic does not flow to a point to a
place where it is not supposed to. This is helpful in making sure changes
to the code do not screw up the logical flow of the program.

\end{fulllineitems}\end{savenotes}

\index{PracticalityError@\spxentry{PracticalityError}}

\begin{savenotes}\begin{fulllineitems}
\phantomsection\label{\detokenize{code/opihiexarata.library.error:opihiexarata.library.error.PracticalityError}}
\pysigstartsignatures
\pysiglinewithargsret{\sphinxbfcode{\sphinxupquote{exception\DUrole{w}{  }}}\sphinxcode{\sphinxupquote{opihiexarata.library.error.}}\sphinxbfcode{\sphinxupquote{PracticalityError}}}{\emph{\DUrole{n}{message}\DUrole{p}{:}\DUrole{w}{  }\DUrole{n}{Optional\DUrole{p}{{[}}str\DUrole{p}{{]}}}\DUrole{w}{  }\DUrole{o}{=}\DUrole{w}{  }\DUrole{default_value}{None}}}{}
\pysigstopsignatures
\sphinxAtStartPar
Bases: {\hyperref[\detokenize{code/opihiexarata.library.error:opihiexarata.library.error.ExarataBaseException}]{\sphinxcrossref{\sphinxcode{\sphinxupquote{ExarataBaseException}}}}}

\sphinxAtStartPar
This is an error to be used when what is trying to be done does not
seem reasonable. Usually warnings are the better vehicle for this but
this error is used when the assumptions for reasonability guided
development and what the user is trying to do is not currently supported
by the software.

\end{fulllineitems}\end{savenotes}

\index{ReadOnlyError@\spxentry{ReadOnlyError}}

\begin{savenotes}\begin{fulllineitems}
\phantomsection\label{\detokenize{code/opihiexarata.library.error:opihiexarata.library.error.ReadOnlyError}}
\pysigstartsignatures
\pysiglinewithargsret{\sphinxbfcode{\sphinxupquote{exception\DUrole{w}{  }}}\sphinxcode{\sphinxupquote{opihiexarata.library.error.}}\sphinxbfcode{\sphinxupquote{ReadOnlyError}}}{\emph{\DUrole{n}{message}\DUrole{p}{:}\DUrole{w}{  }\DUrole{n}{Optional\DUrole{p}{{[}}str\DUrole{p}{{]}}}\DUrole{w}{  }\DUrole{o}{=}\DUrole{w}{  }\DUrole{default_value}{None}}}{}
\pysigstopsignatures
\sphinxAtStartPar
Bases: {\hyperref[\detokenize{code/opihiexarata.library.error:opihiexarata.library.error.ExarataException}]{\sphinxcrossref{\sphinxcode{\sphinxupquote{ExarataException}}}}}

\sphinxAtStartPar
An error where variables or files are assumed to be read only, this
enforces that notion.

\end{fulllineitems}\end{savenotes}

\index{SequentialOrderError@\spxentry{SequentialOrderError}}

\begin{savenotes}\begin{fulllineitems}
\phantomsection\label{\detokenize{code/opihiexarata.library.error:opihiexarata.library.error.SequentialOrderError}}
\pysigstartsignatures
\pysiglinewithargsret{\sphinxbfcode{\sphinxupquote{exception\DUrole{w}{  }}}\sphinxcode{\sphinxupquote{opihiexarata.library.error.}}\sphinxbfcode{\sphinxupquote{SequentialOrderError}}}{\emph{\DUrole{n}{message}\DUrole{p}{:}\DUrole{w}{  }\DUrole{n}{Optional\DUrole{p}{{[}}str\DUrole{p}{{]}}}\DUrole{w}{  }\DUrole{o}{=}\DUrole{w}{  }\DUrole{default_value}{None}}}{}
\pysigstopsignatures
\sphinxAtStartPar
Bases: {\hyperref[\detokenize{code/opihiexarata.library.error:opihiexarata.library.error.ExarataException}]{\sphinxcrossref{\sphinxcode{\sphinxupquote{ExarataException}}}}}

\sphinxAtStartPar
An error used when something is happening out of the expected required
order. This order being in place for specific publically communicated
reasons.

\end{fulllineitems}\end{savenotes}

\index{UndiscoveredError@\spxentry{UndiscoveredError}}

\begin{savenotes}\begin{fulllineitems}
\phantomsection\label{\detokenize{code/opihiexarata.library.error:opihiexarata.library.error.UndiscoveredError}}
\pysigstartsignatures
\pysiglinewithargsret{\sphinxbfcode{\sphinxupquote{exception\DUrole{w}{  }}}\sphinxcode{\sphinxupquote{opihiexarata.library.error.}}\sphinxbfcode{\sphinxupquote{UndiscoveredError}}}{\emph{\DUrole{n}{message}\DUrole{p}{:}\DUrole{w}{  }\DUrole{n}{Optional\DUrole{p}{{[}}str\DUrole{p}{{]}}}\DUrole{w}{  }\DUrole{o}{=}\DUrole{w}{  }\DUrole{default_value}{None}}}{}
\pysigstopsignatures
\sphinxAtStartPar
Bases: {\hyperref[\detokenize{code/opihiexarata.library.error:opihiexarata.library.error.ExarataBaseException}]{\sphinxcrossref{\sphinxcode{\sphinxupquote{ExarataBaseException}}}}}

\sphinxAtStartPar
This is an error used in cases where the source of the error has not
been determined and so a more helpful error message or mitigation strategy
cannot be devised.

\end{fulllineitems}\end{savenotes}

\index{WebRequestError@\spxentry{WebRequestError}}

\begin{savenotes}\begin{fulllineitems}
\phantomsection\label{\detokenize{code/opihiexarata.library.error:opihiexarata.library.error.WebRequestError}}
\pysigstartsignatures
\pysiglinewithargsret{\sphinxbfcode{\sphinxupquote{exception\DUrole{w}{  }}}\sphinxcode{\sphinxupquote{opihiexarata.library.error.}}\sphinxbfcode{\sphinxupquote{WebRequestError}}}{\emph{\DUrole{n}{message}\DUrole{p}{:}\DUrole{w}{  }\DUrole{n}{Optional\DUrole{p}{{[}}str\DUrole{p}{{]}}}\DUrole{w}{  }\DUrole{o}{=}\DUrole{w}{  }\DUrole{default_value}{None}}}{}
\pysigstopsignatures
\sphinxAtStartPar
Bases: {\hyperref[\detokenize{code/opihiexarata.library.error:opihiexarata.library.error.ExarataException}]{\sphinxcrossref{\sphinxcode{\sphinxupquote{ExarataException}}}}}

\sphinxAtStartPar
An error to be used when a web request to some API fails, either because
of something from their end, or our end.

\end{fulllineitems}\end{savenotes}

\index{warn() (in module opihiexarata.library.error)@\spxentry{warn()}\spxextra{in module opihiexarata.library.error}}

\begin{savenotes}\begin{fulllineitems}
\phantomsection\label{\detokenize{code/opihiexarata.library.error:opihiexarata.library.error.warn}}
\pysigstartsignatures
\pysiglinewithargsret{\sphinxcode{\sphinxupquote{opihiexarata.library.error.}}\sphinxbfcode{\sphinxupquote{warn}}}{\emph{\DUrole{n}{warn\_class: type{[}opihiexarata.library.error.ExarataWarning{]} = <class 'opihiexarata.library.error.ExarataWarning'>}}, \emph{\DUrole{n}{message: str = ''}}, \emph{\DUrole{n}{stacklevel: int = 2}}}{}
\pysigstopsignatures
\sphinxAtStartPar
The common method to use to warn for any OpihiExarata based warnings.

\sphinxAtStartPar
This is used because it has better context manager wrappers.
\begin{quote}\begin{description}
\sphinxlineitem{Parameters}\begin{itemize}
\item {} 
\sphinxAtStartPar
\sphinxstyleliteralstrong{\sphinxupquote{warn\_class}} (\sphinxstyleliteralemphasis{\sphinxupquote{type}}\sphinxstyleliteralemphasis{\sphinxupquote{, }}\sphinxstyleliteralemphasis{\sphinxupquote{default = ExarataWarning}}) – The warning class, it must be a subtype of a user warning.

\item {} 
\sphinxAtStartPar
\sphinxstyleliteralstrong{\sphinxupquote{message}} (\sphinxstyleliteralemphasis{\sphinxupquote{string}}\sphinxstyleliteralemphasis{\sphinxupquote{, }}\sphinxstyleliteralemphasis{\sphinxupquote{default = ""}}) – The warning message.

\item {} 
\sphinxAtStartPar
\sphinxstyleliteralstrong{\sphinxupquote{stacklevel}} (\sphinxstyleliteralemphasis{\sphinxupquote{integer}}\sphinxstyleliteralemphasis{\sphinxupquote{, }}\sphinxstyleliteralemphasis{\sphinxupquote{default = 2}}) – The location in the stack that the warning call will highlight.

\end{itemize}

\sphinxlineitem{Return type}
\sphinxAtStartPar
None

\end{description}\end{quote}

\end{fulllineitems}\end{savenotes}


\sphinxstepscope


\subparagraph{opihiexarata.library.fits module}
\label{\detokenize{code/opihiexarata.library.fits:module-opihiexarata.library.fits}}\label{\detokenize{code/opihiexarata.library.fits:opihiexarata-library-fits-module}}\label{\detokenize{code/opihiexarata.library.fits::doc}}\index{module@\spxentry{module}!opihiexarata.library.fits@\spxentry{opihiexarata.library.fits}}\index{opihiexarata.library.fits@\spxentry{opihiexarata.library.fits}!module@\spxentry{module}}
\sphinxAtStartPar
Fits file based operations. These are kind of like convince functions.
\index{read\_fits\_header() (in module opihiexarata.library.fits)@\spxentry{read\_fits\_header()}\spxextra{in module opihiexarata.library.fits}}

\begin{savenotes}\begin{fulllineitems}
\phantomsection\label{\detokenize{code/opihiexarata.library.fits:opihiexarata.library.fits.read_fits_header}}
\pysigstartsignatures
\pysiglinewithargsret{\sphinxcode{\sphinxupquote{opihiexarata.library.fits.}}\sphinxbfcode{\sphinxupquote{read\_fits\_header}}}{\emph{\DUrole{n}{filename}\DUrole{p}{:}\DUrole{w}{  }\DUrole{n}{str}}, \emph{\DUrole{n}{extension}\DUrole{p}{:}\DUrole{w}{  }\DUrole{n}{Union\DUrole{p}{{[}}int\DUrole{p}{,}\DUrole{w}{  }str\DUrole{p}{{]}}}\DUrole{w}{  }\DUrole{o}{=}\DUrole{w}{  }\DUrole{default_value}{0}}}{{ $\rightarrow$ Header}}
\pysigstopsignatures
\sphinxAtStartPar
This reads the header of fits files only. This should be used only if
there is no data.

\sphinxAtStartPar
Really, this is just a wrapper around Astropy, but it is made for
consistency and to avoid the usage of the convince functions.
\begin{quote}\begin{description}
\sphinxlineitem{Parameters}\begin{itemize}
\item {} 
\sphinxAtStartPar
\sphinxstyleliteralstrong{\sphinxupquote{filename}} (\sphinxstyleliteralemphasis{\sphinxupquote{string}}) – The filename that the fits image file is at.

\item {} 
\sphinxAtStartPar
\sphinxstyleliteralstrong{\sphinxupquote{extension}} (\sphinxstyleliteralemphasis{\sphinxupquote{int}}\sphinxstyleliteralemphasis{\sphinxupquote{ or }}\sphinxstyleliteralemphasis{\sphinxupquote{string}}\sphinxstyleliteralemphasis{\sphinxupquote{, }}\sphinxstyleliteralemphasis{\sphinxupquote{default = 0}}) – The fits extension that is desired to be opened.

\end{itemize}

\sphinxlineitem{Returns}
\sphinxAtStartPar
\sphinxstylestrong{header} – The header of the fits file.

\sphinxlineitem{Return type}
\sphinxAtStartPar
Astropy Header

\end{description}\end{quote}

\end{fulllineitems}\end{savenotes}

\index{read\_fits\_image\_file() (in module opihiexarata.library.fits)@\spxentry{read\_fits\_image\_file()}\spxextra{in module opihiexarata.library.fits}}

\begin{savenotes}\begin{fulllineitems}
\phantomsection\label{\detokenize{code/opihiexarata.library.fits:opihiexarata.library.fits.read_fits_image_file}}
\pysigstartsignatures
\pysiglinewithargsret{\sphinxcode{\sphinxupquote{opihiexarata.library.fits.}}\sphinxbfcode{\sphinxupquote{read\_fits\_image\_file}}}{\emph{\DUrole{n}{filename}\DUrole{p}{:}\DUrole{w}{  }\DUrole{n}{str}}, \emph{\DUrole{n}{extension}\DUrole{p}{:}\DUrole{w}{  }\DUrole{n}{Union\DUrole{p}{{[}}int\DUrole{p}{,}\DUrole{w}{  }str\DUrole{p}{{]}}}\DUrole{w}{  }\DUrole{o}{=}\DUrole{w}{  }\DUrole{default_value}{0}}}{{ $\rightarrow$ tuple\DUrole{p}{{[}}astropy.io.fits.header.Header\DUrole{p}{,}\DUrole{w}{  }numpy.ndarray\DUrole{p}{{]}}}}
\pysigstopsignatures
\sphinxAtStartPar
This reads fits files, assuming that the fits file is an image. It is a
wrapper function around the astropy functions.
\begin{quote}\begin{description}
\sphinxlineitem{Parameters}\begin{itemize}
\item {} 
\sphinxAtStartPar
\sphinxstyleliteralstrong{\sphinxupquote{filename}} (\sphinxstyleliteralemphasis{\sphinxupquote{string}}) – The filename that the fits image file is at.

\item {} 
\sphinxAtStartPar
\sphinxstyleliteralstrong{\sphinxupquote{extension}} (\sphinxstyleliteralemphasis{\sphinxupquote{int}}\sphinxstyleliteralemphasis{\sphinxupquote{ or }}\sphinxstyleliteralemphasis{\sphinxupquote{string}}\sphinxstyleliteralemphasis{\sphinxupquote{, }}\sphinxstyleliteralemphasis{\sphinxupquote{default = 0}}) – The fits extension that is desired to be opened.

\end{itemize}

\sphinxlineitem{Returns}
\sphinxAtStartPar
\begin{itemize}
\item {} 
\sphinxAtStartPar
\sphinxstylestrong{header} (\sphinxstyleemphasis{Astropy Header}) – The header of the fits file.

\item {} 
\sphinxAtStartPar
\sphinxstylestrong{data} (\sphinxstyleemphasis{array}) – The data image of the fits file.

\end{itemize}


\end{description}\end{quote}

\end{fulllineitems}\end{savenotes}

\index{read\_fits\_table\_file() (in module opihiexarata.library.fits)@\spxentry{read\_fits\_table\_file()}\spxextra{in module opihiexarata.library.fits}}

\begin{savenotes}\begin{fulllineitems}
\phantomsection\label{\detokenize{code/opihiexarata.library.fits:opihiexarata.library.fits.read_fits_table_file}}
\pysigstartsignatures
\pysiglinewithargsret{\sphinxcode{\sphinxupquote{opihiexarata.library.fits.}}\sphinxbfcode{\sphinxupquote{read\_fits\_table\_file}}}{\emph{\DUrole{n}{filename}\DUrole{p}{:}\DUrole{w}{  }\DUrole{n}{str}}, \emph{\DUrole{n}{extension}\DUrole{p}{:}\DUrole{w}{  }\DUrole{n}{Union\DUrole{p}{{[}}int\DUrole{p}{,}\DUrole{w}{  }str\DUrole{p}{{]}}}\DUrole{w}{  }\DUrole{o}{=}\DUrole{w}{  }\DUrole{default_value}{0}}}{{ $\rightarrow$ tuple\DUrole{p}{{[}}astropy.io.fits.header.Header\DUrole{p}{,}\DUrole{w}{  }astropy.table.table.Table\DUrole{p}{{]}}}}
\pysigstopsignatures
\sphinxAtStartPar
This reads fits files, assuming that the fits file is a binary table.
It is a wrapper function around the astropy functions.
\begin{quote}\begin{description}
\sphinxlineitem{Parameters}\begin{itemize}
\item {} 
\sphinxAtStartPar
\sphinxstyleliteralstrong{\sphinxupquote{filename}} (\sphinxstyleliteralemphasis{\sphinxupquote{string}}) – The filename that the fits image file is at.

\item {} 
\sphinxAtStartPar
\sphinxstyleliteralstrong{\sphinxupquote{extension}} (\sphinxstyleliteralemphasis{\sphinxupquote{int}}\sphinxstyleliteralemphasis{\sphinxupquote{ or }}\sphinxstyleliteralemphasis{\sphinxupquote{string}}\sphinxstyleliteralemphasis{\sphinxupquote{, }}\sphinxstyleliteralemphasis{\sphinxupquote{default = 0}}) – The fits extension that is desired to be opened.

\end{itemize}

\sphinxlineitem{Returns}
\sphinxAtStartPar
\begin{itemize}
\item {} 
\sphinxAtStartPar
\sphinxstylestrong{header} (\sphinxstyleemphasis{Astropy Header}) – The header of the fits file.

\item {} 
\sphinxAtStartPar
\sphinxstylestrong{table} (\sphinxstyleemphasis{Astropy Table}) – The data table of the fits file.

\end{itemize}


\end{description}\end{quote}

\end{fulllineitems}\end{savenotes}

\index{update\_fits\_header() (in module opihiexarata.library.fits)@\spxentry{update\_fits\_header()}\spxextra{in module opihiexarata.library.fits}}

\begin{savenotes}\begin{fulllineitems}
\phantomsection\label{\detokenize{code/opihiexarata.library.fits:opihiexarata.library.fits.update_fits_header}}
\pysigstartsignatures
\pysiglinewithargsret{\sphinxcode{\sphinxupquote{opihiexarata.library.fits.}}\sphinxbfcode{\sphinxupquote{update\_fits\_header}}}{\emph{\DUrole{n}{header}\DUrole{p}{:}\DUrole{w}{  }\DUrole{n}{Header}}, \emph{\DUrole{n}{entries}\DUrole{p}{:}\DUrole{w}{  }\DUrole{n}{dict}}, \emph{\DUrole{n}{comments}\DUrole{p}{:}\DUrole{w}{  }\DUrole{n}{dict}\DUrole{w}{  }\DUrole{o}{=}\DUrole{w}{  }\DUrole{default_value}{\{\}}}}{{ $\rightarrow$ Header}}
\pysigstopsignatures
\sphinxAtStartPar
This appends entries from a dictionary to an Astropy header.

\sphinxAtStartPar
This function is preferred to adding using standard methods as it performs
checks to make sure it only uses header keys reserved for OpihiExarata.
This function raises an error upon attempting to add an entry which does
not conform to fits standards and is not keyed with the “OX\#\#\#\#\#\#”
template.
\begin{quote}\begin{description}
\sphinxlineitem{Parameters}\begin{itemize}
\item {} 
\sphinxAtStartPar
\sphinxstyleliteralstrong{\sphinxupquote{header}} (\sphinxstyleliteralemphasis{\sphinxupquote{Astropy Header}}) – The header which the entries will be added to.

\item {} 
\sphinxAtStartPar
\sphinxstyleliteralstrong{\sphinxupquote{entries}} (\sphinxstyleliteralemphasis{\sphinxupquote{dictionary}}) – The new entries to the header.

\item {} 
\sphinxAtStartPar
\sphinxstyleliteralstrong{\sphinxupquote{comments}} (\sphinxstyleliteralemphasis{\sphinxupquote{dictionary}}\sphinxstyleliteralemphasis{\sphinxupquote{, }}\sphinxstyleliteralemphasis{\sphinxupquote{default = \{\}}}) – If comments are to be added to data entries, then they may be
provided as a dictionary here with keys exactly the same as the
data entries. This is not for comment cards.

\end{itemize}

\end{description}\end{quote}

\end{fulllineitems}\end{savenotes}

\index{write\_fits\_image\_file() (in module opihiexarata.library.fits)@\spxentry{write\_fits\_image\_file()}\spxextra{in module opihiexarata.library.fits}}

\begin{savenotes}\begin{fulllineitems}
\phantomsection\label{\detokenize{code/opihiexarata.library.fits:opihiexarata.library.fits.write_fits_image_file}}
\pysigstartsignatures
\pysiglinewithargsret{\sphinxcode{\sphinxupquote{opihiexarata.library.fits.}}\sphinxbfcode{\sphinxupquote{write\_fits\_image\_file}}}{\emph{\DUrole{n}{filename}\DUrole{p}{:}\DUrole{w}{  }\DUrole{n}{str}}, \emph{\DUrole{n}{header}\DUrole{p}{:}\DUrole{w}{  }\DUrole{n}{Header}}, \emph{\DUrole{n}{data}\DUrole{p}{:}\DUrole{w}{  }\DUrole{n}{ndarray}}, \emph{\DUrole{n}{overwrite}\DUrole{p}{:}\DUrole{w}{  }\DUrole{n}{bool}\DUrole{w}{  }\DUrole{o}{=}\DUrole{w}{  }\DUrole{default_value}{False}}}{{ $\rightarrow$ None}}
\pysigstopsignatures
\sphinxAtStartPar
This writes fits image files to disk. Acting as a wrapper around the
fits functionality of astropy.
\begin{quote}\begin{description}
\sphinxlineitem{Parameters}\begin{itemize}
\item {} 
\sphinxAtStartPar
\sphinxstyleliteralstrong{\sphinxupquote{filename}} (\sphinxstyleliteralemphasis{\sphinxupquote{string}}) – The filename that the fits image file will be written to.

\item {} 
\sphinxAtStartPar
\sphinxstyleliteralstrong{\sphinxupquote{header}} (\sphinxstyleliteralemphasis{\sphinxupquote{Astropy Header}}) – The header of the fits file.

\item {} 
\sphinxAtStartPar
\sphinxstyleliteralstrong{\sphinxupquote{data}} (\sphinxstyleliteralemphasis{\sphinxupquote{array\sphinxhyphen{}like}}) – The data image of the fits file.

\item {} 
\sphinxAtStartPar
\sphinxstyleliteralstrong{\sphinxupquote{overwrite}} (\sphinxstyleliteralemphasis{\sphinxupquote{boolean}}\sphinxstyleliteralemphasis{\sphinxupquote{, }}\sphinxstyleliteralemphasis{\sphinxupquote{default = False}}) – Decides if to overwrite the file if it already exists.

\end{itemize}

\sphinxlineitem{Return type}
\sphinxAtStartPar
None

\end{description}\end{quote}

\end{fulllineitems}\end{savenotes}

\index{write\_fits\_table\_file() (in module opihiexarata.library.fits)@\spxentry{write\_fits\_table\_file()}\spxextra{in module opihiexarata.library.fits}}

\begin{savenotes}\begin{fulllineitems}
\phantomsection\label{\detokenize{code/opihiexarata.library.fits:opihiexarata.library.fits.write_fits_table_file}}
\pysigstartsignatures
\pysiglinewithargsret{\sphinxcode{\sphinxupquote{opihiexarata.library.fits.}}\sphinxbfcode{\sphinxupquote{write\_fits\_table\_file}}}{\emph{\DUrole{n}{filename}\DUrole{p}{:}\DUrole{w}{  }\DUrole{n}{str}}, \emph{\DUrole{n}{header}\DUrole{p}{:}\DUrole{w}{  }\DUrole{n}{Header}}, \emph{\DUrole{n}{data}\DUrole{p}{:}\DUrole{w}{  }\DUrole{n}{Table}}, \emph{\DUrole{n}{overwrite}\DUrole{p}{:}\DUrole{w}{  }\DUrole{n}{bool}\DUrole{w}{  }\DUrole{o}{=}\DUrole{w}{  }\DUrole{default_value}{False}}}{{ $\rightarrow$ None}}
\pysigstopsignatures
\sphinxAtStartPar
This writes fits table files to disk. Acting as a wrapper around the
fits functionality of astropy.
\begin{quote}\begin{description}
\sphinxlineitem{Parameters}\begin{itemize}
\item {} 
\sphinxAtStartPar
\sphinxstyleliteralstrong{\sphinxupquote{filename}} (\sphinxstyleliteralemphasis{\sphinxupquote{string}}) – The filename that the fits image file will be written to.

\item {} 
\sphinxAtStartPar
\sphinxstyleliteralstrong{\sphinxupquote{header}} (\sphinxstyleliteralemphasis{\sphinxupquote{Astropy Header}}) – The header of the fits file.

\item {} 
\sphinxAtStartPar
\sphinxstyleliteralstrong{\sphinxupquote{data}} (\sphinxstyleliteralemphasis{\sphinxupquote{Astropy Table}}) – The data table of the table file.

\item {} 
\sphinxAtStartPar
\sphinxstyleliteralstrong{\sphinxupquote{overwrite}} (\sphinxstyleliteralemphasis{\sphinxupquote{boolean}}\sphinxstyleliteralemphasis{\sphinxupquote{, }}\sphinxstyleliteralemphasis{\sphinxupquote{default = False}}) – Decides if to overwrite the file if it already exists.

\end{itemize}

\sphinxlineitem{Return type}
\sphinxAtStartPar
None

\end{description}\end{quote}

\end{fulllineitems}\end{savenotes}


\sphinxstepscope


\subparagraph{opihiexarata.library.hint module}
\label{\detokenize{code/opihiexarata.library.hint:module-opihiexarata.library.hint}}\label{\detokenize{code/opihiexarata.library.hint:opihiexarata-library-hint-module}}\label{\detokenize{code/opihiexarata.library.hint::doc}}\index{module@\spxentry{module}!opihiexarata.library.hint@\spxentry{opihiexarata.library.hint}}\index{opihiexarata.library.hint@\spxentry{opihiexarata.library.hint}!module@\spxentry{module}}
\sphinxAtStartPar
These are redefinitions and wrapping variables for type hints. Its purpose
is for just uniform hinting types.

\sphinxAtStartPar
This should only be used for types which are otherwise not native and would
require an import, including the typing module. The whole point of this is to
be a central collection of types for the purpose of type hinting.

\sphinxAtStartPar
This module should never be used for anything other than hinting. Use proper
imports to access these classes. Otherwise, you will likely get circular
imports and other nasty things.

\sphinxstepscope


\subparagraph{opihiexarata.library.http module}
\label{\detokenize{code/opihiexarata.library.http:module-opihiexarata.library.http}}\label{\detokenize{code/opihiexarata.library.http:opihiexarata-library-http-module}}\label{\detokenize{code/opihiexarata.library.http::doc}}\index{module@\spxentry{module}!opihiexarata.library.http@\spxentry{opihiexarata.library.http}}\index{opihiexarata.library.http@\spxentry{opihiexarata.library.http}!module@\spxentry{module}}
\sphinxAtStartPar
Functions and methods which allow for ease of interacting with web based
resources. Included here are functions which download files, query web resources
and other things. This interacts mostly with HTTP based services.
\index{api\_request\_sleep() (in module opihiexarata.library.http)@\spxentry{api\_request\_sleep()}\spxextra{in module opihiexarata.library.http}}

\begin{savenotes}\begin{fulllineitems}
\phantomsection\label{\detokenize{code/opihiexarata.library.http:opihiexarata.library.http.api_request_sleep}}
\pysigstartsignatures
\pysiglinewithargsret{\sphinxcode{\sphinxupquote{opihiexarata.library.http.}}\sphinxbfcode{\sphinxupquote{api\_request\_sleep}}}{\emph{\DUrole{n}{seconds}\DUrole{p}{:}\DUrole{w}{  }\DUrole{n}{Optional\DUrole{p}{{[}}float\DUrole{p}{{]}}}\DUrole{w}{  }\DUrole{o}{=}\DUrole{w}{  }\DUrole{default_value}{None}}}{{ $\rightarrow$ None}}
\pysigstopsignatures
\sphinxAtStartPar
Sleep for the time, specified in the configuration file, for API
requests. This function exists to ensure uniformity in application.
\begin{quote}\begin{description}
\sphinxlineitem{Parameters}\begin{itemize}
\item {} 
\sphinxAtStartPar
\sphinxstyleliteralstrong{\sphinxupquote{seconds}} (\sphinxstyleliteralemphasis{\sphinxupquote{float}}\sphinxstyleliteralemphasis{\sphinxupquote{, }}\sphinxstyleliteralemphasis{\sphinxupquote{default = None}}) – The number of seconds that the program should sleep for. If not
provided, then it defaults to the configuration value.

\item {} 
\sphinxAtStartPar
\sphinxstyleliteralstrong{\sphinxupquote{Results}} – 

\item {} 
\sphinxAtStartPar
\sphinxstyleliteralstrong{\sphinxupquote{\sphinxhyphen{}\sphinxhyphen{}\sphinxhyphen{}\sphinxhyphen{}\sphinxhyphen{}\sphinxhyphen{}\sphinxhyphen{}}} – 

\item {} 
\sphinxAtStartPar
\sphinxstyleliteralstrong{\sphinxupquote{None}} – 

\end{itemize}

\end{description}\end{quote}

\end{fulllineitems}\end{savenotes}

\index{download\_file\_from\_url() (in module opihiexarata.library.http)@\spxentry{download\_file\_from\_url()}\spxextra{in module opihiexarata.library.http}}

\begin{savenotes}\begin{fulllineitems}
\phantomsection\label{\detokenize{code/opihiexarata.library.http:opihiexarata.library.http.download_file_from_url}}
\pysigstartsignatures
\pysiglinewithargsret{\sphinxcode{\sphinxupquote{opihiexarata.library.http.}}\sphinxbfcode{\sphinxupquote{download\_file\_from\_url}}}{\emph{\DUrole{n}{url}\DUrole{p}{:}\DUrole{w}{  }\DUrole{n}{str}}, \emph{\DUrole{n}{filename}\DUrole{p}{:}\DUrole{w}{  }\DUrole{n}{str}}, \emph{\DUrole{n}{overwrite}\DUrole{p}{:}\DUrole{w}{  }\DUrole{n}{bool}\DUrole{w}{  }\DUrole{o}{=}\DUrole{w}{  }\DUrole{default_value}{False}}}{{ $\rightarrow$ None}}
\pysigstopsignatures
\sphinxAtStartPar
Download a file from a URL to disk.

\sphinxAtStartPar
..warning:: The backend of this function relies on a function which may be
depreciated in the future. This function may need to be rewritten.
\begin{quote}\begin{description}
\sphinxlineitem{Parameters}\begin{itemize}
\item {} 
\sphinxAtStartPar
\sphinxstyleliteralstrong{\sphinxupquote{url}} (\sphinxstyleliteralemphasis{\sphinxupquote{string}}) – The url which the file will be downloaded from.

\item {} 
\sphinxAtStartPar
\sphinxstyleliteralstrong{\sphinxupquote{filename}} (\sphinxstyleliteralemphasis{\sphinxupquote{string}}) – The filename where the file will be saved.

\item {} 
\sphinxAtStartPar
\sphinxstyleliteralstrong{\sphinxupquote{overwrite}} (\sphinxstyleliteralemphasis{\sphinxupquote{bool}}\sphinxstyleliteralemphasis{\sphinxupquote{, }}\sphinxstyleliteralemphasis{\sphinxupquote{default = False}}) – If the file already exists, overwrite it. If False, it would raise
an error instead.

\end{itemize}

\end{description}\end{quote}

\end{fulllineitems}\end{savenotes}

\index{get\_http\_status\_code() (in module opihiexarata.library.http)@\spxentry{get\_http\_status\_code()}\spxextra{in module opihiexarata.library.http}}

\begin{savenotes}\begin{fulllineitems}
\phantomsection\label{\detokenize{code/opihiexarata.library.http:opihiexarata.library.http.get_http_status_code}}
\pysigstartsignatures
\pysiglinewithargsret{\sphinxcode{\sphinxupquote{opihiexarata.library.http.}}\sphinxbfcode{\sphinxupquote{get\_http\_status\_code}}}{\emph{\DUrole{n}{url}\DUrole{p}{:}\DUrole{w}{  }\DUrole{n}{str}}}{{ $\rightarrow$ int}}
\pysigstopsignatures
\sphinxAtStartPar
This gets the http status code of a web resource.
\begin{quote}\begin{description}
\sphinxlineitem{Parameters}
\sphinxAtStartPar
\sphinxstyleliteralstrong{\sphinxupquote{url}} (\sphinxstyleliteralemphasis{\sphinxupquote{string}}) – The url which the http status code will try and obtain.

\sphinxlineitem{Returns}
\sphinxAtStartPar
\sphinxstylestrong{status\_code} – The status code.

\sphinxlineitem{Return type}
\sphinxAtStartPar
int

\end{description}\end{quote}

\end{fulllineitems}\end{savenotes}


\sphinxstepscope


\subparagraph{opihiexarata.library.image module}
\label{\detokenize{code/opihiexarata.library.image:module-opihiexarata.library.image}}\label{\detokenize{code/opihiexarata.library.image:opihiexarata-library-image-module}}\label{\detokenize{code/opihiexarata.library.image::doc}}\index{module@\spxentry{module}!opihiexarata.library.image@\spxentry{opihiexarata.library.image}}\index{opihiexarata.library.image@\spxentry{opihiexarata.library.image}!module@\spxentry{module}}
\sphinxAtStartPar
Functions to help with image and array manipulations.
\index{create\_circular\_mask() (in module opihiexarata.library.image)@\spxentry{create\_circular\_mask()}\spxextra{in module opihiexarata.library.image}}

\begin{savenotes}\begin{fulllineitems}
\phantomsection\label{\detokenize{code/opihiexarata.library.image:opihiexarata.library.image.create_circular_mask}}
\pysigstartsignatures
\pysiglinewithargsret{\sphinxcode{\sphinxupquote{opihiexarata.library.image.}}\sphinxbfcode{\sphinxupquote{create\_circular\_mask}}}{\emph{\DUrole{n}{array}\DUrole{p}{:}\DUrole{w}{  }\DUrole{n}{ndarray}}, \emph{\DUrole{n}{center\_x}\DUrole{p}{:}\DUrole{w}{  }\DUrole{n}{int}}, \emph{\DUrole{n}{center\_y}\DUrole{p}{:}\DUrole{w}{  }\DUrole{n}{int}}, \emph{\DUrole{n}{radius}\DUrole{p}{:}\DUrole{w}{  }\DUrole{n}{float}}}{{ $\rightarrow$ ndarray}}
\pysigstopsignatures
\sphinxAtStartPar
Creates an array which is a circular mask of some radius centered at a
custom index value location. This process is a little intensive so using
smaller subsets of arrays are preferred.

\sphinxAtStartPar
Method inspired by \sphinxurl{https://stackoverflow.com/a/44874588}.
\begin{quote}\begin{description}
\sphinxlineitem{Parameters}\begin{itemize}
\item {} 
\sphinxAtStartPar
\sphinxstyleliteralstrong{\sphinxupquote{array}} (\sphinxstyleliteralemphasis{\sphinxupquote{array\sphinxhyphen{}like}}) – The data array which the mask will base itself off of. The data in the
array is not actually modified but it is required for the shape
definition.

\item {} 
\sphinxAtStartPar
\sphinxstyleliteralstrong{\sphinxupquote{center\_x}} (\sphinxstyleliteralemphasis{\sphinxupquote{integer}}) – The x\sphinxhyphen{}axis coordinate where the mask will be centered.

\item {} 
\sphinxAtStartPar
\sphinxstyleliteralstrong{\sphinxupquote{center\_y}} (\sphinxstyleliteralemphasis{\sphinxupquote{integer}}) – The y\sphinxhyphen{}axis coordinate where the mask will be centered.

\item {} 
\sphinxAtStartPar
\sphinxstyleliteralstrong{\sphinxupquote{radius}} (\sphinxstyleliteralemphasis{\sphinxupquote{float}}) – The radius of the circle of the mask in pixels.

\end{itemize}

\sphinxlineitem{Returns}
\sphinxAtStartPar
\sphinxstylestrong{circular\_mask} – The mask; it is the same dimensions of the input data array. If True,
the the mask should be applied.

\sphinxlineitem{Return type}
\sphinxAtStartPar
array\sphinxhyphen{}like

\end{description}\end{quote}

\end{fulllineitems}\end{savenotes}

\index{save\_array\_as\_png\_grayscale() (in module opihiexarata.library.image)@\spxentry{save\_array\_as\_png\_grayscale()}\spxextra{in module opihiexarata.library.image}}

\begin{savenotes}\begin{fulllineitems}
\phantomsection\label{\detokenize{code/opihiexarata.library.image:opihiexarata.library.image.save_array_as_png_grayscale}}
\pysigstartsignatures
\pysiglinewithargsret{\sphinxcode{\sphinxupquote{opihiexarata.library.image.}}\sphinxbfcode{\sphinxupquote{save\_array\_as\_png\_grayscale}}}{\emph{\DUrole{n}{array}\DUrole{p}{:}\DUrole{w}{  }\DUrole{n}{ndarray}}, \emph{\DUrole{n}{filename}\DUrole{p}{:}\DUrole{w}{  }\DUrole{n}{str}}, \emph{\DUrole{n}{overwrite}\DUrole{p}{:}\DUrole{w}{  }\DUrole{n}{bool}\DUrole{w}{  }\DUrole{o}{=}\DUrole{w}{  }\DUrole{default_value}{False}}}{{ $\rightarrow$ None}}
\pysigstopsignatures
\sphinxAtStartPar
This converts an array to a grayscale PNG file.

\sphinxAtStartPar
The PNG specification requires that the data values be integer. Note that
if you are saving an array as a PNG, then data may be lost during the
conversion between float to integer.
\begin{quote}\begin{description}
\sphinxlineitem{Parameters}\begin{itemize}
\item {} 
\sphinxAtStartPar
\sphinxstyleliteralstrong{\sphinxupquote{array}} (\sphinxstyleliteralemphasis{\sphinxupquote{array\sphinxhyphen{}like}}) – The array that will be saved as a png.

\item {} 
\sphinxAtStartPar
\sphinxstyleliteralstrong{\sphinxupquote{filename}} (\sphinxstyleliteralemphasis{\sphinxupquote{string}}) – The filename where the png will be saved. If the filename does not have
the appropriate filename extension, it will be appended.

\item {} 
\sphinxAtStartPar
\sphinxstyleliteralstrong{\sphinxupquote{overwrite}} (\sphinxstyleliteralemphasis{\sphinxupquote{boolean}}) – If the file already exists, should it be overwritten?

\end{itemize}

\end{description}\end{quote}

\end{fulllineitems}\end{savenotes}

\index{scale\_image\_array() (in module opihiexarata.library.image)@\spxentry{scale\_image\_array()}\spxextra{in module opihiexarata.library.image}}

\begin{savenotes}\begin{fulllineitems}
\phantomsection\label{\detokenize{code/opihiexarata.library.image:opihiexarata.library.image.scale_image_array}}
\pysigstartsignatures
\pysiglinewithargsret{\sphinxcode{\sphinxupquote{opihiexarata.library.image.}}\sphinxbfcode{\sphinxupquote{scale\_image\_array}}}{\emph{\DUrole{n}{array}\DUrole{p}{:}\DUrole{w}{  }\DUrole{n}{ndarray}}, \emph{\DUrole{n}{minimum}\DUrole{p}{:}\DUrole{w}{  }\DUrole{n}{float}}, \emph{\DUrole{n}{maximum}\DUrole{p}{:}\DUrole{w}{  }\DUrole{n}{float}}, \emph{\DUrole{n}{lower\_percent\_cut}\DUrole{p}{:}\DUrole{w}{  }\DUrole{n}{float}\DUrole{w}{  }\DUrole{o}{=}\DUrole{w}{  }\DUrole{default_value}{0}}, \emph{\DUrole{n}{upper\_percent\_cut}\DUrole{p}{:}\DUrole{w}{  }\DUrole{n}{float}\DUrole{w}{  }\DUrole{o}{=}\DUrole{w}{  }\DUrole{default_value}{0}}}{{ $\rightarrow$ ndarray}}
\pysigstopsignatures
\sphinxAtStartPar
This function scales the array to the provided minimum and maximum
ranges after the percentile masks are taken.
\begin{quote}\begin{description}
\sphinxlineitem{Parameters}\begin{itemize}
\item {} 
\sphinxAtStartPar
\sphinxstyleliteralstrong{\sphinxupquote{array}} (\sphinxstyleliteralemphasis{\sphinxupquote{array\sphinxhyphen{}like}}) – The array to be scaled.

\item {} 
\sphinxAtStartPar
\sphinxstyleliteralstrong{\sphinxupquote{minimum}} (\sphinxstyleliteralemphasis{\sphinxupquote{float}}) – The minimum value of the scaling axis. This will be equal to the
minimum value of the scaled array after accounting for the percentile
cuts.

\item {} 
\sphinxAtStartPar
\sphinxstyleliteralstrong{\sphinxupquote{maximum}} (\sphinxstyleliteralemphasis{\sphinxupquote{float}}) – The maximum value of the scaling axis. This will be equal to the
maximum value of the scaled array after accounting for the percentile
cuts.

\item {} 
\sphinxAtStartPar
\sphinxstyleliteralstrong{\sphinxupquote{lower\_percent\_cut}} (\sphinxstyleliteralemphasis{\sphinxupquote{float}}) – The percent of values that will be masked from the lower end. Must be
between 0\sphinxhyphen{}100.

\item {} 
\sphinxAtStartPar
\sphinxstyleliteralstrong{\sphinxupquote{upper\_percent\_cut}} (\sphinxstyleliteralemphasis{\sphinxupquote{float}}) – The percent of values that will be masked from the upper end. Must be
between 0\sphinxhyphen{}100.

\end{itemize}

\sphinxlineitem{Returns}
\sphinxAtStartPar
\sphinxstylestrong{scaled\_array} – The array, after the scaling.

\sphinxlineitem{Return type}
\sphinxAtStartPar
array\sphinxhyphen{}like

\end{description}\end{quote}

\end{fulllineitems}\end{savenotes}

\index{slice\_array\_boundary() (in module opihiexarata.library.image)@\spxentry{slice\_array\_boundary()}\spxextra{in module opihiexarata.library.image}}

\begin{savenotes}\begin{fulllineitems}
\phantomsection\label{\detokenize{code/opihiexarata.library.image:opihiexarata.library.image.slice_array_boundary}}
\pysigstartsignatures
\pysiglinewithargsret{\sphinxcode{\sphinxupquote{opihiexarata.library.image.}}\sphinxbfcode{\sphinxupquote{slice\_array\_boundary}}}{\emph{\DUrole{n}{array}\DUrole{p}{:}\DUrole{w}{  }\DUrole{n}{ndarray}}, \emph{\DUrole{n}{x\_min}\DUrole{p}{:}\DUrole{w}{  }\DUrole{n}{int}}, \emph{\DUrole{n}{x\_max}\DUrole{p}{:}\DUrole{w}{  }\DUrole{n}{int}}, \emph{\DUrole{n}{y\_min}\DUrole{p}{:}\DUrole{w}{  }\DUrole{n}{int}}, \emph{\DUrole{n}{y\_max}\DUrole{p}{:}\DUrole{w}{  }\DUrole{n}{int}}}{{ $\rightarrow$ ndarray}}
\pysigstopsignatures
\sphinxAtStartPar
Slice an image array such that it stops at the boundaries and does not
exceed past it. This function basically handels runtime slicing, but it
returns a copy.

\sphinxAtStartPar
This function does not wrap around slices.
\begin{quote}\begin{description}
\sphinxlineitem{Parameters}\begin{itemize}
\item {} 
\sphinxAtStartPar
\sphinxstyleliteralstrong{\sphinxupquote{array}} (\sphinxstyleliteralemphasis{\sphinxupquote{array\sphinxhyphen{}like}}) – The base array which the slice will access.

\item {} 
\sphinxAtStartPar
\sphinxstyleliteralstrong{\sphinxupquote{x\_min}} (\sphinxstyleliteralemphasis{\sphinxupquote{int}}) – The lower index bound of the x\sphinxhyphen{}axis slice.

\item {} 
\sphinxAtStartPar
\sphinxstyleliteralstrong{\sphinxupquote{x\_max}} (\sphinxstyleliteralemphasis{\sphinxupquote{int}}) – The upper index bound of the x\sphinxhyphen{}axis slice.

\item {} 
\sphinxAtStartPar
\sphinxstyleliteralstrong{\sphinxupquote{y\_min}} (\sphinxstyleliteralemphasis{\sphinxupquote{int}}) – The lower index bound of the y\sphinxhyphen{}axis slice.

\item {} 
\sphinxAtStartPar
\sphinxstyleliteralstrong{\sphinxupquote{y\_max}} (\sphinxstyleliteralemphasis{\sphinxupquote{int}}) – The upper index bound of the y\sphinxhyphen{}axis slice.

\end{itemize}

\sphinxlineitem{Returns}
\sphinxAtStartPar
\sphinxstylestrong{boundary\_sliced\_array} – The array, sliced while adhering to the boundary of the slices.

\sphinxlineitem{Return type}
\sphinxAtStartPar
array\sphinxhyphen{}like

\end{description}\end{quote}

\end{fulllineitems}\end{savenotes}

\index{translate\_image\_array() (in module opihiexarata.library.image)@\spxentry{translate\_image\_array()}\spxextra{in module opihiexarata.library.image}}

\begin{savenotes}\begin{fulllineitems}
\phantomsection\label{\detokenize{code/opihiexarata.library.image:opihiexarata.library.image.translate_image_array}}
\pysigstartsignatures
\pysiglinewithargsret{\sphinxcode{\sphinxupquote{opihiexarata.library.image.}}\sphinxbfcode{\sphinxupquote{translate\_image\_array}}}{\emph{\DUrole{n}{array}\DUrole{p}{:}\DUrole{w}{  }\DUrole{n}{ndarray}}, \emph{\DUrole{n}{shift\_x}\DUrole{p}{:}\DUrole{w}{  }\DUrole{n}{float}\DUrole{w}{  }\DUrole{o}{=}\DUrole{w}{  }\DUrole{default_value}{0}}, \emph{\DUrole{n}{shift\_y}\DUrole{p}{:}\DUrole{w}{  }\DUrole{n}{float}\DUrole{w}{  }\DUrole{o}{=}\DUrole{w}{  }\DUrole{default_value}{0}}, \emph{\DUrole{n}{pad\_value}\DUrole{p}{:}\DUrole{w}{  }\DUrole{n}{float}\DUrole{w}{  }\DUrole{o}{=}\DUrole{w}{  }\DUrole{default_value}{nan}}}{{ $\rightarrow$ ndarray}}
\pysigstopsignatures
\sphinxAtStartPar
This function translates an image or array in some direction. The image
is treated as value padded so pixels beyond the scope of the image after
translation are given by the value specified.
\begin{quote}\begin{description}
\sphinxlineitem{Parameters}\begin{itemize}
\item {} 
\sphinxAtStartPar
\sphinxstyleliteralstrong{\sphinxupquote{array}} (\sphinxstyleliteralemphasis{\sphinxupquote{array}}) – The image array which is going to be translated.

\item {} 
\sphinxAtStartPar
\sphinxstyleliteralstrong{\sphinxupquote{shift\_x}} (\sphinxstyleliteralemphasis{\sphinxupquote{float}}\sphinxstyleliteralemphasis{\sphinxupquote{, }}\sphinxstyleliteralemphasis{\sphinxupquote{default = 0}}) – The number of pixels the image will be shifted in the x direction.

\item {} 
\sphinxAtStartPar
\sphinxstyleliteralstrong{\sphinxupquote{shift\_y}} (\sphinxstyleliteralemphasis{\sphinxupquote{float}}\sphinxstyleliteralemphasis{\sphinxupquote{, }}\sphinxstyleliteralemphasis{\sphinxupquote{default = 0}}) – The number of pixels the image will be shifted in the y direction.

\item {} 
\sphinxAtStartPar
\sphinxstyleliteralstrong{\sphinxupquote{pad\_value}} (\sphinxstyleliteralemphasis{\sphinxupquote{float}}\sphinxstyleliteralemphasis{\sphinxupquote{, }}\sphinxstyleliteralemphasis{\sphinxupquote{default = np.nan}}) – The value to pad around the image.

\end{itemize}

\sphinxlineitem{Returns}
\sphinxAtStartPar
\sphinxstylestrong{shifted\_image} – The image array after shifting.

\sphinxlineitem{Return type}
\sphinxAtStartPar
array

\end{description}\end{quote}

\end{fulllineitems}\end{savenotes}


\sphinxstepscope


\subparagraph{opihiexarata.library.json module}
\label{\detokenize{code/opihiexarata.library.json:module-opihiexarata.library.json}}\label{\detokenize{code/opihiexarata.library.json:opihiexarata-library-json-module}}\label{\detokenize{code/opihiexarata.library.json::doc}}\index{module@\spxentry{module}!opihiexarata.library.json@\spxentry{opihiexarata.library.json}}\index{opihiexarata.library.json@\spxentry{opihiexarata.library.json}!module@\spxentry{module}}
\sphinxAtStartPar
A collection of functions to deal with JSON input and handling. For the
most part, these functions are just wrappers around the built\sphinxhyphen{}in JSON handling.
\index{dictionary\_to\_json() (in module opihiexarata.library.json)@\spxentry{dictionary\_to\_json()}\spxextra{in module opihiexarata.library.json}}

\begin{savenotes}\begin{fulllineitems}
\phantomsection\label{\detokenize{code/opihiexarata.library.json:opihiexarata.library.json.dictionary_to_json}}
\pysigstartsignatures
\pysiglinewithargsret{\sphinxcode{\sphinxupquote{opihiexarata.library.json.}}\sphinxbfcode{\sphinxupquote{dictionary\_to\_json}}}{\emph{\DUrole{n}{dictionary}\DUrole{p}{:}\DUrole{w}{  }\DUrole{n}{dict}}}{{ $\rightarrow$ str}}
\pysigstopsignatures
\sphinxAtStartPar
Converts a Python dictionary to a JSON string.
\begin{quote}\begin{description}
\sphinxlineitem{Parameters}
\sphinxAtStartPar
\sphinxstyleliteralstrong{\sphinxupquote{dictionary}} (\sphinxstyleliteralemphasis{\sphinxupquote{dict}}) – The Python dictionary which will be converted to a JSON string.

\sphinxlineitem{Returns}
\sphinxAtStartPar
\sphinxstylestrong{json\_string} – The JSON string.

\sphinxlineitem{Return type}
\sphinxAtStartPar
str

\end{description}\end{quote}

\end{fulllineitems}\end{savenotes}

\index{json\_to\_dictionary() (in module opihiexarata.library.json)@\spxentry{json\_to\_dictionary()}\spxextra{in module opihiexarata.library.json}}

\begin{savenotes}\begin{fulllineitems}
\phantomsection\label{\detokenize{code/opihiexarata.library.json:opihiexarata.library.json.json_to_dictionary}}
\pysigstartsignatures
\pysiglinewithargsret{\sphinxcode{\sphinxupquote{opihiexarata.library.json.}}\sphinxbfcode{\sphinxupquote{json\_to\_dictionary}}}{\emph{\DUrole{n}{json\_string}\DUrole{p}{:}\DUrole{w}{  }\DUrole{n}{str}}}{{ $\rightarrow$ dict}}
\pysigstopsignatures
\sphinxAtStartPar
Converts a JSON string to a dictionary.
\begin{quote}\begin{description}
\sphinxlineitem{Parameters}
\sphinxAtStartPar
\sphinxstyleliteralstrong{\sphinxupquote{json\_string}} (\sphinxstyleliteralemphasis{\sphinxupquote{str}}) – The JSON string.

\sphinxlineitem{Returns}
\sphinxAtStartPar
\sphinxstylestrong{dictionary} – The Python dictionary which will be converted to a JSON string.

\sphinxlineitem{Return type}
\sphinxAtStartPar
dict

\end{description}\end{quote}

\end{fulllineitems}\end{savenotes}


\sphinxstepscope


\subparagraph{opihiexarata.library.mpcrecord module}
\label{\detokenize{code/opihiexarata.library.mpcrecord:module-opihiexarata.library.mpcrecord}}\label{\detokenize{code/opihiexarata.library.mpcrecord:opihiexarata-library-mpcrecord-module}}\label{\detokenize{code/opihiexarata.library.mpcrecord::doc}}\index{module@\spxentry{module}!opihiexarata.library.mpcrecord@\spxentry{opihiexarata.library.mpcrecord}}\index{opihiexarata.library.mpcrecord@\spxentry{opihiexarata.library.mpcrecord}!module@\spxentry{module}}\index{blank\_minor\_planet\_table() (in module opihiexarata.library.mpcrecord)@\spxentry{blank\_minor\_planet\_table()}\spxextra{in module opihiexarata.library.mpcrecord}}

\begin{savenotes}\begin{fulllineitems}
\phantomsection\label{\detokenize{code/opihiexarata.library.mpcrecord:opihiexarata.library.mpcrecord.blank_minor_planet_table}}
\pysigstartsignatures
\pysiglinewithargsret{\sphinxcode{\sphinxupquote{opihiexarata.library.mpcrecord.}}\sphinxbfcode{\sphinxupquote{blank\_minor\_planet\_table}}}{}{{ $\rightarrow$ Table}}
\pysigstopsignatures
\sphinxAtStartPar
Creates a blank table which contains the columns which are recognized by
the MPC standard 80\sphinxhyphen{}column record format.
\begin{quote}\begin{description}
\sphinxlineitem{Parameters}
\sphinxAtStartPar
\sphinxstyleliteralstrong{\sphinxupquote{None}} – 

\sphinxlineitem{Returns}
\sphinxAtStartPar
\sphinxstylestrong{blank\_table} – The table with only the column headings; no records are in the table.

\sphinxlineitem{Return type}
\sphinxAtStartPar
Astropy Table

\end{description}\end{quote}

\end{fulllineitems}\end{savenotes}

\index{minor\_planet\_record\_to\_table() (in module opihiexarata.library.mpcrecord)@\spxentry{minor\_planet\_record\_to\_table()}\spxextra{in module opihiexarata.library.mpcrecord}}

\begin{savenotes}\begin{fulllineitems}
\phantomsection\label{\detokenize{code/opihiexarata.library.mpcrecord:opihiexarata.library.mpcrecord.minor_planet_record_to_table}}
\pysigstartsignatures
\pysiglinewithargsret{\sphinxcode{\sphinxupquote{opihiexarata.library.mpcrecord.}}\sphinxbfcode{\sphinxupquote{minor\_planet\_record\_to\_table}}}{\emph{\DUrole{n}{records}\DUrole{p}{:}\DUrole{w}{  }\DUrole{n}{list\DUrole{p}{{[}}str\DUrole{p}{{]}}}}}{{ $\rightarrow$ Table}}
\pysigstopsignatures
\sphinxAtStartPar
This converts an 80 column record for minor planets to a table
representing the same data.

\sphinxAtStartPar
The documentation for how the columns are assigned is provided by the
Minor Planet Center:
\sphinxurl{https://www.minorplanetcenter.net/iau/info/OpticalObs.html}
\begin{quote}\begin{description}
\sphinxlineitem{Parameters}
\sphinxAtStartPar
\sphinxstyleliteralstrong{\sphinxupquote{records}} (\sphinxstyleliteralemphasis{\sphinxupquote{list}}) – The records in MPC format. Each entry of the list should be an 80
column string representing an observed record, a single line.

\sphinxlineitem{Returns}
\sphinxAtStartPar
\sphinxstylestrong{table} – A table containing the same information that is in the MPC record
format in an easier interface.

\sphinxlineitem{Return type}
\sphinxAtStartPar
Astropy Table

\end{description}\end{quote}

\end{fulllineitems}\end{savenotes}

\index{minor\_planet\_table\_to\_record() (in module opihiexarata.library.mpcrecord)@\spxentry{minor\_planet\_table\_to\_record()}\spxextra{in module opihiexarata.library.mpcrecord}}

\begin{savenotes}\begin{fulllineitems}
\phantomsection\label{\detokenize{code/opihiexarata.library.mpcrecord:opihiexarata.library.mpcrecord.minor_planet_table_to_record}}
\pysigstartsignatures
\pysiglinewithargsret{\sphinxcode{\sphinxupquote{opihiexarata.library.mpcrecord.}}\sphinxbfcode{\sphinxupquote{minor\_planet\_table\_to\_record}}}{\emph{\DUrole{n}{table}\DUrole{p}{:}\DUrole{w}{  }\DUrole{n}{Table}}}{{ $\rightarrow$ list\DUrole{p}{{[}}str\DUrole{p}{{]}}}}
\pysigstopsignatures
\sphinxAtStartPar
This converts an 80 column record for minor planets to a table
representing the same data.

\sphinxAtStartPar
This function provides a minimal amount of verification that the input
table is correct. If the provided entry of the table is too long, the text
is striped of whitespace and then ususally cut.

\sphinxAtStartPar
The documentation for how the columns are assigned is provided by the
Minor Planet Center:
\sphinxurl{https://www.minorplanetcenter.net/iau/info/OpticalObs.html}
\begin{quote}\begin{description}
\sphinxlineitem{Parameters}
\sphinxAtStartPar
\sphinxstyleliteralstrong{\sphinxupquote{table}} (\sphinxstyleliteralemphasis{\sphinxupquote{Astropy Table}}) – A table containing the same information that can be written as the
standard 80\sphinxhyphen{}column record.

\sphinxlineitem{Returns}
\sphinxAtStartPar
\sphinxstylestrong{records} – The records in MPC format. Each entry is 80 characters long and are in
the standard format. The entries are derived from the provided table
with information cut to fit into the format.

\sphinxlineitem{Return type}
\sphinxAtStartPar
list

\end{description}\end{quote}

\end{fulllineitems}\end{savenotes}


\sphinxstepscope


\subparagraph{opihiexarata.library.path module}
\label{\detokenize{code/opihiexarata.library.path:module-opihiexarata.library.path}}\label{\detokenize{code/opihiexarata.library.path:opihiexarata-library-path-module}}\label{\detokenize{code/opihiexarata.library.path::doc}}\index{module@\spxentry{module}!opihiexarata.library.path@\spxentry{opihiexarata.library.path}}\index{opihiexarata.library.path@\spxentry{opihiexarata.library.path}!module@\spxentry{module}}
\sphinxAtStartPar
This module is just functions to deal with different common pathname manipulations.
As Exarata is going to be cross platform, this is a nice abstraction.
\index{get\_directory() (in module opihiexarata.library.path)@\spxentry{get\_directory()}\spxextra{in module opihiexarata.library.path}}

\begin{savenotes}\begin{fulllineitems}
\phantomsection\label{\detokenize{code/opihiexarata.library.path:opihiexarata.library.path.get_directory}}
\pysigstartsignatures
\pysiglinewithargsret{\sphinxcode{\sphinxupquote{opihiexarata.library.path.}}\sphinxbfcode{\sphinxupquote{get\_directory}}}{\emph{\DUrole{n}{pathname}\DUrole{p}{:}\DUrole{w}{  }\DUrole{n}{str}}}{{ $\rightarrow$ str}}
\pysigstopsignatures
\sphinxAtStartPar
Get the directory from the pathname without the file or the extension.
\begin{quote}\begin{description}
\sphinxlineitem{Parameters}
\sphinxAtStartPar
\sphinxstyleliteralstrong{\sphinxupquote{pathname}} (\sphinxstyleliteralemphasis{\sphinxupquote{string}}) – The pathname which the directory will be extracted.

\sphinxlineitem{Returns}
\sphinxAtStartPar
\sphinxstylestrong{directory} – The directory which belongs to the pathname.

\sphinxlineitem{Return type}
\sphinxAtStartPar
string

\end{description}\end{quote}

\end{fulllineitems}\end{savenotes}

\index{get\_file\_extension() (in module opihiexarata.library.path)@\spxentry{get\_file\_extension()}\spxextra{in module opihiexarata.library.path}}

\begin{savenotes}\begin{fulllineitems}
\phantomsection\label{\detokenize{code/opihiexarata.library.path:opihiexarata.library.path.get_file_extension}}
\pysigstartsignatures
\pysiglinewithargsret{\sphinxcode{\sphinxupquote{opihiexarata.library.path.}}\sphinxbfcode{\sphinxupquote{get\_file\_extension}}}{\emph{\DUrole{n}{pathname}\DUrole{p}{:}\DUrole{w}{  }\DUrole{n}{str}}}{{ $\rightarrow$ str}}
\pysigstopsignatures
\sphinxAtStartPar
Get the file extension only from the pathname.
\begin{quote}\begin{description}
\sphinxlineitem{Parameters}
\sphinxAtStartPar
\sphinxstyleliteralstrong{\sphinxupquote{pathname}} (\sphinxstyleliteralemphasis{\sphinxupquote{string}}) – The pathname which the file extension will be extracted.

\sphinxlineitem{Returns}
\sphinxAtStartPar
\sphinxstylestrong{extension} – The file extension only.

\sphinxlineitem{Return type}
\sphinxAtStartPar
string

\end{description}\end{quote}

\end{fulllineitems}\end{savenotes}

\index{get\_filename\_with\_extension() (in module opihiexarata.library.path)@\spxentry{get\_filename\_with\_extension()}\spxextra{in module opihiexarata.library.path}}

\begin{savenotes}\begin{fulllineitems}
\phantomsection\label{\detokenize{code/opihiexarata.library.path:opihiexarata.library.path.get_filename_with_extension}}
\pysigstartsignatures
\pysiglinewithargsret{\sphinxcode{\sphinxupquote{opihiexarata.library.path.}}\sphinxbfcode{\sphinxupquote{get\_filename\_with\_extension}}}{\emph{\DUrole{n}{pathname}\DUrole{p}{:}\DUrole{w}{  }\DUrole{n}{str}}}{{ $\rightarrow$ str}}
\pysigstopsignatures
\sphinxAtStartPar
Get the filename from the pathname with the file extension.
\begin{quote}\begin{description}
\sphinxlineitem{Parameters}
\sphinxAtStartPar
\sphinxstyleliteralstrong{\sphinxupquote{pathname}} (\sphinxstyleliteralemphasis{\sphinxupquote{string}}) – The pathname which the filename will be extracted.

\sphinxlineitem{Returns}
\sphinxAtStartPar
\sphinxstylestrong{filename} – The filename with the file extension.

\sphinxlineitem{Return type}
\sphinxAtStartPar
string

\end{description}\end{quote}

\end{fulllineitems}\end{savenotes}

\index{get\_filename\_without\_extension() (in module opihiexarata.library.path)@\spxentry{get\_filename\_without\_extension()}\spxextra{in module opihiexarata.library.path}}

\begin{savenotes}\begin{fulllineitems}
\phantomsection\label{\detokenize{code/opihiexarata.library.path:opihiexarata.library.path.get_filename_without_extension}}
\pysigstartsignatures
\pysiglinewithargsret{\sphinxcode{\sphinxupquote{opihiexarata.library.path.}}\sphinxbfcode{\sphinxupquote{get\_filename\_without\_extension}}}{\emph{\DUrole{n}{pathname}\DUrole{p}{:}\DUrole{w}{  }\DUrole{n}{str}}}{{ $\rightarrow$ str}}
\pysigstopsignatures
\sphinxAtStartPar
Get the filename from the pathname without the file extension.
\begin{quote}\begin{description}
\sphinxlineitem{Parameters}
\sphinxAtStartPar
\sphinxstyleliteralstrong{\sphinxupquote{pathname}} (\sphinxstyleliteralemphasis{\sphinxupquote{string}}) – The pathname which the filename will be extracted.

\sphinxlineitem{Returns}
\sphinxAtStartPar
\sphinxstylestrong{filename} – The filename without the file extension.

\sphinxlineitem{Return type}
\sphinxAtStartPar
string

\end{description}\end{quote}

\end{fulllineitems}\end{savenotes}

\index{get\_most\_recent\_filename\_in\_directory() (in module opihiexarata.library.path)@\spxentry{get\_most\_recent\_filename\_in\_directory()}\spxextra{in module opihiexarata.library.path}}

\begin{savenotes}\begin{fulllineitems}
\phantomsection\label{\detokenize{code/opihiexarata.library.path:opihiexarata.library.path.get_most_recent_filename_in_directory}}
\pysigstartsignatures
\pysiglinewithargsret{\sphinxcode{\sphinxupquote{opihiexarata.library.path.}}\sphinxbfcode{\sphinxupquote{get\_most\_recent\_filename\_in\_directory}}}{\emph{\DUrole{n}{directory}\DUrole{p}{:}\DUrole{w}{  }\DUrole{n}{str}}, \emph{\DUrole{n}{extension}\DUrole{p}{:}\DUrole{w}{  }\DUrole{n}{Optional\DUrole{p}{{[}}Union\DUrole{p}{{[}}str\DUrole{p}{,}\DUrole{w}{  }list\DUrole{p}{{]}}\DUrole{p}{{]}}}\DUrole{w}{  }\DUrole{o}{=}\DUrole{w}{  }\DUrole{default_value}{None}}}{{ $\rightarrow$ str}}
\pysigstopsignatures
\sphinxAtStartPar
This gets the most recent filename from a directory.

\sphinxAtStartPar
Because of issues with different operating systems having differing
issues with storing the creation time of a file, this function sorts based
off of modification time. The most recent modified file is thus
\begin{quote}\begin{description}
\sphinxlineitem{Parameters}\begin{itemize}
\item {} 
\sphinxAtStartPar
\sphinxstyleliteralstrong{\sphinxupquote{directory}} (\sphinxstyleliteralemphasis{\sphinxupquote{string}}) – The directory by which the most recent file will be derived from.

\item {} 
\sphinxAtStartPar
\sphinxstyleliteralstrong{\sphinxupquote{extension}} (\sphinxstyleliteralemphasis{\sphinxupquote{string}}\sphinxstyleliteralemphasis{\sphinxupquote{ or }}\sphinxstyleliteralemphasis{\sphinxupquote{list}}) – The extension by which to filter for. It is often the case that some
files are created but the most recent file of some type is desired.
Only files which match the included extensions will be considered.

\end{itemize}

\sphinxlineitem{Returns}
\sphinxAtStartPar
\sphinxstylestrong{recent\_filename} – The filename of the most recent file, by modification time, in the
directory.

\sphinxlineitem{Return type}
\sphinxAtStartPar
string

\end{description}\end{quote}

\end{fulllineitems}\end{savenotes}

\index{merge\_pathname() (in module opihiexarata.library.path)@\spxentry{merge\_pathname()}\spxextra{in module opihiexarata.library.path}}

\begin{savenotes}\begin{fulllineitems}
\phantomsection\label{\detokenize{code/opihiexarata.library.path:opihiexarata.library.path.merge_pathname}}
\pysigstartsignatures
\pysiglinewithargsret{\sphinxcode{\sphinxupquote{opihiexarata.library.path.}}\sphinxbfcode{\sphinxupquote{merge\_pathname}}}{\emph{\DUrole{n}{directory}\DUrole{p}{:}\DUrole{w}{  }\DUrole{n}{Optional\DUrole{p}{{[}}Union\DUrole{p}{{[}}str\DUrole{p}{,}\DUrole{w}{  }list\DUrole{p}{{]}}\DUrole{p}{{]}}}\DUrole{w}{  }\DUrole{o}{=}\DUrole{w}{  }\DUrole{default_value}{None}}, \emph{\DUrole{n}{filename}\DUrole{p}{:}\DUrole{w}{  }\DUrole{n}{Optional\DUrole{p}{{[}}str\DUrole{p}{{]}}}\DUrole{w}{  }\DUrole{o}{=}\DUrole{w}{  }\DUrole{default_value}{None}}, \emph{\DUrole{n}{extension}\DUrole{p}{:}\DUrole{w}{  }\DUrole{n}{Optional\DUrole{p}{{[}}str\DUrole{p}{{]}}}\DUrole{w}{  }\DUrole{o}{=}\DUrole{w}{  }\DUrole{default_value}{None}}}{{ $\rightarrow$ str}}
\pysigstopsignatures
\sphinxAtStartPar
Joins directories, filenames, and file extensions into one pathname.
\begin{quote}\begin{description}
\sphinxlineitem{Parameters}\begin{itemize}
\item {} 
\sphinxAtStartPar
\sphinxstyleliteralstrong{\sphinxupquote{directory}} (\sphinxstyleliteralemphasis{\sphinxupquote{string}}\sphinxstyleliteralemphasis{\sphinxupquote{ or }}\sphinxstyleliteralemphasis{\sphinxupquote{list}}\sphinxstyleliteralemphasis{\sphinxupquote{, }}\sphinxstyleliteralemphasis{\sphinxupquote{default = None}}) – The directory(s) which is going to be used. If it is a list,
then the paths within it are combined.

\item {} 
\sphinxAtStartPar
\sphinxstyleliteralstrong{\sphinxupquote{filename}} (\sphinxstyleliteralemphasis{\sphinxupquote{string}}\sphinxstyleliteralemphasis{\sphinxupquote{, }}\sphinxstyleliteralemphasis{\sphinxupquote{default = None}}) – The filename that is going to be used for path construction.

\item {} 
\sphinxAtStartPar
\sphinxstyleliteralstrong{\sphinxupquote{extension}} (\sphinxstyleliteralemphasis{\sphinxupquote{string}}\sphinxstyleliteralemphasis{\sphinxupquote{, }}\sphinxstyleliteralemphasis{\sphinxupquote{default = None}}) – The filename extension that is going to be used.

\end{itemize}

\sphinxlineitem{Returns}
\sphinxAtStartPar
\sphinxstylestrong{pathname} – The combined pathname.

\sphinxlineitem{Return type}
\sphinxAtStartPar
string

\end{description}\end{quote}

\end{fulllineitems}\end{savenotes}

\index{split\_pathname() (in module opihiexarata.library.path)@\spxentry{split\_pathname()}\spxextra{in module opihiexarata.library.path}}

\begin{savenotes}\begin{fulllineitems}
\phantomsection\label{\detokenize{code/opihiexarata.library.path:opihiexarata.library.path.split_pathname}}
\pysigstartsignatures
\pysiglinewithargsret{\sphinxcode{\sphinxupquote{opihiexarata.library.path.}}\sphinxbfcode{\sphinxupquote{split\_pathname}}}{\emph{\DUrole{n}{pathname}\DUrole{p}{:}\DUrole{w}{  }\DUrole{n}{str}}}{{ $\rightarrow$ tuple\DUrole{p}{{[}}str\DUrole{p}{,}\DUrole{w}{  }str\DUrole{p}{,}\DUrole{w}{  }str\DUrole{p}{{]}}}}
\pysigstopsignatures
\sphinxAtStartPar
Splits a path into a directory, filename, and file extension.

\sphinxAtStartPar
This is a wrapper function around the more elementry functions
\sphinxtitleref{get\_directory}, \sphinxtitleref{get\_filename\_without\_extension}, and
\sphinxtitleref{get\_file\_extension}.
\begin{quote}\begin{description}
\sphinxlineitem{Parameters}
\sphinxAtStartPar
\sphinxstyleliteralstrong{\sphinxupquote{pathname}} (\sphinxstyleliteralemphasis{\sphinxupquote{string}}) – The combined pathname which to be split.

\sphinxlineitem{Returns}
\sphinxAtStartPar
\begin{itemize}
\item {} 
\sphinxAtStartPar
\sphinxstylestrong{directory} (\sphinxstyleemphasis{string}) – The directory which was split from the pathname.

\item {} 
\sphinxAtStartPar
\sphinxstylestrong{filename} (\sphinxstyleemphasis{string}) – The filename which was split from the pathname.

\item {} 
\sphinxAtStartPar
\sphinxstylestrong{extension} (\sphinxstyleemphasis{string}) – The filename extension which was split from the pathname.

\end{itemize}


\end{description}\end{quote}

\end{fulllineitems}\end{savenotes}


\sphinxstepscope


\subparagraph{opihiexarata.library.phototable module}
\label{\detokenize{code/opihiexarata.library.phototable:module-opihiexarata.library.phototable}}\label{\detokenize{code/opihiexarata.library.phototable:opihiexarata-library-phototable-module}}\label{\detokenize{code/opihiexarata.library.phototable::doc}}\index{module@\spxentry{module}!opihiexarata.library.phototable@\spxentry{opihiexarata.library.phototable}}\index{opihiexarata.library.phototable@\spxentry{opihiexarata.library.phototable}!module@\spxentry{module}}
\sphinxAtStartPar
This is a module for uniform handling of the photometric star table.
Many different databases will have their own standards and so functions and
classes helpful for unifying all of their different entries into one uniform
table which the software can expect are detailed here.
\index{blank\_photometry\_table() (in module opihiexarata.library.phototable)@\spxentry{blank\_photometry\_table()}\spxextra{in module opihiexarata.library.phototable}}

\begin{savenotes}\begin{fulllineitems}
\phantomsection\label{\detokenize{code/opihiexarata.library.phototable:opihiexarata.library.phototable.blank_photometry_table}}
\pysigstartsignatures
\pysiglinewithargsret{\sphinxcode{\sphinxupquote{opihiexarata.library.phototable.}}\sphinxbfcode{\sphinxupquote{blank\_photometry\_table}}}{}{{ $\rightarrow$ Table}}
\pysigstopsignatures
\sphinxAtStartPar
Creates a blank table which contains the columns which are the unified
photometry table.
\begin{quote}\begin{description}
\sphinxlineitem{Parameters}
\sphinxAtStartPar
\sphinxstyleliteralstrong{\sphinxupquote{None}} – 

\sphinxlineitem{Returns}
\sphinxAtStartPar
\sphinxstylestrong{blank\_table} – The table with only the column headings; no records are in the table.

\sphinxlineitem{Return type}
\sphinxAtStartPar
Astropy Table

\end{description}\end{quote}

\end{fulllineitems}\end{savenotes}

\index{fill\_incomplete\_photometry\_table() (in module opihiexarata.library.phototable)@\spxentry{fill\_incomplete\_photometry\_table()}\spxextra{in module opihiexarata.library.phototable}}

\begin{savenotes}\begin{fulllineitems}
\phantomsection\label{\detokenize{code/opihiexarata.library.phototable:opihiexarata.library.phototable.fill_incomplete_photometry_table}}
\pysigstartsignatures
\pysiglinewithargsret{\sphinxcode{\sphinxupquote{opihiexarata.library.phototable.}}\sphinxbfcode{\sphinxupquote{fill\_incomplete\_photometry\_table}}}{\emph{\DUrole{n}{partial\_table}\DUrole{p}{:}\DUrole{w}{  }\DUrole{n}{Table}}}{{ $\rightarrow$ Table}}
\pysigstopsignatures
\sphinxAtStartPar
This function takes a photometry table which is partially standardized
but may be missing a few columns or rows and fills in the missing values
with NaNs so that the resulting table is a standardized table for the
OpihiExarata software.

\sphinxAtStartPar
Metadata is added to the resulting table to signify that it is a complete
photometry table as standardized by OpihiExarata.
\begin{quote}\begin{description}
\sphinxlineitem{Parameters}\begin{itemize}
\item {} 
\sphinxAtStartPar
\sphinxstyleliteralstrong{\sphinxupquote{partial\_table}} (\sphinxstyleliteralemphasis{\sphinxupquote{Table}}) – A table which partially covers the photometry table standard but is
otherwise missing a few records.

\item {} 
\sphinxAtStartPar
\sphinxstyleliteralstrong{\sphinxupquote{complete\_table}} (\sphinxstyleliteralemphasis{\sphinxupquote{Table}}) – A completed form of the partial table which is standardized to the
expectations of the photometry table.

\end{itemize}

\end{description}\end{quote}

\end{fulllineitems}\end{savenotes}


\sphinxstepscope


\subparagraph{opihiexarata.library.temporary module}
\label{\detokenize{code/opihiexarata.library.temporary:module-opihiexarata.library.temporary}}\label{\detokenize{code/opihiexarata.library.temporary:opihiexarata-library-temporary-module}}\label{\detokenize{code/opihiexarata.library.temporary::doc}}\index{module@\spxentry{module}!opihiexarata.library.temporary@\spxentry{opihiexarata.library.temporary}}\index{opihiexarata.library.temporary@\spxentry{opihiexarata.library.temporary}!module@\spxentry{module}}
\sphinxAtStartPar
This is where functions dealing with the temporary files and temporary
directory of the OpihiExarata system. Temporary files are helpful because they
may also contain information useful to the user. These functions thus serve the
same purpose as Python’s build\sphinxhyphen{}in functions, but it is more restricted to
OpihiExarata and it is also more persistant.
\index{create\_temporary\_directory() (in module opihiexarata.library.temporary)@\spxentry{create\_temporary\_directory()}\spxextra{in module opihiexarata.library.temporary}}

\begin{savenotes}\begin{fulllineitems}
\phantomsection\label{\detokenize{code/opihiexarata.library.temporary:opihiexarata.library.temporary.create_temporary_directory}}
\pysigstartsignatures
\pysiglinewithargsret{\sphinxcode{\sphinxupquote{opihiexarata.library.temporary.}}\sphinxbfcode{\sphinxupquote{create\_temporary\_directory}}}{\emph{\DUrole{n}{unique}\DUrole{p}{:}\DUrole{w}{  }\DUrole{n}{Optional\DUrole{p}{{[}}bool\DUrole{p}{{]}}}\DUrole{w}{  }\DUrole{o}{=}\DUrole{w}{  }\DUrole{default_value}{None}}}{{ $\rightarrow$ None}}
\pysigstopsignatures
\sphinxAtStartPar
Make the temporary directory.
\begin{quote}\begin{description}
\sphinxlineitem{Parameters}
\sphinxAtStartPar
\sphinxstyleliteralstrong{\sphinxupquote{unique}} (\sphinxstyleliteralemphasis{\sphinxupquote{bool}}\sphinxstyleliteralemphasis{\sphinxupquote{, }}\sphinxstyleliteralemphasis{\sphinxupquote{default = None}}) – Require a check on the creation of the directory to require it to be
unique. If True, this will raise if the directory already exists
otherwise it does not care. Will defer to the configuration file
if None.

\sphinxlineitem{Return type}
\sphinxAtStartPar
None

\end{description}\end{quote}

\end{fulllineitems}\end{savenotes}

\index{delete\_temporary\_directory() (in module opihiexarata.library.temporary)@\spxentry{delete\_temporary\_directory()}\spxextra{in module opihiexarata.library.temporary}}

\begin{savenotes}\begin{fulllineitems}
\phantomsection\label{\detokenize{code/opihiexarata.library.temporary:opihiexarata.library.temporary.delete_temporary_directory}}
\pysigstartsignatures
\pysiglinewithargsret{\sphinxcode{\sphinxupquote{opihiexarata.library.temporary.}}\sphinxbfcode{\sphinxupquote{delete\_temporary\_directory}}}{}{{ $\rightarrow$ None}}
\pysigstopsignatures
\sphinxAtStartPar
Delete the temporary directory. If the directory does not exist, this
function will do nothing. If the directory exists, but contains files, then
this function will fail.
\begin{quote}\begin{description}
\sphinxlineitem{Parameters}
\sphinxAtStartPar
\sphinxstyleliteralstrong{\sphinxupquote{None}} – 

\sphinxlineitem{Return type}
\sphinxAtStartPar
None

\end{description}\end{quote}

\end{fulllineitems}\end{savenotes}

\index{make\_temporary\_directory\_path() (in module opihiexarata.library.temporary)@\spxentry{make\_temporary\_directory\_path()}\spxextra{in module opihiexarata.library.temporary}}

\begin{savenotes}\begin{fulllineitems}
\phantomsection\label{\detokenize{code/opihiexarata.library.temporary:opihiexarata.library.temporary.make_temporary_directory_path}}
\pysigstartsignatures
\pysiglinewithargsret{\sphinxcode{\sphinxupquote{opihiexarata.library.temporary.}}\sphinxbfcode{\sphinxupquote{make\_temporary\_directory\_path}}}{\emph{\DUrole{n}{filename}\DUrole{p}{:}\DUrole{w}{  }\DUrole{n}{str}}}{{ $\rightarrow$ str}}
\pysigstopsignatures
\sphinxAtStartPar
Creates a full filename path to use. This function basically adds the
temporary directory path prefix to place the filename path into it.
\begin{quote}\begin{description}
\sphinxlineitem{Parameters}
\sphinxAtStartPar
\sphinxstyleliteralstrong{\sphinxupquote{filename}} (\sphinxstyleliteralemphasis{\sphinxupquote{string}}) – The filename of the target to be placed in the temporary directory.

\sphinxlineitem{Returns}
\sphinxAtStartPar
\sphinxstylestrong{full\_path} – The full path of the file, as it would be stored in the temporary
directory.

\sphinxlineitem{Return type}
\sphinxAtStartPar
string

\end{description}\end{quote}

\end{fulllineitems}\end{savenotes}

\index{purge\_temporary\_directory() (in module opihiexarata.library.temporary)@\spxentry{purge\_temporary\_directory()}\spxextra{in module opihiexarata.library.temporary}}

\begin{savenotes}\begin{fulllineitems}
\phantomsection\label{\detokenize{code/opihiexarata.library.temporary:opihiexarata.library.temporary.purge_temporary_directory}}
\pysigstartsignatures
\pysiglinewithargsret{\sphinxcode{\sphinxupquote{opihiexarata.library.temporary.}}\sphinxbfcode{\sphinxupquote{purge\_temporary\_directory}}}{}{{ $\rightarrow$ None}}
\pysigstopsignatures
\sphinxAtStartPar
Delete or purge all files in the temporary directory. This does it
recursively.
\begin{quote}\begin{description}
\sphinxlineitem{Parameters}
\sphinxAtStartPar
\sphinxstyleliteralstrong{\sphinxupquote{None}} – 

\sphinxlineitem{Return type}
\sphinxAtStartPar
None

\end{description}\end{quote}

\end{fulllineitems}\end{savenotes}



\subparagraph{Module contents}
\label{\detokenize{code/opihiexarata.library:module-opihiexarata.library}}\label{\detokenize{code/opihiexarata.library:module-contents}}\index{module@\spxentry{module}!opihiexarata.library@\spxentry{opihiexarata.library}}\index{opihiexarata.library@\spxentry{opihiexarata.library}!module@\spxentry{module}}
\sphinxAtStartPar
Common routines which are important functions of Exarata.

\sphinxstepscope


\paragraph{opihiexarata.opihi package}
\label{\detokenize{code/opihiexarata.opihi:opihiexarata-opihi-package}}\label{\detokenize{code/opihiexarata.opihi::doc}}

\subparagraph{Submodules}
\label{\detokenize{code/opihiexarata.opihi:submodules}}
\sphinxstepscope


\subparagraph{opihiexarata.opihi.preprocess module}
\label{\detokenize{code/opihiexarata.opihi.preprocess:module-opihiexarata.opihi.preprocess}}\label{\detokenize{code/opihiexarata.opihi.preprocess:opihiexarata-opihi-preprocess-module}}\label{\detokenize{code/opihiexarata.opihi.preprocess::doc}}\index{module@\spxentry{module}!opihiexarata.opihi.preprocess@\spxentry{opihiexarata.opihi.preprocess}}\index{opihiexarata.opihi.preprocess@\spxentry{opihiexarata.opihi.preprocess}!module@\spxentry{module}}
\sphinxAtStartPar
A data wrapper class which takes in raw Opihi data, flats, and darks and
produces a valid reduced image.
\index{OpihiPreprocessSolution (class in opihiexarata.opihi.preprocess)@\spxentry{OpihiPreprocessSolution}\spxextra{class in opihiexarata.opihi.preprocess}}

\begin{savenotes}\begin{fulllineitems}
\phantomsection\label{\detokenize{code/opihiexarata.opihi.preprocess:opihiexarata.opihi.preprocess.OpihiPreprocessSolution}}
\pysigstartsignatures
\pysiglinewithargsret{\sphinxbfcode{\sphinxupquote{class\DUrole{w}{  }}}\sphinxcode{\sphinxupquote{opihiexarata.opihi.preprocess.}}\sphinxbfcode{\sphinxupquote{OpihiPreprocessSolution}}}{\emph{\DUrole{n}{mask\_c\_fits\_filename}\DUrole{p}{:}\DUrole{w}{  }\DUrole{n}{str}}, \emph{\DUrole{n}{mask\_g\_fits\_filename}\DUrole{p}{:}\DUrole{w}{  }\DUrole{n}{str}}, \emph{\DUrole{n}{mask\_r\_fits\_filename}\DUrole{p}{:}\DUrole{w}{  }\DUrole{n}{str}}, \emph{\DUrole{n}{mask\_i\_fits\_filename}\DUrole{p}{:}\DUrole{w}{  }\DUrole{n}{str}}, \emph{\DUrole{n}{mask\_z\_fits\_filename}\DUrole{p}{:}\DUrole{w}{  }\DUrole{n}{str}}, \emph{\DUrole{n}{mask\_1\_fits\_filename}\DUrole{p}{:}\DUrole{w}{  }\DUrole{n}{str}}, \emph{\DUrole{n}{mask\_2\_fits\_filename}\DUrole{p}{:}\DUrole{w}{  }\DUrole{n}{str}}, \emph{\DUrole{n}{mask\_3\_fits\_filename}\DUrole{p}{:}\DUrole{w}{  }\DUrole{n}{str}}, \emph{\DUrole{n}{flat\_c\_fits\_filename}\DUrole{p}{:}\DUrole{w}{  }\DUrole{n}{str}}, \emph{\DUrole{n}{flat\_g\_fits\_filename}\DUrole{p}{:}\DUrole{w}{  }\DUrole{n}{str}}, \emph{\DUrole{n}{flat\_r\_fits\_filename}\DUrole{p}{:}\DUrole{w}{  }\DUrole{n}{str}}, \emph{\DUrole{n}{flat\_i\_fits\_filename}\DUrole{p}{:}\DUrole{w}{  }\DUrole{n}{str}}, \emph{\DUrole{n}{flat\_z\_fits\_filename}\DUrole{p}{:}\DUrole{w}{  }\DUrole{n}{str}}, \emph{\DUrole{n}{flat\_1\_fits\_filename}\DUrole{p}{:}\DUrole{w}{  }\DUrole{n}{str}}, \emph{\DUrole{n}{flat\_2\_fits\_filename}\DUrole{p}{:}\DUrole{w}{  }\DUrole{n}{str}}, \emph{\DUrole{n}{flat\_3\_fits\_filename}\DUrole{p}{:}\DUrole{w}{  }\DUrole{n}{str}}, \emph{\DUrole{n}{bias\_fits\_filename}\DUrole{p}{:}\DUrole{w}{  }\DUrole{n}{str}}, \emph{\DUrole{n}{dark\_current\_fits\_filename}\DUrole{p}{:}\DUrole{w}{  }\DUrole{n}{str}}, \emph{\DUrole{n}{linearity\_fits\_filename}\DUrole{p}{:}\DUrole{w}{  }\DUrole{n}{str}}}{}
\pysigstopsignatures
\sphinxAtStartPar
Bases: {\hyperref[\detokenize{code/opihiexarata.library.engine:opihiexarata.library.engine.ExarataSolution}]{\sphinxcrossref{\sphinxcode{\sphinxupquote{ExarataSolution}}}}}

\sphinxAtStartPar
A class which represents the reduction process of Opihi data, having the
raw data corrected using previously and provided derived flats and darks.
The required parameters (such as exposure time) must also be provided.

\sphinxAtStartPar
This class does not have an engine as there is only one way to reduce data
provided the systematics of the Opihi telescope itself; as such the
data is handled straight by this solution class.
\index{\_mask\_c\_fits\_filename (opihiexarata.opihi.preprocess.OpihiPreprocessSolution attribute)@\spxentry{\_mask\_c\_fits\_filename}\spxextra{opihiexarata.opihi.preprocess.OpihiPreprocessSolution attribute}}

\begin{savenotes}\begin{fulllineitems}
\phantomsection\label{\detokenize{code/opihiexarata.opihi.preprocess:opihiexarata.opihi.preprocess.OpihiPreprocessSolution._mask_c_fits_filename}}
\pysigstartsignatures
\pysigline{\sphinxbfcode{\sphinxupquote{\_mask\_c\_fits\_filename}}}
\pysigstopsignatures
\sphinxAtStartPar
The filename for the pixel mask for the clear filter stored in a
fits file.
\begin{quote}\begin{description}
\sphinxlineitem{Type}
\sphinxAtStartPar
string

\end{description}\end{quote}

\end{fulllineitems}\end{savenotes}

\index{\_mask\_g\_fits\_filename (opihiexarata.opihi.preprocess.OpihiPreprocessSolution attribute)@\spxentry{\_mask\_g\_fits\_filename}\spxextra{opihiexarata.opihi.preprocess.OpihiPreprocessSolution attribute}}

\begin{savenotes}\begin{fulllineitems}
\phantomsection\label{\detokenize{code/opihiexarata.opihi.preprocess:opihiexarata.opihi.preprocess.OpihiPreprocessSolution._mask_g_fits_filename}}
\pysigstartsignatures
\pysigline{\sphinxbfcode{\sphinxupquote{\_mask\_g\_fits\_filename}}}
\pysigstopsignatures
\sphinxAtStartPar
The filename for the pixel mask for the g filter stored in a
fits file.
\begin{quote}\begin{description}
\sphinxlineitem{Type}
\sphinxAtStartPar
string

\end{description}\end{quote}

\end{fulllineitems}\end{savenotes}

\index{\_mask\_r\_fits\_filename (opihiexarata.opihi.preprocess.OpihiPreprocessSolution attribute)@\spxentry{\_mask\_r\_fits\_filename}\spxextra{opihiexarata.opihi.preprocess.OpihiPreprocessSolution attribute}}

\begin{savenotes}\begin{fulllineitems}
\phantomsection\label{\detokenize{code/opihiexarata.opihi.preprocess:opihiexarata.opihi.preprocess.OpihiPreprocessSolution._mask_r_fits_filename}}
\pysigstartsignatures
\pysigline{\sphinxbfcode{\sphinxupquote{\_mask\_r\_fits\_filename}}}
\pysigstopsignatures
\sphinxAtStartPar
The filename for the pixel mask for the r filter stored in a
fits file.
\begin{quote}\begin{description}
\sphinxlineitem{Type}
\sphinxAtStartPar
string

\end{description}\end{quote}

\end{fulllineitems}\end{savenotes}

\index{\_mask\_i\_fits\_filename (opihiexarata.opihi.preprocess.OpihiPreprocessSolution attribute)@\spxentry{\_mask\_i\_fits\_filename}\spxextra{opihiexarata.opihi.preprocess.OpihiPreprocessSolution attribute}}

\begin{savenotes}\begin{fulllineitems}
\phantomsection\label{\detokenize{code/opihiexarata.opihi.preprocess:opihiexarata.opihi.preprocess.OpihiPreprocessSolution._mask_i_fits_filename}}
\pysigstartsignatures
\pysigline{\sphinxbfcode{\sphinxupquote{\_mask\_i\_fits\_filename}}}
\pysigstopsignatures
\sphinxAtStartPar
The filename for the pixel mask for the i filter stored in a
fits file.
\begin{quote}\begin{description}
\sphinxlineitem{Type}
\sphinxAtStartPar
string

\end{description}\end{quote}

\end{fulllineitems}\end{savenotes}

\index{\_mask\_z\_fits\_filename (opihiexarata.opihi.preprocess.OpihiPreprocessSolution attribute)@\spxentry{\_mask\_z\_fits\_filename}\spxextra{opihiexarata.opihi.preprocess.OpihiPreprocessSolution attribute}}

\begin{savenotes}\begin{fulllineitems}
\phantomsection\label{\detokenize{code/opihiexarata.opihi.preprocess:opihiexarata.opihi.preprocess.OpihiPreprocessSolution._mask_z_fits_filename}}
\pysigstartsignatures
\pysigline{\sphinxbfcode{\sphinxupquote{\_mask\_z\_fits\_filename}}}
\pysigstopsignatures
\sphinxAtStartPar
The filename for the pixel mask for the z filter stored in a
fits file.
\begin{quote}\begin{description}
\sphinxlineitem{Type}
\sphinxAtStartPar
string

\end{description}\end{quote}

\end{fulllineitems}\end{savenotes}

\index{\_mask\_1\_fits\_filename (opihiexarata.opihi.preprocess.OpihiPreprocessSolution attribute)@\spxentry{\_mask\_1\_fits\_filename}\spxextra{opihiexarata.opihi.preprocess.OpihiPreprocessSolution attribute}}

\begin{savenotes}\begin{fulllineitems}
\phantomsection\label{\detokenize{code/opihiexarata.opihi.preprocess:opihiexarata.opihi.preprocess.OpihiPreprocessSolution._mask_1_fits_filename}}
\pysigstartsignatures
\pysigline{\sphinxbfcode{\sphinxupquote{\_mask\_1\_fits\_filename}}}
\pysigstopsignatures
\sphinxAtStartPar
The filename for the pixel mask for the 1 filter stored in a
fits file.
\begin{quote}\begin{description}
\sphinxlineitem{Type}
\sphinxAtStartPar
string

\end{description}\end{quote}

\end{fulllineitems}\end{savenotes}

\index{\_mask\_2\_fits\_filename (opihiexarata.opihi.preprocess.OpihiPreprocessSolution attribute)@\spxentry{\_mask\_2\_fits\_filename}\spxextra{opihiexarata.opihi.preprocess.OpihiPreprocessSolution attribute}}

\begin{savenotes}\begin{fulllineitems}
\phantomsection\label{\detokenize{code/opihiexarata.opihi.preprocess:opihiexarata.opihi.preprocess.OpihiPreprocessSolution._mask_2_fits_filename}}
\pysigstartsignatures
\pysigline{\sphinxbfcode{\sphinxupquote{\_mask\_2\_fits\_filename}}}
\pysigstopsignatures
\sphinxAtStartPar
The filename for the pixel mask for the 2 filter stored in a
fits file.
\begin{quote}\begin{description}
\sphinxlineitem{Type}
\sphinxAtStartPar
string

\end{description}\end{quote}

\end{fulllineitems}\end{savenotes}

\index{\_mask\_3\_fits\_filename (opihiexarata.opihi.preprocess.OpihiPreprocessSolution attribute)@\spxentry{\_mask\_3\_fits\_filename}\spxextra{opihiexarata.opihi.preprocess.OpihiPreprocessSolution attribute}}

\begin{savenotes}\begin{fulllineitems}
\phantomsection\label{\detokenize{code/opihiexarata.opihi.preprocess:opihiexarata.opihi.preprocess.OpihiPreprocessSolution._mask_3_fits_filename}}
\pysigstartsignatures
\pysigline{\sphinxbfcode{\sphinxupquote{\_mask\_3\_fits\_filename}}}
\pysigstopsignatures
\sphinxAtStartPar
The filename for the pixel mask for the 3 filter stored in a
fits file.
\begin{quote}\begin{description}
\sphinxlineitem{Type}
\sphinxAtStartPar
string

\end{description}\end{quote}

\end{fulllineitems}\end{savenotes}

\index{\_flat\_c\_fits\_filename (opihiexarata.opihi.preprocess.OpihiPreprocessSolution attribute)@\spxentry{\_flat\_c\_fits\_filename}\spxextra{opihiexarata.opihi.preprocess.OpihiPreprocessSolution attribute}}

\begin{savenotes}\begin{fulllineitems}
\phantomsection\label{\detokenize{code/opihiexarata.opihi.preprocess:opihiexarata.opihi.preprocess.OpihiPreprocessSolution._flat_c_fits_filename}}
\pysigstartsignatures
\pysigline{\sphinxbfcode{\sphinxupquote{\_flat\_c\_fits\_filename}}}
\pysigstopsignatures
\sphinxAtStartPar
The filename for the flat field for the clear filter stored in a
fits file.
\begin{quote}\begin{description}
\sphinxlineitem{Type}
\sphinxAtStartPar
string

\end{description}\end{quote}

\end{fulllineitems}\end{savenotes}

\index{\_flat\_g\_fits\_filename (opihiexarata.opihi.preprocess.OpihiPreprocessSolution attribute)@\spxentry{\_flat\_g\_fits\_filename}\spxextra{opihiexarata.opihi.preprocess.OpihiPreprocessSolution attribute}}

\begin{savenotes}\begin{fulllineitems}
\phantomsection\label{\detokenize{code/opihiexarata.opihi.preprocess:opihiexarata.opihi.preprocess.OpihiPreprocessSolution._flat_g_fits_filename}}
\pysigstartsignatures
\pysigline{\sphinxbfcode{\sphinxupquote{\_flat\_g\_fits\_filename}}}
\pysigstopsignatures
\sphinxAtStartPar
The filename for the flat field for the g filter stored in a
fits file.
\begin{quote}\begin{description}
\sphinxlineitem{Type}
\sphinxAtStartPar
string

\end{description}\end{quote}

\end{fulllineitems}\end{savenotes}

\index{\_flat\_r\_fits\_filename (opihiexarata.opihi.preprocess.OpihiPreprocessSolution attribute)@\spxentry{\_flat\_r\_fits\_filename}\spxextra{opihiexarata.opihi.preprocess.OpihiPreprocessSolution attribute}}

\begin{savenotes}\begin{fulllineitems}
\phantomsection\label{\detokenize{code/opihiexarata.opihi.preprocess:opihiexarata.opihi.preprocess.OpihiPreprocessSolution._flat_r_fits_filename}}
\pysigstartsignatures
\pysigline{\sphinxbfcode{\sphinxupquote{\_flat\_r\_fits\_filename}}}
\pysigstopsignatures
\sphinxAtStartPar
The filename for the flat field for the r filter stored in a
fits file.
\begin{quote}\begin{description}
\sphinxlineitem{Type}
\sphinxAtStartPar
string

\end{description}\end{quote}

\end{fulllineitems}\end{savenotes}

\index{\_flat\_i\_fits\_filename (opihiexarata.opihi.preprocess.OpihiPreprocessSolution attribute)@\spxentry{\_flat\_i\_fits\_filename}\spxextra{opihiexarata.opihi.preprocess.OpihiPreprocessSolution attribute}}

\begin{savenotes}\begin{fulllineitems}
\phantomsection\label{\detokenize{code/opihiexarata.opihi.preprocess:opihiexarata.opihi.preprocess.OpihiPreprocessSolution._flat_i_fits_filename}}
\pysigstartsignatures
\pysigline{\sphinxbfcode{\sphinxupquote{\_flat\_i\_fits\_filename}}}
\pysigstopsignatures
\sphinxAtStartPar
The filename for the flat field for the i filter stored in a
fits file.
\begin{quote}\begin{description}
\sphinxlineitem{Type}
\sphinxAtStartPar
string

\end{description}\end{quote}

\end{fulllineitems}\end{savenotes}

\index{\_flat\_z\_fits\_filename (opihiexarata.opihi.preprocess.OpihiPreprocessSolution attribute)@\spxentry{\_flat\_z\_fits\_filename}\spxextra{opihiexarata.opihi.preprocess.OpihiPreprocessSolution attribute}}

\begin{savenotes}\begin{fulllineitems}
\phantomsection\label{\detokenize{code/opihiexarata.opihi.preprocess:opihiexarata.opihi.preprocess.OpihiPreprocessSolution._flat_z_fits_filename}}
\pysigstartsignatures
\pysigline{\sphinxbfcode{\sphinxupquote{\_flat\_z\_fits\_filename}}}
\pysigstopsignatures
\sphinxAtStartPar
The filename for the flat field for the z filter stored in a
fits file.
\begin{quote}\begin{description}
\sphinxlineitem{Type}
\sphinxAtStartPar
string

\end{description}\end{quote}

\end{fulllineitems}\end{savenotes}

\index{\_flat\_1\_fits\_filename (opihiexarata.opihi.preprocess.OpihiPreprocessSolution attribute)@\spxentry{\_flat\_1\_fits\_filename}\spxextra{opihiexarata.opihi.preprocess.OpihiPreprocessSolution attribute}}

\begin{savenotes}\begin{fulllineitems}
\phantomsection\label{\detokenize{code/opihiexarata.opihi.preprocess:opihiexarata.opihi.preprocess.OpihiPreprocessSolution._flat_1_fits_filename}}
\pysigstartsignatures
\pysigline{\sphinxbfcode{\sphinxupquote{\_flat\_1\_fits\_filename}}}
\pysigstopsignatures
\sphinxAtStartPar
The filename for the flat field for the 1 filter stored in a
fits file.
\begin{quote}\begin{description}
\sphinxlineitem{Type}
\sphinxAtStartPar
string

\end{description}\end{quote}

\end{fulllineitems}\end{savenotes}

\index{\_flat\_2\_fits\_filename (opihiexarata.opihi.preprocess.OpihiPreprocessSolution attribute)@\spxentry{\_flat\_2\_fits\_filename}\spxextra{opihiexarata.opihi.preprocess.OpihiPreprocessSolution attribute}}

\begin{savenotes}\begin{fulllineitems}
\phantomsection\label{\detokenize{code/opihiexarata.opihi.preprocess:opihiexarata.opihi.preprocess.OpihiPreprocessSolution._flat_2_fits_filename}}
\pysigstartsignatures
\pysigline{\sphinxbfcode{\sphinxupquote{\_flat\_2\_fits\_filename}}}
\pysigstopsignatures
\sphinxAtStartPar
The filename for the flat field for the 2 filter stored in a
fits file.
\begin{quote}\begin{description}
\sphinxlineitem{Type}
\sphinxAtStartPar
string

\end{description}\end{quote}

\end{fulllineitems}\end{savenotes}

\index{\_flat\_3\_fits\_filename (opihiexarata.opihi.preprocess.OpihiPreprocessSolution attribute)@\spxentry{\_flat\_3\_fits\_filename}\spxextra{opihiexarata.opihi.preprocess.OpihiPreprocessSolution attribute}}

\begin{savenotes}\begin{fulllineitems}
\phantomsection\label{\detokenize{code/opihiexarata.opihi.preprocess:opihiexarata.opihi.preprocess.OpihiPreprocessSolution._flat_3_fits_filename}}
\pysigstartsignatures
\pysigline{\sphinxbfcode{\sphinxupquote{\_flat\_3\_fits\_filename}}}
\pysigstopsignatures
\sphinxAtStartPar
The filename for the flat field for the 3 filter stored in a
fits file.
\begin{quote}\begin{description}
\sphinxlineitem{Type}
\sphinxAtStartPar
string

\end{description}\end{quote}

\end{fulllineitems}\end{savenotes}

\index{\_bias\_fits\_filename (opihiexarata.opihi.preprocess.OpihiPreprocessSolution attribute)@\spxentry{\_bias\_fits\_filename}\spxextra{opihiexarata.opihi.preprocess.OpihiPreprocessSolution attribute}}

\begin{savenotes}\begin{fulllineitems}
\phantomsection\label{\detokenize{code/opihiexarata.opihi.preprocess:opihiexarata.opihi.preprocess.OpihiPreprocessSolution._bias_fits_filename}}
\pysigstartsignatures
\pysigline{\sphinxbfcode{\sphinxupquote{\_bias\_fits\_filename}}}
\pysigstopsignatures
\sphinxAtStartPar
The filename for the per\sphinxhyphen{}pixel bias values of the data,
stored in a fits file.
\begin{quote}\begin{description}
\sphinxlineitem{Type}
\sphinxAtStartPar
string

\end{description}\end{quote}

\end{fulllineitems}\end{savenotes}

\index{\_dark\_current\_fits\_filename (opihiexarata.opihi.preprocess.OpihiPreprocessSolution attribute)@\spxentry{\_dark\_current\_fits\_filename}\spxextra{opihiexarata.opihi.preprocess.OpihiPreprocessSolution attribute}}

\begin{savenotes}\begin{fulllineitems}
\phantomsection\label{\detokenize{code/opihiexarata.opihi.preprocess:opihiexarata.opihi.preprocess.OpihiPreprocessSolution._dark_current_fits_filename}}
\pysigstartsignatures
\pysigline{\sphinxbfcode{\sphinxupquote{\_dark\_current\_fits\_filename}}}
\pysigstopsignatures
\sphinxAtStartPar
The filename for the per\sphinxhyphen{}pixel rate values of the dark data,
stored in a fits file.
\begin{quote}\begin{description}
\sphinxlineitem{Type}
\sphinxAtStartPar
string

\end{description}\end{quote}

\end{fulllineitems}\end{savenotes}

\index{\_linearity\_fits\_filename (opihiexarata.opihi.preprocess.OpihiPreprocessSolution attribute)@\spxentry{\_linearity\_fits\_filename}\spxextra{opihiexarata.opihi.preprocess.OpihiPreprocessSolution attribute}}

\begin{savenotes}\begin{fulllineitems}
\phantomsection\label{\detokenize{code/opihiexarata.opihi.preprocess:opihiexarata.opihi.preprocess.OpihiPreprocessSolution._linearity_fits_filename}}
\pysigstartsignatures
\pysigline{\sphinxbfcode{\sphinxupquote{\_linearity\_fits\_filename}}}
\pysigstopsignatures
\sphinxAtStartPar
The filename for the linearity responce of the CCD. This should be a
1D fits file detailing counts as a function of time for the saturation
curve of the CCD.
\begin{quote}\begin{description}
\sphinxlineitem{Type}
\sphinxAtStartPar
string

\end{description}\end{quote}

\end{fulllineitems}\end{savenotes}

\begin{description}
\sphinxlineitem{mask\_c}{[}array{]}
\sphinxAtStartPar
The pixel mask for the clear filter as determined by the provided
fits file.

\sphinxlineitem{mask\_g}{[}array{]}
\sphinxAtStartPar
The pixel mask for the g filter as determined by the provided
fits file.

\sphinxlineitem{mask\_r}{[}array{]}
\sphinxAtStartPar
The pixel mask for the r filter as determined by the provided
fits file.

\sphinxlineitem{mask\_i}{[}array{]}
\sphinxAtStartPar
The pixel mask for the i filter as determined by the provided
fits file.

\sphinxlineitem{mask\_z}{[}array{]}
\sphinxAtStartPar
The pixel mask for the z filter as determined by the provided
fits file.

\sphinxlineitem{mask\_1}{[}array{]}
\sphinxAtStartPar
The pixel mask for the 1 filter as determined by the provided
fits file.

\sphinxlineitem{mask\_2}{[}array{]}
\sphinxAtStartPar
The pixel mask for the 2 filter as determined by the provided
fits file.

\sphinxlineitem{mask\_3}{[}array{]}
\sphinxAtStartPar
The pixel mask for the 3 filter as determined by the provided
fits file.

\sphinxlineitem{flat\_c}{[}array{]}
\sphinxAtStartPar
The flat field for the clear filter as determined by the provided
fits file.

\sphinxlineitem{flat\_g}{[}array{]}
\sphinxAtStartPar
The flat field for the g filter as determined by the provided
fits file.

\sphinxlineitem{flat\_r}{[}array{]}
\sphinxAtStartPar
The flat field for the r filter as determined by the provided
fits file.

\sphinxlineitem{flat\_i}{[}array{]}
\sphinxAtStartPar
The flat field for the i filter as determined by the provided
fits file.

\sphinxlineitem{flat\_z}{[}array{]}
\sphinxAtStartPar
The flat field for the z filter as determined by the provided
fits file.

\sphinxlineitem{flat\_1}{[}array{]}
\sphinxAtStartPar
The flat field for the 1 filter as determined by the provided
fits file.

\sphinxlineitem{flat\_2}{[}array{]}
\sphinxAtStartPar
The flat field for the 2 filter as determined by the provided
fits file.

\sphinxlineitem{flat\_3}{[}array{]}
\sphinxAtStartPar
The flat field for the 3 filter as determined by the provided
fits file.

\sphinxlineitem{bias}{[}array{]}
\sphinxAtStartPar
The bias array as determined by the provided fits file.

\sphinxlineitem{dark\_current}{[}array{]}
\sphinxAtStartPar
The dark rate, per pixel, as determined by the provided fits file.

\sphinxlineitem{linearity\_factors}{[}array{]}
\sphinxAtStartPar
The polynomial factors of the linearity function starting from the
0th order.

\sphinxlineitem{linearity\_function}{[}function{]}
\sphinxAtStartPar
The linearity function across the whole CCD. It is an average function
across all of the pixels.

\end{description}
\index{\_\_init\_\_() (opihiexarata.opihi.preprocess.OpihiPreprocessSolution method)@\spxentry{\_\_init\_\_()}\spxextra{opihiexarata.opihi.preprocess.OpihiPreprocessSolution method}}

\begin{savenotes}\begin{fulllineitems}
\phantomsection\label{\detokenize{code/opihiexarata.opihi.preprocess:opihiexarata.opihi.preprocess.OpihiPreprocessSolution.__init__}}
\pysigstartsignatures
\pysiglinewithargsret{\sphinxbfcode{\sphinxupquote{\_\_init\_\_}}}{\emph{\DUrole{n}{mask\_c\_fits\_filename}\DUrole{p}{:}\DUrole{w}{  }\DUrole{n}{str}}, \emph{\DUrole{n}{mask\_g\_fits\_filename}\DUrole{p}{:}\DUrole{w}{  }\DUrole{n}{str}}, \emph{\DUrole{n}{mask\_r\_fits\_filename}\DUrole{p}{:}\DUrole{w}{  }\DUrole{n}{str}}, \emph{\DUrole{n}{mask\_i\_fits\_filename}\DUrole{p}{:}\DUrole{w}{  }\DUrole{n}{str}}, \emph{\DUrole{n}{mask\_z\_fits\_filename}\DUrole{p}{:}\DUrole{w}{  }\DUrole{n}{str}}, \emph{\DUrole{n}{mask\_1\_fits\_filename}\DUrole{p}{:}\DUrole{w}{  }\DUrole{n}{str}}, \emph{\DUrole{n}{mask\_2\_fits\_filename}\DUrole{p}{:}\DUrole{w}{  }\DUrole{n}{str}}, \emph{\DUrole{n}{mask\_3\_fits\_filename}\DUrole{p}{:}\DUrole{w}{  }\DUrole{n}{str}}, \emph{\DUrole{n}{flat\_c\_fits\_filename}\DUrole{p}{:}\DUrole{w}{  }\DUrole{n}{str}}, \emph{\DUrole{n}{flat\_g\_fits\_filename}\DUrole{p}{:}\DUrole{w}{  }\DUrole{n}{str}}, \emph{\DUrole{n}{flat\_r\_fits\_filename}\DUrole{p}{:}\DUrole{w}{  }\DUrole{n}{str}}, \emph{\DUrole{n}{flat\_i\_fits\_filename}\DUrole{p}{:}\DUrole{w}{  }\DUrole{n}{str}}, \emph{\DUrole{n}{flat\_z\_fits\_filename}\DUrole{p}{:}\DUrole{w}{  }\DUrole{n}{str}}, \emph{\DUrole{n}{flat\_1\_fits\_filename}\DUrole{p}{:}\DUrole{w}{  }\DUrole{n}{str}}, \emph{\DUrole{n}{flat\_2\_fits\_filename}\DUrole{p}{:}\DUrole{w}{  }\DUrole{n}{str}}, \emph{\DUrole{n}{flat\_3\_fits\_filename}\DUrole{p}{:}\DUrole{w}{  }\DUrole{n}{str}}, \emph{\DUrole{n}{bias\_fits\_filename}\DUrole{p}{:}\DUrole{w}{  }\DUrole{n}{str}}, \emph{\DUrole{n}{dark\_current\_fits\_filename}\DUrole{p}{:}\DUrole{w}{  }\DUrole{n}{str}}, \emph{\DUrole{n}{linearity\_fits\_filename}\DUrole{p}{:}\DUrole{w}{  }\DUrole{n}{str}}}{{ $\rightarrow$ None}}
\pysigstopsignatures
\sphinxAtStartPar
Instantiation of the reduced Opihi data class.
\begin{quote}\begin{description}
\sphinxlineitem{Parameters}\begin{itemize}
\item {} 
\sphinxAtStartPar
\sphinxstyleliteralstrong{\sphinxupquote{mask\_c\_fits\_filename}} (\sphinxstyleliteralemphasis{\sphinxupquote{string}}) – The filename for the pixel mask in the clear filter stored in a
fits file.

\item {} 
\sphinxAtStartPar
\sphinxstyleliteralstrong{\sphinxupquote{mask\_g\_fits\_filename}} (\sphinxstyleliteralemphasis{\sphinxupquote{string}}) – The filename for the pixel mask in the g filter stored in a
fits file.

\item {} 
\sphinxAtStartPar
\sphinxstyleliteralstrong{\sphinxupquote{mask\_r\_fits\_filename}} (\sphinxstyleliteralemphasis{\sphinxupquote{string}}) – The filename for the pixel mask in the r filter stored in a
fits file.

\item {} 
\sphinxAtStartPar
\sphinxstyleliteralstrong{\sphinxupquote{mask\_i\_fits\_filename}} (\sphinxstyleliteralemphasis{\sphinxupquote{string}}) – The filename for the pixel mask in the i filter stored in a
fits file.

\item {} 
\sphinxAtStartPar
\sphinxstyleliteralstrong{\sphinxupquote{mask\_z\_fits\_filename}} (\sphinxstyleliteralemphasis{\sphinxupquote{string}}) – The filename for the pixel mask in the z filter stored in a
fits file.

\item {} 
\sphinxAtStartPar
\sphinxstyleliteralstrong{\sphinxupquote{mask\_1\_fits\_filename}} (\sphinxstyleliteralemphasis{\sphinxupquote{string}}) – The filename for the pixel mask in the 1 filter stored in a
fits file.

\item {} 
\sphinxAtStartPar
\sphinxstyleliteralstrong{\sphinxupquote{mask\_2\_fits\_filename}} (\sphinxstyleliteralemphasis{\sphinxupquote{string}}) – The filename for the pixel mask in the 2 filter stored in a
fits file.

\item {} 
\sphinxAtStartPar
\sphinxstyleliteralstrong{\sphinxupquote{mask\_3\_fits\_filename}} (\sphinxstyleliteralemphasis{\sphinxupquote{string}}) – The filename for the pixel mask in the 3 filter stored in a
fits file.

\item {} 
\sphinxAtStartPar
\sphinxstyleliteralstrong{\sphinxupquote{flat\_c\_fits\_filename}} (\sphinxstyleliteralemphasis{\sphinxupquote{string}}) – The filename for the flat field in the clear filter stored in a
fits file.

\item {} 
\sphinxAtStartPar
\sphinxstyleliteralstrong{\sphinxupquote{flat\_g\_fits\_filename}} (\sphinxstyleliteralemphasis{\sphinxupquote{string}}) – The filename for the flat field in the g filter stored in a
fits file.

\item {} 
\sphinxAtStartPar
\sphinxstyleliteralstrong{\sphinxupquote{flat\_r\_fits\_filename}} (\sphinxstyleliteralemphasis{\sphinxupquote{string}}) – The filename for the flat field in the r filter stored in a
fits file.

\item {} 
\sphinxAtStartPar
\sphinxstyleliteralstrong{\sphinxupquote{flat\_i\_fits\_filename}} (\sphinxstyleliteralemphasis{\sphinxupquote{string}}) – The filename for the flat field in the i filter stored in a
fits file.

\item {} 
\sphinxAtStartPar
\sphinxstyleliteralstrong{\sphinxupquote{flat\_z\_fits\_filename}} (\sphinxstyleliteralemphasis{\sphinxupquote{string}}) – The filename for the flat field in the z filter stored in a
fits file.

\item {} 
\sphinxAtStartPar
\sphinxstyleliteralstrong{\sphinxupquote{flat\_1\_fits\_filename}} (\sphinxstyleliteralemphasis{\sphinxupquote{string}}) – The filename for the flat field in the 1 filter stored in a
fits file.

\item {} 
\sphinxAtStartPar
\sphinxstyleliteralstrong{\sphinxupquote{flat\_2\_fits\_filename}} (\sphinxstyleliteralemphasis{\sphinxupquote{string}}) – The filename for the flat field in the 2 filter stored in a
fits file.

\item {} 
\sphinxAtStartPar
\sphinxstyleliteralstrong{\sphinxupquote{flat\_3\_fits\_filename}} (\sphinxstyleliteralemphasis{\sphinxupquote{string}}) – The filename for the flat field in the 3 filter stored in a
fits file.

\end{itemize}

\end{description}\end{quote}
\begin{description}
\sphinxlineitem{bias\_fits\_filename}{[}string{]}
\sphinxAtStartPar
The filename for the per\sphinxhyphen{}pixel bias values of the data,
stored in a fits file.

\sphinxlineitem{dark\_current\_fits\_filename}{[}string{]}
\sphinxAtStartPar
The filename for the per\sphinxhyphen{}pixel rate values of the dark data,
stored in a fits file.

\sphinxlineitem{linearity\_fits\_filename}{[}string{]}
\sphinxAtStartPar
The filename for the linearity responce of the CCD, stored as a
text file.

\end{description}
\begin{quote}\begin{description}
\sphinxlineitem{Return type}
\sphinxAtStartPar
None

\end{description}\end{quote}

\end{fulllineitems}\end{savenotes}

\index{\_\_init\_read\_flat\_data() (opihiexarata.opihi.preprocess.OpihiPreprocessSolution method)@\spxentry{\_\_init\_read\_flat\_data()}\spxextra{opihiexarata.opihi.preprocess.OpihiPreprocessSolution method}}

\begin{savenotes}\begin{fulllineitems}
\phantomsection\label{\detokenize{code/opihiexarata.opihi.preprocess:opihiexarata.opihi.preprocess.OpihiPreprocessSolution.__init_read_flat_data}}
\pysigstartsignatures
\pysiglinewithargsret{\sphinxbfcode{\sphinxupquote{\_\_init\_read\_flat\_data}}}{}{{ $\rightarrow$ None}}
\pysigstopsignatures
\sphinxAtStartPar
This function just reads all of the fits file data for the
filter\sphinxhyphen{}dependent flat fields and puts it where it belongs per the
documentation.
\begin{quote}\begin{description}
\sphinxlineitem{Parameters}
\sphinxAtStartPar
\sphinxstyleliteralstrong{\sphinxupquote{None}} – 

\sphinxlineitem{Return type}
\sphinxAtStartPar
None

\end{description}\end{quote}

\end{fulllineitems}\end{savenotes}

\index{\_\_init\_read\_linearity\_data() (opihiexarata.opihi.preprocess.OpihiPreprocessSolution method)@\spxentry{\_\_init\_read\_linearity\_data()}\spxextra{opihiexarata.opihi.preprocess.OpihiPreprocessSolution method}}

\begin{savenotes}\begin{fulllineitems}
\phantomsection\label{\detokenize{code/opihiexarata.opihi.preprocess:opihiexarata.opihi.preprocess.OpihiPreprocessSolution.__init_read_linearity_data}}
\pysigstartsignatures
\pysiglinewithargsret{\sphinxbfcode{\sphinxupquote{\_\_init\_read\_linearity\_data}}}{}{}
\pysigstopsignatures
\sphinxAtStartPar
This function reads all of the linearity data and creates a
function for linearity. First order interpolation is done on this data.

\sphinxAtStartPar
It is expected that the data from the linearity filename is of high
enough resolution that first order interpolation is good enough.
\begin{quote}\begin{description}
\sphinxlineitem{Parameters}
\sphinxAtStartPar
\sphinxstyleliteralstrong{\sphinxupquote{None}} – 

\sphinxlineitem{Return type}
\sphinxAtStartPar
None

\end{description}\end{quote}

\end{fulllineitems}\end{savenotes}

\index{\_\_init\_read\_mask\_data() (opihiexarata.opihi.preprocess.OpihiPreprocessSolution method)@\spxentry{\_\_init\_read\_mask\_data()}\spxextra{opihiexarata.opihi.preprocess.OpihiPreprocessSolution method}}

\begin{savenotes}\begin{fulllineitems}
\phantomsection\label{\detokenize{code/opihiexarata.opihi.preprocess:opihiexarata.opihi.preprocess.OpihiPreprocessSolution.__init_read_mask_data}}
\pysigstartsignatures
\pysiglinewithargsret{\sphinxbfcode{\sphinxupquote{\_\_init\_read\_mask\_data}}}{}{{ $\rightarrow$ None}}
\pysigstopsignatures
\sphinxAtStartPar
This function just reads all of the fits file data for the
filter\sphinxhyphen{}dependent pixel masks and puts it where it belongs per the
documentation.
\begin{quote}\begin{description}
\sphinxlineitem{Parameters}
\sphinxAtStartPar
\sphinxstyleliteralstrong{\sphinxupquote{None}} – 

\sphinxlineitem{Return type}
\sphinxAtStartPar
None

\end{description}\end{quote}

\end{fulllineitems}\end{savenotes}

\index{preprocess\_data\_image() (opihiexarata.opihi.preprocess.OpihiPreprocessSolution method)@\spxentry{preprocess\_data\_image()}\spxextra{opihiexarata.opihi.preprocess.OpihiPreprocessSolution method}}

\begin{savenotes}\begin{fulllineitems}
\phantomsection\label{\detokenize{code/opihiexarata.opihi.preprocess:opihiexarata.opihi.preprocess.OpihiPreprocessSolution.preprocess_data_image}}
\pysigstartsignatures
\pysiglinewithargsret{\sphinxbfcode{\sphinxupquote{preprocess\_data\_image}}}{\emph{\DUrole{n}{raw\_data}\DUrole{p}{:}\DUrole{w}{  }\DUrole{n}{ndarray}}, \emph{\DUrole{n}{exposure\_time}\DUrole{p}{:}\DUrole{w}{  }\DUrole{n}{float}}, \emph{\DUrole{n}{filter\_name}\DUrole{p}{:}\DUrole{w}{  }\DUrole{n}{str}}}{{ $\rightarrow$ ndarray}}
\pysigstopsignatures
\sphinxAtStartPar
The formal reduction algorithm for data from Opihi. It follows
preprocessing instructions for CCDs.
\begin{quote}\begin{description}
\sphinxlineitem{Parameters}\begin{itemize}
\item {} 
\sphinxAtStartPar
\sphinxstyleliteralstrong{\sphinxupquote{data}} (\sphinxstyleliteralemphasis{\sphinxupquote{array\sphinxhyphen{}like}}) – The raw image data from the Opihi telescope.

\item {} 
\sphinxAtStartPar
\sphinxstyleliteralstrong{\sphinxupquote{exposure\_time}} (\sphinxstyleliteralemphasis{\sphinxupquote{float}}) – The exposure time of the image in seconds.

\item {} 
\sphinxAtStartPar
\sphinxstyleliteralstrong{\sphinxupquote{filter\_name}} (\sphinxstyleliteralemphasis{\sphinxupquote{string}}) – The name of the filter which the image was taken in, used to
select the correct flat and mask file.

\end{itemize}

\sphinxlineitem{Returns}
\sphinxAtStartPar
\sphinxstylestrong{preprocess\_data} – The data, after it has been preprocessed.

\sphinxlineitem{Return type}
\sphinxAtStartPar
array

\end{description}\end{quote}

\end{fulllineitems}\end{savenotes}

\index{preprocess\_fits\_file() (opihiexarata.opihi.preprocess.OpihiPreprocessSolution method)@\spxentry{preprocess\_fits\_file()}\spxextra{opihiexarata.opihi.preprocess.OpihiPreprocessSolution method}}

\begin{savenotes}\begin{fulllineitems}
\phantomsection\label{\detokenize{code/opihiexarata.opihi.preprocess:opihiexarata.opihi.preprocess.OpihiPreprocessSolution.preprocess_fits_file}}
\pysigstartsignatures
\pysiglinewithargsret{\sphinxbfcode{\sphinxupquote{preprocess\_fits\_file}}}{\emph{\DUrole{n}{raw\_filename}\DUrole{p}{:}\DUrole{w}{  }\DUrole{n}{str}}, \emph{\DUrole{n}{out\_filename}\DUrole{p}{:}\DUrole{w}{  }\DUrole{n}{Optional\DUrole{p}{{[}}str\DUrole{p}{{]}}}\DUrole{w}{  }\DUrole{o}{=}\DUrole{w}{  }\DUrole{default_value}{None}}}{{ $\rightarrow$ tuple\DUrole{p}{{[}}astropy.io.fits.header.Header\DUrole{p}{,}\DUrole{w}{  }numpy.ndarray\DUrole{p}{{]}}}}
\pysigstopsignatures
\sphinxAtStartPar
Preprocess an Opihi image, the provided fits filename is read, the
needed information extracted from it, and it is processed using
historical archive calibration files created per the documentation and
specified by the configuration files.
\begin{quote}\begin{description}
\sphinxlineitem{Parameters}\begin{itemize}
\item {} 
\sphinxAtStartPar
\sphinxstyleliteralstrong{\sphinxupquote{raw\_filename}} (\sphinxstyleliteralemphasis{\sphinxupquote{str}}) – The filename of the raw fits file image from Opihi.

\item {} 
\sphinxAtStartPar
\sphinxstyleliteralstrong{\sphinxupquote{out\_filename}} (\sphinxstyleliteralemphasis{\sphinxupquote{str}}\sphinxstyleliteralemphasis{\sphinxupquote{, }}\sphinxstyleliteralemphasis{\sphinxupquote{default = None}}) – The filename to save the reduced image as a fits file. Some added
entries are added to the header. If this is not provided as
defaults to None, no file is saved.

\end{itemize}

\sphinxlineitem{Returns}
\sphinxAtStartPar
\begin{itemize}
\item {} 
\sphinxAtStartPar
\sphinxstylestrong{preprocess\_header} (\sphinxstyleemphasis{Astropy Header}) – The header of the fits file after preprocessing. Some added
entries are present to document information from preprocessing.

\item {} 
\sphinxAtStartPar
\sphinxstylestrong{preprocess\_data} (\sphinxstyleemphasis{array}) – The data array of the image after the raw image went through the
preprocess reduction.

\end{itemize}


\end{description}\end{quote}

\end{fulllineitems}\end{savenotes}


\end{fulllineitems}\end{savenotes}


\sphinxstepscope


\subparagraph{opihiexarata.opihi.solution module}
\label{\detokenize{code/opihiexarata.opihi.solution:module-opihiexarata.opihi.solution}}\label{\detokenize{code/opihiexarata.opihi.solution:opihiexarata-opihi-solution-module}}\label{\detokenize{code/opihiexarata.opihi.solution::doc}}\index{module@\spxentry{module}!opihiexarata.opihi.solution@\spxentry{opihiexarata.opihi.solution}}\index{opihiexarata.opihi.solution@\spxentry{opihiexarata.opihi.solution}!module@\spxentry{module}}
\sphinxAtStartPar
This is the class for a collection of solutions which the GUI interacts
with and acts as the complete solver. There is not engine as it just shuffles
the solutions.
\index{OpihiSolution (class in opihiexarata.opihi.solution)@\spxentry{OpihiSolution}\spxextra{class in opihiexarata.opihi.solution}}

\begin{savenotes}\begin{fulllineitems}
\phantomsection\label{\detokenize{code/opihiexarata.opihi.solution:opihiexarata.opihi.solution.OpihiSolution}}
\pysigstartsignatures
\pysiglinewithargsret{\sphinxbfcode{\sphinxupquote{class\DUrole{w}{  }}}\sphinxcode{\sphinxupquote{opihiexarata.opihi.solution.}}\sphinxbfcode{\sphinxupquote{OpihiSolution}}}{\emph{\DUrole{n}{fits\_filename}\DUrole{p}{:}\DUrole{w}{  }\DUrole{n}{str}}, \emph{\DUrole{n}{filter\_name}\DUrole{p}{:}\DUrole{w}{  }\DUrole{n}{Optional\DUrole{p}{{[}}str\DUrole{p}{{]}}}\DUrole{w}{  }\DUrole{o}{=}\DUrole{w}{  }\DUrole{default_value}{None}}, \emph{\DUrole{n}{exposure\_time}\DUrole{p}{:}\DUrole{w}{  }\DUrole{n}{Optional\DUrole{p}{{[}}float\DUrole{p}{{]}}}\DUrole{w}{  }\DUrole{o}{=}\DUrole{w}{  }\DUrole{default_value}{None}}, \emph{\DUrole{n}{observing\_time}\DUrole{p}{:}\DUrole{w}{  }\DUrole{n}{Optional\DUrole{p}{{[}}float\DUrole{p}{{]}}}\DUrole{w}{  }\DUrole{o}{=}\DUrole{w}{  }\DUrole{default_value}{None}}, \emph{\DUrole{n}{asteroid\_name}\DUrole{p}{:}\DUrole{w}{  }\DUrole{n}{Optional\DUrole{p}{{[}}str\DUrole{p}{{]}}}\DUrole{w}{  }\DUrole{o}{=}\DUrole{w}{  }\DUrole{default_value}{None}}, \emph{\DUrole{n}{asteroid\_location}\DUrole{p}{:}\DUrole{w}{  }\DUrole{n}{Optional\DUrole{p}{{[}}tuple\DUrole{p}{{[}}float\DUrole{p}{,}\DUrole{w}{  }float\DUrole{p}{{]}}\DUrole{p}{{]}}}\DUrole{w}{  }\DUrole{o}{=}\DUrole{w}{  }\DUrole{default_value}{None}}, \emph{\DUrole{n}{asteroid\_history}\DUrole{p}{:}\DUrole{w}{  }\DUrole{n}{Optional\DUrole{p}{{[}}list\DUrole{p}{{[}}str\DUrole{p}{{]}}\DUrole{p}{{]}}}\DUrole{w}{  }\DUrole{o}{=}\DUrole{w}{  }\DUrole{default_value}{None}}}{}
\pysigstopsignatures
\sphinxAtStartPar
Bases: {\hyperref[\detokenize{code/opihiexarata.library.engine:opihiexarata.library.engine.ExarataSolution}]{\sphinxcrossref{\sphinxcode{\sphinxupquote{ExarataSolution}}}}}

\sphinxAtStartPar
This is the main class which acts as a collection container of
solution classes. It facilitates the interaction between the solution
classes and the GUI.
\index{fits\_filename (opihiexarata.opihi.solution.OpihiSolution attribute)@\spxentry{fits\_filename}\spxextra{opihiexarata.opihi.solution.OpihiSolution attribute}}

\begin{savenotes}\begin{fulllineitems}
\phantomsection\label{\detokenize{code/opihiexarata.opihi.solution:opihiexarata.opihi.solution.OpihiSolution.fits_filename}}
\pysigstartsignatures
\pysigline{\sphinxbfcode{\sphinxupquote{fits\_filename}}}
\pysigstopsignatures
\sphinxAtStartPar
The fits filename of which is the image which this solution is solving.
\begin{quote}\begin{description}
\sphinxlineitem{Type}
\sphinxAtStartPar
str

\end{description}\end{quote}

\end{fulllineitems}\end{savenotes}

\index{filter\_name (opihiexarata.opihi.solution.OpihiSolution attribute)@\spxentry{filter\_name}\spxextra{opihiexarata.opihi.solution.OpihiSolution attribute}}

\begin{savenotes}\begin{fulllineitems}
\phantomsection\label{\detokenize{code/opihiexarata.opihi.solution:opihiexarata.opihi.solution.OpihiSolution.filter_name}}
\pysigstartsignatures
\pysigline{\sphinxbfcode{\sphinxupquote{filter\_name}}}
\pysigstopsignatures
\sphinxAtStartPar
The filter\_name which this image is taken in.
\begin{quote}\begin{description}
\sphinxlineitem{Type}
\sphinxAtStartPar
str

\end{description}\end{quote}

\end{fulllineitems}\end{savenotes}

\index{exposure\_time (opihiexarata.opihi.solution.OpihiSolution attribute)@\spxentry{exposure\_time}\spxextra{opihiexarata.opihi.solution.OpihiSolution attribute}}

\begin{savenotes}\begin{fulllineitems}
\phantomsection\label{\detokenize{code/opihiexarata.opihi.solution:opihiexarata.opihi.solution.OpihiSolution.exposure_time}}
\pysigstartsignatures
\pysigline{\sphinxbfcode{\sphinxupquote{exposure\_time}}}
\pysigstopsignatures
\sphinxAtStartPar
The exposure time of the image, in seconds.
\begin{quote}\begin{description}
\sphinxlineitem{Type}
\sphinxAtStartPar
float

\end{description}\end{quote}

\end{fulllineitems}\end{savenotes}

\index{observing\_time (opihiexarata.opihi.solution.OpihiSolution attribute)@\spxentry{observing\_time}\spxextra{opihiexarata.opihi.solution.OpihiSolution attribute}}

\begin{savenotes}\begin{fulllineitems}
\phantomsection\label{\detokenize{code/opihiexarata.opihi.solution:opihiexarata.opihi.solution.OpihiSolution.observing_time}}
\pysigstartsignatures
\pysigline{\sphinxbfcode{\sphinxupquote{observing\_time}}}
\pysigstopsignatures
\sphinxAtStartPar
The time of observation, this must be a Julian day time.
\begin{quote}\begin{description}
\sphinxlineitem{Type}
\sphinxAtStartPar
float

\end{description}\end{quote}

\end{fulllineitems}\end{savenotes}

\index{asteroid\_name (opihiexarata.opihi.solution.OpihiSolution attribute)@\spxentry{asteroid\_name}\spxextra{opihiexarata.opihi.solution.OpihiSolution attribute}}

\begin{savenotes}\begin{fulllineitems}
\phantomsection\label{\detokenize{code/opihiexarata.opihi.solution:opihiexarata.opihi.solution.OpihiSolution.asteroid_name}}
\pysigstartsignatures
\pysigline{\sphinxbfcode{\sphinxupquote{asteroid\_name}}}
\pysigstopsignatures
\sphinxAtStartPar
The name of the asteroid. This is used to group similar observations
and to also retrieve data from the MPC.
\begin{quote}\begin{description}
\sphinxlineitem{Type}
\sphinxAtStartPar
str

\end{description}\end{quote}

\end{fulllineitems}\end{savenotes}

\index{asteroid\_location (opihiexarata.opihi.solution.OpihiSolution attribute)@\spxentry{asteroid\_location}\spxextra{opihiexarata.opihi.solution.OpihiSolution attribute}}

\begin{savenotes}\begin{fulllineitems}
\phantomsection\label{\detokenize{code/opihiexarata.opihi.solution:opihiexarata.opihi.solution.OpihiSolution.asteroid_location}}
\pysigstartsignatures
\pysigline{\sphinxbfcode{\sphinxupquote{asteroid\_location}}}
\pysigstopsignatures
\sphinxAtStartPar
The pixel location of the asteroid. (Usually determined by a centroid
around a user specified location.) If this is None, then asteroid
calculations are disabled as there is no asteroid.
\begin{quote}\begin{description}
\sphinxlineitem{Type}
\sphinxAtStartPar
tuple

\end{description}\end{quote}

\end{fulllineitems}\end{savenotes}

\index{asteroid\_history (opihiexarata.opihi.solution.OpihiSolution attribute)@\spxentry{asteroid\_history}\spxextra{opihiexarata.opihi.solution.OpihiSolution attribute}}

\begin{savenotes}\begin{fulllineitems}
\phantomsection\label{\detokenize{code/opihiexarata.opihi.solution:opihiexarata.opihi.solution.OpihiSolution.asteroid_history}}
\pysigstartsignatures
\pysigline{\sphinxbfcode{\sphinxupquote{asteroid\_history}}}
\pysigstopsignatures
\sphinxAtStartPar
The total observational history of the asteroid provided. This includes
previous observations done by Opihi and processed by OpihiExarata, but
does not include the current one. This is the 80\sphinxhyphen{}column text file
form of a MPC record. If this is None, then asteroid calculations are
disabled as there is no asteroid.
\begin{quote}\begin{description}
\sphinxlineitem{Type}
\sphinxAtStartPar
list

\end{description}\end{quote}

\end{fulllineitems}\end{savenotes}

\index{asteroid\_observations (opihiexarata.opihi.solution.OpihiSolution attribute)@\spxentry{asteroid\_observations}\spxextra{opihiexarata.opihi.solution.OpihiSolution attribute}}

\begin{savenotes}\begin{fulllineitems}
\phantomsection\label{\detokenize{code/opihiexarata.opihi.solution:opihiexarata.opihi.solution.OpihiSolution.asteroid_observations}}
\pysigstartsignatures
\pysigline{\sphinxbfcode{\sphinxupquote{asteroid\_observations}}}
\pysigstopsignatures
\sphinxAtStartPar
The total observational history of the asteroid provided. This includes
previous observations done by Opihi and processed by OpihiExarata, but
does not include the current data. This is the table form of a MPC
record. If this is None, then asteroid calculations are disabled as
there is no asteroid.
\begin{quote}\begin{description}
\sphinxlineitem{Type}
\sphinxAtStartPar
table

\end{description}\end{quote}

\end{fulllineitems}\end{savenotes}

\index{header (opihiexarata.opihi.solution.OpihiSolution attribute)@\spxentry{header}\spxextra{opihiexarata.opihi.solution.OpihiSolution attribute}}

\begin{savenotes}\begin{fulllineitems}
\phantomsection\label{\detokenize{code/opihiexarata.opihi.solution:opihiexarata.opihi.solution.OpihiSolution.header}}
\pysigstartsignatures
\pysigline{\sphinxbfcode{\sphinxupquote{header}}}
\pysigstopsignatures
\sphinxAtStartPar
The header of the fits file.
\begin{quote}\begin{description}
\sphinxlineitem{Type}
\sphinxAtStartPar
Astropy Header

\end{description}\end{quote}

\end{fulllineitems}\end{savenotes}

\index{data (opihiexarata.opihi.solution.OpihiSolution attribute)@\spxentry{data}\spxextra{opihiexarata.opihi.solution.OpihiSolution attribute}}

\begin{savenotes}\begin{fulllineitems}
\phantomsection\label{\detokenize{code/opihiexarata.opihi.solution:opihiexarata.opihi.solution.OpihiSolution.data}}
\pysigstartsignatures
\pysigline{\sphinxbfcode{\sphinxupquote{data}}}
\pysigstopsignatures
\sphinxAtStartPar
The image data of the fits file itself.
\begin{quote}\begin{description}
\sphinxlineitem{Type}
\sphinxAtStartPar
array

\end{description}\end{quote}

\end{fulllineitems}\end{savenotes}

\begin{description}
\sphinxlineitem{astrometrics}{[}AstrometricSolution{]}
\sphinxAtStartPar
The astrometric solution; if it has not been solved yet, this is None.

\sphinxlineitem{photometrics}{[}PhotometricSolution{]}
\sphinxAtStartPar
The photometric solution; if it has not been solved yet, this is None.

\sphinxlineitem{propagatives}{[}PropagativeSolution{]}
\sphinxAtStartPar
The propagation solution; if it has not been solved yet, this is None.

\sphinxlineitem{orbitals}{[}OrbitalSolution{]}
\sphinxAtStartPar
The orbit solution; if it has not been solved yet, this is None.

\sphinxlineitem{ephemeritics}{[}EphemeriticSolution{]}
\sphinxAtStartPar
The ephemeris solution; if it has not been solved yet, this is None.

\sphinxlineitem{astrometrics\_status}{[}bool, None{]}
\sphinxAtStartPar
The status of the solving of the astrometric solution. It is True or
False based on the success of the solve, None if a solve has not
been attempted.

\sphinxlineitem{photometrics\_status}{[}bool, None{]}
\sphinxAtStartPar
The status of the solving of the photometric solution. It is True or
False based on the success of the solve, None if a solve has not
been attempted.

\sphinxlineitem{propagatives\_status}{[}bool, None{]}
\sphinxAtStartPar
The status of the solving of the propagative solution. It is True or
False based on the success of the solve, None if a solve has not
been attempted.

\sphinxlineitem{orbitals\_status}{[}bool, None{]}
\sphinxAtStartPar
The status of the solving of the orbital solution. It is True or
False based on the success of the solve, None if a solve has not
been attempted.

\sphinxlineitem{ephemeritics\_status}{[}bool, None{]}
\sphinxAtStartPar
The status of the solving of the ephemeris solution. It is True or
False based on the success of the solve, None if a solve has not
been attempted.

\end{description}
\index{\_\_init\_\_() (opihiexarata.opihi.solution.OpihiSolution method)@\spxentry{\_\_init\_\_()}\spxextra{opihiexarata.opihi.solution.OpihiSolution method}}

\begin{savenotes}\begin{fulllineitems}
\phantomsection\label{\detokenize{code/opihiexarata.opihi.solution:opihiexarata.opihi.solution.OpihiSolution.__init__}}
\pysigstartsignatures
\pysiglinewithargsret{\sphinxbfcode{\sphinxupquote{\_\_init\_\_}}}{\emph{\DUrole{n}{fits\_filename}\DUrole{p}{:}\DUrole{w}{  }\DUrole{n}{str}}, \emph{\DUrole{n}{filter\_name}\DUrole{p}{:}\DUrole{w}{  }\DUrole{n}{Optional\DUrole{p}{{[}}str\DUrole{p}{{]}}}\DUrole{w}{  }\DUrole{o}{=}\DUrole{w}{  }\DUrole{default_value}{None}}, \emph{\DUrole{n}{exposure\_time}\DUrole{p}{:}\DUrole{w}{  }\DUrole{n}{Optional\DUrole{p}{{[}}float\DUrole{p}{{]}}}\DUrole{w}{  }\DUrole{o}{=}\DUrole{w}{  }\DUrole{default_value}{None}}, \emph{\DUrole{n}{observing\_time}\DUrole{p}{:}\DUrole{w}{  }\DUrole{n}{Optional\DUrole{p}{{[}}float\DUrole{p}{{]}}}\DUrole{w}{  }\DUrole{o}{=}\DUrole{w}{  }\DUrole{default_value}{None}}, \emph{\DUrole{n}{asteroid\_name}\DUrole{p}{:}\DUrole{w}{  }\DUrole{n}{Optional\DUrole{p}{{[}}str\DUrole{p}{{]}}}\DUrole{w}{  }\DUrole{o}{=}\DUrole{w}{  }\DUrole{default_value}{None}}, \emph{\DUrole{n}{asteroid\_location}\DUrole{p}{:}\DUrole{w}{  }\DUrole{n}{Optional\DUrole{p}{{[}}tuple\DUrole{p}{{[}}float\DUrole{p}{,}\DUrole{w}{  }float\DUrole{p}{{]}}\DUrole{p}{{]}}}\DUrole{w}{  }\DUrole{o}{=}\DUrole{w}{  }\DUrole{default_value}{None}}, \emph{\DUrole{n}{asteroid\_history}\DUrole{p}{:}\DUrole{w}{  }\DUrole{n}{Optional\DUrole{p}{{[}}list\DUrole{p}{{[}}str\DUrole{p}{{]}}\DUrole{p}{{]}}}\DUrole{w}{  }\DUrole{o}{=}\DUrole{w}{  }\DUrole{default_value}{None}}}{{ $\rightarrow$ None}}
\pysigstopsignatures
\sphinxAtStartPar
Creating the main solution class.

\sphinxAtStartPar
All of the data which is needed to derive the other solutions should
be provided. The solutions, however, are only done when called.
Overriding parameters can be applied when calling the solutions.

\sphinxAtStartPar
If the asteroid input values are not provided, then this class will
prohibit calculations meant for asteroids because of the lack
of an asteroid.
\begin{quote}\begin{description}
\sphinxlineitem{Parameters}\begin{itemize}
\item {} 
\sphinxAtStartPar
\sphinxstyleliteralstrong{\sphinxupquote{fits\_filename}} (\sphinxstyleliteralemphasis{\sphinxupquote{str}}) – The fits filename of which is the image which this solution is
solving.

\item {} 
\sphinxAtStartPar
\sphinxstyleliteralstrong{\sphinxupquote{filter\_name}} (\sphinxstyleliteralemphasis{\sphinxupquote{string}}\sphinxstyleliteralemphasis{\sphinxupquote{, }}\sphinxstyleliteralemphasis{\sphinxupquote{default=None}}) – The filter\_name of the image which is contained within the data
array. If None, we attempt to pull the value from the fits file.

\item {} 
\sphinxAtStartPar
\sphinxstyleliteralstrong{\sphinxupquote{exposure\_time}} (\sphinxstyleliteralemphasis{\sphinxupquote{float}}\sphinxstyleliteralemphasis{\sphinxupquote{, }}\sphinxstyleliteralemphasis{\sphinxupquote{default=None}}) – The exposure time of the image, in seconds.
If None, we attempt to pull the value from the fits file.

\item {} 
\sphinxAtStartPar
\sphinxstyleliteralstrong{\sphinxupquote{observing\_time}} (\sphinxstyleliteralemphasis{\sphinxupquote{float}}\sphinxstyleliteralemphasis{\sphinxupquote{, }}\sphinxstyleliteralemphasis{\sphinxupquote{default=None}}) – The time of observation, this time must in Julian day.
If None, we attempt to pull the value from the fits file.

\item {} 
\sphinxAtStartPar
\sphinxstyleliteralstrong{\sphinxupquote{asteroid\_name}} (\sphinxstyleliteralemphasis{\sphinxupquote{str}}\sphinxstyleliteralemphasis{\sphinxupquote{, }}\sphinxstyleliteralemphasis{\sphinxupquote{default = None}}) – The name of the asteroid.

\item {} 
\sphinxAtStartPar
\sphinxstyleliteralstrong{\sphinxupquote{asteroid\_location}} (\sphinxstyleliteralemphasis{\sphinxupquote{tuple}}\sphinxstyleliteralemphasis{\sphinxupquote{, }}\sphinxstyleliteralemphasis{\sphinxupquote{default = None}}) – The pixel location of the asteroid.

\item {} 
\sphinxAtStartPar
\sphinxstyleliteralstrong{\sphinxupquote{asteroid\_history}} (\sphinxstyleliteralemphasis{\sphinxupquote{list}}\sphinxstyleliteralemphasis{\sphinxupquote{, }}\sphinxstyleliteralemphasis{\sphinxupquote{default = None}}) – The history of observations of an asteroid written in a standard
80\sphinxhyphen{}column MPC record.

\end{itemize}

\end{description}\end{quote}

\end{fulllineitems}\end{savenotes}

\index{mpc\_record\_row() (opihiexarata.opihi.solution.OpihiSolution method)@\spxentry{mpc\_record\_row()}\spxextra{opihiexarata.opihi.solution.OpihiSolution method}}

\begin{savenotes}\begin{fulllineitems}
\phantomsection\label{\detokenize{code/opihiexarata.opihi.solution:opihiexarata.opihi.solution.OpihiSolution.mpc_record_row}}
\pysigstartsignatures
\pysiglinewithargsret{\sphinxbfcode{\sphinxupquote{mpc\_record\_row}}}{}{{ $\rightarrow$ str}}
\pysigstopsignatures
\sphinxAtStartPar
Returns an 80\sphinxhyphen{}character record describing the observation of this
object assuming it is an asteroid. It only uses information
that is provided and does not attempt to compute any solutions.
\begin{quote}\begin{description}
\sphinxlineitem{Parameters}
\sphinxAtStartPar
\sphinxstyleliteralstrong{\sphinxupquote{None}} – 

\sphinxlineitem{Returns}
\sphinxAtStartPar
\sphinxstylestrong{record\_row} – The 80\sphinxhyphen{}character record as determined by the MPC specification.

\sphinxlineitem{Return type}
\sphinxAtStartPar
str

\end{description}\end{quote}

\end{fulllineitems}\end{savenotes}

\index{mpc\_table\_row() (opihiexarata.opihi.solution.OpihiSolution method)@\spxentry{mpc\_table\_row()}\spxextra{opihiexarata.opihi.solution.OpihiSolution method}}

\begin{savenotes}\begin{fulllineitems}
\phantomsection\label{\detokenize{code/opihiexarata.opihi.solution:opihiexarata.opihi.solution.OpihiSolution.mpc_table_row}}
\pysigstartsignatures
\pysiglinewithargsret{\sphinxbfcode{\sphinxupquote{mpc\_table\_row}}}{}{{ $\rightarrow$ Table}}
\pysigstopsignatures
\sphinxAtStartPar
An MPC table of the current observation with information provided
by solved solutions. This routine does not attempt to do any solutions.
\begin{quote}\begin{description}
\sphinxlineitem{Parameters}
\sphinxAtStartPar
\sphinxstyleliteralstrong{\sphinxupquote{None}} – 

\sphinxlineitem{Returns}
\sphinxAtStartPar
\sphinxstylestrong{table\_row} – The MPC table of the information. It is a single row table.

\sphinxlineitem{Return type}
\sphinxAtStartPar
Astropy Table

\end{description}\end{quote}

\end{fulllineitems}\end{savenotes}

\index{solve\_astrometry() (opihiexarata.opihi.solution.OpihiSolution method)@\spxentry{solve\_astrometry()}\spxextra{opihiexarata.opihi.solution.OpihiSolution method}}

\begin{savenotes}\begin{fulllineitems}
\phantomsection\label{\detokenize{code/opihiexarata.opihi.solution:opihiexarata.opihi.solution.OpihiSolution.solve_astrometry}}
\pysigstartsignatures
\pysiglinewithargsret{\sphinxbfcode{\sphinxupquote{solve\_astrometry}}}{\emph{\DUrole{n}{solver\_engine}\DUrole{p}{:}\DUrole{w}{  }\DUrole{n}{{\hyperref[\detokenize{code/opihiexarata.library.engine:opihiexarata.library.engine.AstrometryEngine}]{\sphinxcrossref{AstrometryEngine}}}}}, \emph{\DUrole{n}{overwrite}\DUrole{p}{:}\DUrole{w}{  }\DUrole{n}{bool}\DUrole{w}{  }\DUrole{o}{=}\DUrole{w}{  }\DUrole{default_value}{True}}, \emph{\DUrole{n}{raise\_on\_error}\DUrole{p}{:}\DUrole{w}{  }\DUrole{n}{bool}\DUrole{w}{  }\DUrole{o}{=}\DUrole{w}{  }\DUrole{default_value}{False}}, \emph{\DUrole{n}{vehicle\_args}\DUrole{p}{:}\DUrole{w}{  }\DUrole{n}{dict}\DUrole{w}{  }\DUrole{o}{=}\DUrole{w}{  }\DUrole{default_value}{\{\}}}}{{ $\rightarrow$ {\hyperref[\detokenize{code/opihiexarata.astrometry.solution:opihiexarata.astrometry.solution.AstrometricSolution}]{\sphinxcrossref{AstrometricSolution}}}}}
\pysigstopsignatures
\sphinxAtStartPar
Solve the image astrometry by using an astrometric engine.
\begin{quote}\begin{description}
\sphinxlineitem{Parameters}\begin{itemize}
\item {} 
\sphinxAtStartPar
\sphinxstyleliteralstrong{\sphinxupquote{solver\_engine}} ({\hyperref[\detokenize{code/opihiexarata.library.engine:opihiexarata.library.engine.AstrometryEngine}]{\sphinxcrossref{\sphinxstyleliteralemphasis{\sphinxupquote{AstrometryEngine}}}}}) – The astrometric engine which the astrometry solver will use.

\item {} 
\sphinxAtStartPar
\sphinxstyleliteralstrong{\sphinxupquote{overwrite}} (\sphinxstyleliteralemphasis{\sphinxupquote{bool}}\sphinxstyleliteralemphasis{\sphinxupquote{, }}\sphinxstyleliteralemphasis{\sphinxupquote{default = True}}) – Overwrite and replace the information of this class with the new
values. If False, the returned solution is not also applied.

\item {} 
\sphinxAtStartPar
\sphinxstyleliteralstrong{\sphinxupquote{raise\_on\_error}} (\sphinxstyleliteralemphasis{\sphinxupquote{bool}}\sphinxstyleliteralemphasis{\sphinxupquote{, }}\sphinxstyleliteralemphasis{\sphinxupquote{default = False}}) – If True, this disables the error handing and allows for errors from
the solving engines/solutions to be propagated out.

\item {} 
\sphinxAtStartPar
\sphinxstyleliteralstrong{\sphinxupquote{vehicle\_args}} (\sphinxstyleliteralemphasis{\sphinxupquote{dictionary}}\sphinxstyleliteralemphasis{\sphinxupquote{, }}\sphinxstyleliteralemphasis{\sphinxupquote{default = \{\}}}) – If the vehicle function for the provided solver engine needs
extra parameters not otherwise provided by the standard input,
they are given here.

\end{itemize}

\sphinxlineitem{Returns}
\sphinxAtStartPar
\begin{itemize}
\item {} 
\sphinxAtStartPar
\sphinxstylestrong{astrometric\_solution} (\sphinxstyleemphasis{AstrometricSolution}) – The astrometry solution for the image.

\item {} 
\sphinxAtStartPar
\sphinxstylestrong{solve\_status} (\sphinxstyleemphasis{bool}) – The status of the solve. If True, the solving was successful.

\end{itemize}


\end{description}\end{quote}

\end{fulllineitems}\end{savenotes}

\index{solve\_ephemeris() (opihiexarata.opihi.solution.OpihiSolution method)@\spxentry{solve\_ephemeris()}\spxextra{opihiexarata.opihi.solution.OpihiSolution method}}

\begin{savenotes}\begin{fulllineitems}
\phantomsection\label{\detokenize{code/opihiexarata.opihi.solution:opihiexarata.opihi.solution.OpihiSolution.solve_ephemeris}}
\pysigstartsignatures
\pysiglinewithargsret{\sphinxbfcode{\sphinxupquote{solve\_ephemeris}}}{\emph{\DUrole{n}{solver\_engine}\DUrole{p}{:}\DUrole{w}{  }\DUrole{n}{{\hyperref[\detokenize{code/opihiexarata.library.engine:opihiexarata.library.engine.EphemerisEngine}]{\sphinxcrossref{EphemerisEngine}}}}}, \emph{\DUrole{n}{overwrite}\DUrole{p}{:}\DUrole{w}{  }\DUrole{n}{bool}\DUrole{w}{  }\DUrole{o}{=}\DUrole{w}{  }\DUrole{default_value}{True}}, \emph{\DUrole{n}{raise\_on\_error}\DUrole{p}{:}\DUrole{w}{  }\DUrole{n}{bool}\DUrole{w}{  }\DUrole{o}{=}\DUrole{w}{  }\DUrole{default_value}{False}}, \emph{\DUrole{n}{vehicle\_args}\DUrole{p}{:}\DUrole{w}{  }\DUrole{n}{dict}\DUrole{w}{  }\DUrole{o}{=}\DUrole{w}{  }\DUrole{default_value}{\{\}}}}{}
\pysigstopsignatures
\sphinxAtStartPar
Solve for the ephemeris solution an asteroid using previous
observations and derived orbital elements.
\begin{quote}\begin{description}
\sphinxlineitem{Parameters}\begin{itemize}
\item {} 
\sphinxAtStartPar
\sphinxstyleliteralstrong{\sphinxupquote{solver\_engine}} ({\hyperref[\detokenize{code/opihiexarata.library.engine:opihiexarata.library.engine.EphemerisEngine}]{\sphinxcrossref{\sphinxstyleliteralemphasis{\sphinxupquote{EphemerisEngine}}}}}) – The ephemeris engine which the ephemeris solver will use.

\item {} 
\sphinxAtStartPar
\sphinxstyleliteralstrong{\sphinxupquote{overwrite}} (\sphinxstyleliteralemphasis{\sphinxupquote{bool}}\sphinxstyleliteralemphasis{\sphinxupquote{, }}\sphinxstyleliteralemphasis{\sphinxupquote{default = True}}) – Overwrite and replace the information of this class with the new
values. If False, the returned solution is not also applied.

\item {} 
\sphinxAtStartPar
\sphinxstyleliteralstrong{\sphinxupquote{raise\_on\_error}} (\sphinxstyleliteralemphasis{\sphinxupquote{bool}}\sphinxstyleliteralemphasis{\sphinxupquote{, }}\sphinxstyleliteralemphasis{\sphinxupquote{default = False}}) – If True, this disables the error handing and allows for errors from
the solving engines/solutions to be propagated out.

\item {} 
\sphinxAtStartPar
\sphinxstyleliteralstrong{\sphinxupquote{vehicle\_args}} (\sphinxstyleliteralemphasis{\sphinxupquote{dictionary}}\sphinxstyleliteralemphasis{\sphinxupquote{, }}\sphinxstyleliteralemphasis{\sphinxupquote{default = \{\}}}) – If the vehicle function for the provided solver engine needs
extra parameters not otherwise provided by the standard input,
they are given here.

\end{itemize}

\sphinxlineitem{Returns}
\sphinxAtStartPar
\begin{itemize}
\item {} 
\sphinxAtStartPar
\sphinxstylestrong{ephemeritics\_solution} (\sphinxstyleemphasis{EphemeriticSolution}) – The orbit solution for the asteroid and image.

\item {} 
\sphinxAtStartPar
\sphinxstylestrong{solve\_status} (\sphinxstyleemphasis{bool}) – The status of the solve. If True, the solving was successful.

\item {} 
\sphinxAtStartPar
\sphinxstyleemphasis{Warning ..} – This requires that the orbital solution be computed
before\sphinxhyphen{}hand. It will not be precomputed automatically; without it
being called explicitly, this will instead raise an error.

\end{itemize}


\end{description}\end{quote}

\end{fulllineitems}\end{savenotes}

\index{solve\_orbit() (opihiexarata.opihi.solution.OpihiSolution method)@\spxentry{solve\_orbit()}\spxextra{opihiexarata.opihi.solution.OpihiSolution method}}

\begin{savenotes}\begin{fulllineitems}
\phantomsection\label{\detokenize{code/opihiexarata.opihi.solution:opihiexarata.opihi.solution.OpihiSolution.solve_orbit}}
\pysigstartsignatures
\pysiglinewithargsret{\sphinxbfcode{\sphinxupquote{solve\_orbit}}}{\emph{\DUrole{n}{solver\_engine}\DUrole{p}{:}\DUrole{w}{  }\DUrole{n}{{\hyperref[\detokenize{code/opihiexarata.library.engine:opihiexarata.library.engine.OrbitEngine}]{\sphinxcrossref{OrbitEngine}}}}}, \emph{\DUrole{n}{overwrite}\DUrole{p}{:}\DUrole{w}{  }\DUrole{n}{bool}\DUrole{w}{  }\DUrole{o}{=}\DUrole{w}{  }\DUrole{default_value}{True}}, \emph{\DUrole{n}{raise\_on\_error}\DUrole{p}{:}\DUrole{w}{  }\DUrole{n}{bool}\DUrole{w}{  }\DUrole{o}{=}\DUrole{w}{  }\DUrole{default_value}{False}}, \emph{\DUrole{n}{asteroid\_location}\DUrole{p}{:}\DUrole{w}{  }\DUrole{n}{Optional\DUrole{p}{{[}}tuple\DUrole{p}{{]}}}\DUrole{w}{  }\DUrole{o}{=}\DUrole{w}{  }\DUrole{default_value}{None}}, \emph{\DUrole{n}{vehicle\_args}\DUrole{p}{:}\DUrole{w}{  }\DUrole{n}{dict}\DUrole{w}{  }\DUrole{o}{=}\DUrole{w}{  }\DUrole{default_value}{\{\}}}}{}
\pysigstopsignatures
\sphinxAtStartPar
Solve for the orbital elements an asteroid using previous
observations.
\begin{quote}\begin{description}
\sphinxlineitem{Parameters}\begin{itemize}
\item {} 
\sphinxAtStartPar
\sphinxstyleliteralstrong{\sphinxupquote{solver\_engine}} ({\hyperref[\detokenize{code/opihiexarata.library.engine:opihiexarata.library.engine.OrbitEngine}]{\sphinxcrossref{\sphinxstyleliteralemphasis{\sphinxupquote{OrbitEngine}}}}}) – The orbital engine which the orbit solver will use.

\item {} 
\sphinxAtStartPar
\sphinxstyleliteralstrong{\sphinxupquote{overwrite}} (\sphinxstyleliteralemphasis{\sphinxupquote{bool}}\sphinxstyleliteralemphasis{\sphinxupquote{, }}\sphinxstyleliteralemphasis{\sphinxupquote{default = True}}) – Overwrite and replace the information of this class with the new
values. If False, the returned solution is not also applied.

\item {} 
\sphinxAtStartPar
\sphinxstyleliteralstrong{\sphinxupquote{asteroid\_location}} (\sphinxstyleliteralemphasis{\sphinxupquote{tuple}}\sphinxstyleliteralemphasis{\sphinxupquote{, }}\sphinxstyleliteralemphasis{\sphinxupquote{default = None}}) – The pixel location of the asteroid in the image. Defaults to the
value provided at instantiation.

\item {} 
\sphinxAtStartPar
\sphinxstyleliteralstrong{\sphinxupquote{raise\_on\_error}} (\sphinxstyleliteralemphasis{\sphinxupquote{bool}}\sphinxstyleliteralemphasis{\sphinxupquote{, }}\sphinxstyleliteralemphasis{\sphinxupquote{default = False}}) – If True, this disables the error handing and allows for errors from
the solving engines/solutions to be propagated out.

\item {} 
\sphinxAtStartPar
\sphinxstyleliteralstrong{\sphinxupquote{vehicle\_args}} (\sphinxstyleliteralemphasis{\sphinxupquote{dictionary}}\sphinxstyleliteralemphasis{\sphinxupquote{, }}\sphinxstyleliteralemphasis{\sphinxupquote{default = \{\}}}) – If the vehicle function for the provided solver engine needs
extra parameters not otherwise provided by the standard input,
they are given here.

\end{itemize}

\sphinxlineitem{Returns}
\sphinxAtStartPar
\begin{itemize}
\item {} 
\sphinxAtStartPar
\sphinxstylestrong{orbital\_solution} (\sphinxstyleemphasis{OrbitalSolution}) – The orbit solution for the asteroid and image.

\item {} 
\sphinxAtStartPar
\sphinxstylestrong{solve\_status} (\sphinxstyleemphasis{bool}) – The status of the solve. If True, the solving was successful.

\item {} 
\sphinxAtStartPar
\sphinxstyleemphasis{Warning ..} – This requires that the astrometric solution be computed
before\sphinxhyphen{}hand. It will not be precomputed automatically; without it
being called explicitly, this will instead raise an error.

\end{itemize}


\end{description}\end{quote}

\end{fulllineitems}\end{savenotes}

\index{solve\_photometry() (opihiexarata.opihi.solution.OpihiSolution method)@\spxentry{solve\_photometry()}\spxextra{opihiexarata.opihi.solution.OpihiSolution method}}

\begin{savenotes}\begin{fulllineitems}
\phantomsection\label{\detokenize{code/opihiexarata.opihi.solution:opihiexarata.opihi.solution.OpihiSolution.solve_photometry}}
\pysigstartsignatures
\pysiglinewithargsret{\sphinxbfcode{\sphinxupquote{solve\_photometry}}}{\emph{\DUrole{n}{solver\_engine}\DUrole{p}{:}\DUrole{w}{  }\DUrole{n}{{\hyperref[\detokenize{code/opihiexarata.library.engine:opihiexarata.library.engine.PhotometryEngine}]{\sphinxcrossref{PhotometryEngine}}}}}, \emph{\DUrole{n}{overwrite}\DUrole{p}{:}\DUrole{w}{  }\DUrole{n}{bool}\DUrole{w}{  }\DUrole{o}{=}\DUrole{w}{  }\DUrole{default_value}{True}}, \emph{\DUrole{n}{raise\_on\_error}\DUrole{p}{:}\DUrole{w}{  }\DUrole{n}{bool}\DUrole{w}{  }\DUrole{o}{=}\DUrole{w}{  }\DUrole{default_value}{False}}, \emph{\DUrole{n}{filter\_name}\DUrole{p}{:}\DUrole{w}{  }\DUrole{n}{Optional\DUrole{p}{{[}}str\DUrole{p}{{]}}}\DUrole{w}{  }\DUrole{o}{=}\DUrole{w}{  }\DUrole{default_value}{None}}, \emph{\DUrole{n}{exposure\_time}\DUrole{p}{:}\DUrole{w}{  }\DUrole{n}{Optional\DUrole{p}{{[}}float\DUrole{p}{{]}}}\DUrole{w}{  }\DUrole{o}{=}\DUrole{w}{  }\DUrole{default_value}{None}}, \emph{\DUrole{n}{vehicle\_args}\DUrole{p}{:}\DUrole{w}{  }\DUrole{n}{dict}\DUrole{w}{  }\DUrole{o}{=}\DUrole{w}{  }\DUrole{default_value}{\{\}}}}{{ $\rightarrow$ {\hyperref[\detokenize{code/opihiexarata.photometry.solution:opihiexarata.photometry.solution.PhotometricSolution}]{\sphinxcrossref{PhotometricSolution}}}}}
\pysigstopsignatures
\sphinxAtStartPar
Solve the image photometry by using a photometric engine.
\begin{quote}\begin{description}
\sphinxlineitem{Parameters}\begin{itemize}
\item {} 
\sphinxAtStartPar
\sphinxstyleliteralstrong{\sphinxupquote{solver\_engine}} ({\hyperref[\detokenize{code/opihiexarata.library.engine:opihiexarata.library.engine.PhotometryEngine}]{\sphinxcrossref{\sphinxstyleliteralemphasis{\sphinxupquote{PhotometryEngine}}}}}) – The photometric engine which the photometry solver will use.

\item {} 
\sphinxAtStartPar
\sphinxstyleliteralstrong{\sphinxupquote{overwrite}} (\sphinxstyleliteralemphasis{\sphinxupquote{bool}}\sphinxstyleliteralemphasis{\sphinxupquote{, }}\sphinxstyleliteralemphasis{\sphinxupquote{default = True}}) – Overwrite and replace the information of this class with the new
values. If False, the returned solution is not also applied.

\item {} 
\sphinxAtStartPar
\sphinxstyleliteralstrong{\sphinxupquote{filter\_name}} (\sphinxstyleliteralemphasis{\sphinxupquote{string}}\sphinxstyleliteralemphasis{\sphinxupquote{, }}\sphinxstyleliteralemphasis{\sphinxupquote{default = None}}) – The filter name of the image, defaults to the value provided at
instantiation.

\item {} 
\sphinxAtStartPar
\sphinxstyleliteralstrong{\sphinxupquote{exposure\_time}} (\sphinxstyleliteralemphasis{\sphinxupquote{float}}\sphinxstyleliteralemphasis{\sphinxupquote{, }}\sphinxstyleliteralemphasis{\sphinxupquote{default = None}}) – The exposure time of the image, in seconds. Defaults to the value
provided at instantiation.

\item {} 
\sphinxAtStartPar
\sphinxstyleliteralstrong{\sphinxupquote{raise\_on\_error}} (\sphinxstyleliteralemphasis{\sphinxupquote{bool}}\sphinxstyleliteralemphasis{\sphinxupquote{, }}\sphinxstyleliteralemphasis{\sphinxupquote{default = False}}) – If True, this disables the error handing and allows for errors from
the solving engines/solutions to be propagated out.

\item {} 
\sphinxAtStartPar
\sphinxstyleliteralstrong{\sphinxupquote{vehicle\_args}} (\sphinxstyleliteralemphasis{\sphinxupquote{dictionary}}\sphinxstyleliteralemphasis{\sphinxupquote{, }}\sphinxstyleliteralemphasis{\sphinxupquote{default = \{\}}}) – If the vehicle function for the provided solver engine needs
extra parameters not otherwise provided by the standard input,
they are given here.

\end{itemize}

\sphinxlineitem{Returns}
\sphinxAtStartPar
\begin{itemize}
\item {} 
\sphinxAtStartPar
\sphinxstylestrong{photometric\_solution} (\sphinxstyleemphasis{PhotometrySolution}) – The photometry solution for the image.

\item {} 
\sphinxAtStartPar
\sphinxstylestrong{solve\_status} (\sphinxstyleemphasis{bool}) – The status of the solve. If True, the solving was successful.

\item {} 
\sphinxAtStartPar
\sphinxstyleemphasis{Warning ..} – This requires that the astrometric solution be computed
before\sphinxhyphen{}hand. It will not be precomputed automatically; without it
being called explicitly, this will instead raise an error.

\end{itemize}


\end{description}\end{quote}

\end{fulllineitems}\end{savenotes}

\index{solve\_propagate() (opihiexarata.opihi.solution.OpihiSolution method)@\spxentry{solve\_propagate()}\spxextra{opihiexarata.opihi.solution.OpihiSolution method}}

\begin{savenotes}\begin{fulllineitems}
\phantomsection\label{\detokenize{code/opihiexarata.opihi.solution:opihiexarata.opihi.solution.OpihiSolution.solve_propagate}}
\pysigstartsignatures
\pysiglinewithargsret{\sphinxbfcode{\sphinxupquote{solve\_propagate}}}{\emph{\DUrole{n}{solver\_engine}\DUrole{p}{:}\DUrole{w}{  }\DUrole{n}{{\hyperref[\detokenize{code/opihiexarata.library.engine:opihiexarata.library.engine.PropagationEngine}]{\sphinxcrossref{PropagationEngine}}}}}, \emph{\DUrole{n}{overwrite}\DUrole{p}{:}\DUrole{w}{  }\DUrole{n}{bool}\DUrole{w}{  }\DUrole{o}{=}\DUrole{w}{  }\DUrole{default_value}{True}}, \emph{\DUrole{n}{raise\_on\_error}\DUrole{p}{:}\DUrole{w}{  }\DUrole{n}{bool}\DUrole{w}{  }\DUrole{o}{=}\DUrole{w}{  }\DUrole{default_value}{False}}, \emph{\DUrole{n}{asteroid\_location}\DUrole{p}{:}\DUrole{w}{  }\DUrole{n}{Optional\DUrole{p}{{[}}tuple\DUrole{p}{{[}}float\DUrole{p}{,}\DUrole{w}{  }float\DUrole{p}{{]}}\DUrole{p}{{]}}}\DUrole{w}{  }\DUrole{o}{=}\DUrole{w}{  }\DUrole{default_value}{None}}, \emph{\DUrole{n}{vehicle\_args}\DUrole{p}{:}\DUrole{w}{  }\DUrole{n}{dict}\DUrole{w}{  }\DUrole{o}{=}\DUrole{w}{  }\DUrole{default_value}{\{\}}}}{{ $\rightarrow$ {\hyperref[\detokenize{code/opihiexarata.propagate.solution:opihiexarata.propagate.solution.PropagativeSolution}]{\sphinxcrossref{PropagativeSolution}}}}}
\pysigstopsignatures
\sphinxAtStartPar
Solve for the location of an asteroid using a method of propagation.
\begin{quote}\begin{description}
\sphinxlineitem{Parameters}\begin{itemize}
\item {} 
\sphinxAtStartPar
\sphinxstyleliteralstrong{\sphinxupquote{solver\_engine}} ({\hyperref[\detokenize{code/opihiexarata.library.engine:opihiexarata.library.engine.PropagationEngine}]{\sphinxcrossref{\sphinxstyleliteralemphasis{\sphinxupquote{PropagationEngine}}}}}) – The propagative engine which the propagation solver will use.

\item {} 
\sphinxAtStartPar
\sphinxstyleliteralstrong{\sphinxupquote{overwrite}} (\sphinxstyleliteralemphasis{\sphinxupquote{bool}}\sphinxstyleliteralemphasis{\sphinxupquote{, }}\sphinxstyleliteralemphasis{\sphinxupquote{default = True}}) – Overwrite and replace the information of this class with the new
values. If False, the returned solution is not also applied.

\item {} 
\sphinxAtStartPar
\sphinxstyleliteralstrong{\sphinxupquote{asteroid\_location}} (\sphinxstyleliteralemphasis{\sphinxupquote{tuple}}\sphinxstyleliteralemphasis{\sphinxupquote{, }}\sphinxstyleliteralemphasis{\sphinxupquote{default = None}}) – The pixel location of the asteroid in the image. Defaults to the
value provided at instantiation.

\item {} 
\sphinxAtStartPar
\sphinxstyleliteralstrong{\sphinxupquote{raise\_on\_error}} (\sphinxstyleliteralemphasis{\sphinxupquote{bool}}\sphinxstyleliteralemphasis{\sphinxupquote{, }}\sphinxstyleliteralemphasis{\sphinxupquote{default = False}}) – If True, this disables the error handing and allows for errors from
the solving engines/solutions to be propagated out.

\item {} 
\sphinxAtStartPar
\sphinxstyleliteralstrong{\sphinxupquote{vehicle\_args}} (\sphinxstyleliteralemphasis{\sphinxupquote{dictionary}}\sphinxstyleliteralemphasis{\sphinxupquote{, }}\sphinxstyleliteralemphasis{\sphinxupquote{default = \{\}}}) – If the vehicle function for the provided solver engine needs
extra parameters not otherwise provided by the standard input,
they are given here.

\end{itemize}

\sphinxlineitem{Returns}
\sphinxAtStartPar
\begin{itemize}
\item {} 
\sphinxAtStartPar
\sphinxstylestrong{propagative\_solution} (\sphinxstyleemphasis{PropagativeSolution}) – The propagation solution for the asteroid and image.

\item {} 
\sphinxAtStartPar
\sphinxstylestrong{solve\_status} (\sphinxstyleemphasis{bool}) – The status of the solve. If True, the solving was successful.

\item {} 
\sphinxAtStartPar
\sphinxstyleemphasis{Warning ..} – This requires that the astrometric solution be computed
before\sphinxhyphen{}hand. It will not be precomputed automatically; without it
being called explicitly, this will instead raise an error.

\end{itemize}


\end{description}\end{quote}

\end{fulllineitems}\end{savenotes}


\end{fulllineitems}\end{savenotes}



\subparagraph{Module contents}
\label{\detokenize{code/opihiexarata.opihi:module-opihiexarata.opihi}}\label{\detokenize{code/opihiexarata.opihi:module-contents}}\index{module@\spxentry{module}!opihiexarata.opihi@\spxentry{opihiexarata.opihi}}\index{opihiexarata.opihi@\spxentry{opihiexarata.opihi}!module@\spxentry{module}}
\sphinxAtStartPar
The class for the collection of Exarata solutions.

\sphinxstepscope


\paragraph{opihiexarata.orbit package}
\label{\detokenize{code/opihiexarata.orbit:opihiexarata-orbit-package}}\label{\detokenize{code/opihiexarata.orbit::doc}}

\subparagraph{Submodules}
\label{\detokenize{code/opihiexarata.orbit:submodules}}
\sphinxstepscope


\subparagraph{opihiexarata.orbit.custom module}
\label{\detokenize{code/opihiexarata.orbit.custom:module-opihiexarata.orbit.custom}}\label{\detokenize{code/opihiexarata.orbit.custom:opihiexarata-orbit-custom-module}}\label{\detokenize{code/opihiexarata.orbit.custom::doc}}\index{module@\spxentry{module}!opihiexarata.orbit.custom@\spxentry{opihiexarata.orbit.custom}}\index{opihiexarata.orbit.custom@\spxentry{opihiexarata.orbit.custom}!module@\spxentry{module}}
\sphinxAtStartPar
This is a class which defines a custom orbit. A user supplies the orbital
elements to this engine and the vehicle function.
\index{CustomOrbitEngine (class in opihiexarata.orbit.custom)@\spxentry{CustomOrbitEngine}\spxextra{class in opihiexarata.orbit.custom}}

\begin{savenotes}\begin{fulllineitems}
\phantomsection\label{\detokenize{code/opihiexarata.orbit.custom:opihiexarata.orbit.custom.CustomOrbitEngine}}
\pysigstartsignatures
\pysiglinewithargsret{\sphinxbfcode{\sphinxupquote{class\DUrole{w}{  }}}\sphinxcode{\sphinxupquote{opihiexarata.orbit.custom.}}\sphinxbfcode{\sphinxupquote{CustomOrbitEngine}}}{\emph{\DUrole{n}{semimajor\_axis}\DUrole{p}{:}\DUrole{w}{  }\DUrole{n}{float}}, \emph{\DUrole{n}{eccentricity}\DUrole{p}{:}\DUrole{w}{  }\DUrole{n}{float}}, \emph{\DUrole{n}{inclination}\DUrole{p}{:}\DUrole{w}{  }\DUrole{n}{float}}, \emph{\DUrole{n}{longitude\_ascending\_node}\DUrole{p}{:}\DUrole{w}{  }\DUrole{n}{float}}, \emph{\DUrole{n}{argument\_perihelion}\DUrole{p}{:}\DUrole{w}{  }\DUrole{n}{float}}, \emph{\DUrole{n}{mean\_anomaly}\DUrole{p}{:}\DUrole{w}{  }\DUrole{n}{float}}, \emph{\DUrole{n}{epoch\_julian\_day}\DUrole{p}{:}\DUrole{w}{  }\DUrole{n}{float}}, \emph{\DUrole{n}{semimajor\_axis\_error}\DUrole{p}{:}\DUrole{w}{  }\DUrole{n}{Optional\DUrole{p}{{[}}float\DUrole{p}{{]}}}\DUrole{w}{  }\DUrole{o}{=}\DUrole{w}{  }\DUrole{default_value}{None}}, \emph{\DUrole{n}{eccentricity\_error}\DUrole{p}{:}\DUrole{w}{  }\DUrole{n}{Optional\DUrole{p}{{[}}float\DUrole{p}{{]}}}\DUrole{w}{  }\DUrole{o}{=}\DUrole{w}{  }\DUrole{default_value}{None}}, \emph{\DUrole{n}{inclination\_error}\DUrole{p}{:}\DUrole{w}{  }\DUrole{n}{Optional\DUrole{p}{{[}}float\DUrole{p}{{]}}}\DUrole{w}{  }\DUrole{o}{=}\DUrole{w}{  }\DUrole{default_value}{None}}, \emph{\DUrole{n}{longitude\_ascending\_node\_error}\DUrole{p}{:}\DUrole{w}{  }\DUrole{n}{Optional\DUrole{p}{{[}}float\DUrole{p}{{]}}}\DUrole{w}{  }\DUrole{o}{=}\DUrole{w}{  }\DUrole{default_value}{None}}, \emph{\DUrole{n}{argument\_perihelion\_error}\DUrole{p}{:}\DUrole{w}{  }\DUrole{n}{Optional\DUrole{p}{{[}}float\DUrole{p}{{]}}}\DUrole{w}{  }\DUrole{o}{=}\DUrole{w}{  }\DUrole{default_value}{None}}, \emph{\DUrole{n}{mean\_anomaly\_error}\DUrole{p}{:}\DUrole{w}{  }\DUrole{n}{Optional\DUrole{p}{{[}}float\DUrole{p}{{]}}}\DUrole{w}{  }\DUrole{o}{=}\DUrole{w}{  }\DUrole{default_value}{None}}}{}
\pysigstopsignatures
\sphinxAtStartPar
Bases: {\hyperref[\detokenize{code/opihiexarata.library.engine:opihiexarata.library.engine.OrbitEngine}]{\sphinxcrossref{\sphinxcode{\sphinxupquote{OrbitEngine}}}}}

\sphinxAtStartPar
This engine is just a wrapper for when a custom orbit is desired to be
specified.
\index{semimajor\_axis (opihiexarata.orbit.custom.CustomOrbitEngine attribute)@\spxentry{semimajor\_axis}\spxextra{opihiexarata.orbit.custom.CustomOrbitEngine attribute}}

\begin{savenotes}\begin{fulllineitems}
\phantomsection\label{\detokenize{code/opihiexarata.orbit.custom:opihiexarata.orbit.custom.CustomOrbitEngine.semimajor_axis}}
\pysigstartsignatures
\pysigline{\sphinxbfcode{\sphinxupquote{semimajor\_axis}}}
\pysigstopsignatures
\sphinxAtStartPar
The semi\sphinxhyphen{}major axis of the orbit provided, in AU.
\begin{quote}\begin{description}
\sphinxlineitem{Type}
\sphinxAtStartPar
float

\end{description}\end{quote}

\end{fulllineitems}\end{savenotes}

\index{semimajor\_axis\_error (opihiexarata.orbit.custom.CustomOrbitEngine attribute)@\spxentry{semimajor\_axis\_error}\spxextra{opihiexarata.orbit.custom.CustomOrbitEngine attribute}}

\begin{savenotes}\begin{fulllineitems}
\phantomsection\label{\detokenize{code/opihiexarata.orbit.custom:opihiexarata.orbit.custom.CustomOrbitEngine.semimajor_axis_error}}
\pysigstartsignatures
\pysigline{\sphinxbfcode{\sphinxupquote{semimajor\_axis\_error}}}
\pysigstopsignatures
\sphinxAtStartPar
The error on the semi\sphinxhyphen{}major axis of the orbit provided, in AU.
\begin{quote}\begin{description}
\sphinxlineitem{Type}
\sphinxAtStartPar
float

\end{description}\end{quote}

\end{fulllineitems}\end{savenotes}

\index{eccentricity (opihiexarata.orbit.custom.CustomOrbitEngine attribute)@\spxentry{eccentricity}\spxextra{opihiexarata.orbit.custom.CustomOrbitEngine attribute}}

\begin{savenotes}\begin{fulllineitems}
\phantomsection\label{\detokenize{code/opihiexarata.orbit.custom:opihiexarata.orbit.custom.CustomOrbitEngine.eccentricity}}
\pysigstartsignatures
\pysigline{\sphinxbfcode{\sphinxupquote{eccentricity}}}
\pysigstopsignatures
\sphinxAtStartPar
The eccentricity of the orbit provided.
\begin{quote}\begin{description}
\sphinxlineitem{Type}
\sphinxAtStartPar
float

\end{description}\end{quote}

\end{fulllineitems}\end{savenotes}

\index{eccentricity\_error (opihiexarata.orbit.custom.CustomOrbitEngine attribute)@\spxentry{eccentricity\_error}\spxextra{opihiexarata.orbit.custom.CustomOrbitEngine attribute}}

\begin{savenotes}\begin{fulllineitems}
\phantomsection\label{\detokenize{code/opihiexarata.orbit.custom:opihiexarata.orbit.custom.CustomOrbitEngine.eccentricity_error}}
\pysigstartsignatures
\pysigline{\sphinxbfcode{\sphinxupquote{eccentricity\_error}}}
\pysigstopsignatures
\sphinxAtStartPar
The error on the eccentricity of the orbit provided.
\begin{quote}\begin{description}
\sphinxlineitem{Type}
\sphinxAtStartPar
float

\end{description}\end{quote}

\end{fulllineitems}\end{savenotes}

\index{inclination (opihiexarata.orbit.custom.CustomOrbitEngine attribute)@\spxentry{inclination}\spxextra{opihiexarata.orbit.custom.CustomOrbitEngine attribute}}

\begin{savenotes}\begin{fulllineitems}
\phantomsection\label{\detokenize{code/opihiexarata.orbit.custom:opihiexarata.orbit.custom.CustomOrbitEngine.inclination}}
\pysigstartsignatures
\pysigline{\sphinxbfcode{\sphinxupquote{inclination}}}
\pysigstopsignatures
\sphinxAtStartPar
The angle of inclination of the orbit provided, in degrees.
\begin{quote}\begin{description}
\sphinxlineitem{Type}
\sphinxAtStartPar
float

\end{description}\end{quote}

\end{fulllineitems}\end{savenotes}

\index{inclination\_error (opihiexarata.orbit.custom.CustomOrbitEngine attribute)@\spxentry{inclination\_error}\spxextra{opihiexarata.orbit.custom.CustomOrbitEngine attribute}}

\begin{savenotes}\begin{fulllineitems}
\phantomsection\label{\detokenize{code/opihiexarata.orbit.custom:opihiexarata.orbit.custom.CustomOrbitEngine.inclination_error}}
\pysigstartsignatures
\pysigline{\sphinxbfcode{\sphinxupquote{inclination\_error}}}
\pysigstopsignatures
\sphinxAtStartPar
The error on the angle of inclination of the orbit provided, in degrees.
\begin{quote}\begin{description}
\sphinxlineitem{Type}
\sphinxAtStartPar
float

\end{description}\end{quote}

\end{fulllineitems}\end{savenotes}

\index{longitude\_ascending\_node (opihiexarata.orbit.custom.CustomOrbitEngine attribute)@\spxentry{longitude\_ascending\_node}\spxextra{opihiexarata.orbit.custom.CustomOrbitEngine attribute}}

\begin{savenotes}\begin{fulllineitems}
\phantomsection\label{\detokenize{code/opihiexarata.orbit.custom:opihiexarata.orbit.custom.CustomOrbitEngine.longitude_ascending_node}}
\pysigstartsignatures
\pysigline{\sphinxbfcode{\sphinxupquote{longitude\_ascending\_node}}}
\pysigstopsignatures
\sphinxAtStartPar
The longitude of the ascending node of the orbit provided, in degrees.
\begin{quote}\begin{description}
\sphinxlineitem{Type}
\sphinxAtStartPar
float

\end{description}\end{quote}

\end{fulllineitems}\end{savenotes}

\index{longitude\_ascending\_node\_error (opihiexarata.orbit.custom.CustomOrbitEngine attribute)@\spxentry{longitude\_ascending\_node\_error}\spxextra{opihiexarata.orbit.custom.CustomOrbitEngine attribute}}

\begin{savenotes}\begin{fulllineitems}
\phantomsection\label{\detokenize{code/opihiexarata.orbit.custom:opihiexarata.orbit.custom.CustomOrbitEngine.longitude_ascending_node_error}}
\pysigstartsignatures
\pysigline{\sphinxbfcode{\sphinxupquote{longitude\_ascending\_node\_error}}}
\pysigstopsignatures
\sphinxAtStartPar
The error on the longitude of the ascending node of the orbit
provided, in degrees.
\begin{quote}\begin{description}
\sphinxlineitem{Type}
\sphinxAtStartPar
float

\end{description}\end{quote}

\end{fulllineitems}\end{savenotes}

\index{argument\_perihelion (opihiexarata.orbit.custom.CustomOrbitEngine attribute)@\spxentry{argument\_perihelion}\spxextra{opihiexarata.orbit.custom.CustomOrbitEngine attribute}}

\begin{savenotes}\begin{fulllineitems}
\phantomsection\label{\detokenize{code/opihiexarata.orbit.custom:opihiexarata.orbit.custom.CustomOrbitEngine.argument_perihelion}}
\pysigstartsignatures
\pysigline{\sphinxbfcode{\sphinxupquote{argument\_perihelion}}}
\pysigstopsignatures
\sphinxAtStartPar
The argument of perihelion of the orbit provided, in degrees.
\begin{quote}\begin{description}
\sphinxlineitem{Type}
\sphinxAtStartPar
float

\end{description}\end{quote}

\end{fulllineitems}\end{savenotes}

\index{argument\_perihelion\_error (opihiexarata.orbit.custom.CustomOrbitEngine attribute)@\spxentry{argument\_perihelion\_error}\spxextra{opihiexarata.orbit.custom.CustomOrbitEngine attribute}}

\begin{savenotes}\begin{fulllineitems}
\phantomsection\label{\detokenize{code/opihiexarata.orbit.custom:opihiexarata.orbit.custom.CustomOrbitEngine.argument_perihelion_error}}
\pysigstartsignatures
\pysigline{\sphinxbfcode{\sphinxupquote{argument\_perihelion\_error}}}
\pysigstopsignatures
\sphinxAtStartPar
The error on the argument of perihelion of the orbit
provided, in degrees.
\begin{quote}\begin{description}
\sphinxlineitem{Type}
\sphinxAtStartPar
float

\end{description}\end{quote}

\end{fulllineitems}\end{savenotes}

\index{mean\_anomaly (opihiexarata.orbit.custom.CustomOrbitEngine attribute)@\spxentry{mean\_anomaly}\spxextra{opihiexarata.orbit.custom.CustomOrbitEngine attribute}}

\begin{savenotes}\begin{fulllineitems}
\phantomsection\label{\detokenize{code/opihiexarata.orbit.custom:opihiexarata.orbit.custom.CustomOrbitEngine.mean_anomaly}}
\pysigstartsignatures
\pysigline{\sphinxbfcode{\sphinxupquote{mean\_anomaly}}}
\pysigstopsignatures
\sphinxAtStartPar
The mean anomaly of the orbit provided, in degrees.
\begin{quote}\begin{description}
\sphinxlineitem{Type}
\sphinxAtStartPar
float

\end{description}\end{quote}

\end{fulllineitems}\end{savenotes}

\index{mean\_anomaly\_error (opihiexarata.orbit.custom.CustomOrbitEngine attribute)@\spxentry{mean\_anomaly\_error}\spxextra{opihiexarata.orbit.custom.CustomOrbitEngine attribute}}

\begin{savenotes}\begin{fulllineitems}
\phantomsection\label{\detokenize{code/opihiexarata.orbit.custom:opihiexarata.orbit.custom.CustomOrbitEngine.mean_anomaly_error}}
\pysigstartsignatures
\pysigline{\sphinxbfcode{\sphinxupquote{mean\_anomaly\_error}}}
\pysigstopsignatures
\sphinxAtStartPar
The error on the mean anomaly of the orbit provided, in degrees.
\begin{quote}\begin{description}
\sphinxlineitem{Type}
\sphinxAtStartPar
float

\end{description}\end{quote}

\end{fulllineitems}\end{savenotes}

\index{epoch\_julian\_day (opihiexarata.orbit.custom.CustomOrbitEngine attribute)@\spxentry{epoch\_julian\_day}\spxextra{opihiexarata.orbit.custom.CustomOrbitEngine attribute}}

\begin{savenotes}\begin{fulllineitems}
\phantomsection\label{\detokenize{code/opihiexarata.orbit.custom:opihiexarata.orbit.custom.CustomOrbitEngine.epoch_julian_day}}
\pysigstartsignatures
\pysigline{\sphinxbfcode{\sphinxupquote{epoch\_julian\_day}}}
\pysigstopsignatures
\sphinxAtStartPar
The epoch where for these osculating orbital elements. This value is
in Julian days.
\begin{quote}\begin{description}
\sphinxlineitem{Type}
\sphinxAtStartPar
float

\end{description}\end{quote}

\end{fulllineitems}\end{savenotes}

\index{\_\_init\_\_() (opihiexarata.orbit.custom.CustomOrbitEngine method)@\spxentry{\_\_init\_\_()}\spxextra{opihiexarata.orbit.custom.CustomOrbitEngine method}}

\begin{savenotes}\begin{fulllineitems}
\phantomsection\label{\detokenize{code/opihiexarata.orbit.custom:opihiexarata.orbit.custom.CustomOrbitEngine.__init__}}
\pysigstartsignatures
\pysiglinewithargsret{\sphinxbfcode{\sphinxupquote{\_\_init\_\_}}}{\emph{\DUrole{n}{semimajor\_axis}\DUrole{p}{:}\DUrole{w}{  }\DUrole{n}{float}}, \emph{\DUrole{n}{eccentricity}\DUrole{p}{:}\DUrole{w}{  }\DUrole{n}{float}}, \emph{\DUrole{n}{inclination}\DUrole{p}{:}\DUrole{w}{  }\DUrole{n}{float}}, \emph{\DUrole{n}{longitude\_ascending\_node}\DUrole{p}{:}\DUrole{w}{  }\DUrole{n}{float}}, \emph{\DUrole{n}{argument\_perihelion}\DUrole{p}{:}\DUrole{w}{  }\DUrole{n}{float}}, \emph{\DUrole{n}{mean\_anomaly}\DUrole{p}{:}\DUrole{w}{  }\DUrole{n}{float}}, \emph{\DUrole{n}{epoch\_julian\_day}\DUrole{p}{:}\DUrole{w}{  }\DUrole{n}{float}}, \emph{\DUrole{n}{semimajor\_axis\_error}\DUrole{p}{:}\DUrole{w}{  }\DUrole{n}{Optional\DUrole{p}{{[}}float\DUrole{p}{{]}}}\DUrole{w}{  }\DUrole{o}{=}\DUrole{w}{  }\DUrole{default_value}{None}}, \emph{\DUrole{n}{eccentricity\_error}\DUrole{p}{:}\DUrole{w}{  }\DUrole{n}{Optional\DUrole{p}{{[}}float\DUrole{p}{{]}}}\DUrole{w}{  }\DUrole{o}{=}\DUrole{w}{  }\DUrole{default_value}{None}}, \emph{\DUrole{n}{inclination\_error}\DUrole{p}{:}\DUrole{w}{  }\DUrole{n}{Optional\DUrole{p}{{[}}float\DUrole{p}{{]}}}\DUrole{w}{  }\DUrole{o}{=}\DUrole{w}{  }\DUrole{default_value}{None}}, \emph{\DUrole{n}{longitude\_ascending\_node\_error}\DUrole{p}{:}\DUrole{w}{  }\DUrole{n}{Optional\DUrole{p}{{[}}float\DUrole{p}{{]}}}\DUrole{w}{  }\DUrole{o}{=}\DUrole{w}{  }\DUrole{default_value}{None}}, \emph{\DUrole{n}{argument\_perihelion\_error}\DUrole{p}{:}\DUrole{w}{  }\DUrole{n}{Optional\DUrole{p}{{[}}float\DUrole{p}{{]}}}\DUrole{w}{  }\DUrole{o}{=}\DUrole{w}{  }\DUrole{default_value}{None}}, \emph{\DUrole{n}{mean\_anomaly\_error}\DUrole{p}{:}\DUrole{w}{  }\DUrole{n}{Optional\DUrole{p}{{[}}float\DUrole{p}{{]}}}\DUrole{w}{  }\DUrole{o}{=}\DUrole{w}{  }\DUrole{default_value}{None}}}{{ $\rightarrow$ None}}
\pysigstopsignatures
\sphinxAtStartPar
The orbital elements are already provided for this custom
solution. If errors may optionally be provided.
\begin{quote}\begin{description}
\sphinxlineitem{Parameters}\begin{itemize}
\item {} 
\sphinxAtStartPar
\sphinxstyleliteralstrong{\sphinxupquote{semimajor\_axis}} (\sphinxstyleliteralemphasis{\sphinxupquote{float}}) – The semi\sphinxhyphen{}major axis of the orbit provided, in AU.

\item {} 
\sphinxAtStartPar
\sphinxstyleliteralstrong{\sphinxupquote{eccentricity}} (\sphinxstyleliteralemphasis{\sphinxupquote{float}}) – The eccentricity of the orbit provided.

\item {} 
\sphinxAtStartPar
\sphinxstyleliteralstrong{\sphinxupquote{inclination}} (\sphinxstyleliteralemphasis{\sphinxupquote{float}}) – The angle of inclination of the orbit provided, in degrees.

\item {} 
\sphinxAtStartPar
\sphinxstyleliteralstrong{\sphinxupquote{longitude\_ascending\_node}} (\sphinxstyleliteralemphasis{\sphinxupquote{float}}) – The longitude of the ascending node of the orbit
provided, in degrees.

\item {} 
\sphinxAtStartPar
\sphinxstyleliteralstrong{\sphinxupquote{argument\_perihelion}} (\sphinxstyleliteralemphasis{\sphinxupquote{float}}) – The argument of perihelion of the orbit provided, in degrees.

\item {} 
\sphinxAtStartPar
\sphinxstyleliteralstrong{\sphinxupquote{mean\_anomaly}} (\sphinxstyleliteralemphasis{\sphinxupquote{float}}) – The mean anomaly of the orbit provided, in degrees.

\item {} 
\sphinxAtStartPar
\sphinxstyleliteralstrong{\sphinxupquote{epoch\_julian\_day}} (\sphinxstyleliteralemphasis{\sphinxupquote{float}}) – The epoch where for these osculating orbital elements. This value is
in Julian days.

\item {} 
\sphinxAtStartPar
\sphinxstyleliteralstrong{\sphinxupquote{semimajor\_axis\_error}} (\sphinxstyleliteralemphasis{\sphinxupquote{float}}\sphinxstyleliteralemphasis{\sphinxupquote{, }}\sphinxstyleliteralemphasis{\sphinxupquote{default = None}}) – The error on the semi\sphinxhyphen{}major axis of the orbit provided, in AU.

\item {} 
\sphinxAtStartPar
\sphinxstyleliteralstrong{\sphinxupquote{eccentricity\_error}} (\sphinxstyleliteralemphasis{\sphinxupquote{float}}\sphinxstyleliteralemphasis{\sphinxupquote{, }}\sphinxstyleliteralemphasis{\sphinxupquote{default = None}}) – The error on the eccentricity of the orbit provided.

\item {} 
\sphinxAtStartPar
\sphinxstyleliteralstrong{\sphinxupquote{inclination\_error}} (\sphinxstyleliteralemphasis{\sphinxupquote{float}}\sphinxstyleliteralemphasis{\sphinxupquote{, }}\sphinxstyleliteralemphasis{\sphinxupquote{default = None}}) – The error on the angle of inclination of the orbit provided, in degrees.

\item {} 
\sphinxAtStartPar
\sphinxstyleliteralstrong{\sphinxupquote{longitude\_ascending\_node\_error}} (\sphinxstyleliteralemphasis{\sphinxupquote{float}}\sphinxstyleliteralemphasis{\sphinxupquote{, }}\sphinxstyleliteralemphasis{\sphinxupquote{default = None}}) – The error on the longitude of the ascending node of the orbit
provided, in degrees.

\item {} 
\sphinxAtStartPar
\sphinxstyleliteralstrong{\sphinxupquote{argument\_perihelion\_error}} (\sphinxstyleliteralemphasis{\sphinxupquote{float}}\sphinxstyleliteralemphasis{\sphinxupquote{, }}\sphinxstyleliteralemphasis{\sphinxupquote{default = None}}) – The error on the argument of perihelion of the orbit
provided, in degrees.

\item {} 
\sphinxAtStartPar
\sphinxstyleliteralstrong{\sphinxupquote{mean\_anomaly\_error}} (\sphinxstyleliteralemphasis{\sphinxupquote{float}}\sphinxstyleliteralemphasis{\sphinxupquote{, }}\sphinxstyleliteralemphasis{\sphinxupquote{default = None}}) – The error on the mean anomaly of the orbit provided, in degrees.

\end{itemize}

\sphinxlineitem{Return type}
\sphinxAtStartPar
None

\end{description}\end{quote}

\end{fulllineitems}\end{savenotes}


\end{fulllineitems}\end{savenotes}


\sphinxstepscope


\subparagraph{opihiexarata.orbit.orbfit module}
\label{\detokenize{code/opihiexarata.orbit.orbfit:module-opihiexarata.orbit.orbfit}}\label{\detokenize{code/opihiexarata.orbit.orbfit:opihiexarata-orbit-orbfit-module}}\label{\detokenize{code/opihiexarata.orbit.orbfit::doc}}\index{module@\spxentry{module}!opihiexarata.orbit.orbfit@\spxentry{opihiexarata.orbit.orbfit}}\index{opihiexarata.orbit.orbfit@\spxentry{opihiexarata.orbit.orbfit}!module@\spxentry{module}}
\sphinxAtStartPar
This contains the Python wrapper class around Orbfit, assuming the
installation procedure was followed.
\index{OrbfitOrbitDeterminerEngine (class in opihiexarata.orbit.orbfit)@\spxentry{OrbfitOrbitDeterminerEngine}\spxextra{class in opihiexarata.orbit.orbfit}}

\begin{savenotes}\begin{fulllineitems}
\phantomsection\label{\detokenize{code/opihiexarata.orbit.orbfit:opihiexarata.orbit.orbfit.OrbfitOrbitDeterminerEngine}}
\pysigstartsignatures
\pysigline{\sphinxbfcode{\sphinxupquote{class\DUrole{w}{  }}}\sphinxcode{\sphinxupquote{opihiexarata.orbit.orbfit.}}\sphinxbfcode{\sphinxupquote{OrbfitOrbitDeterminerEngine}}}
\pysigstopsignatures
\sphinxAtStartPar
Bases: {\hyperref[\detokenize{code/opihiexarata.library.engine:opihiexarata.library.engine.OrbitEngine}]{\sphinxcrossref{\sphinxcode{\sphinxupquote{OrbitEngine}}}}}

\sphinxAtStartPar
Uses the Orbfit package to determine the orbital elements of an astroid
provided observations. This assumes that the installation instructions
provided were followed.
\index{orbital\_elements (opihiexarata.orbit.orbfit.OrbfitOrbitDeterminerEngine attribute)@\spxentry{orbital\_elements}\spxextra{opihiexarata.orbit.orbfit.OrbfitOrbitDeterminerEngine attribute}}

\begin{savenotes}\begin{fulllineitems}
\phantomsection\label{\detokenize{code/opihiexarata.orbit.orbfit:opihiexarata.orbit.orbfit.OrbfitOrbitDeterminerEngine.orbital_elements}}
\pysigstartsignatures
\pysigline{\sphinxbfcode{\sphinxupquote{orbital\_elements}}}
\pysigstopsignatures
\sphinxAtStartPar
The six Keplarian orbital elements.
\begin{quote}\begin{description}
\sphinxlineitem{Type}
\sphinxAtStartPar
dict

\end{description}\end{quote}

\end{fulllineitems}\end{savenotes}

\index{orbital\_elements\_errors (opihiexarata.orbit.orbfit.OrbfitOrbitDeterminerEngine attribute)@\spxentry{orbital\_elements\_errors}\spxextra{opihiexarata.orbit.orbfit.OrbfitOrbitDeterminerEngine attribute}}

\begin{savenotes}\begin{fulllineitems}
\phantomsection\label{\detokenize{code/opihiexarata.orbit.orbfit:opihiexarata.orbit.orbfit.OrbfitOrbitDeterminerEngine.orbital_elements_errors}}
\pysigstartsignatures
\pysigline{\sphinxbfcode{\sphinxupquote{orbital\_elements\_errors}}}
\pysigstopsignatures
\sphinxAtStartPar
The errors of the orbital elements.
\begin{quote}\begin{description}
\sphinxlineitem{Type}
\sphinxAtStartPar
dict

\end{description}\end{quote}

\end{fulllineitems}\end{savenotes}

\index{\_\_check\_installation() (opihiexarata.orbit.orbfit.OrbfitOrbitDeterminerEngine class method)@\spxentry{\_\_check\_installation()}\spxextra{opihiexarata.orbit.orbfit.OrbfitOrbitDeterminerEngine class method}}

\begin{savenotes}\begin{fulllineitems}
\phantomsection\label{\detokenize{code/opihiexarata.orbit.orbfit:opihiexarata.orbit.orbfit.OrbfitOrbitDeterminerEngine.__check_installation}}
\pysigstartsignatures
\pysiglinewithargsret{\sphinxbfcode{\sphinxupquote{classmethod\DUrole{w}{  }}}\sphinxbfcode{\sphinxupquote{\_\_check\_installation}}}{\emph{\DUrole{n}{no\_raise}\DUrole{p}{:}\DUrole{w}{  }\DUrole{n}{bool}\DUrole{w}{  }\DUrole{o}{=}\DUrole{w}{  }\DUrole{default_value}{False}}}{{ $\rightarrow$ bool}}
\pysigstopsignatures
\sphinxAtStartPar
Check if the installation was completed according to the
documentation provided. This functions only checks for the existence
of the template files.
\begin{quote}\begin{description}
\sphinxlineitem{Parameters}
\sphinxAtStartPar
\sphinxstyleliteralstrong{\sphinxupquote{no\_raise}} (\sphinxstyleliteralemphasis{\sphinxupquote{bool}}\sphinxstyleliteralemphasis{\sphinxupquote{, }}\sphinxstyleliteralemphasis{\sphinxupquote{default = False}}) – By default, an invalid install will raise. Set this if a False
return without interrupting the flow of the program is desired.

\sphinxlineitem{Returns}
\sphinxAtStartPar
\sphinxstylestrong{valid\_install} – If the installation is detected to be valid, then this returns True.

\sphinxlineitem{Return type}
\sphinxAtStartPar
bool

\end{description}\end{quote}

\end{fulllineitems}\end{savenotes}

\index{\_\_init\_\_() (opihiexarata.orbit.orbfit.OrbfitOrbitDeterminerEngine method)@\spxentry{\_\_init\_\_()}\spxextra{opihiexarata.orbit.orbfit.OrbfitOrbitDeterminerEngine method}}

\begin{savenotes}\begin{fulllineitems}
\phantomsection\label{\detokenize{code/opihiexarata.orbit.orbfit:opihiexarata.orbit.orbfit.OrbfitOrbitDeterminerEngine.__init__}}
\pysigstartsignatures
\pysiglinewithargsret{\sphinxbfcode{\sphinxupquote{\_\_init\_\_}}}{}{{ $\rightarrow$ None}}
\pysigstopsignatures
\sphinxAtStartPar
Instantiation of the orbfit package.
\begin{quote}\begin{description}
\sphinxlineitem{Parameters}
\sphinxAtStartPar
\sphinxstyleliteralstrong{\sphinxupquote{None}} – 

\sphinxlineitem{Return type}
\sphinxAtStartPar
None

\end{description}\end{quote}

\end{fulllineitems}\end{savenotes}

\index{\_clean\_orbfit\_files() (opihiexarata.orbit.orbfit.OrbfitOrbitDeterminerEngine method)@\spxentry{\_clean\_orbfit\_files()}\spxextra{opihiexarata.orbit.orbfit.OrbfitOrbitDeterminerEngine method}}

\begin{savenotes}\begin{fulllineitems}
\phantomsection\label{\detokenize{code/opihiexarata.orbit.orbfit:opihiexarata.orbit.orbfit.OrbfitOrbitDeterminerEngine._clean_orbfit_files}}
\pysigstartsignatures
\pysiglinewithargsret{\sphinxbfcode{\sphinxupquote{\_clean\_orbfit\_files}}}{}{{ $\rightarrow$ None}}
\pysigstopsignatures
\sphinxAtStartPar
This function cleans up the operational directory of Orbfit.
If there are leftover files, the program may try to use them in a
manner which is not desired. It is usually better to start from
scratch each time to avoid these issues.
\begin{quote}\begin{description}
\sphinxlineitem{Parameters}
\sphinxAtStartPar
\sphinxstyleliteralstrong{\sphinxupquote{None}} – 

\sphinxlineitem{Return type}
\sphinxAtStartPar
None

\end{description}\end{quote}

\end{fulllineitems}\end{savenotes}

\index{\_prepare\_orbfit\_files() (opihiexarata.orbit.orbfit.OrbfitOrbitDeterminerEngine method)@\spxentry{\_prepare\_orbfit\_files()}\spxextra{opihiexarata.orbit.orbfit.OrbfitOrbitDeterminerEngine method}}

\begin{savenotes}\begin{fulllineitems}
\phantomsection\label{\detokenize{code/opihiexarata.orbit.orbfit:opihiexarata.orbit.orbfit.OrbfitOrbitDeterminerEngine._prepare_orbfit_files}}
\pysigstartsignatures
\pysiglinewithargsret{\sphinxbfcode{\sphinxupquote{\_prepare\_orbfit\_files}}}{}{{ $\rightarrow$ None}}
\pysigstopsignatures
\sphinxAtStartPar
This function prepares the operational directory for Orbfit inside
of the temporary directory. This allows for files to be generated for
useage by the binary orbfit.
\begin{quote}\begin{description}
\sphinxlineitem{Parameters}
\sphinxAtStartPar
\sphinxstyleliteralstrong{\sphinxupquote{None}} – 

\sphinxlineitem{Return type}
\sphinxAtStartPar
None

\end{description}\end{quote}

\end{fulllineitems}\end{savenotes}

\index{\_solve\_single\_orbit() (opihiexarata.orbit.orbfit.OrbfitOrbitDeterminerEngine method)@\spxentry{\_solve\_single\_orbit()}\spxextra{opihiexarata.orbit.orbfit.OrbfitOrbitDeterminerEngine method}}

\begin{savenotes}\begin{fulllineitems}
\phantomsection\label{\detokenize{code/opihiexarata.orbit.orbfit:opihiexarata.orbit.orbfit.OrbfitOrbitDeterminerEngine._solve_single_orbit}}
\pysigstartsignatures
\pysiglinewithargsret{\sphinxbfcode{\sphinxupquote{\_solve\_single\_orbit}}}{\emph{\DUrole{n}{observation\_table}\DUrole{p}{:}\DUrole{w}{  }\DUrole{n}{Table}}}{{ $\rightarrow$ tuple\DUrole{p}{{[}}dict\DUrole{p}{,}\DUrole{w}{  }dict\DUrole{p}{,}\DUrole{w}{  }float\DUrole{p}{{]}}}}
\pysigstopsignatures
\sphinxAtStartPar
This uses the Orbfit program to an orbit provided a record of
observations. If it cannot be solved, an error is raised.

\sphinxAtStartPar
This function is not intended to solve an entire set of observations,
but instead a subset of them. The rational being that orbfit tends to
fail on larger sets; averaging the values of smaller sets is a little
more robust against failure. Use the non\sphinxhyphen{}hidden function to compute
orbital elements for a full range of observations.
\begin{quote}\begin{description}
\sphinxlineitem{Parameters}
\sphinxAtStartPar
\sphinxstyleliteralstrong{\sphinxupquote{observation\_table}} (\sphinxstyleliteralemphasis{\sphinxupquote{Astropy Table}}) – The table of observations; this will be converted to the
required formats for processing.

\sphinxlineitem{Returns}
\sphinxAtStartPar
\begin{itemize}
\item {} 
\sphinxAtStartPar
\sphinxstylestrong{kepler\_elements} (\sphinxstyleemphasis{dict}) – The Keplarian orbital elements.

\item {} 
\sphinxAtStartPar
\sphinxstylestrong{kepler\_error} (\sphinxstyleemphasis{dict}) – The error on the Keplarian orbital elements.

\item {} 
\sphinxAtStartPar
\sphinxstylestrong{modified\_julian\_date} (\sphinxstyleemphasis{float}) – The modified Julian date corresponding to the osculating orbit and
the Keplerian orbital parameters provided.

\end{itemize}


\end{description}\end{quote}

\end{fulllineitems}\end{savenotes}

\index{solve\_orbit() (opihiexarata.orbit.orbfit.OrbfitOrbitDeterminerEngine method)@\spxentry{solve\_orbit()}\spxextra{opihiexarata.orbit.orbfit.OrbfitOrbitDeterminerEngine method}}

\begin{savenotes}\begin{fulllineitems}
\phantomsection\label{\detokenize{code/opihiexarata.orbit.orbfit:opihiexarata.orbit.orbfit.OrbfitOrbitDeterminerEngine.solve_orbit}}
\pysigstartsignatures
\pysiglinewithargsret{\sphinxbfcode{\sphinxupquote{solve\_orbit}}}{\emph{\DUrole{n}{observation\_table}\DUrole{p}{:}\DUrole{w}{  }\DUrole{n}{Table}}}{{ $\rightarrow$ tuple\DUrole{p}{{[}}dict\DUrole{p}{,}\DUrole{w}{  }dict\DUrole{p}{,}\DUrole{w}{  }float\DUrole{p}{{]}}}}
\pysigstopsignatures
\sphinxAtStartPar
Attempts to compute Keplarian orbits provided a table of observations.

\sphinxAtStartPar
This function attempts to compute the orbit using the entire
observation table. If it is unable to, then the observations are split
into subsets based on the year of observations. The derived orbital
elements are averaged and errors propagated. If no orbit is found,
then an error is raised.
\begin{quote}\begin{description}
\sphinxlineitem{Parameters}
\sphinxAtStartPar
\sphinxstyleliteralstrong{\sphinxupquote{observation\_record}} (\sphinxstyleliteralemphasis{\sphinxupquote{Astropy Table}}) – The table of observational records; this will be converted to the
required MPC 80 column format.

\sphinxlineitem{Returns}
\sphinxAtStartPar
\begin{itemize}
\item {} 
\sphinxAtStartPar
\sphinxstylestrong{kepler\_elements} (\sphinxstyleemphasis{dict}) – The Keplarian orbital elements.

\item {} 
\sphinxAtStartPar
\sphinxstylestrong{kepler\_error} (\sphinxstyleemphasis{dict}) – The error on the Keplarian orbital elements.

\item {} 
\sphinxAtStartPar
\sphinxstylestrong{modified\_julian\_date} (\sphinxstyleemphasis{float}) – The modified Julian date corresponding to the osculating orbit and
the Keplerian orbital parameters provided.

\end{itemize}


\end{description}\end{quote}

\end{fulllineitems}\end{savenotes}

\index{solve\_orbit\_via\_record() (opihiexarata.orbit.orbfit.OrbfitOrbitDeterminerEngine method)@\spxentry{solve\_orbit\_via\_record()}\spxextra{opihiexarata.orbit.orbfit.OrbfitOrbitDeterminerEngine method}}

\begin{savenotes}\begin{fulllineitems}
\phantomsection\label{\detokenize{code/opihiexarata.orbit.orbfit:opihiexarata.orbit.orbfit.OrbfitOrbitDeterminerEngine.solve_orbit_via_record}}
\pysigstartsignatures
\pysiglinewithargsret{\sphinxbfcode{\sphinxupquote{solve\_orbit\_via\_record}}}{\emph{\DUrole{n}{observation\_record}\DUrole{p}{:}\DUrole{w}{  }\DUrole{n}{list\DUrole{p}{{[}}str\DUrole{p}{{]}}}}}{{ $\rightarrow$ tuple\DUrole{p}{{[}}dict\DUrole{p}{,}\DUrole{w}{  }dict\DUrole{p}{,}\DUrole{w}{  }float\DUrole{p}{{]}}}}
\pysigstopsignatures\begin{description}
\sphinxlineitem{Attempts to compute Keplarian orbits provided a standard 80\sphinxhyphen{}column}
\sphinxAtStartPar
record of observations.

\end{description}

\sphinxAtStartPar
This function calls and depends on \sphinxtitleref{solve\_orbfit}.
\begin{quote}\begin{description}
\sphinxlineitem{Parameters}
\sphinxAtStartPar
\sphinxstyleliteralstrong{\sphinxupquote{observation\_record}} (\sphinxstyleliteralemphasis{\sphinxupquote{Astropy Table}}) – The record of observations in the standard 80\sphinxhyphen{}column format.

\sphinxlineitem{Returns}
\sphinxAtStartPar
\begin{itemize}
\item {} 
\sphinxAtStartPar
\sphinxstylestrong{kepler\_elements} (\sphinxstyleemphasis{dict}) – The Keplarian orbital elements.

\item {} 
\sphinxAtStartPar
\sphinxstylestrong{kepler\_error} (\sphinxstyleemphasis{dict}) – The error on the Keplarian orbital elements.

\item {} 
\sphinxAtStartPar
\sphinxstylestrong{modified\_julian\_date} (\sphinxstyleemphasis{float}) – The modified Julian date corresponding to the osculating orbit and
the Keplerian orbital parameters provided.

\end{itemize}


\end{description}\end{quote}

\end{fulllineitems}\end{savenotes}


\end{fulllineitems}\end{savenotes}


\sphinxstepscope


\subparagraph{opihiexarata.orbit.solution module}
\label{\detokenize{code/opihiexarata.orbit.solution:module-opihiexarata.orbit.solution}}\label{\detokenize{code/opihiexarata.orbit.solution:opihiexarata-orbit-solution-module}}\label{\detokenize{code/opihiexarata.orbit.solution::doc}}\index{module@\spxentry{module}!opihiexarata.orbit.solution@\spxentry{opihiexarata.orbit.solution}}\index{opihiexarata.orbit.solution@\spxentry{opihiexarata.orbit.solution}!module@\spxentry{module}}
\sphinxAtStartPar
The orbit solution class.
\index{OrbitalSolution (class in opihiexarata.orbit.solution)@\spxentry{OrbitalSolution}\spxextra{class in opihiexarata.orbit.solution}}

\begin{savenotes}\begin{fulllineitems}
\phantomsection\label{\detokenize{code/opihiexarata.orbit.solution:opihiexarata.orbit.solution.OrbitalSolution}}
\pysigstartsignatures
\pysiglinewithargsret{\sphinxbfcode{\sphinxupquote{class\DUrole{w}{  }}}\sphinxcode{\sphinxupquote{opihiexarata.orbit.solution.}}\sphinxbfcode{\sphinxupquote{OrbitalSolution}}}{\emph{\DUrole{n}{observation\_record}\DUrole{p}{:}\DUrole{w}{  }\DUrole{n}{list\DUrole{p}{{[}}str\DUrole{p}{{]}}}}, \emph{\DUrole{n}{solver\_engine}\DUrole{p}{:}\DUrole{w}{  }\DUrole{n}{{\hyperref[\detokenize{code/opihiexarata.library.engine:opihiexarata.library.engine.OrbitEngine}]{\sphinxcrossref{OrbitEngine}}}}}, \emph{\DUrole{n}{vehicle\_args}\DUrole{p}{:}\DUrole{w}{  }\DUrole{n}{dict}\DUrole{w}{  }\DUrole{o}{=}\DUrole{w}{  }\DUrole{default_value}{\{\}}}}{}
\pysigstopsignatures
\sphinxAtStartPar
Bases: {\hyperref[\detokenize{code/opihiexarata.library.engine:opihiexarata.library.engine.ExarataSolution}]{\sphinxcrossref{\sphinxcode{\sphinxupquote{ExarataSolution}}}}}

\sphinxAtStartPar
This is the class which solves a record of observations to derive the
orbital parameters of asteroids or objects in general. A record of
observations must be provided.
\index{semimajor\_axis (opihiexarata.orbit.solution.OrbitalSolution attribute)@\spxentry{semimajor\_axis}\spxextra{opihiexarata.orbit.solution.OrbitalSolution attribute}}

\begin{savenotes}\begin{fulllineitems}
\phantomsection\label{\detokenize{code/opihiexarata.orbit.solution:opihiexarata.orbit.solution.OrbitalSolution.semimajor_axis}}
\pysigstartsignatures
\pysigline{\sphinxbfcode{\sphinxupquote{semimajor\_axis}}}
\pysigstopsignatures
\sphinxAtStartPar
The semi\sphinxhyphen{}major axis of the orbit solved, in AU.
\begin{quote}\begin{description}
\sphinxlineitem{Type}
\sphinxAtStartPar
float

\end{description}\end{quote}

\end{fulllineitems}\end{savenotes}

\index{semimajor\_axis\_error (opihiexarata.orbit.solution.OrbitalSolution attribute)@\spxentry{semimajor\_axis\_error}\spxextra{opihiexarata.orbit.solution.OrbitalSolution attribute}}

\begin{savenotes}\begin{fulllineitems}
\phantomsection\label{\detokenize{code/opihiexarata.orbit.solution:opihiexarata.orbit.solution.OrbitalSolution.semimajor_axis_error}}
\pysigstartsignatures
\pysigline{\sphinxbfcode{\sphinxupquote{semimajor\_axis\_error}}}
\pysigstopsignatures
\sphinxAtStartPar
The error on the semi\sphinxhyphen{}major axis of the orbit solved, in AU.
\begin{quote}\begin{description}
\sphinxlineitem{Type}
\sphinxAtStartPar
float

\end{description}\end{quote}

\end{fulllineitems}\end{savenotes}

\index{eccentricity (opihiexarata.orbit.solution.OrbitalSolution attribute)@\spxentry{eccentricity}\spxextra{opihiexarata.orbit.solution.OrbitalSolution attribute}}

\begin{savenotes}\begin{fulllineitems}
\phantomsection\label{\detokenize{code/opihiexarata.orbit.solution:opihiexarata.orbit.solution.OrbitalSolution.eccentricity}}
\pysigstartsignatures
\pysigline{\sphinxbfcode{\sphinxupquote{eccentricity}}}
\pysigstopsignatures
\sphinxAtStartPar
The eccentricity of the orbit solved.
\begin{quote}\begin{description}
\sphinxlineitem{Type}
\sphinxAtStartPar
float

\end{description}\end{quote}

\end{fulllineitems}\end{savenotes}

\index{eccentricity\_error (opihiexarata.orbit.solution.OrbitalSolution attribute)@\spxentry{eccentricity\_error}\spxextra{opihiexarata.orbit.solution.OrbitalSolution attribute}}

\begin{savenotes}\begin{fulllineitems}
\phantomsection\label{\detokenize{code/opihiexarata.orbit.solution:opihiexarata.orbit.solution.OrbitalSolution.eccentricity_error}}
\pysigstartsignatures
\pysigline{\sphinxbfcode{\sphinxupquote{eccentricity\_error}}}
\pysigstopsignatures
\sphinxAtStartPar
The error on the eccentricity of the orbit solved.
\begin{quote}\begin{description}
\sphinxlineitem{Type}
\sphinxAtStartPar
float

\end{description}\end{quote}

\end{fulllineitems}\end{savenotes}

\index{inclination (opihiexarata.orbit.solution.OrbitalSolution attribute)@\spxentry{inclination}\spxextra{opihiexarata.orbit.solution.OrbitalSolution attribute}}

\begin{savenotes}\begin{fulllineitems}
\phantomsection\label{\detokenize{code/opihiexarata.orbit.solution:opihiexarata.orbit.solution.OrbitalSolution.inclination}}
\pysigstartsignatures
\pysigline{\sphinxbfcode{\sphinxupquote{inclination}}}
\pysigstopsignatures
\sphinxAtStartPar
The angle of inclination of the orbit solved, in degrees.
\begin{quote}\begin{description}
\sphinxlineitem{Type}
\sphinxAtStartPar
float

\end{description}\end{quote}

\end{fulllineitems}\end{savenotes}

\index{inclination\_error (opihiexarata.orbit.solution.OrbitalSolution attribute)@\spxentry{inclination\_error}\spxextra{opihiexarata.orbit.solution.OrbitalSolution attribute}}

\begin{savenotes}\begin{fulllineitems}
\phantomsection\label{\detokenize{code/opihiexarata.orbit.solution:opihiexarata.orbit.solution.OrbitalSolution.inclination_error}}
\pysigstartsignatures
\pysigline{\sphinxbfcode{\sphinxupquote{inclination\_error}}}
\pysigstopsignatures
\sphinxAtStartPar
The error on the angle of inclination of the orbit solved, in degrees.
\begin{quote}\begin{description}
\sphinxlineitem{Type}
\sphinxAtStartPar
float

\end{description}\end{quote}

\end{fulllineitems}\end{savenotes}

\index{longitude\_ascending\_node (opihiexarata.orbit.solution.OrbitalSolution attribute)@\spxentry{longitude\_ascending\_node}\spxextra{opihiexarata.orbit.solution.OrbitalSolution attribute}}

\begin{savenotes}\begin{fulllineitems}
\phantomsection\label{\detokenize{code/opihiexarata.orbit.solution:opihiexarata.orbit.solution.OrbitalSolution.longitude_ascending_node}}
\pysigstartsignatures
\pysigline{\sphinxbfcode{\sphinxupquote{longitude\_ascending\_node}}}
\pysigstopsignatures
\sphinxAtStartPar
The longitude of the ascending node of the orbit solved, in degrees.
\begin{quote}\begin{description}
\sphinxlineitem{Type}
\sphinxAtStartPar
float

\end{description}\end{quote}

\end{fulllineitems}\end{savenotes}

\index{longitude\_ascending\_node\_error (opihiexarata.orbit.solution.OrbitalSolution attribute)@\spxentry{longitude\_ascending\_node\_error}\spxextra{opihiexarata.orbit.solution.OrbitalSolution attribute}}

\begin{savenotes}\begin{fulllineitems}
\phantomsection\label{\detokenize{code/opihiexarata.orbit.solution:opihiexarata.orbit.solution.OrbitalSolution.longitude_ascending_node_error}}
\pysigstartsignatures
\pysigline{\sphinxbfcode{\sphinxupquote{longitude\_ascending\_node\_error}}}
\pysigstopsignatures
\sphinxAtStartPar
The error on the longitude of the ascending node of the orbit solved, in degrees.
\begin{quote}\begin{description}
\sphinxlineitem{Type}
\sphinxAtStartPar
float

\end{description}\end{quote}

\end{fulllineitems}\end{savenotes}

\index{argument\_perihelion (opihiexarata.orbit.solution.OrbitalSolution attribute)@\spxentry{argument\_perihelion}\spxextra{opihiexarata.orbit.solution.OrbitalSolution attribute}}

\begin{savenotes}\begin{fulllineitems}
\phantomsection\label{\detokenize{code/opihiexarata.orbit.solution:opihiexarata.orbit.solution.OrbitalSolution.argument_perihelion}}
\pysigstartsignatures
\pysigline{\sphinxbfcode{\sphinxupquote{argument\_perihelion}}}
\pysigstopsignatures
\sphinxAtStartPar
The argument of perihelion of the orbit solved, in degrees.
\begin{quote}\begin{description}
\sphinxlineitem{Type}
\sphinxAtStartPar
float

\end{description}\end{quote}

\end{fulllineitems}\end{savenotes}

\index{argument\_perihelion\_error (opihiexarata.orbit.solution.OrbitalSolution attribute)@\spxentry{argument\_perihelion\_error}\spxextra{opihiexarata.orbit.solution.OrbitalSolution attribute}}

\begin{savenotes}\begin{fulllineitems}
\phantomsection\label{\detokenize{code/opihiexarata.orbit.solution:opihiexarata.orbit.solution.OrbitalSolution.argument_perihelion_error}}
\pysigstartsignatures
\pysigline{\sphinxbfcode{\sphinxupquote{argument\_perihelion\_error}}}
\pysigstopsignatures
\sphinxAtStartPar
The error on the argument of perihelion of the orbit solved, in degrees.
\begin{quote}\begin{description}
\sphinxlineitem{Type}
\sphinxAtStartPar
float

\end{description}\end{quote}

\end{fulllineitems}\end{savenotes}

\index{mean\_anomaly (opihiexarata.orbit.solution.OrbitalSolution attribute)@\spxentry{mean\_anomaly}\spxextra{opihiexarata.orbit.solution.OrbitalSolution attribute}}

\begin{savenotes}\begin{fulllineitems}
\phantomsection\label{\detokenize{code/opihiexarata.orbit.solution:opihiexarata.orbit.solution.OrbitalSolution.mean_anomaly}}
\pysigstartsignatures
\pysigline{\sphinxbfcode{\sphinxupquote{mean\_anomaly}}}
\pysigstopsignatures
\sphinxAtStartPar
The mean anomaly of the orbit solved, in degrees.
\begin{quote}\begin{description}
\sphinxlineitem{Type}
\sphinxAtStartPar
float

\end{description}\end{quote}

\end{fulllineitems}\end{savenotes}

\index{mean\_anomaly\_error (opihiexarata.orbit.solution.OrbitalSolution attribute)@\spxentry{mean\_anomaly\_error}\spxextra{opihiexarata.orbit.solution.OrbitalSolution attribute}}

\begin{savenotes}\begin{fulllineitems}
\phantomsection\label{\detokenize{code/opihiexarata.orbit.solution:opihiexarata.orbit.solution.OrbitalSolution.mean_anomaly_error}}
\pysigstartsignatures
\pysigline{\sphinxbfcode{\sphinxupquote{mean\_anomaly\_error}}}
\pysigstopsignatures
\sphinxAtStartPar
The error on the mean anomaly of the orbit solved, in degrees.
\begin{quote}\begin{description}
\sphinxlineitem{Type}
\sphinxAtStartPar
float

\end{description}\end{quote}

\end{fulllineitems}\end{savenotes}

\index{true\_anomaly (opihiexarata.orbit.solution.OrbitalSolution attribute)@\spxentry{true\_anomaly}\spxextra{opihiexarata.orbit.solution.OrbitalSolution attribute}}

\begin{savenotes}\begin{fulllineitems}
\phantomsection\label{\detokenize{code/opihiexarata.orbit.solution:opihiexarata.orbit.solution.OrbitalSolution.true_anomaly}}
\pysigstartsignatures
\pysigline{\sphinxbfcode{\sphinxupquote{true\_anomaly}}}
\pysigstopsignatures
\sphinxAtStartPar
The true anomaly of the orbit solved, in degrees. This value is
calculated from the mean anomaly.
\begin{quote}\begin{description}
\sphinxlineitem{Type}
\sphinxAtStartPar
float

\end{description}\end{quote}

\end{fulllineitems}\end{savenotes}

\index{true\_anomaly\_error (opihiexarata.orbit.solution.OrbitalSolution attribute)@\spxentry{true\_anomaly\_error}\spxextra{opihiexarata.orbit.solution.OrbitalSolution attribute}}

\begin{savenotes}\begin{fulllineitems}
\phantomsection\label{\detokenize{code/opihiexarata.orbit.solution:opihiexarata.orbit.solution.OrbitalSolution.true_anomaly_error}}
\pysigstartsignatures
\pysigline{\sphinxbfcode{\sphinxupquote{true\_anomaly\_error}}}
\pysigstopsignatures
\sphinxAtStartPar
The error on the true anomaly of the orbit solved, in degrees. This
value is calculated from the error on the mean anomaly.
\begin{quote}\begin{description}
\sphinxlineitem{Type}
\sphinxAtStartPar
float

\end{description}\end{quote}

\end{fulllineitems}\end{savenotes}

\index{epoch\_julian\_day (opihiexarata.orbit.solution.OrbitalSolution attribute)@\spxentry{epoch\_julian\_day}\spxextra{opihiexarata.orbit.solution.OrbitalSolution attribute}}

\begin{savenotes}\begin{fulllineitems}
\phantomsection\label{\detokenize{code/opihiexarata.orbit.solution:opihiexarata.orbit.solution.OrbitalSolution.epoch_julian_day}}
\pysigstartsignatures
\pysigline{\sphinxbfcode{\sphinxupquote{epoch\_julian\_day}}}
\pysigstopsignatures
\sphinxAtStartPar
The epoch where for these osculating orbital elements. This value is
in Julian days.
\begin{quote}\begin{description}
\sphinxlineitem{Type}
\sphinxAtStartPar
float

\end{description}\end{quote}

\end{fulllineitems}\end{savenotes}

\index{\_\_calculate\_eccentric\_anomaly() (opihiexarata.orbit.solution.OrbitalSolution method)@\spxentry{\_\_calculate\_eccentric\_anomaly()}\spxextra{opihiexarata.orbit.solution.OrbitalSolution method}}

\begin{savenotes}\begin{fulllineitems}
\phantomsection\label{\detokenize{code/opihiexarata.orbit.solution:opihiexarata.orbit.solution.OrbitalSolution.__calculate_eccentric_anomaly}}
\pysigstartsignatures
\pysiglinewithargsret{\sphinxbfcode{\sphinxupquote{\_\_calculate\_eccentric\_anomaly}}}{}{{ $\rightarrow$ tuple\DUrole{p}{{[}}float\DUrole{p}{,}\DUrole{w}{  }float\DUrole{p}{{]}}}}
\pysigstopsignatures
\sphinxAtStartPar
Calculating the eccentric anomaly and error from the mean anomaly.
\begin{quote}\begin{description}
\sphinxlineitem{Parameters}
\sphinxAtStartPar
\sphinxstyleliteralstrong{\sphinxupquote{None}} – 

\sphinxlineitem{Returns}
\sphinxAtStartPar
\begin{itemize}
\item {} 
\sphinxAtStartPar
\sphinxstylestrong{eccentric\_anomaly} (\sphinxstyleemphasis{float}) – The eccentric anomaly as derived from the mean anomaly.

\item {} 
\sphinxAtStartPar
\sphinxstylestrong{eccentric\_anomaly\_error} (\sphinxstyleemphasis{float}) – The eccentric anomaly error derived as an average from the upper
and lower ranges of the mean anomaly.

\end{itemize}


\end{description}\end{quote}

\end{fulllineitems}\end{savenotes}

\index{\_\_calculate\_true\_anomaly() (opihiexarata.orbit.solution.OrbitalSolution method)@\spxentry{\_\_calculate\_true\_anomaly()}\spxextra{opihiexarata.orbit.solution.OrbitalSolution method}}

\begin{savenotes}\begin{fulllineitems}
\phantomsection\label{\detokenize{code/opihiexarata.orbit.solution:opihiexarata.orbit.solution.OrbitalSolution.__calculate_true_anomaly}}
\pysigstartsignatures
\pysiglinewithargsret{\sphinxbfcode{\sphinxupquote{\_\_calculate\_true\_anomaly}}}{}{{ $\rightarrow$ tuple\DUrole{p}{{[}}float\DUrole{p}{,}\DUrole{w}{  }float\DUrole{p}{{]}}}}
\pysigstopsignatures
\sphinxAtStartPar
Calculating the true anomaly and error from the eccentric anomaly.
\begin{quote}\begin{description}
\sphinxlineitem{Parameters}
\sphinxAtStartPar
\sphinxstyleliteralstrong{\sphinxupquote{None}} – 

\sphinxlineitem{Returns}
\sphinxAtStartPar
\begin{itemize}
\item {} 
\sphinxAtStartPar
\sphinxstylestrong{true\_anomaly} (\sphinxstyleemphasis{float}) – The true anomaly as derived from the mean anomaly.

\item {} 
\sphinxAtStartPar
\sphinxstylestrong{true\_anomaly\_error} (\sphinxstyleemphasis{float}) – The true anomaly error derived as an average from the upper
and lower ranges of the eccentric anomaly.

\end{itemize}


\end{description}\end{quote}

\end{fulllineitems}\end{savenotes}

\index{\_\_init\_\_() (opihiexarata.orbit.solution.OrbitalSolution method)@\spxentry{\_\_init\_\_()}\spxextra{opihiexarata.orbit.solution.OrbitalSolution method}}

\begin{savenotes}\begin{fulllineitems}
\phantomsection\label{\detokenize{code/opihiexarata.orbit.solution:opihiexarata.orbit.solution.OrbitalSolution.__init__}}
\pysigstartsignatures
\pysiglinewithargsret{\sphinxbfcode{\sphinxupquote{\_\_init\_\_}}}{\emph{\DUrole{n}{observation\_record}\DUrole{p}{:}\DUrole{w}{  }\DUrole{n}{list\DUrole{p}{{[}}str\DUrole{p}{{]}}}}, \emph{\DUrole{n}{solver\_engine}\DUrole{p}{:}\DUrole{w}{  }\DUrole{n}{{\hyperref[\detokenize{code/opihiexarata.library.engine:opihiexarata.library.engine.OrbitEngine}]{\sphinxcrossref{OrbitEngine}}}}}, \emph{\DUrole{n}{vehicle\_args}\DUrole{p}{:}\DUrole{w}{  }\DUrole{n}{dict}\DUrole{w}{  }\DUrole{o}{=}\DUrole{w}{  }\DUrole{default_value}{\{\}}}}{{ $\rightarrow$ None}}
\pysigstopsignatures
\sphinxAtStartPar
The initialization function. Provided the list of observations,
solves the orbit for the Keplarian orbits.
\begin{quote}\begin{description}
\sphinxlineitem{Parameters}\begin{itemize}
\item {} 
\sphinxAtStartPar
\sphinxstyleliteralstrong{\sphinxupquote{observation\_record}} (\sphinxstyleliteralemphasis{\sphinxupquote{list}}) – A list of the standard MPC 80\sphinxhyphen{}column observation records. Each
element of the list should be a string representing the observation.

\item {} 
\sphinxAtStartPar
\sphinxstyleliteralstrong{\sphinxupquote{solver\_engine}} ({\hyperref[\detokenize{code/opihiexarata.library.engine:opihiexarata.library.engine.OrbitEngine}]{\sphinxcrossref{\sphinxstyleliteralemphasis{\sphinxupquote{OrbitEngine}}}}}) – The engine which will be used to complete the orbital elements
from the observation record.

\item {} 
\sphinxAtStartPar
\sphinxstyleliteralstrong{\sphinxupquote{vehicle\_args}} (\sphinxstyleliteralemphasis{\sphinxupquote{dictionary}}) – If the vehicle function for the provided solver engine needs
extra arguments not otherwise provided by the standard input,
they are given here.

\end{itemize}

\sphinxlineitem{Return type}
\sphinxAtStartPar
None

\end{description}\end{quote}

\end{fulllineitems}\end{savenotes}


\end{fulllineitems}\end{savenotes}

\index{\_calculate\_eccentric\_anomaly() (in module opihiexarata.orbit.solution)@\spxentry{\_calculate\_eccentric\_anomaly()}\spxextra{in module opihiexarata.orbit.solution}}

\begin{savenotes}\begin{fulllineitems}
\phantomsection\label{\detokenize{code/opihiexarata.orbit.solution:opihiexarata.orbit.solution._calculate_eccentric_anomaly}}
\pysigstartsignatures
\pysiglinewithargsret{\sphinxcode{\sphinxupquote{opihiexarata.orbit.solution.}}\sphinxbfcode{\sphinxupquote{\_calculate\_eccentric\_anomaly}}}{\emph{\DUrole{n}{mean\_anomaly}\DUrole{p}{:}\DUrole{w}{  }\DUrole{n}{float}}, \emph{\DUrole{n}{eccentricity}\DUrole{p}{:}\DUrole{w}{  }\DUrole{n}{float}}}{{ $\rightarrow$ float}}
\pysigstopsignatures
\sphinxAtStartPar
Calculate the eccentric anomaly from the mean anomaly and eccentricity
of an orbit. This is found iteratively using Newton’s method.
\begin{quote}\begin{description}
\sphinxlineitem{Parameters}
\sphinxAtStartPar
\sphinxstyleliteralstrong{\sphinxupquote{mean\_anomaly}} (\sphinxstyleliteralemphasis{\sphinxupquote{float}}) – The mean anomaly of the orbit, in degrees.

\sphinxlineitem{Returns}
\sphinxAtStartPar
\sphinxstylestrong{eccentric\_anomaly} – The eccentric anomaly as derived from the mean anomaly, in degrees.

\sphinxlineitem{Return type}
\sphinxAtStartPar
float

\end{description}\end{quote}

\end{fulllineitems}\end{savenotes}

\index{\_calculate\_true\_anomaly() (in module opihiexarata.orbit.solution)@\spxentry{\_calculate\_true\_anomaly()}\spxextra{in module opihiexarata.orbit.solution}}

\begin{savenotes}\begin{fulllineitems}
\phantomsection\label{\detokenize{code/opihiexarata.orbit.solution:opihiexarata.orbit.solution._calculate_true_anomaly}}
\pysigstartsignatures
\pysiglinewithargsret{\sphinxcode{\sphinxupquote{opihiexarata.orbit.solution.}}\sphinxbfcode{\sphinxupquote{\_calculate\_true\_anomaly}}}{\emph{\DUrole{n}{eccentric\_anomaly}\DUrole{p}{:}\DUrole{w}{  }\DUrole{n}{float}}, \emph{\DUrole{n}{eccentricity}\DUrole{p}{:}\DUrole{w}{  }\DUrole{n}{float}}}{{ $\rightarrow$ float}}
\pysigstopsignatures
\sphinxAtStartPar
Calculate the true anomaly from the mean anomaly and eccentricity
of an orbit.
\begin{quote}\begin{description}
\sphinxlineitem{Parameters}
\sphinxAtStartPar
\sphinxstyleliteralstrong{\sphinxupquote{eccentric\_anomaly}} (\sphinxstyleliteralemphasis{\sphinxupquote{float}}) – The eccentric anomaly of the orbit, in degrees.

\sphinxlineitem{Returns}
\sphinxAtStartPar
\sphinxstylestrong{true\_anomaly} – The true anomaly as derived from the eccentric anomaly, in degrees.

\sphinxlineitem{Return type}
\sphinxAtStartPar
float

\end{description}\end{quote}

\end{fulllineitems}\end{savenotes}

\index{\_vehicle\_custom\_orbit() (in module opihiexarata.orbit.solution)@\spxentry{\_vehicle\_custom\_orbit()}\spxextra{in module opihiexarata.orbit.solution}}

\begin{savenotes}\begin{fulllineitems}
\phantomsection\label{\detokenize{code/opihiexarata.orbit.solution:opihiexarata.orbit.solution._vehicle_custom_orbit}}
\pysigstartsignatures
\pysiglinewithargsret{\sphinxcode{\sphinxupquote{opihiexarata.orbit.solution.}}\sphinxbfcode{\sphinxupquote{\_vehicle\_custom\_orbit}}}{\emph{\DUrole{n}{observation\_record}\DUrole{p}{:}\DUrole{w}{  }\DUrole{n}{list\DUrole{p}{{[}}str\DUrole{p}{{]}}}}, \emph{\DUrole{n}{vehicle\_args}\DUrole{p}{:}\DUrole{w}{  }\DUrole{n}{dict}}}{{ $\rightarrow$ dict}}
\pysigstopsignatures
\sphinxAtStartPar
This is the vehicle function for the specification of a custom orbit.

\sphinxAtStartPar
A check is done for the extra vehicle arguments to ensure that the orbital
elements desired are within the arguments. The observation record is
not of concern for this vehicle.
\begin{quote}\begin{description}
\sphinxlineitem{Parameters}\begin{itemize}
\item {} 
\sphinxAtStartPar
\sphinxstyleliteralstrong{\sphinxupquote{observation\_record}} (\sphinxstyleliteralemphasis{\sphinxupquote{list}}) – The MPC standard 80\sphinxhyphen{}column record for observations of the asteroid
by which the orbit shall be computed from.

\item {} 
\sphinxAtStartPar
\sphinxstyleliteralstrong{\sphinxupquote{vehicle\_args}} (\sphinxstyleliteralemphasis{\sphinxupquote{dict}}) – The arguments to be passed to the Engine class to help its creation
and solving abilities. In this case, it is just the orbital elements
as defined.

\end{itemize}

\end{description}\end{quote}

\end{fulllineitems}\end{savenotes}

\index{\_vehicle\_orbfit\_orbit\_determiner() (in module opihiexarata.orbit.solution)@\spxentry{\_vehicle\_orbfit\_orbit\_determiner()}\spxextra{in module opihiexarata.orbit.solution}}

\begin{savenotes}\begin{fulllineitems}
\phantomsection\label{\detokenize{code/opihiexarata.orbit.solution:opihiexarata.orbit.solution._vehicle_orbfit_orbit_determiner}}
\pysigstartsignatures
\pysiglinewithargsret{\sphinxcode{\sphinxupquote{opihiexarata.orbit.solution.}}\sphinxbfcode{\sphinxupquote{\_vehicle\_orbfit\_orbit\_determiner}}}{\emph{\DUrole{n}{observation\_record}\DUrole{p}{:}\DUrole{w}{  }\DUrole{n}{list\DUrole{p}{{[}}str\DUrole{p}{{]}}}}}{{ $\rightarrow$ dict}}
\pysigstopsignatures
\sphinxAtStartPar
This uses the Orbfit engine to calculate orbital elements from the
observation record. The results are then returned to be managed by
the main class.
\begin{quote}\begin{description}
\sphinxlineitem{Parameters}
\sphinxAtStartPar
\sphinxstyleliteralstrong{\sphinxupquote{observation\_record}} (\sphinxstyleliteralemphasis{\sphinxupquote{list}}) – The MPC standard 80\sphinxhyphen{}column record for observations of the asteroid
by which the orbit shall be computed from.

\sphinxlineitem{Returns}
\sphinxAtStartPar
\sphinxstylestrong{orbit\_results} – The results of the orbit computation using the Orbfit engine. Namely,
this returns the 6 classical Kepler elements, using mean anomaly.

\sphinxlineitem{Return type}
\sphinxAtStartPar
dict

\end{description}\end{quote}

\end{fulllineitems}\end{savenotes}



\subparagraph{Module contents}
\label{\detokenize{code/opihiexarata.orbit:module-opihiexarata.orbit}}\label{\detokenize{code/opihiexarata.orbit:module-contents}}\index{module@\spxentry{module}!opihiexarata.orbit@\spxentry{opihiexarata.orbit}}\index{opihiexarata.orbit@\spxentry{opihiexarata.orbit}!module@\spxentry{module}}
\sphinxstepscope


\paragraph{opihiexarata.photometry package}
\label{\detokenize{code/opihiexarata.photometry:opihiexarata-photometry-package}}\label{\detokenize{code/opihiexarata.photometry::doc}}

\subparagraph{Submodules}
\label{\detokenize{code/opihiexarata.photometry:submodules}}
\sphinxstepscope


\subparagraph{opihiexarata.photometry.panstarrs module}
\label{\detokenize{code/opihiexarata.photometry.panstarrs:module-opihiexarata.photometry.panstarrs}}\label{\detokenize{code/opihiexarata.photometry.panstarrs:opihiexarata-photometry-panstarrs-module}}\label{\detokenize{code/opihiexarata.photometry.panstarrs::doc}}\index{module@\spxentry{module}!opihiexarata.photometry.panstarrs@\spxentry{opihiexarata.photometry.panstarrs}}\index{opihiexarata.photometry.panstarrs@\spxentry{opihiexarata.photometry.panstarrs}!module@\spxentry{module}}
\sphinxAtStartPar
Photometric database access using PANSTARRS data. There are a few ways and
they are implemented here.
\index{PanstarrsMastWebAPIEngine (class in opihiexarata.photometry.panstarrs)@\spxentry{PanstarrsMastWebAPIEngine}\spxextra{class in opihiexarata.photometry.panstarrs}}

\begin{savenotes}\begin{fulllineitems}
\phantomsection\label{\detokenize{code/opihiexarata.photometry.panstarrs:opihiexarata.photometry.panstarrs.PanstarrsMastWebAPIEngine}}
\pysigstartsignatures
\pysiglinewithargsret{\sphinxbfcode{\sphinxupquote{class\DUrole{w}{  }}}\sphinxcode{\sphinxupquote{opihiexarata.photometry.panstarrs.}}\sphinxbfcode{\sphinxupquote{PanstarrsMastWebAPIEngine}}}{\emph{\DUrole{n}{verify\_ssl}\DUrole{p}{:}\DUrole{w}{  }\DUrole{n}{bool}\DUrole{w}{  }\DUrole{o}{=}\DUrole{w}{  }\DUrole{default_value}{True}}}{}
\pysigstopsignatures
\sphinxAtStartPar
Bases: {\hyperref[\detokenize{code/opihiexarata.library.engine:opihiexarata.library.engine.PhotometryEngine}]{\sphinxcrossref{\sphinxcode{\sphinxupquote{PhotometryEngine}}}}}

\sphinxAtStartPar
This is a photometric data extractor using PanSTARRS data obtained from
their catalogs via the MAST API.

\sphinxAtStartPar
See \sphinxurl{https://catalogs.mast.stsci.edu/docs/panstarrs.html} for more
information.
\index{\_\_init\_\_() (opihiexarata.photometry.panstarrs.PanstarrsMastWebAPIEngine method)@\spxentry{\_\_init\_\_()}\spxextra{opihiexarata.photometry.panstarrs.PanstarrsMastWebAPIEngine method}}

\begin{savenotes}\begin{fulllineitems}
\phantomsection\label{\detokenize{code/opihiexarata.photometry.panstarrs:opihiexarata.photometry.panstarrs.PanstarrsMastWebAPIEngine.__init__}}
\pysigstartsignatures
\pysiglinewithargsret{\sphinxbfcode{\sphinxupquote{\_\_init\_\_}}}{\emph{\DUrole{n}{verify\_ssl}\DUrole{p}{:}\DUrole{w}{  }\DUrole{n}{bool}\DUrole{w}{  }\DUrole{o}{=}\DUrole{w}{  }\DUrole{default_value}{True}}}{{ $\rightarrow$ None}}
\pysigstopsignatures
\sphinxAtStartPar
Create the instance of the API.
\begin{quote}\begin{description}
\sphinxlineitem{Parameters}
\sphinxAtStartPar
\sphinxstyleliteralstrong{\sphinxupquote{verify\_ssl}} (\sphinxstyleliteralemphasis{\sphinxupquote{boolean}}\sphinxstyleliteralemphasis{\sphinxupquote{, }}\sphinxstyleliteralemphasis{\sphinxupquote{default = True}}) – Connecting to the MAST API usually uses SSL verification via HTTPS,
set to False to allow bypassing this.

\sphinxlineitem{Return type}
\sphinxAtStartPar
None

\end{description}\end{quote}

\end{fulllineitems}\end{savenotes}

\index{\_mask\_table\_data() (opihiexarata.photometry.panstarrs.PanstarrsMastWebAPIEngine method)@\spxentry{\_mask\_table\_data()}\spxextra{opihiexarata.photometry.panstarrs.PanstarrsMastWebAPIEngine method}}

\begin{savenotes}\begin{fulllineitems}
\phantomsection\label{\detokenize{code/opihiexarata.photometry.panstarrs:opihiexarata.photometry.panstarrs.PanstarrsMastWebAPIEngine._mask_table_data}}
\pysigstartsignatures
\pysiglinewithargsret{\sphinxbfcode{\sphinxupquote{\_mask\_table\_data}}}{\emph{\DUrole{n}{data\_table}\DUrole{p}{:}\DUrole{w}{  }\DUrole{n}{Table}}}{{ $\rightarrow$ Table}}
\pysigstopsignatures
\sphinxAtStartPar
This masks the raw data derived from PanSTARRS, implementing the
masking/null value specifics of the PanSTARRS system.

\sphinxAtStartPar
The point of this is to make masked or invalid data more typical to
the user by abstracting the idiosyncrasies of PanSTARRS.
\begin{quote}\begin{description}
\sphinxlineitem{Parameters}
\sphinxAtStartPar
\sphinxstyleliteralstrong{\sphinxupquote{data\_table}} (\sphinxstyleliteralemphasis{\sphinxupquote{Astropy Table}}) – The data table to be cleaned up.

\sphinxlineitem{Returns}
\sphinxAtStartPar
\sphinxstylestrong{masked\_data\_table} – The masked table.

\sphinxlineitem{Return type}
\sphinxAtStartPar
Astropy Table

\end{description}\end{quote}

\end{fulllineitems}\end{savenotes}

\index{cone\_search() (opihiexarata.photometry.panstarrs.PanstarrsMastWebAPIEngine method)@\spxentry{cone\_search()}\spxextra{opihiexarata.photometry.panstarrs.PanstarrsMastWebAPIEngine method}}

\begin{savenotes}\begin{fulllineitems}
\phantomsection\label{\detokenize{code/opihiexarata.photometry.panstarrs:opihiexarata.photometry.panstarrs.PanstarrsMastWebAPIEngine.cone_search}}
\pysigstartsignatures
\pysiglinewithargsret{\sphinxbfcode{\sphinxupquote{cone\_search}}}{\emph{\DUrole{n}{ra}\DUrole{p}{:}\DUrole{w}{  }\DUrole{n}{float}}, \emph{\DUrole{n}{dec}\DUrole{p}{:}\DUrole{w}{  }\DUrole{n}{float}}, \emph{\DUrole{n}{radius}\DUrole{p}{:}\DUrole{w}{  }\DUrole{n}{float}}, \emph{\DUrole{n}{detections}\DUrole{p}{:}\DUrole{w}{  }\DUrole{n}{int}\DUrole{w}{  }\DUrole{o}{=}\DUrole{w}{  }\DUrole{default_value}{3}}, \emph{\DUrole{n}{color\_detections}\DUrole{p}{:}\DUrole{w}{  }\DUrole{n}{int}\DUrole{w}{  }\DUrole{o}{=}\DUrole{w}{  }\DUrole{default_value}{1}}, \emph{\DUrole{n}{columns}\DUrole{p}{:}\DUrole{w}{  }\DUrole{n}{Optional\DUrole{p}{{[}}list\DUrole{p}{{[}}str\DUrole{p}{{]}}\DUrole{p}{{]}}}\DUrole{w}{  }\DUrole{o}{=}\DUrole{w}{  }\DUrole{default_value}{None}}, \emph{\DUrole{n}{max\_rows}\DUrole{p}{:}\DUrole{w}{  }\DUrole{n}{int}\DUrole{w}{  }\DUrole{o}{=}\DUrole{w}{  }\DUrole{default_value}{1000}}, \emph{\DUrole{n}{data\_release}\DUrole{p}{:}\DUrole{w}{  }\DUrole{n}{int}\DUrole{w}{  }\DUrole{o}{=}\DUrole{w}{  }\DUrole{default_value}{2}}}{{ $\rightarrow$ Table}}
\pysigstopsignatures
\sphinxAtStartPar
Search the PanSTARRS database for targets within a cone region
specified.

\sphinxAtStartPar
The table data returned from this function is not processed and is raw
from the fetching of the results.
\begin{quote}\begin{description}
\sphinxlineitem{Parameters}\begin{itemize}
\item {} 
\sphinxAtStartPar
\sphinxstyleliteralstrong{\sphinxupquote{ra}} (\sphinxstyleliteralemphasis{\sphinxupquote{float}}) – The right ascension of the center point of the cone search, in
degrees.

\item {} 
\sphinxAtStartPar
\sphinxstyleliteralstrong{\sphinxupquote{dec}} (\sphinxstyleliteralemphasis{\sphinxupquote{float}}) – The declination of the center point of the cone search, in
degrees.

\item {} 
\sphinxAtStartPar
\sphinxstyleliteralstrong{\sphinxupquote{radius}} (\sphinxstyleliteralemphasis{\sphinxupquote{float}}) – The radius from the center point of the cone search in which to
search, in degrees.

\item {} 
\sphinxAtStartPar
\sphinxstyleliteralstrong{\sphinxupquote{detections}} (\sphinxstyleliteralemphasis{\sphinxupquote{int}}\sphinxstyleliteralemphasis{\sphinxupquote{, }}\sphinxstyleliteralemphasis{\sphinxupquote{default = 3}}) – The minimum number of detections each object needs to have to be
included.

\item {} 
\sphinxAtStartPar
\sphinxstyleliteralstrong{\sphinxupquote{color\_detections}} (\sphinxstyleliteralemphasis{\sphinxupquote{int}}\sphinxstyleliteralemphasis{\sphinxupquote{, }}\sphinxstyleliteralemphasis{\sphinxupquote{default = 1}}) – The minimum number of detections for the g, r, i, z filters of the
Sloan filters of PanSTARRS. As this is a photometric engine for
OpihiExarata, the filters should be the ones pertinent to the
telescope.

\item {} 
\sphinxAtStartPar
\sphinxstyleliteralstrong{\sphinxupquote{columns}} (\sphinxstyleliteralemphasis{\sphinxupquote{list}}) – The columns that are desired to be pulled. The purpose of this is
to lighten the data download load. If None, then it defaults to
all columns.

\item {} 
\sphinxAtStartPar
\sphinxstyleliteralstrong{\sphinxupquote{max\_rows}} (\sphinxstyleliteralemphasis{\sphinxupquote{int}}\sphinxstyleliteralemphasis{\sphinxupquote{, }}\sphinxstyleliteralemphasis{\sphinxupquote{default = 1000}}) – The maximum entries that will be pulled from the server.

\item {} 
\sphinxAtStartPar
\sphinxstyleliteralstrong{\sphinxupquote{data\_release}} (\sphinxstyleliteralemphasis{\sphinxupquote{int}}\sphinxstyleliteralemphasis{\sphinxupquote{, }}\sphinxstyleliteralemphasis{\sphinxupquote{default = 2}}) – The PanSTARRS data release version from which to take the data from.

\end{itemize}

\sphinxlineitem{Returns}
\sphinxAtStartPar
\sphinxstylestrong{catalog\_results} – The result of the cone search, pulled from the PanSTARRS catalog.

\sphinxlineitem{Return type}
\sphinxAtStartPar
Astropy Table

\end{description}\end{quote}

\end{fulllineitems}\end{savenotes}

\index{masked\_cone\_search() (opihiexarata.photometry.panstarrs.PanstarrsMastWebAPIEngine method)@\spxentry{masked\_cone\_search()}\spxextra{opihiexarata.photometry.panstarrs.PanstarrsMastWebAPIEngine method}}

\begin{savenotes}\begin{fulllineitems}
\phantomsection\label{\detokenize{code/opihiexarata.photometry.panstarrs:opihiexarata.photometry.panstarrs.PanstarrsMastWebAPIEngine.masked_cone_search}}
\pysigstartsignatures
\pysiglinewithargsret{\sphinxbfcode{\sphinxupquote{masked\_cone\_search}}}{\emph{\DUrole{o}{*}\DUrole{n}{args}}, \emph{\DUrole{o}{**}\DUrole{n}{kwargs}}}{{ $\rightarrow$ Table}}
\pysigstopsignatures
\sphinxAtStartPar
The same as cone\_search, but it also masks the data based on the
masking idiosyncrasies of PanSTARRS.
\begin{quote}\begin{description}
\sphinxlineitem{Parameters}
\sphinxAtStartPar
\sphinxstyleliteralstrong{\sphinxupquote{cone\_search}}\sphinxstyleliteralstrong{\sphinxupquote{)}} (\sphinxstyleliteralemphasis{\sphinxupquote{(}}\sphinxstyleliteralemphasis{\sphinxupquote{see}}) – 

\sphinxlineitem{Returns}
\sphinxAtStartPar
\sphinxstylestrong{masked\_catalog\_results} – The data from the cone search with the entries masked where
appropriate.

\sphinxlineitem{Return type}
\sphinxAtStartPar
Astropy Table

\end{description}\end{quote}

\end{fulllineitems}\end{savenotes}


\end{fulllineitems}\end{savenotes}


\sphinxstepscope


\subparagraph{opihiexarata.photometry.solution module}
\label{\detokenize{code/opihiexarata.photometry.solution:module-opihiexarata.photometry.solution}}\label{\detokenize{code/opihiexarata.photometry.solution:opihiexarata-photometry-solution-module}}\label{\detokenize{code/opihiexarata.photometry.solution::doc}}\index{module@\spxentry{module}!opihiexarata.photometry.solution@\spxentry{opihiexarata.photometry.solution}}\index{opihiexarata.photometry.solution@\spxentry{opihiexarata.photometry.solution}!module@\spxentry{module}}
\sphinxAtStartPar
The general photometric solver.
\index{PhotometricSolution (class in opihiexarata.photometry.solution)@\spxentry{PhotometricSolution}\spxextra{class in opihiexarata.photometry.solution}}

\begin{savenotes}\begin{fulllineitems}
\phantomsection\label{\detokenize{code/opihiexarata.photometry.solution:opihiexarata.photometry.solution.PhotometricSolution}}
\pysigstartsignatures
\pysiglinewithargsret{\sphinxbfcode{\sphinxupquote{class\DUrole{w}{  }}}\sphinxcode{\sphinxupquote{opihiexarata.photometry.solution.}}\sphinxbfcode{\sphinxupquote{PhotometricSolution}}}{\emph{\DUrole{n}{fits\_filename}\DUrole{p}{:}\DUrole{w}{  }\DUrole{n}{str}}, \emph{\DUrole{n}{solver\_engine}\DUrole{p}{:}\DUrole{w}{  }\DUrole{n}{{\hyperref[\detokenize{code/opihiexarata.library.engine:opihiexarata.library.engine.PhotometryEngine}]{\sphinxcrossref{PhotometryEngine}}}}}, \emph{\DUrole{n}{astrometrics}\DUrole{p}{:}\DUrole{w}{  }\DUrole{n}{{\hyperref[\detokenize{code/opihiexarata.astrometry.solution:opihiexarata.astrometry.solution.AstrometricSolution}]{\sphinxcrossref{AstrometricSolution}}}}}, \emph{\DUrole{n}{exposure\_time}\DUrole{p}{:}\DUrole{w}{  }\DUrole{n}{Optional\DUrole{p}{{[}}float\DUrole{p}{{]}}}\DUrole{w}{  }\DUrole{o}{=}\DUrole{w}{  }\DUrole{default_value}{None}}, \emph{\DUrole{n}{filter\_name}\DUrole{p}{:}\DUrole{w}{  }\DUrole{n}{Optional\DUrole{p}{{[}}str\DUrole{p}{{]}}}\DUrole{w}{  }\DUrole{o}{=}\DUrole{w}{  }\DUrole{default_value}{None}}, \emph{\DUrole{n}{vehicle\_args}\DUrole{p}{:}\DUrole{w}{  }\DUrole{n}{dict}\DUrole{w}{  }\DUrole{o}{=}\DUrole{w}{  }\DUrole{default_value}{\{\}}}}{}
\pysigstopsignatures
\sphinxAtStartPar
Bases: {\hyperref[\detokenize{code/opihiexarata.library.engine:opihiexarata.library.engine.ExarataSolution}]{\sphinxcrossref{\sphinxcode{\sphinxupquote{ExarataSolution}}}}}

\sphinxAtStartPar
The primary class describing an photometric solution, based on an image
provided and catalog data provided from the photometric engine.

\sphinxAtStartPar
This class is the middlewere class between the engines which solve the
photometry, and the rest of the OpihiExarata code.
\index{\_original\_filename (opihiexarata.photometry.solution.PhotometricSolution attribute)@\spxentry{\_original\_filename}\spxextra{opihiexarata.photometry.solution.PhotometricSolution attribute}}

\begin{savenotes}\begin{fulllineitems}
\phantomsection\label{\detokenize{code/opihiexarata.photometry.solution:opihiexarata.photometry.solution.PhotometricSolution._original_filename}}
\pysigstartsignatures
\pysigline{\sphinxbfcode{\sphinxupquote{\_original\_filename}}}
\pysigstopsignatures
\sphinxAtStartPar
The original filename where the fits file is stored at, or copied to.
\begin{quote}\begin{description}
\sphinxlineitem{Type}
\sphinxAtStartPar
string

\end{description}\end{quote}

\end{fulllineitems}\end{savenotes}

\index{\_original\_header (opihiexarata.photometry.solution.PhotometricSolution attribute)@\spxentry{\_original\_header}\spxextra{opihiexarata.photometry.solution.PhotometricSolution attribute}}

\begin{savenotes}\begin{fulllineitems}
\phantomsection\label{\detokenize{code/opihiexarata.photometry.solution:opihiexarata.photometry.solution.PhotometricSolution._original_header}}
\pysigstartsignatures
\pysigline{\sphinxbfcode{\sphinxupquote{\_original\_header}}}
\pysigstopsignatures
\sphinxAtStartPar
The original header of the fits file that was pulled to solve for this
astrometric solution.
\begin{quote}\begin{description}
\sphinxlineitem{Type}
\sphinxAtStartPar
Header

\end{description}\end{quote}

\end{fulllineitems}\end{savenotes}

\index{\_original\_data (opihiexarata.photometry.solution.PhotometricSolution attribute)@\spxentry{\_original\_data}\spxextra{opihiexarata.photometry.solution.PhotometricSolution attribute}}

\begin{savenotes}\begin{fulllineitems}
\phantomsection\label{\detokenize{code/opihiexarata.photometry.solution:opihiexarata.photometry.solution.PhotometricSolution._original_data}}
\pysigstartsignatures
\pysigline{\sphinxbfcode{\sphinxupquote{\_original\_data}}}
\pysigstopsignatures
\sphinxAtStartPar
The original data of the fits file that was pulled to solve for this
photometric solution.
\begin{quote}\begin{description}
\sphinxlineitem{Type}
\sphinxAtStartPar
array\sphinxhyphen{}like

\end{description}\end{quote}

\end{fulllineitems}\end{savenotes}

\index{astrometrics (opihiexarata.photometry.solution.PhotometricSolution attribute)@\spxentry{astrometrics}\spxextra{opihiexarata.photometry.solution.PhotometricSolution attribute}}

\begin{savenotes}\begin{fulllineitems}
\phantomsection\label{\detokenize{code/opihiexarata.photometry.solution:opihiexarata.photometry.solution.PhotometricSolution.astrometrics}}
\pysigstartsignatures
\pysigline{\sphinxbfcode{\sphinxupquote{astrometrics}}}
\pysigstopsignatures
\sphinxAtStartPar
The astrometric solution that is required for the photometric solution.
\begin{quote}\begin{description}
\sphinxlineitem{Type}
\sphinxAtStartPar
{\hyperref[\detokenize{code/opihiexarata.astrometry.solution:opihiexarata.astrometry.solution.AstrometricSolution}]{\sphinxcrossref{AstrometricSolution}}}

\end{description}\end{quote}

\end{fulllineitems}\end{savenotes}

\index{sky\_counts (opihiexarata.photometry.solution.PhotometricSolution attribute)@\spxentry{sky\_counts}\spxextra{opihiexarata.photometry.solution.PhotometricSolution attribute}}

\begin{savenotes}\begin{fulllineitems}
\phantomsection\label{\detokenize{code/opihiexarata.photometry.solution:opihiexarata.photometry.solution.PhotometricSolution.sky_counts}}
\pysigstartsignatures
\pysigline{\sphinxbfcode{\sphinxupquote{sky\_counts}}}
\pysigstopsignatures
\sphinxAtStartPar
The average sky contribution per pixel.
\begin{quote}\begin{description}
\sphinxlineitem{Type}
\sphinxAtStartPar
float

\end{description}\end{quote}

\end{fulllineitems}\end{savenotes}

\index{star\_table (opihiexarata.photometry.solution.PhotometricSolution attribute)@\spxentry{star\_table}\spxextra{opihiexarata.photometry.solution.PhotometricSolution attribute}}

\begin{savenotes}\begin{fulllineitems}
\phantomsection\label{\detokenize{code/opihiexarata.photometry.solution:opihiexarata.photometry.solution.PhotometricSolution.star_table}}
\pysigstartsignatures
\pysigline{\sphinxbfcode{\sphinxupquote{star\_table}}}
\pysigstopsignatures
\sphinxAtStartPar
A table of stars around the image with their RA, DEC, and filter
magnitudes. It is not guaranteed that this star table and the
astrometric star table is correlated.
\begin{quote}\begin{description}
\sphinxlineitem{Type}
\sphinxAtStartPar
Table

\end{description}\end{quote}

\end{fulllineitems}\end{savenotes}

\index{intersection\_star\_table (opihiexarata.photometry.solution.PhotometricSolution attribute)@\spxentry{intersection\_star\_table}\spxextra{opihiexarata.photometry.solution.PhotometricSolution attribute}}

\begin{savenotes}\begin{fulllineitems}
\phantomsection\label{\detokenize{code/opihiexarata.photometry.solution:opihiexarata.photometry.solution.PhotometricSolution.intersection_star_table}}
\pysigstartsignatures
\pysigline{\sphinxbfcode{\sphinxupquote{intersection\_star\_table}}}
\pysigstopsignatures
\sphinxAtStartPar
A table of stars with both the astrometric and photometric RA and DEC
coordinates found by the astrometric solution and the photometric
engine. The filter magnitudes of these stars are also provided. It is
guaranteed that the stars within this table are correlated.
\begin{quote}\begin{description}
\sphinxlineitem{Type}
\sphinxAtStartPar
Table

\end{description}\end{quote}

\end{fulllineitems}\end{savenotes}

\index{exposure\_time (opihiexarata.photometry.solution.PhotometricSolution attribute)@\spxentry{exposure\_time}\spxextra{opihiexarata.photometry.solution.PhotometricSolution attribute}}

\begin{savenotes}\begin{fulllineitems}
\phantomsection\label{\detokenize{code/opihiexarata.photometry.solution:opihiexarata.photometry.solution.PhotometricSolution.exposure_time}}
\pysigstartsignatures
\pysigline{\sphinxbfcode{\sphinxupquote{exposure\_time}}}
\pysigstopsignatures
\sphinxAtStartPar
How long, in seconds, the image in question was exposed for.
\begin{quote}\begin{description}
\sphinxlineitem{Type}
\sphinxAtStartPar
float

\end{description}\end{quote}

\end{fulllineitems}\end{savenotes}

\index{filter\_name (opihiexarata.photometry.solution.PhotometricSolution attribute)@\spxentry{filter\_name}\spxextra{opihiexarata.photometry.solution.PhotometricSolution attribute}}

\begin{savenotes}\begin{fulllineitems}
\phantomsection\label{\detokenize{code/opihiexarata.photometry.solution:opihiexarata.photometry.solution.PhotometricSolution.filter_name}}
\pysigstartsignatures
\pysigline{\sphinxbfcode{\sphinxupquote{filter\_name}}}
\pysigstopsignatures
\sphinxAtStartPar
A single character string describing the name of the filter band that
this image was taken in. Currently, it assumes the MKO/SDSS visual
filters.
\begin{quote}\begin{description}
\sphinxlineitem{Type}
\sphinxAtStartPar
string

\end{description}\end{quote}

\end{fulllineitems}\end{savenotes}

\index{available\_filters (opihiexarata.photometry.solution.PhotometricSolution attribute)@\spxentry{available\_filters}\spxextra{opihiexarata.photometry.solution.PhotometricSolution attribute}}

\begin{savenotes}\begin{fulllineitems}
\phantomsection\label{\detokenize{code/opihiexarata.photometry.solution:opihiexarata.photometry.solution.PhotometricSolution.available_filters}}
\pysigstartsignatures
\pysigline{\sphinxbfcode{\sphinxupquote{available\_filters}}}
\pysigstopsignatures
\sphinxAtStartPar
The list of filter names which the star table currently covers and has
data for.
\begin{quote}\begin{description}
\sphinxlineitem{Type}
\sphinxAtStartPar
tuple

\end{description}\end{quote}

\end{fulllineitems}\end{savenotes}

\index{zero\_point (opihiexarata.photometry.solution.PhotometricSolution attribute)@\spxentry{zero\_point}\spxextra{opihiexarata.photometry.solution.PhotometricSolution attribute}}

\begin{savenotes}\begin{fulllineitems}
\phantomsection\label{\detokenize{code/opihiexarata.photometry.solution:opihiexarata.photometry.solution.PhotometricSolution.zero_point}}
\pysigstartsignatures
\pysigline{\sphinxbfcode{\sphinxupquote{zero\_point}}}
\pysigstopsignatures
\sphinxAtStartPar
The zero point of the image.
\begin{quote}\begin{description}
\sphinxlineitem{Type}
\sphinxAtStartPar
float

\end{description}\end{quote}

\end{fulllineitems}\end{savenotes}

\index{zero\_point\_error (opihiexarata.photometry.solution.PhotometricSolution attribute)@\spxentry{zero\_point\_error}\spxextra{opihiexarata.photometry.solution.PhotometricSolution attribute}}

\begin{savenotes}\begin{fulllineitems}
\phantomsection\label{\detokenize{code/opihiexarata.photometry.solution:opihiexarata.photometry.solution.PhotometricSolution.zero_point_error}}
\pysigstartsignatures
\pysigline{\sphinxbfcode{\sphinxupquote{zero\_point\_error}}}
\pysigstopsignatures
\sphinxAtStartPar
The standard deviation of the error point mean as calculated using
many stars.
\begin{quote}\begin{description}
\sphinxlineitem{Type}
\sphinxAtStartPar
float

\end{description}\end{quote}

\end{fulllineitems}\end{savenotes}

\index{\_\_calculate\_intersection\_star\_table() (opihiexarata.photometry.solution.PhotometricSolution method)@\spxentry{\_\_calculate\_intersection\_star\_table()}\spxextra{opihiexarata.photometry.solution.PhotometricSolution method}}

\begin{savenotes}\begin{fulllineitems}
\phantomsection\label{\detokenize{code/opihiexarata.photometry.solution:opihiexarata.photometry.solution.PhotometricSolution.__calculate_intersection_star_table}}
\pysigstartsignatures
\pysiglinewithargsret{\sphinxbfcode{\sphinxupquote{\_\_calculate\_intersection\_star\_table}}}{}{{ $\rightarrow$ Table}}
\pysigstopsignatures
\sphinxAtStartPar
This function determines the intersection star table.

\sphinxAtStartPar
Basically this function matches the entries in the astrometric star
table with the photometric star table. This function accomplishes that
by simply associating the closest entires as the same star. The
distance function assumes a tangent sky projection.
\begin{quote}\begin{description}
\sphinxlineitem{Parameters}
\sphinxAtStartPar
\sphinxstyleliteralstrong{\sphinxupquote{None}} – 

\sphinxlineitem{Returns}
\sphinxAtStartPar
\sphinxstylestrong{intersection\_table} – The intersection of the astrometric and photometric star tables
giving the correlated star entries between them.

\sphinxlineitem{Return type}
\sphinxAtStartPar
Table

\end{description}\end{quote}

\end{fulllineitems}\end{savenotes}

\index{\_\_calculate\_sky\_counts\_mask() (opihiexarata.photometry.solution.PhotometricSolution method)@\spxentry{\_\_calculate\_sky\_counts\_mask()}\spxextra{opihiexarata.photometry.solution.PhotometricSolution method}}

\begin{savenotes}\begin{fulllineitems}
\phantomsection\label{\detokenize{code/opihiexarata.photometry.solution:opihiexarata.photometry.solution.PhotometricSolution.__calculate_sky_counts_mask}}
\pysigstartsignatures
\pysiglinewithargsret{\sphinxbfcode{\sphinxupquote{\_\_calculate\_sky\_counts\_mask}}}{}{{ $\rightarrow$ ndarray}}
\pysigstopsignatures
\sphinxAtStartPar
Calculate a mask which blocks out all but the sky for sky counts
determination.

\sphinxAtStartPar
The method used is to exclude the regions where stars exist (as
determined by the star tables) and also the central region
of the image (as it is expected that there is a science object there).
\begin{quote}\begin{description}
\sphinxlineitem{Parameters}
\sphinxAtStartPar
\sphinxstyleliteralstrong{\sphinxupquote{None}} – 

\sphinxlineitem{Returns}
\sphinxAtStartPar
\sphinxstylestrong{sky\_counts\_mask} – The mask which masks out which is not interesting regarding sky
count calculations.

\sphinxlineitem{Return type}
\sphinxAtStartPar
float

\end{description}\end{quote}

\end{fulllineitems}\end{savenotes}

\index{\_\_calculate\_sky\_counts\_value() (opihiexarata.photometry.solution.PhotometricSolution method)@\spxentry{\_\_calculate\_sky\_counts\_value()}\spxextra{opihiexarata.photometry.solution.PhotometricSolution method}}

\begin{savenotes}\begin{fulllineitems}
\phantomsection\label{\detokenize{code/opihiexarata.photometry.solution:opihiexarata.photometry.solution.PhotometricSolution.__calculate_sky_counts_value}}
\pysigstartsignatures
\pysiglinewithargsret{\sphinxbfcode{\sphinxupquote{\_\_calculate\_sky\_counts\_value}}}{}{{ $\rightarrow$ float}}
\pysigstopsignatures
\sphinxAtStartPar
Calculate the background sky value, in counts, from the image.
Obviously needed for photometric calibrations.

\sphinxAtStartPar
The regions outside of the sky mask represent the sky and the sky
counts is extracted from that.
\begin{quote}\begin{description}
\sphinxlineitem{Parameters}
\sphinxAtStartPar
\sphinxstyleliteralstrong{\sphinxupquote{None}} – 

\sphinxlineitem{Returns}
\sphinxAtStartPar
\sphinxstylestrong{sky\_counts} – The total number of counts, in DN that, on average, the sky
contributes per pixel.

\sphinxlineitem{Return type}
\sphinxAtStartPar
float

\end{description}\end{quote}

\end{fulllineitems}\end{savenotes}

\index{\_\_init\_\_() (opihiexarata.photometry.solution.PhotometricSolution method)@\spxentry{\_\_init\_\_()}\spxextra{opihiexarata.photometry.solution.PhotometricSolution method}}

\begin{savenotes}\begin{fulllineitems}
\phantomsection\label{\detokenize{code/opihiexarata.photometry.solution:opihiexarata.photometry.solution.PhotometricSolution.__init__}}
\pysigstartsignatures
\pysiglinewithargsret{\sphinxbfcode{\sphinxupquote{\_\_init\_\_}}}{\emph{\DUrole{n}{fits\_filename}\DUrole{p}{:}\DUrole{w}{  }\DUrole{n}{str}}, \emph{\DUrole{n}{solver\_engine}\DUrole{p}{:}\DUrole{w}{  }\DUrole{n}{{\hyperref[\detokenize{code/opihiexarata.library.engine:opihiexarata.library.engine.PhotometryEngine}]{\sphinxcrossref{PhotometryEngine}}}}}, \emph{\DUrole{n}{astrometrics}\DUrole{p}{:}\DUrole{w}{  }\DUrole{n}{{\hyperref[\detokenize{code/opihiexarata.astrometry.solution:opihiexarata.astrometry.solution.AstrometricSolution}]{\sphinxcrossref{AstrometricSolution}}}}}, \emph{\DUrole{n}{exposure\_time}\DUrole{p}{:}\DUrole{w}{  }\DUrole{n}{Optional\DUrole{p}{{[}}float\DUrole{p}{{]}}}\DUrole{w}{  }\DUrole{o}{=}\DUrole{w}{  }\DUrole{default_value}{None}}, \emph{\DUrole{n}{filter\_name}\DUrole{p}{:}\DUrole{w}{  }\DUrole{n}{Optional\DUrole{p}{{[}}str\DUrole{p}{{]}}}\DUrole{w}{  }\DUrole{o}{=}\DUrole{w}{  }\DUrole{default_value}{None}}, \emph{\DUrole{n}{vehicle\_args}\DUrole{p}{:}\DUrole{w}{  }\DUrole{n}{dict}\DUrole{w}{  }\DUrole{o}{=}\DUrole{w}{  }\DUrole{default_value}{\{\}}}}{{ $\rightarrow$ None}}
\pysigstopsignatures
\sphinxAtStartPar
Initialization of the photometric solution.
\begin{quote}\begin{description}
\sphinxlineitem{Parameters}\begin{itemize}
\item {} 
\sphinxAtStartPar
\sphinxstyleliteralstrong{\sphinxupquote{fits\_filename}} (\sphinxstyleliteralemphasis{\sphinxupquote{string}}) – The path of the fits file that contains the data for the astrometric
solution.

\item {} 
\sphinxAtStartPar
\sphinxstyleliteralstrong{\sphinxupquote{solver\_engine}} ({\hyperref[\detokenize{code/opihiexarata.library.engine:opihiexarata.library.engine.PhotometryEngine}]{\sphinxcrossref{\sphinxstyleliteralemphasis{\sphinxupquote{PhotometryEngine}}}}}) – The photometric solver engine class. This is what will act as the
“behind the scenes” and solve the field, using this middlewhere to
translate it into something that is easier.

\item {} 
\sphinxAtStartPar
\sphinxstyleliteralstrong{\sphinxupquote{astrometrics}} ({\hyperref[\detokenize{code/opihiexarata.astrometry.solution:opihiexarata.astrometry.solution.AstrometricSolution}]{\sphinxcrossref{\sphinxstyleliteralemphasis{\sphinxupquote{AstrometricSolution}}}}}\sphinxstyleliteralemphasis{\sphinxupquote{, }}\sphinxstyleliteralemphasis{\sphinxupquote{default = None}}) – A precomputed astrometric solution which belongs to this image.

\item {} 
\sphinxAtStartPar
\sphinxstyleliteralstrong{\sphinxupquote{exposure\_time}} (\sphinxstyleliteralemphasis{\sphinxupquote{float}}\sphinxstyleliteralemphasis{\sphinxupquote{, }}\sphinxstyleliteralemphasis{\sphinxupquote{default = None}}) – How long, in seconds, the image in question was exposed for. If
not provided, calculation of the zero\sphinxhyphen{}point is skipped.

\item {} 
\sphinxAtStartPar
\sphinxstyleliteralstrong{\sphinxupquote{filter\_name}} (\sphinxstyleliteralemphasis{\sphinxupquote{string}}\sphinxstyleliteralemphasis{\sphinxupquote{, }}\sphinxstyleliteralemphasis{\sphinxupquote{default = None}}) – A single character string describing the name of the filter band that
this image was taken in. Currently, it assumes the MKO/SDSS visual
filters. If it is None, calculation of the zero\sphinxhyphen{}point is skipped.

\item {} 
\sphinxAtStartPar
\sphinxstyleliteralstrong{\sphinxupquote{vehicle\_args}} (\sphinxstyleliteralemphasis{\sphinxupquote{dictionary}}) – If the vehicle function for the provided solver engine needs
extra parameters not otherwise provided by the standard input,
they are given here.

\end{itemize}

\sphinxlineitem{Return type}
\sphinxAtStartPar
None

\end{description}\end{quote}

\end{fulllineitems}\end{savenotes}

\index{\_calculate\_star\_photon\_counts\_coordinate() (opihiexarata.photometry.solution.PhotometricSolution method)@\spxentry{\_calculate\_star\_photon\_counts\_coordinate()}\spxextra{opihiexarata.photometry.solution.PhotometricSolution method}}

\begin{savenotes}\begin{fulllineitems}
\phantomsection\label{\detokenize{code/opihiexarata.photometry.solution:opihiexarata.photometry.solution.PhotometricSolution._calculate_star_photon_counts_coordinate}}
\pysigstartsignatures
\pysiglinewithargsret{\sphinxbfcode{\sphinxupquote{\_calculate\_star\_photon\_counts\_coordinate}}}{\emph{\DUrole{n}{ra}\DUrole{p}{:}\DUrole{w}{  }\DUrole{n}{float}}, \emph{\DUrole{n}{dec}\DUrole{p}{:}\DUrole{w}{  }\DUrole{n}{float}}, \emph{\DUrole{n}{radius}\DUrole{p}{:}\DUrole{w}{  }\DUrole{n}{float}}}{{ $\rightarrow$ float}}
\pysigstopsignatures
\sphinxAtStartPar
Calculate the total number of photometric counts at an RA DEC. The
counts are already corrected for the sky counts.

\sphinxAtStartPar
This function does not check if a star is actually there. This function
is a wrapper around its pixel version, converting via the WCS solution.
\begin{quote}\begin{description}
\sphinxlineitem{Parameters}\begin{itemize}
\item {} 
\sphinxAtStartPar
\sphinxstyleliteralstrong{\sphinxupquote{ra}} (\sphinxstyleliteralemphasis{\sphinxupquote{float}}) – The right ascension in degrees.

\item {} 
\sphinxAtStartPar
\sphinxstyleliteralstrong{\sphinxupquote{dec}} (\sphinxstyleliteralemphasis{\sphinxupquote{float}}) – The declination in degrees.

\item {} 
\sphinxAtStartPar
\sphinxstyleliteralstrong{\sphinxupquote{radius}} (\sphinxstyleliteralemphasis{\sphinxupquote{float}}) – The radius of the circular aperture to be considered, in degrees.

\end{itemize}

\sphinxlineitem{Returns}
\sphinxAtStartPar
\sphinxstylestrong{photon\_counts} – The sum of the sky corrected counts for the region defined.

\sphinxlineitem{Return type}
\sphinxAtStartPar
float

\end{description}\end{quote}

\end{fulllineitems}\end{savenotes}

\index{\_calculate\_star\_photon\_counts\_pixel() (opihiexarata.photometry.solution.PhotometricSolution method)@\spxentry{\_calculate\_star\_photon\_counts\_pixel()}\spxextra{opihiexarata.photometry.solution.PhotometricSolution method}}

\begin{savenotes}\begin{fulllineitems}
\phantomsection\label{\detokenize{code/opihiexarata.photometry.solution:opihiexarata.photometry.solution.PhotometricSolution._calculate_star_photon_counts_pixel}}
\pysigstartsignatures
\pysiglinewithargsret{\sphinxbfcode{\sphinxupquote{\_calculate\_star\_photon\_counts\_pixel}}}{\emph{\DUrole{n}{pixel\_x}\DUrole{p}{:}\DUrole{w}{  }\DUrole{n}{int}}, \emph{\DUrole{n}{pixel\_y}\DUrole{p}{:}\DUrole{w}{  }\DUrole{n}{int}}, \emph{\DUrole{n}{radius}\DUrole{p}{:}\DUrole{w}{  }\DUrole{n}{float}}}{{ $\rightarrow$ float}}
\pysigstopsignatures
\sphinxAtStartPar
Calculate the total number of photometric counts at a pixel
location. The counts are already corrected for the sky counts.

\sphinxAtStartPar
This function does not check if a star is actually there.
\begin{quote}\begin{description}
\sphinxlineitem{Parameters}\begin{itemize}
\item {} 
\sphinxAtStartPar
\sphinxstyleliteralstrong{\sphinxupquote{pixel\_x}} (\sphinxstyleliteralemphasis{\sphinxupquote{int}}) – The x coordinate of the center pixel.

\item {} 
\sphinxAtStartPar
\sphinxstyleliteralstrong{\sphinxupquote{pixel\_y}} (\sphinxstyleliteralemphasis{\sphinxupquote{int}}) – The y coordinate of the center pixel.

\item {} 
\sphinxAtStartPar
\sphinxstyleliteralstrong{\sphinxupquote{radius}} (\sphinxstyleliteralemphasis{\sphinxupquote{float}}) – The radius of the circular aperture to be considered in pixel
counts.

\end{itemize}

\sphinxlineitem{Returns}
\sphinxAtStartPar
\sphinxstylestrong{photon\_counts} – The sum of the sky corrected counts for the region defined.

\sphinxlineitem{Return type}
\sphinxAtStartPar
float

\end{description}\end{quote}

\end{fulllineitems}\end{savenotes}

\index{\_calculate\_zero\_point() (opihiexarata.photometry.solution.PhotometricSolution method)@\spxentry{\_calculate\_zero\_point()}\spxextra{opihiexarata.photometry.solution.PhotometricSolution method}}

\begin{savenotes}\begin{fulllineitems}
\phantomsection\label{\detokenize{code/opihiexarata.photometry.solution:opihiexarata.photometry.solution.PhotometricSolution._calculate_zero_point}}
\pysigstartsignatures
\pysiglinewithargsret{\sphinxbfcode{\sphinxupquote{\_calculate\_zero\_point}}}{\emph{\DUrole{n}{exposure\_time}\DUrole{p}{:}\DUrole{w}{  }\DUrole{n}{float}}, \emph{\DUrole{n}{filter\_name}\DUrole{p}{:}\DUrole{w}{  }\DUrole{n}{Optional\DUrole{p}{{[}}str\DUrole{p}{{]}}}\DUrole{w}{  }\DUrole{o}{=}\DUrole{w}{  }\DUrole{default_value}{None}}}{{ $\rightarrow$ float}}
\pysigstopsignatures
\sphinxAtStartPar
This function calculates the photometric zero\sphinxhyphen{}point of the image
provided the data in the intersection star table.

\sphinxAtStartPar
This function uses the set exposure time and the intersection star
table. The band is also assumed from the initial parameters.
\begin{quote}\begin{description}
\sphinxlineitem{Parameters}\begin{itemize}
\item {} 
\sphinxAtStartPar
\sphinxstyleliteralstrong{\sphinxupquote{exposure\_time}} (\sphinxstyleliteralemphasis{\sphinxupquote{float}}) – How long, in seconds, the image in question was exposed for.

\item {} 
\sphinxAtStartPar
\sphinxstyleliteralstrong{\sphinxupquote{filter\_name}} (\sphinxstyleliteralemphasis{\sphinxupquote{string}}\sphinxstyleliteralemphasis{\sphinxupquote{, }}\sphinxstyleliteralemphasis{\sphinxupquote{default = None}}) – A single character string describing the name of the filter band that
this image was taken in. Currently, it assumes the MKO/SDSS visual
filters. If it is None, then this function does nothing.

\end{itemize}

\sphinxlineitem{Returns}
\sphinxAtStartPar
\begin{itemize}
\item {} 
\sphinxAtStartPar
\sphinxstylestrong{zero\_point} (\sphinxstyleemphasis{float}) – The zero point of this image. This is computed as a mean of all of
the calculated zero points.

\item {} 
\sphinxAtStartPar
\sphinxstylestrong{zero\_point\_error} (\sphinxstyleemphasis{float}) – The standard deviation of the zero points calculated.

\end{itemize}


\end{description}\end{quote}

\end{fulllineitems}\end{savenotes}


\end{fulllineitems}\end{savenotes}

\index{\_vehicle\_panstarrs\_mast\_web\_api() (in module opihiexarata.photometry.solution)@\spxentry{\_vehicle\_panstarrs\_mast\_web\_api()}\spxextra{in module opihiexarata.photometry.solution}}

\begin{savenotes}\begin{fulllineitems}
\phantomsection\label{\detokenize{code/opihiexarata.photometry.solution:opihiexarata.photometry.solution._vehicle_panstarrs_mast_web_api}}
\pysigstartsignatures
\pysiglinewithargsret{\sphinxcode{\sphinxupquote{opihiexarata.photometry.solution.}}\sphinxbfcode{\sphinxupquote{\_vehicle\_panstarrs\_mast\_web\_api}}}{\emph{\DUrole{n}{ra}\DUrole{p}{:}\DUrole{w}{  }\DUrole{n}{float}}, \emph{\DUrole{n}{dec}\DUrole{p}{:}\DUrole{w}{  }\DUrole{n}{float}}, \emph{\DUrole{n}{radius}\DUrole{p}{:}\DUrole{w}{  }\DUrole{n}{float}}}{{ $\rightarrow$ dict}}
\pysigstopsignatures
\sphinxAtStartPar
A vehicle function for photometric solutions. Extract photometric
data using the PanSTARRS database accessed via the MAST API.
\begin{quote}\begin{description}
\sphinxlineitem{Parameters}\begin{itemize}
\item {} 
\sphinxAtStartPar
\sphinxstyleliteralstrong{\sphinxupquote{ra}} (\sphinxstyleliteralemphasis{\sphinxupquote{float}}) – The right ascension of the center of the area to extract from,
in degrees.

\item {} 
\sphinxAtStartPar
\sphinxstyleliteralstrong{\sphinxupquote{dec}} (\sphinxstyleliteralemphasis{\sphinxupquote{float}}) – The declination of the center of the area to extract from,
in degrees.

\item {} 
\sphinxAtStartPar
\sphinxstyleliteralstrong{\sphinxupquote{radius}} (\sphinxstyleliteralemphasis{\sphinxupquote{float}}) – The search radius from the center that defines the search area.

\end{itemize}

\sphinxlineitem{Returns}
\sphinxAtStartPar
\sphinxstylestrong{photometry\_results} – The results of the photometry engine.

\sphinxlineitem{Return type}
\sphinxAtStartPar
dictionary

\end{description}\end{quote}

\end{fulllineitems}\end{savenotes}



\subparagraph{Module contents}
\label{\detokenize{code/opihiexarata.photometry:module-opihiexarata.photometry}}\label{\detokenize{code/opihiexarata.photometry:module-contents}}\index{module@\spxentry{module}!opihiexarata.photometry@\spxentry{opihiexarata.photometry}}\index{opihiexarata.photometry@\spxentry{opihiexarata.photometry}!module@\spxentry{module}}
\sphinxstepscope


\paragraph{opihiexarata.propagate package}
\label{\detokenize{code/opihiexarata.propagate:opihiexarata-propagate-package}}\label{\detokenize{code/opihiexarata.propagate::doc}}

\subparagraph{Submodules}
\label{\detokenize{code/opihiexarata.propagate:submodules}}
\sphinxstepscope


\subparagraph{opihiexarata.propagate.polynomial module}
\label{\detokenize{code/opihiexarata.propagate.polynomial:module-opihiexarata.propagate.polynomial}}\label{\detokenize{code/opihiexarata.propagate.polynomial:opihiexarata-propagate-polynomial-module}}\label{\detokenize{code/opihiexarata.propagate.polynomial::doc}}\index{module@\spxentry{module}!opihiexarata.propagate.polynomial@\spxentry{opihiexarata.propagate.polynomial}}\index{opihiexarata.propagate.polynomial@\spxentry{opihiexarata.propagate.polynomial}!module@\spxentry{module}}
\sphinxAtStartPar
For polynomial fitting propagation, using approximations of 1st or 2nd order
terms but ignoring some spherical effects.

\sphinxAtStartPar
Although this could be easily implemented in a better method using subclassing
rather than having two classes, as having a 3rd order is not really feasible,
and for the sake of readability and stability, two separate copy\sphinxhyphen{}like classes
are written.
\index{LinearPropagationEngine (class in opihiexarata.propagate.polynomial)@\spxentry{LinearPropagationEngine}\spxextra{class in opihiexarata.propagate.polynomial}}

\begin{savenotes}\begin{fulllineitems}
\phantomsection\label{\detokenize{code/opihiexarata.propagate.polynomial:opihiexarata.propagate.polynomial.LinearPropagationEngine}}
\pysigstartsignatures
\pysiglinewithargsret{\sphinxbfcode{\sphinxupquote{class\DUrole{w}{  }}}\sphinxcode{\sphinxupquote{opihiexarata.propagate.polynomial.}}\sphinxbfcode{\sphinxupquote{LinearPropagationEngine}}}{\emph{\DUrole{n}{ra}\DUrole{p}{:}\DUrole{w}{  }\DUrole{n}{ndarray}}, \emph{\DUrole{n}{dec}\DUrole{p}{:}\DUrole{w}{  }\DUrole{n}{ndarray}}, \emph{\DUrole{n}{obs\_time}\DUrole{p}{:}\DUrole{w}{  }\DUrole{n}{ndarray}}}{}
\pysigstopsignatures
\sphinxAtStartPar
Bases: {\hyperref[\detokenize{code/opihiexarata.library.engine:opihiexarata.library.engine.PropagationEngine}]{\sphinxcrossref{\sphinxcode{\sphinxupquote{PropagationEngine}}}}}

\sphinxAtStartPar
A simple propagation engine which uses 1st order extrapolation of
RA DEC points independently to determine future location.
\index{ra\_array (opihiexarata.propagate.polynomial.LinearPropagationEngine attribute)@\spxentry{ra\_array}\spxextra{opihiexarata.propagate.polynomial.LinearPropagationEngine attribute}}

\begin{savenotes}\begin{fulllineitems}
\phantomsection\label{\detokenize{code/opihiexarata.propagate.polynomial:opihiexarata.propagate.polynomial.LinearPropagationEngine.ra_array}}
\pysigstartsignatures
\pysigline{\sphinxbfcode{\sphinxupquote{ra\_array}}}
\pysigstopsignatures
\sphinxAtStartPar
The array of right ascension measurements to extrapolate to.
\begin{quote}\begin{description}
\sphinxlineitem{Type}
\sphinxAtStartPar
ndarray

\end{description}\end{quote}

\end{fulllineitems}\end{savenotes}

\index{dec\_array (opihiexarata.propagate.polynomial.LinearPropagationEngine attribute)@\spxentry{dec\_array}\spxextra{opihiexarata.propagate.polynomial.LinearPropagationEngine attribute}}

\begin{savenotes}\begin{fulllineitems}
\phantomsection\label{\detokenize{code/opihiexarata.propagate.polynomial:opihiexarata.propagate.polynomial.LinearPropagationEngine.dec_array}}
\pysigstartsignatures
\pysigline{\sphinxbfcode{\sphinxupquote{dec\_array}}}
\pysigstopsignatures
\sphinxAtStartPar
The array of declinations measurements to extrapolate to.
\begin{quote}\begin{description}
\sphinxlineitem{Type}
\sphinxAtStartPar
ndarray

\end{description}\end{quote}

\end{fulllineitems}\end{savenotes}

\index{obs\_time\_array (opihiexarata.propagate.polynomial.LinearPropagationEngine attribute)@\spxentry{obs\_time\_array}\spxextra{opihiexarata.propagate.polynomial.LinearPropagationEngine attribute}}

\begin{savenotes}\begin{fulllineitems}
\phantomsection\label{\detokenize{code/opihiexarata.propagate.polynomial:opihiexarata.propagate.polynomial.LinearPropagationEngine.obs_time_array}}
\pysigstartsignatures
\pysigline{\sphinxbfcode{\sphinxupquote{obs\_time\_array}}}
\pysigstopsignatures
\sphinxAtStartPar
The array of observation times which the RA and DEC measurements were
taken at. The values are in Julian days.
\begin{quote}\begin{description}
\sphinxlineitem{Type}
\sphinxAtStartPar
ndarray

\end{description}\end{quote}

\end{fulllineitems}\end{savenotes}

\index{ra\_poly\_param (opihiexarata.propagate.polynomial.LinearPropagationEngine attribute)@\spxentry{ra\_poly\_param}\spxextra{opihiexarata.propagate.polynomial.LinearPropagationEngine attribute}}

\begin{savenotes}\begin{fulllineitems}
\phantomsection\label{\detokenize{code/opihiexarata.propagate.polynomial:opihiexarata.propagate.polynomial.LinearPropagationEngine.ra_poly_param}}
\pysigstartsignatures
\pysigline{\sphinxbfcode{\sphinxupquote{ra\_poly\_param}}}
\pysigstopsignatures
\sphinxAtStartPar
The polynomial fit parameters for the RA(time) propagation.
\begin{quote}\begin{description}
\sphinxlineitem{Type}
\sphinxAtStartPar
tuple

\end{description}\end{quote}

\end{fulllineitems}\end{savenotes}

\index{dec\_poly\_param (opihiexarata.propagate.polynomial.LinearPropagationEngine attribute)@\spxentry{dec\_poly\_param}\spxextra{opihiexarata.propagate.polynomial.LinearPropagationEngine attribute}}

\begin{savenotes}\begin{fulllineitems}
\phantomsection\label{\detokenize{code/opihiexarata.propagate.polynomial:opihiexarata.propagate.polynomial.LinearPropagationEngine.dec_poly_param}}
\pysigstartsignatures
\pysigline{\sphinxbfcode{\sphinxupquote{dec\_poly\_param}}}
\pysigstopsignatures
\sphinxAtStartPar
The polynomial fit parameters for the DEC(time) propagation.
\begin{quote}\begin{description}
\sphinxlineitem{Type}
\sphinxAtStartPar
tuple

\end{description}\end{quote}

\end{fulllineitems}\end{savenotes}

\index{\_\_fit\_polynomial\_function() (opihiexarata.propagate.polynomial.LinearPropagationEngine method)@\spxentry{\_\_fit\_polynomial\_function()}\spxextra{opihiexarata.propagate.polynomial.LinearPropagationEngine method}}

\begin{savenotes}\begin{fulllineitems}
\phantomsection\label{\detokenize{code/opihiexarata.propagate.polynomial:opihiexarata.propagate.polynomial.LinearPropagationEngine.__fit_polynomial_function}}
\pysigstartsignatures
\pysiglinewithargsret{\sphinxbfcode{\sphinxupquote{\_\_fit\_polynomial\_function}}}{\emph{\DUrole{n}{fit\_x}\DUrole{p}{:}\DUrole{w}{  }\DUrole{n}{ndarray}}, \emph{\DUrole{n}{fit\_y}\DUrole{p}{:}\DUrole{w}{  }\DUrole{n}{ndarray}}}{{ $\rightarrow$ tuple\DUrole{p}{{[}}tuple\DUrole{p}{,}\DUrole{w}{  }tuple\DUrole{p}{{]}}}}
\pysigstopsignatures
\sphinxAtStartPar
A wrapper class for fitting the defined specific polynomial function.
\begin{quote}\begin{description}
\sphinxlineitem{Parameters}\begin{itemize}
\item {} 
\sphinxAtStartPar
\sphinxstyleliteralstrong{\sphinxupquote{fix\_x}} (\sphinxstyleliteralemphasis{\sphinxupquote{array\sphinxhyphen{}like}}) – The x values which shall be fit.

\item {} 
\sphinxAtStartPar
\sphinxstyleliteralstrong{\sphinxupquote{fix\_y}} (\sphinxstyleliteralemphasis{\sphinxupquote{array\sphinxhyphen{}like}}) – The y values which shall be fit.

\end{itemize}

\sphinxlineitem{Returns}
\sphinxAtStartPar
\begin{itemize}
\item {} 
\sphinxAtStartPar
\sphinxstylestrong{fit\_param} (\sphinxstyleemphasis{tuple}) – The parameters of the polynomial that corresponded to the best fit.
Determined by the order of the polynomial function.

\item {} 
\sphinxAtStartPar
\sphinxstylestrong{fit\_error} (\sphinxstyleemphasis{tuple}) – The error on the parameters of the fit.

\end{itemize}


\end{description}\end{quote}

\end{fulllineitems}\end{savenotes}

\index{\_\_init\_\_() (opihiexarata.propagate.polynomial.LinearPropagationEngine method)@\spxentry{\_\_init\_\_()}\spxextra{opihiexarata.propagate.polynomial.LinearPropagationEngine method}}

\begin{savenotes}\begin{fulllineitems}
\phantomsection\label{\detokenize{code/opihiexarata.propagate.polynomial:opihiexarata.propagate.polynomial.LinearPropagationEngine.__init__}}
\pysigstartsignatures
\pysiglinewithargsret{\sphinxbfcode{\sphinxupquote{\_\_init\_\_}}}{\emph{\DUrole{n}{ra}\DUrole{p}{:}\DUrole{w}{  }\DUrole{n}{ndarray}}, \emph{\DUrole{n}{dec}\DUrole{p}{:}\DUrole{w}{  }\DUrole{n}{ndarray}}, \emph{\DUrole{n}{obs\_time}\DUrole{p}{:}\DUrole{w}{  }\DUrole{n}{ndarray}}}{{ $\rightarrow$ None}}
\pysigstopsignatures
\sphinxAtStartPar
Instantiation of the propagation engine.
\begin{quote}\begin{description}
\sphinxlineitem{Parameters}\begin{itemize}
\item {} 
\sphinxAtStartPar
\sphinxstyleliteralstrong{\sphinxupquote{ra}} (\sphinxstyleliteralemphasis{\sphinxupquote{array\sphinxhyphen{}like}}) – An array of right ascensions to fit and extrapolate to, must be in
degrees.

\item {} 
\sphinxAtStartPar
\sphinxstyleliteralstrong{\sphinxupquote{dec}} (\sphinxstyleliteralemphasis{\sphinxupquote{array\sphinxhyphen{}like}}) – An array of declinations to fit and extrapolate to, must be in
degrees.

\item {} 
\sphinxAtStartPar
\sphinxstyleliteralstrong{\sphinxupquote{obs\_time}} (\sphinxstyleliteralemphasis{\sphinxupquote{array\sphinxhyphen{}like}}) – An array of observation times which the RA and DEC measurements
were taken at. This should be in Julian days.

\end{itemize}

\sphinxlineitem{Return type}
\sphinxAtStartPar
None

\end{description}\end{quote}

\end{fulllineitems}\end{savenotes}

\index{\_\_linear\_function() (opihiexarata.propagate.polynomial.LinearPropagationEngine method)@\spxentry{\_\_linear\_function()}\spxextra{opihiexarata.propagate.polynomial.LinearPropagationEngine method}}

\begin{savenotes}\begin{fulllineitems}
\phantomsection\label{\detokenize{code/opihiexarata.propagate.polynomial:opihiexarata.propagate.polynomial.LinearPropagationEngine.__linear_function}}
\pysigstartsignatures
\pysiglinewithargsret{\sphinxbfcode{\sphinxupquote{\_\_linear\_function}}}{\emph{\DUrole{n}{c0}\DUrole{p}{:}\DUrole{w}{  }\DUrole{n}{float}}, \emph{\DUrole{n}{c1}\DUrole{p}{:}\DUrole{w}{  }\DUrole{n}{float}}}{{ $\rightarrow$ ndarray}}
\pysigstopsignatures
\sphinxAtStartPar
The linear polynomial function that will be used.

\sphinxAtStartPar
This function is hard coded to be a specific order on purpose. The
order may be changed between versions if need be, but should not be
changed via configuration.
\begin{quote}\begin{description}
\sphinxlineitem{Parameters}\begin{itemize}
\item {} 
\sphinxAtStartPar
\sphinxstyleliteralstrong{\sphinxupquote{x}} (\sphinxstyleliteralemphasis{\sphinxupquote{array\sphinxhyphen{}like}}) – The input for computing the polynomial.

\item {} 
\sphinxAtStartPar
\sphinxstyleliteralstrong{\sphinxupquote{c0}} (\sphinxstyleliteralemphasis{\sphinxupquote{float}}) – Coefficient for order 0.

\item {} 
\sphinxAtStartPar
\sphinxstyleliteralstrong{\sphinxupquote{c1}} (\sphinxstyleliteralemphasis{\sphinxupquote{float}}) – Coefficient for order 1.

\end{itemize}

\sphinxlineitem{Returns}
\sphinxAtStartPar
\sphinxstylestrong{y} – The output after computing the polynomial with the provided
coefficients.

\sphinxlineitem{Return type}
\sphinxAtStartPar
array\sphinxhyphen{}like

\end{description}\end{quote}

\end{fulllineitems}\end{savenotes}

\index{forward\_propagate() (opihiexarata.propagate.polynomial.LinearPropagationEngine method)@\spxentry{forward\_propagate()}\spxextra{opihiexarata.propagate.polynomial.LinearPropagationEngine method}}

\begin{savenotes}\begin{fulllineitems}
\phantomsection\label{\detokenize{code/opihiexarata.propagate.polynomial:opihiexarata.propagate.polynomial.LinearPropagationEngine.forward_propagate}}
\pysigstartsignatures
\pysiglinewithargsret{\sphinxbfcode{\sphinxupquote{forward\_propagate}}}{\emph{\DUrole{n}{future\_time}\DUrole{p}{:}\DUrole{w}{  }\DUrole{n}{ndarray}}}{{ $\rightarrow$ tuple\DUrole{p}{{[}}numpy.ndarray\DUrole{p}{,}\DUrole{w}{  }numpy.ndarray\DUrole{p}{{]}}}}
\pysigstopsignatures
\sphinxAtStartPar
Determine a new location(s) based on the polynomial propagation,
providing new times to locate in the future.
\begin{quote}\begin{description}
\sphinxlineitem{Parameters}
\sphinxAtStartPar
\sphinxstyleliteralstrong{\sphinxupquote{future\_time}} (\sphinxstyleliteralemphasis{\sphinxupquote{array\sphinxhyphen{}like}}) – The set of future times which to derive new RA and DEC coordinates.
The time must be in Julian days.

\sphinxlineitem{Returns}
\sphinxAtStartPar
\begin{itemize}
\item {} 
\sphinxAtStartPar
\sphinxstylestrong{future\_ra} (\sphinxstyleemphasis{ndarray}) – The set of right ascensions that corresponds to the future times,
in degrees.

\item {} 
\sphinxAtStartPar
\sphinxstylestrong{future\_dec} (\sphinxstyleemphasis{ndarray}) – The set of declinations that corresponds to the future times, in
degrees.

\end{itemize}


\end{description}\end{quote}

\end{fulllineitems}\end{savenotes}


\end{fulllineitems}\end{savenotes}

\index{QuadraticPropagationEngine (class in opihiexarata.propagate.polynomial)@\spxentry{QuadraticPropagationEngine}\spxextra{class in opihiexarata.propagate.polynomial}}

\begin{savenotes}\begin{fulllineitems}
\phantomsection\label{\detokenize{code/opihiexarata.propagate.polynomial:opihiexarata.propagate.polynomial.QuadraticPropagationEngine}}
\pysigstartsignatures
\pysiglinewithargsret{\sphinxbfcode{\sphinxupquote{class\DUrole{w}{  }}}\sphinxcode{\sphinxupquote{opihiexarata.propagate.polynomial.}}\sphinxbfcode{\sphinxupquote{QuadraticPropagationEngine}}}{\emph{\DUrole{n}{ra}\DUrole{p}{:}\DUrole{w}{  }\DUrole{n}{ndarray}}, \emph{\DUrole{n}{dec}\DUrole{p}{:}\DUrole{w}{  }\DUrole{n}{ndarray}}, \emph{\DUrole{n}{obs\_time}\DUrole{p}{:}\DUrole{w}{  }\DUrole{n}{ndarray}}}{}
\pysigstopsignatures
\sphinxAtStartPar
Bases: {\hyperref[\detokenize{code/opihiexarata.library.engine:opihiexarata.library.engine.PropagationEngine}]{\sphinxcrossref{\sphinxcode{\sphinxupquote{PropagationEngine}}}}}

\sphinxAtStartPar
A simple propagation engine which uses 2nd order extrapolation of
RA DEC points independently to determine future location.
\index{ra\_array (opihiexarata.propagate.polynomial.QuadraticPropagationEngine attribute)@\spxentry{ra\_array}\spxextra{opihiexarata.propagate.polynomial.QuadraticPropagationEngine attribute}}

\begin{savenotes}\begin{fulllineitems}
\phantomsection\label{\detokenize{code/opihiexarata.propagate.polynomial:opihiexarata.propagate.polynomial.QuadraticPropagationEngine.ra_array}}
\pysigstartsignatures
\pysigline{\sphinxbfcode{\sphinxupquote{ra\_array}}}
\pysigstopsignatures
\sphinxAtStartPar
The array of right ascension measurements to extrapolate to.
\begin{quote}\begin{description}
\sphinxlineitem{Type}
\sphinxAtStartPar
ndarray

\end{description}\end{quote}

\end{fulllineitems}\end{savenotes}

\index{dec\_array (opihiexarata.propagate.polynomial.QuadraticPropagationEngine attribute)@\spxentry{dec\_array}\spxextra{opihiexarata.propagate.polynomial.QuadraticPropagationEngine attribute}}

\begin{savenotes}\begin{fulllineitems}
\phantomsection\label{\detokenize{code/opihiexarata.propagate.polynomial:opihiexarata.propagate.polynomial.QuadraticPropagationEngine.dec_array}}
\pysigstartsignatures
\pysigline{\sphinxbfcode{\sphinxupquote{dec\_array}}}
\pysigstopsignatures
\sphinxAtStartPar
The array of declinations measurements to extrapolate to.
\begin{quote}\begin{description}
\sphinxlineitem{Type}
\sphinxAtStartPar
ndarray

\end{description}\end{quote}

\end{fulllineitems}\end{savenotes}

\index{obs\_time\_array (opihiexarata.propagate.polynomial.QuadraticPropagationEngine attribute)@\spxentry{obs\_time\_array}\spxextra{opihiexarata.propagate.polynomial.QuadraticPropagationEngine attribute}}

\begin{savenotes}\begin{fulllineitems}
\phantomsection\label{\detokenize{code/opihiexarata.propagate.polynomial:opihiexarata.propagate.polynomial.QuadraticPropagationEngine.obs_time_array}}
\pysigstartsignatures
\pysigline{\sphinxbfcode{\sphinxupquote{obs\_time\_array}}}
\pysigstopsignatures
\sphinxAtStartPar
The array of observation times which the RA and DEC measurements were
taken at. The values are in Julian days.
\begin{quote}\begin{description}
\sphinxlineitem{Type}
\sphinxAtStartPar
ndarray

\end{description}\end{quote}

\end{fulllineitems}\end{savenotes}

\index{ra\_poly\_param (opihiexarata.propagate.polynomial.QuadraticPropagationEngine attribute)@\spxentry{ra\_poly\_param}\spxextra{opihiexarata.propagate.polynomial.QuadraticPropagationEngine attribute}}

\begin{savenotes}\begin{fulllineitems}
\phantomsection\label{\detokenize{code/opihiexarata.propagate.polynomial:opihiexarata.propagate.polynomial.QuadraticPropagationEngine.ra_poly_param}}
\pysigstartsignatures
\pysigline{\sphinxbfcode{\sphinxupquote{ra\_poly\_param}}}
\pysigstopsignatures
\sphinxAtStartPar
The polynomial fit parameters for the RA(time) propagation.
\begin{quote}\begin{description}
\sphinxlineitem{Type}
\sphinxAtStartPar
tuple

\end{description}\end{quote}

\end{fulllineitems}\end{savenotes}

\index{dec\_poly\_param (opihiexarata.propagate.polynomial.QuadraticPropagationEngine attribute)@\spxentry{dec\_poly\_param}\spxextra{opihiexarata.propagate.polynomial.QuadraticPropagationEngine attribute}}

\begin{savenotes}\begin{fulllineitems}
\phantomsection\label{\detokenize{code/opihiexarata.propagate.polynomial:opihiexarata.propagate.polynomial.QuadraticPropagationEngine.dec_poly_param}}
\pysigstartsignatures
\pysigline{\sphinxbfcode{\sphinxupquote{dec\_poly\_param}}}
\pysigstopsignatures
\sphinxAtStartPar
The polynomial fit parameters for the DEC(time) propagation.
\begin{quote}\begin{description}
\sphinxlineitem{Type}
\sphinxAtStartPar
tuple

\end{description}\end{quote}

\end{fulllineitems}\end{savenotes}

\index{\_\_fit\_polynomial\_function() (opihiexarata.propagate.polynomial.QuadraticPropagationEngine method)@\spxentry{\_\_fit\_polynomial\_function()}\spxextra{opihiexarata.propagate.polynomial.QuadraticPropagationEngine method}}

\begin{savenotes}\begin{fulllineitems}
\phantomsection\label{\detokenize{code/opihiexarata.propagate.polynomial:opihiexarata.propagate.polynomial.QuadraticPropagationEngine.__fit_polynomial_function}}
\pysigstartsignatures
\pysiglinewithargsret{\sphinxbfcode{\sphinxupquote{\_\_fit\_polynomial\_function}}}{\emph{\DUrole{n}{fit\_x}\DUrole{p}{:}\DUrole{w}{  }\DUrole{n}{ndarray}}, \emph{\DUrole{n}{fit\_y}\DUrole{p}{:}\DUrole{w}{  }\DUrole{n}{ndarray}}}{{ $\rightarrow$ tuple\DUrole{p}{{[}}tuple\DUrole{p}{,}\DUrole{w}{  }tuple\DUrole{p}{{]}}}}
\pysigstopsignatures
\sphinxAtStartPar
A wrapper class for fitting the defined specific polynomial function.
\begin{quote}\begin{description}
\sphinxlineitem{Parameters}\begin{itemize}
\item {} 
\sphinxAtStartPar
\sphinxstyleliteralstrong{\sphinxupquote{fix\_x}} (\sphinxstyleliteralemphasis{\sphinxupquote{array\sphinxhyphen{}like}}) – The x values which shall be fit.

\item {} 
\sphinxAtStartPar
\sphinxstyleliteralstrong{\sphinxupquote{fix\_y}} (\sphinxstyleliteralemphasis{\sphinxupquote{array\sphinxhyphen{}like}}) – The y values which shall be fit.

\end{itemize}

\sphinxlineitem{Returns}
\sphinxAtStartPar
\begin{itemize}
\item {} 
\sphinxAtStartPar
\sphinxstylestrong{fit\_param} (\sphinxstyleemphasis{tuple}) – The parameters of the polynomial that corresponded to the best fit.
Determined by the order of the polynomial function.

\item {} 
\sphinxAtStartPar
\sphinxstylestrong{fit\_error} (\sphinxstyleemphasis{tuple}) – The error on the parameters of the fit.

\end{itemize}


\end{description}\end{quote}

\end{fulllineitems}\end{savenotes}

\index{\_\_init\_\_() (opihiexarata.propagate.polynomial.QuadraticPropagationEngine method)@\spxentry{\_\_init\_\_()}\spxextra{opihiexarata.propagate.polynomial.QuadraticPropagationEngine method}}

\begin{savenotes}\begin{fulllineitems}
\phantomsection\label{\detokenize{code/opihiexarata.propagate.polynomial:opihiexarata.propagate.polynomial.QuadraticPropagationEngine.__init__}}
\pysigstartsignatures
\pysiglinewithargsret{\sphinxbfcode{\sphinxupquote{\_\_init\_\_}}}{\emph{\DUrole{n}{ra}\DUrole{p}{:}\DUrole{w}{  }\DUrole{n}{ndarray}}, \emph{\DUrole{n}{dec}\DUrole{p}{:}\DUrole{w}{  }\DUrole{n}{ndarray}}, \emph{\DUrole{n}{obs\_time}\DUrole{p}{:}\DUrole{w}{  }\DUrole{n}{ndarray}}}{{ $\rightarrow$ None}}
\pysigstopsignatures
\sphinxAtStartPar
Instantiation of the propagation engine.
\begin{quote}\begin{description}
\sphinxlineitem{Parameters}\begin{itemize}
\item {} 
\sphinxAtStartPar
\sphinxstyleliteralstrong{\sphinxupquote{ra}} (\sphinxstyleliteralemphasis{\sphinxupquote{array\sphinxhyphen{}like}}) – An array of right ascensions to fit and extrapolate to, must be in
degrees.

\item {} 
\sphinxAtStartPar
\sphinxstyleliteralstrong{\sphinxupquote{dec}} (\sphinxstyleliteralemphasis{\sphinxupquote{array\sphinxhyphen{}like}}) – An array of declinations to fit and extrapolate to, must be in
degrees.

\item {} 
\sphinxAtStartPar
\sphinxstyleliteralstrong{\sphinxupquote{obs\_time}} (\sphinxstyleliteralemphasis{\sphinxupquote{array\sphinxhyphen{}like}}) – An array of observation times which the RA and DEC measurements
were taken at. This should be in Julian days.

\end{itemize}

\sphinxlineitem{Return type}
\sphinxAtStartPar
None

\end{description}\end{quote}

\end{fulllineitems}\end{savenotes}

\index{\_\_quadratic\_function() (opihiexarata.propagate.polynomial.QuadraticPropagationEngine method)@\spxentry{\_\_quadratic\_function()}\spxextra{opihiexarata.propagate.polynomial.QuadraticPropagationEngine method}}

\begin{savenotes}\begin{fulllineitems}
\phantomsection\label{\detokenize{code/opihiexarata.propagate.polynomial:opihiexarata.propagate.polynomial.QuadraticPropagationEngine.__quadratic_function}}
\pysigstartsignatures
\pysiglinewithargsret{\sphinxbfcode{\sphinxupquote{\_\_quadratic\_function}}}{\emph{\DUrole{n}{c0}\DUrole{p}{:}\DUrole{w}{  }\DUrole{n}{float}}, \emph{\DUrole{n}{c1}\DUrole{p}{:}\DUrole{w}{  }\DUrole{n}{float}}, \emph{\DUrole{n}{c2}\DUrole{p}{:}\DUrole{w}{  }\DUrole{n}{float}}}{{ $\rightarrow$ ndarray}}
\pysigstopsignatures
\sphinxAtStartPar
The polynomial function that will be used.

\sphinxAtStartPar
This function is hard coded to be a specific order on purpose. The
order may be changed between versions if need be, but should not be
changed via configuration.
\begin{quote}\begin{description}
\sphinxlineitem{Parameters}\begin{itemize}
\item {} 
\sphinxAtStartPar
\sphinxstyleliteralstrong{\sphinxupquote{x}} (\sphinxstyleliteralemphasis{\sphinxupquote{array\sphinxhyphen{}like}}) – The input for computing the polynomial.

\item {} 
\sphinxAtStartPar
\sphinxstyleliteralstrong{\sphinxupquote{c0}} (\sphinxstyleliteralemphasis{\sphinxupquote{float}}) – Coefficient for order 0.

\item {} 
\sphinxAtStartPar
\sphinxstyleliteralstrong{\sphinxupquote{c1}} (\sphinxstyleliteralemphasis{\sphinxupquote{float}}) – Coefficient for order 1.

\item {} 
\sphinxAtStartPar
\sphinxstyleliteralstrong{\sphinxupquote{c2}} (\sphinxstyleliteralemphasis{\sphinxupquote{float}}) – Coefficient for order 2.

\end{itemize}

\sphinxlineitem{Returns}
\sphinxAtStartPar
\sphinxstylestrong{y} – The output after computing the polynomial with the provided
coefficients.

\sphinxlineitem{Return type}
\sphinxAtStartPar
array\sphinxhyphen{}like

\end{description}\end{quote}

\end{fulllineitems}\end{savenotes}

\index{forward\_propagate() (opihiexarata.propagate.polynomial.QuadraticPropagationEngine method)@\spxentry{forward\_propagate()}\spxextra{opihiexarata.propagate.polynomial.QuadraticPropagationEngine method}}

\begin{savenotes}\begin{fulllineitems}
\phantomsection\label{\detokenize{code/opihiexarata.propagate.polynomial:opihiexarata.propagate.polynomial.QuadraticPropagationEngine.forward_propagate}}
\pysigstartsignatures
\pysiglinewithargsret{\sphinxbfcode{\sphinxupquote{forward\_propagate}}}{\emph{\DUrole{n}{future\_time}\DUrole{p}{:}\DUrole{w}{  }\DUrole{n}{ndarray}}}{{ $\rightarrow$ tuple\DUrole{p}{{[}}numpy.ndarray\DUrole{p}{,}\DUrole{w}{  }numpy.ndarray\DUrole{p}{{]}}}}
\pysigstopsignatures
\sphinxAtStartPar
Determine a new location(s) based on the polynomial propagation,
providing new times to locate in the future.
\begin{quote}\begin{description}
\sphinxlineitem{Parameters}
\sphinxAtStartPar
\sphinxstyleliteralstrong{\sphinxupquote{future\_time}} (\sphinxstyleliteralemphasis{\sphinxupquote{array\sphinxhyphen{}like}}) – The set of future times which to derive new RA and DEC coordinates.
The time must be in Julian days.

\sphinxlineitem{Returns}
\sphinxAtStartPar
\begin{itemize}
\item {} 
\sphinxAtStartPar
\sphinxstylestrong{future\_ra} (\sphinxstyleemphasis{ndarray}) – The set of right ascensions that corresponds to the future times,
in degrees.

\item {} 
\sphinxAtStartPar
\sphinxstylestrong{future\_dec} (\sphinxstyleemphasis{ndarray}) – The set of declinations that corresponds to the future times, in
degrees.

\end{itemize}


\end{description}\end{quote}

\end{fulllineitems}\end{savenotes}


\end{fulllineitems}\end{savenotes}


\sphinxstepscope


\subparagraph{opihiexarata.propagate.solution module}
\label{\detokenize{code/opihiexarata.propagate.solution:module-opihiexarata.propagate.solution}}\label{\detokenize{code/opihiexarata.propagate.solution:opihiexarata-propagate-solution-module}}\label{\detokenize{code/opihiexarata.propagate.solution::doc}}\index{module@\spxentry{module}!opihiexarata.propagate.solution@\spxentry{opihiexarata.propagate.solution}}\index{opihiexarata.propagate.solution@\spxentry{opihiexarata.propagate.solution}!module@\spxentry{module}}
\sphinxAtStartPar
The main solution class for propagations.
\index{PropagativeSolution (class in opihiexarata.propagate.solution)@\spxentry{PropagativeSolution}\spxextra{class in opihiexarata.propagate.solution}}

\begin{savenotes}\begin{fulllineitems}
\phantomsection\label{\detokenize{code/opihiexarata.propagate.solution:opihiexarata.propagate.solution.PropagativeSolution}}
\pysigstartsignatures
\pysiglinewithargsret{\sphinxbfcode{\sphinxupquote{class\DUrole{w}{  }}}\sphinxcode{\sphinxupquote{opihiexarata.propagate.solution.}}\sphinxbfcode{\sphinxupquote{PropagativeSolution}}}{\emph{\DUrole{n}{ra}\DUrole{p}{:}\DUrole{w}{  }\DUrole{n}{ndarray}}, \emph{\DUrole{n}{dec}\DUrole{p}{:}\DUrole{w}{  }\DUrole{n}{ndarray}}, \emph{\DUrole{n}{obs\_time}\DUrole{p}{:}\DUrole{w}{  }\DUrole{n}{list}}, \emph{\DUrole{n}{solver\_engine}\DUrole{p}{:}\DUrole{w}{  }\DUrole{n}{{\hyperref[\detokenize{code/opihiexarata.library.engine:opihiexarata.library.engine.PropagationEngine}]{\sphinxcrossref{PropagationEngine}}}}}, \emph{\DUrole{n}{vehicle\_args}\DUrole{p}{:}\DUrole{w}{  }\DUrole{n}{dict}\DUrole{w}{  }\DUrole{o}{=}\DUrole{w}{  }\DUrole{default_value}{\{\}}}}{}
\pysigstopsignatures
\sphinxAtStartPar
Bases: {\hyperref[\detokenize{code/opihiexarata.library.engine:opihiexarata.library.engine.ExarataSolution}]{\sphinxcrossref{\sphinxcode{\sphinxupquote{ExarataSolution}}}}}

\sphinxAtStartPar
The general solution class for asteroid propagation.

\sphinxAtStartPar
This uses the recent past location of asteroids to determine their
future location. For determination based on orbital elements and
ephemerids, use the OrbitalSolution and EphemeriticSolution classes
respectively.
\index{ra\_array (opihiexarata.propagate.solution.PropagativeSolution attribute)@\spxentry{ra\_array}\spxextra{opihiexarata.propagate.solution.PropagativeSolution attribute}}

\begin{savenotes}\begin{fulllineitems}
\phantomsection\label{\detokenize{code/opihiexarata.propagate.solution:opihiexarata.propagate.solution.PropagativeSolution.ra_array}}
\pysigstartsignatures
\pysigline{\sphinxbfcode{\sphinxupquote{ra\_array}}}
\pysigstopsignatures
\sphinxAtStartPar
The array of right ascensions used fit and extrapolate to,
in degrees.
\begin{quote}\begin{description}
\sphinxlineitem{Type}
\sphinxAtStartPar
array\sphinxhyphen{}like

\end{description}\end{quote}

\end{fulllineitems}\end{savenotes}

\index{dec\_array (opihiexarata.propagate.solution.PropagativeSolution attribute)@\spxentry{dec\_array}\spxextra{opihiexarata.propagate.solution.PropagativeSolution attribute}}

\begin{savenotes}\begin{fulllineitems}
\phantomsection\label{\detokenize{code/opihiexarata.propagate.solution:opihiexarata.propagate.solution.PropagativeSolution.dec_array}}
\pysigstartsignatures
\pysigline{\sphinxbfcode{\sphinxupquote{dec\_array}}}
\pysigstopsignatures
\sphinxAtStartPar
The array of declinations used fit and extrapolate to, in degrees.
\begin{quote}\begin{description}
\sphinxlineitem{Type}
\sphinxAtStartPar
array\sphinxhyphen{}like

\end{description}\end{quote}

\end{fulllineitems}\end{savenotes}

\index{obs\_time\_array (opihiexarata.propagate.solution.PropagativeSolution attribute)@\spxentry{obs\_time\_array}\spxextra{opihiexarata.propagate.solution.PropagativeSolution attribute}}

\begin{savenotes}\begin{fulllineitems}
\phantomsection\label{\detokenize{code/opihiexarata.propagate.solution:opihiexarata.propagate.solution.PropagativeSolution.obs_time_array}}
\pysigstartsignatures
\pysigline{\sphinxbfcode{\sphinxupquote{obs\_time\_array}}}
\pysigstopsignatures
\sphinxAtStartPar
An array of observation times which the RA and DEC measurements
were taken at. The values are in Julian days.
\begin{quote}\begin{description}
\sphinxlineitem{Type}
\sphinxAtStartPar
array\sphinxhyphen{}like

\end{description}\end{quote}

\end{fulllineitems}\end{savenotes}

\index{raw\_ra\_velocity (opihiexarata.propagate.solution.PropagativeSolution attribute)@\spxentry{raw\_ra\_velocity}\spxextra{opihiexarata.propagate.solution.PropagativeSolution attribute}}

\begin{savenotes}\begin{fulllineitems}
\phantomsection\label{\detokenize{code/opihiexarata.propagate.solution:opihiexarata.propagate.solution.PropagativeSolution.raw_ra_velocity}}
\pysigstartsignatures
\pysigline{\sphinxbfcode{\sphinxupquote{raw\_ra\_velocity}}}
\pysigstopsignatures
\sphinxAtStartPar
The right ascension angular velocity of the target, in degrees per
second. These values are derived straight from the data and not the
propagation engine.
\begin{quote}\begin{description}
\sphinxlineitem{Type}
\sphinxAtStartPar
float

\end{description}\end{quote}

\end{fulllineitems}\end{savenotes}

\index{raw\_dec\_velocity (opihiexarata.propagate.solution.PropagativeSolution attribute)@\spxentry{raw\_dec\_velocity}\spxextra{opihiexarata.propagate.solution.PropagativeSolution attribute}}

\begin{savenotes}\begin{fulllineitems}
\phantomsection\label{\detokenize{code/opihiexarata.propagate.solution:opihiexarata.propagate.solution.PropagativeSolution.raw_dec_velocity}}
\pysigstartsignatures
\pysigline{\sphinxbfcode{\sphinxupquote{raw\_dec\_velocity}}}
\pysigstopsignatures
\sphinxAtStartPar
The declination angular velocity of the target, in degrees per
second. These values are derived straight from the data and not the
propagation engine.
\begin{quote}\begin{description}
\sphinxlineitem{Type}
\sphinxAtStartPar
float

\end{description}\end{quote}

\end{fulllineitems}\end{savenotes}

\index{raw\_ra\_acceleration (opihiexarata.propagate.solution.PropagativeSolution attribute)@\spxentry{raw\_ra\_acceleration}\spxextra{opihiexarata.propagate.solution.PropagativeSolution attribute}}

\begin{savenotes}\begin{fulllineitems}
\phantomsection\label{\detokenize{code/opihiexarata.propagate.solution:opihiexarata.propagate.solution.PropagativeSolution.raw_ra_acceleration}}
\pysigstartsignatures
\pysigline{\sphinxbfcode{\sphinxupquote{raw\_ra\_acceleration}}}
\pysigstopsignatures
\sphinxAtStartPar
The right ascension angular acceleration of the target, in degrees per
second squared. These values are derived straight from the data and
not the propagation engine.
\begin{quote}\begin{description}
\sphinxlineitem{Type}
\sphinxAtStartPar
float

\end{description}\end{quote}

\end{fulllineitems}\end{savenotes}

\index{raw\_dec\_acceleration (opihiexarata.propagate.solution.PropagativeSolution attribute)@\spxentry{raw\_dec\_acceleration}\spxextra{opihiexarata.propagate.solution.PropagativeSolution attribute}}

\begin{savenotes}\begin{fulllineitems}
\phantomsection\label{\detokenize{code/opihiexarata.propagate.solution:opihiexarata.propagate.solution.PropagativeSolution.raw_dec_acceleration}}
\pysigstartsignatures
\pysigline{\sphinxbfcode{\sphinxupquote{raw\_dec\_acceleration}}}
\pysigstopsignatures
\sphinxAtStartPar
The declination angular acceleration of the target, in degrees per
second squared. These values are derived straight from the data
and not the propagation engine.
\begin{quote}\begin{description}
\sphinxlineitem{Type}
\sphinxAtStartPar
float

\end{description}\end{quote}

\end{fulllineitems}\end{savenotes}

\index{ra\_velocity (opihiexarata.propagate.solution.PropagativeSolution attribute)@\spxentry{ra\_velocity}\spxextra{opihiexarata.propagate.solution.PropagativeSolution attribute}}

\begin{savenotes}\begin{fulllineitems}
\phantomsection\label{\detokenize{code/opihiexarata.propagate.solution:opihiexarata.propagate.solution.PropagativeSolution.ra_velocity}}
\pysigstartsignatures
\pysigline{\sphinxbfcode{\sphinxupquote{ra\_velocity}}}
\pysigstopsignatures
\sphinxAtStartPar
The right ascension angular velocity of the target, in degrees per
second. These values are derived from the engine.
\begin{quote}\begin{description}
\sphinxlineitem{Type}
\sphinxAtStartPar
float

\end{description}\end{quote}

\end{fulllineitems}\end{savenotes}

\index{dec\_velocity (opihiexarata.propagate.solution.PropagativeSolution attribute)@\spxentry{dec\_velocity}\spxextra{opihiexarata.propagate.solution.PropagativeSolution attribute}}

\begin{savenotes}\begin{fulllineitems}
\phantomsection\label{\detokenize{code/opihiexarata.propagate.solution:opihiexarata.propagate.solution.PropagativeSolution.dec_velocity}}
\pysigstartsignatures
\pysigline{\sphinxbfcode{\sphinxupquote{dec\_velocity}}}
\pysigstopsignatures
\sphinxAtStartPar
The declination angular velocity of the target, in degrees per
second. These values are derived from the engine.
\begin{quote}\begin{description}
\sphinxlineitem{Type}
\sphinxAtStartPar
float

\end{description}\end{quote}

\end{fulllineitems}\end{savenotes}

\index{ra\_acceleration (opihiexarata.propagate.solution.PropagativeSolution attribute)@\spxentry{ra\_acceleration}\spxextra{opihiexarata.propagate.solution.PropagativeSolution attribute}}

\begin{savenotes}\begin{fulllineitems}
\phantomsection\label{\detokenize{code/opihiexarata.propagate.solution:opihiexarata.propagate.solution.PropagativeSolution.ra_acceleration}}
\pysigstartsignatures
\pysigline{\sphinxbfcode{\sphinxupquote{ra\_acceleration}}}
\pysigstopsignatures
\sphinxAtStartPar
The right ascension angular acceleration of the target, in degrees per
second squared. These values are derived from the engine.
\begin{quote}\begin{description}
\sphinxlineitem{Type}
\sphinxAtStartPar
float

\end{description}\end{quote}

\end{fulllineitems}\end{savenotes}

\index{dec\_acceleration (opihiexarata.propagate.solution.PropagativeSolution attribute)@\spxentry{dec\_acceleration}\spxextra{opihiexarata.propagate.solution.PropagativeSolution attribute}}

\begin{savenotes}\begin{fulllineitems}
\phantomsection\label{\detokenize{code/opihiexarata.propagate.solution:opihiexarata.propagate.solution.PropagativeSolution.dec_acceleration}}
\pysigstartsignatures
\pysigline{\sphinxbfcode{\sphinxupquote{dec\_acceleration}}}
\pysigstopsignatures
\sphinxAtStartPar
The declination angular acceleration of the target, in degrees per
second squared. These values are derived from the engine.
\begin{quote}\begin{description}
\sphinxlineitem{Type}
\sphinxAtStartPar
float

\end{description}\end{quote}

\end{fulllineitems}\end{savenotes}

\index{\_\_init\_\_() (opihiexarata.propagate.solution.PropagativeSolution method)@\spxentry{\_\_init\_\_()}\spxextra{opihiexarata.propagate.solution.PropagativeSolution method}}

\begin{savenotes}\begin{fulllineitems}
\phantomsection\label{\detokenize{code/opihiexarata.propagate.solution:opihiexarata.propagate.solution.PropagativeSolution.__init__}}
\pysigstartsignatures
\pysiglinewithargsret{\sphinxbfcode{\sphinxupquote{\_\_init\_\_}}}{\emph{\DUrole{n}{ra}\DUrole{p}{:}\DUrole{w}{  }\DUrole{n}{ndarray}}, \emph{\DUrole{n}{dec}\DUrole{p}{:}\DUrole{w}{  }\DUrole{n}{ndarray}}, \emph{\DUrole{n}{obs\_time}\DUrole{p}{:}\DUrole{w}{  }\DUrole{n}{list}}, \emph{\DUrole{n}{solver\_engine}\DUrole{p}{:}\DUrole{w}{  }\DUrole{n}{{\hyperref[\detokenize{code/opihiexarata.library.engine:opihiexarata.library.engine.PropagationEngine}]{\sphinxcrossref{PropagationEngine}}}}}, \emph{\DUrole{n}{vehicle\_args}\DUrole{p}{:}\DUrole{w}{  }\DUrole{n}{dict}\DUrole{w}{  }\DUrole{o}{=}\DUrole{w}{  }\DUrole{default_value}{\{\}}}}{}
\pysigstopsignatures
\sphinxAtStartPar
The instantiation of the propagation solution.
\begin{quote}\begin{description}
\sphinxlineitem{Parameters}\begin{itemize}
\item {} 
\sphinxAtStartPar
\sphinxstyleliteralstrong{\sphinxupquote{ra}} (\sphinxstyleliteralemphasis{\sphinxupquote{array\sphinxhyphen{}like}}) – An array of right ascensions to fit and extrapolate to, must be in
degrees.

\item {} 
\sphinxAtStartPar
\sphinxstyleliteralstrong{\sphinxupquote{dec}} (\sphinxstyleliteralemphasis{\sphinxupquote{array\sphinxhyphen{}like}}) – An array of declinations to fit and extrapolate to, must be in
degrees.

\item {} 
\sphinxAtStartPar
\sphinxstyleliteralstrong{\sphinxupquote{obs\_time}} (\sphinxstyleliteralemphasis{\sphinxupquote{array\sphinxhyphen{}like}}) – An array of observation times which the RA and DEC measurements
were taken at. Must be Julian days.

\item {} 
\sphinxAtStartPar
\sphinxstyleliteralstrong{\sphinxupquote{solver\_engine}} ({\hyperref[\detokenize{code/opihiexarata.library.engine:opihiexarata.library.engine.PropagationEngine}]{\sphinxcrossref{\sphinxstyleliteralemphasis{\sphinxupquote{PropagationEngine}}}}}) – The propagation solver engine class that will be used to compute
the propagation solution.

\item {} 
\sphinxAtStartPar
\sphinxstyleliteralstrong{\sphinxupquote{vehicle\_args}} (\sphinxstyleliteralemphasis{\sphinxupquote{dictionary}}) – If the vehicle function for the provided solver engine needs
extra parameters not otherwise provided by the standard input,
they are given here.

\end{itemize}

\sphinxlineitem{Return type}
\sphinxAtStartPar
None

\end{description}\end{quote}

\end{fulllineitems}\end{savenotes}

\index{\_\_init\_compute\_propagation\_motion() (opihiexarata.propagate.solution.PropagativeSolution method)@\spxentry{\_\_init\_compute\_propagation\_motion()}\spxextra{opihiexarata.propagate.solution.PropagativeSolution method}}

\begin{savenotes}\begin{fulllineitems}
\phantomsection\label{\detokenize{code/opihiexarata.propagate.solution:opihiexarata.propagate.solution.PropagativeSolution.__init_compute_propagation_motion}}
\pysigstartsignatures
\pysiglinewithargsret{\sphinxbfcode{\sphinxupquote{\_\_init\_compute\_propagation\_motion}}}{\emph{\DUrole{n}{obs\_time\_array}\DUrole{p}{:}\DUrole{w}{  }\DUrole{n}{ndarray}}}{{ $\rightarrow$ tuple\DUrole{p}{{[}}float\DUrole{p}{,}\DUrole{w}{  }float\DUrole{p}{,}\DUrole{w}{  }float\DUrole{p}{,}\DUrole{w}{  }float\DUrole{p}{{]}}}}
\pysigstopsignatures
\sphinxAtStartPar
Compute the raw velocities and accelerations of RA and DEC.

\sphinxAtStartPar
This function prioritizes calculating the raw motion using the most
recent observations only.
\begin{quote}\begin{description}
\sphinxlineitem{Parameters}
\sphinxAtStartPar
\sphinxstyleliteralstrong{\sphinxupquote{obs\_time\_array}} (\sphinxstyleliteralemphasis{\sphinxupquote{array\sphinxhyphen{}like}}) – An array of observation times which the RA and DEC measurements
were taken at. The values are in Julian days.

\sphinxlineitem{Returns}
\sphinxAtStartPar
\begin{itemize}
\item {} 
\sphinxAtStartPar
\sphinxstylestrong{propagate\_ra\_velocity} (\sphinxstyleemphasis{float}) – The propagative right ascension angular velocity of the target, in degrees
per second.

\item {} 
\sphinxAtStartPar
\sphinxstylestrong{propagate\_dec\_velocity} (\sphinxstyleemphasis{float}) – The propagative declination angular velocity of the target, in degrees per
second.

\item {} 
\sphinxAtStartPar
\sphinxstylestrong{propagate\_ra\_acceleration} (\sphinxstyleemphasis{float}) – The propagative right ascension angular acceleration of the target, in
degrees per second squared.
propagation engine.

\item {} 
\sphinxAtStartPar
\sphinxstylestrong{propagate\_dec\_acceleration} (\sphinxstyleemphasis{float}) – The propagative declination angular acceleration of the target, in
degrees per second squared.

\end{itemize}


\end{description}\end{quote}

\end{fulllineitems}\end{savenotes}

\index{\_\_init\_compute\_raw\_motion() (opihiexarata.propagate.solution.PropagativeSolution method)@\spxentry{\_\_init\_compute\_raw\_motion()}\spxextra{opihiexarata.propagate.solution.PropagativeSolution method}}

\begin{savenotes}\begin{fulllineitems}
\phantomsection\label{\detokenize{code/opihiexarata.propagate.solution:opihiexarata.propagate.solution.PropagativeSolution.__init_compute_raw_motion}}
\pysigstartsignatures
\pysiglinewithargsret{\sphinxbfcode{\sphinxupquote{\_\_init\_compute\_raw\_motion}}}{\emph{\DUrole{n}{ra\_array}\DUrole{p}{:}\DUrole{w}{  }\DUrole{n}{ndarray}}, \emph{\DUrole{n}{dec\_array}\DUrole{p}{:}\DUrole{w}{  }\DUrole{n}{ndarray}}, \emph{\DUrole{n}{obs\_time\_array}\DUrole{p}{:}\DUrole{w}{  }\DUrole{n}{ndarray}}}{{ $\rightarrow$ tuple\DUrole{p}{{[}}float\DUrole{p}{,}\DUrole{w}{  }float\DUrole{p}{,}\DUrole{w}{  }float\DUrole{p}{,}\DUrole{w}{  }float\DUrole{p}{{]}}}}
\pysigstopsignatures
\sphinxAtStartPar
Compute the raw velocities and accelerations of RA and DEC.

\sphinxAtStartPar
This function prioritizes calculating the raw motion using the most
recent observations only.
\begin{quote}\begin{description}
\sphinxlineitem{Parameters}\begin{itemize}
\item {} 
\sphinxAtStartPar
\sphinxstyleliteralstrong{\sphinxupquote{ra\_array}} (\sphinxstyleliteralemphasis{\sphinxupquote{array\sphinxhyphen{}like}}) – The array of right ascensions used fit and extrapolate to,
in degrees.

\item {} 
\sphinxAtStartPar
\sphinxstyleliteralstrong{\sphinxupquote{dec\_array}} (\sphinxstyleliteralemphasis{\sphinxupquote{array\sphinxhyphen{}like}}) – The array of declinations used fit and extrapolate to, in degrees.

\item {} 
\sphinxAtStartPar
\sphinxstyleliteralstrong{\sphinxupquote{obs\_time\_array}} (\sphinxstyleliteralemphasis{\sphinxupquote{array\sphinxhyphen{}like}}) – An array of observation times which the RA and DEC measurements
were taken at. The values are in Julian days.

\end{itemize}

\sphinxlineitem{Returns}
\sphinxAtStartPar
\begin{itemize}
\item {} 
\sphinxAtStartPar
\sphinxstylestrong{raw\_ra\_velocity} (\sphinxstyleemphasis{float}) – The raw right ascension angular velocity of the target, in degrees
per second.

\item {} 
\sphinxAtStartPar
\sphinxstylestrong{raw\_dec\_velocity} (\sphinxstyleemphasis{float}) – The raw declination angular velocity of the target, in degrees per
second.

\item {} 
\sphinxAtStartPar
\sphinxstylestrong{raw\_ra\_acceleration} (\sphinxstyleemphasis{float}) – The raw right ascension angular acceleration of the target, in
degrees per second squared.
propagation engine.

\item {} 
\sphinxAtStartPar
\sphinxstylestrong{raw\_dec\_acceleration} (\sphinxstyleemphasis{float}) – The raw declination angular acceleration of the target, in
degrees per second squared.

\end{itemize}


\end{description}\end{quote}

\end{fulllineitems}\end{savenotes}

\index{forward\_propagate() (opihiexarata.propagate.solution.PropagativeSolution method)@\spxentry{forward\_propagate()}\spxextra{opihiexarata.propagate.solution.PropagativeSolution method}}

\begin{savenotes}\begin{fulllineitems}
\phantomsection\label{\detokenize{code/opihiexarata.propagate.solution:opihiexarata.propagate.solution.PropagativeSolution.forward_propagate}}
\pysigstartsignatures
\pysiglinewithargsret{\sphinxbfcode{\sphinxupquote{forward\_propagate}}}{\emph{\DUrole{n}{future\_time}\DUrole{p}{:}\DUrole{w}{  }\DUrole{n}{ndarray}}}{{ $\rightarrow$ tuple\DUrole{p}{{[}}numpy.ndarray\DUrole{p}{,}\DUrole{w}{  }numpy.ndarray\DUrole{p}{{]}}}}
\pysigstopsignatures
\sphinxAtStartPar
A wrapper call around the engine’s propagation function. This
allows the computation of future positions at a future time using
propagation.
\begin{quote}\begin{description}
\sphinxlineitem{Parameters}
\sphinxAtStartPar
\sphinxstyleliteralstrong{\sphinxupquote{future\_time}} (\sphinxstyleliteralemphasis{\sphinxupquote{array\sphinxhyphen{}like}}) – The set of future times which to derive new RA and DEC coordinates.
The time must be in Julian days.

\sphinxlineitem{Returns}
\sphinxAtStartPar
\begin{itemize}
\item {} 
\sphinxAtStartPar
\sphinxstylestrong{future\_ra} (\sphinxstyleemphasis{ndarray}) – The set of right ascensions that corresponds to the future times,
in degrees.

\item {} 
\sphinxAtStartPar
\sphinxstylestrong{future\_dec} (\sphinxstyleemphasis{ndarray}) – The set of declinations that corresponds to the future times, in
degrees.

\end{itemize}


\end{description}\end{quote}

\end{fulllineitems}\end{savenotes}


\end{fulllineitems}\end{savenotes}

\index{\_vehicle\_linear\_propagation() (in module opihiexarata.propagate.solution)@\spxentry{\_vehicle\_linear\_propagation()}\spxextra{in module opihiexarata.propagate.solution}}

\begin{savenotes}\begin{fulllineitems}
\phantomsection\label{\detokenize{code/opihiexarata.propagate.solution:opihiexarata.propagate.solution._vehicle_linear_propagation}}
\pysigstartsignatures
\pysiglinewithargsret{\sphinxcode{\sphinxupquote{opihiexarata.propagate.solution.}}\sphinxbfcode{\sphinxupquote{\_vehicle\_linear\_propagation}}}{\emph{\DUrole{n}{ra\_array}\DUrole{p}{:}\DUrole{w}{  }\DUrole{n}{ndarray}}, \emph{\DUrole{n}{dec\_array}\DUrole{p}{:}\DUrole{w}{  }\DUrole{n}{ndarray}}, \emph{\DUrole{n}{obs\_time\_array}\DUrole{p}{:}\DUrole{w}{  }\DUrole{n}{ndarray}}}{{ $\rightarrow$ dict}}
\pysigstopsignatures
\sphinxAtStartPar
Derive the propagation from 1st order polynomial extrapolation methods.
\begin{quote}\begin{description}
\sphinxlineitem{Parameters}\begin{itemize}
\item {} 
\sphinxAtStartPar
\sphinxstyleliteralstrong{\sphinxupquote{ra\_array}} (\sphinxstyleliteralemphasis{\sphinxupquote{array\sphinxhyphen{}like}}) – The array of right ascensions used fit and extrapolate to,
in degrees.

\item {} 
\sphinxAtStartPar
\sphinxstyleliteralstrong{\sphinxupquote{dec\_array}} (\sphinxstyleliteralemphasis{\sphinxupquote{array\sphinxhyphen{}like}}) – The array of declinations used fit and extrapolate to, in degrees.

\item {} 
\sphinxAtStartPar
\sphinxstyleliteralstrong{\sphinxupquote{obs\_time\_array}} (\sphinxstyleliteralemphasis{\sphinxupquote{array\sphinxhyphen{}like}}) – An array of observation times which the RA and DEC measurements
were taken at. The values are in Julian days.

\end{itemize}

\sphinxlineitem{Returns}
\sphinxAtStartPar
\sphinxstylestrong{solution\_results} – The results of the propagation engine which then gets integrated into
the solution.

\sphinxlineitem{Return type}
\sphinxAtStartPar
dictionary

\end{description}\end{quote}

\end{fulllineitems}\end{savenotes}

\index{\_vehicle\_quadratic\_propagation() (in module opihiexarata.propagate.solution)@\spxentry{\_vehicle\_quadratic\_propagation()}\spxextra{in module opihiexarata.propagate.solution}}

\begin{savenotes}\begin{fulllineitems}
\phantomsection\label{\detokenize{code/opihiexarata.propagate.solution:opihiexarata.propagate.solution._vehicle_quadratic_propagation}}
\pysigstartsignatures
\pysiglinewithargsret{\sphinxcode{\sphinxupquote{opihiexarata.propagate.solution.}}\sphinxbfcode{\sphinxupquote{\_vehicle\_quadratic\_propagation}}}{\emph{\DUrole{n}{ra\_array}\DUrole{p}{:}\DUrole{w}{  }\DUrole{n}{ndarray}}, \emph{\DUrole{n}{dec\_array}\DUrole{p}{:}\DUrole{w}{  }\DUrole{n}{ndarray}}, \emph{\DUrole{n}{obs\_time\_array}\DUrole{p}{:}\DUrole{w}{  }\DUrole{n}{ndarray}}}{{ $\rightarrow$ dict}}
\pysigstopsignatures
\sphinxAtStartPar
Derive the propagation from 2nd order polynomial extrapolation methods.
\begin{quote}\begin{description}
\sphinxlineitem{Parameters}\begin{itemize}
\item {} 
\sphinxAtStartPar
\sphinxstyleliteralstrong{\sphinxupquote{ra\_array}} (\sphinxstyleliteralemphasis{\sphinxupquote{array\sphinxhyphen{}like}}) – The array of right ascensions used fit and extrapolate to,
in degrees.

\item {} 
\sphinxAtStartPar
\sphinxstyleliteralstrong{\sphinxupquote{dec\_array}} (\sphinxstyleliteralemphasis{\sphinxupquote{array\sphinxhyphen{}like}}) – The array of declinations used fit and extrapolate to, in degrees.

\item {} 
\sphinxAtStartPar
\sphinxstyleliteralstrong{\sphinxupquote{obs\_time\_array}} (\sphinxstyleliteralemphasis{\sphinxupquote{array\sphinxhyphen{}like}}) – An array of observation times which the RA and DEC measurements
were taken at. The values are in Julian days.

\end{itemize}

\sphinxlineitem{Returns}
\sphinxAtStartPar
\sphinxstylestrong{solution\_results} – The results of the propagation engine which then gets integrated into
the solution.

\sphinxlineitem{Return type}
\sphinxAtStartPar
dictionary

\end{description}\end{quote}

\end{fulllineitems}\end{savenotes}



\subparagraph{Module contents}
\label{\detokenize{code/opihiexarata.propagate:module-opihiexarata.propagate}}\label{\detokenize{code/opihiexarata.propagate:module-contents}}\index{module@\spxentry{module}!opihiexarata.propagate@\spxentry{opihiexarata.propagate}}\index{opihiexarata.propagate@\spxentry{opihiexarata.propagate}!module@\spxentry{module}}
\sphinxAtStartPar
Solution methods and engines for asteroid propagation.


\subsubsection{Submodules}
\label{\detokenize{code/opihiexarata:submodules}}
\sphinxstepscope


\paragraph{opihiexarata.\_\_main\_\_ module}
\label{\detokenize{code/opihiexarata.__main__:module-opihiexarata.__main__}}\label{\detokenize{code/opihiexarata.__main__:opihiexarata-main-module}}\label{\detokenize{code/opihiexarata.__main__::doc}}\index{module@\spxentry{module}!opihiexarata.\_\_main\_\_@\spxentry{opihiexarata.\_\_main\_\_}}\index{opihiexarata.\_\_main\_\_@\spxentry{opihiexarata.\_\_main\_\_}!module@\spxentry{module}}
\sphinxAtStartPar
Just a small hook for the main execution. This section parses arguments
which is then passed to execution to do exactly as expected by the commands.

\sphinxAtStartPar
The actual execution is done in the command.py file so that this file
does not get too large.
\index{\_\_main\_execute\_arguments() (in module opihiexarata.\_\_main\_\_)@\spxentry{\_\_main\_execute\_arguments()}\spxextra{in module opihiexarata.\_\_main\_\_}}

\begin{savenotes}\begin{fulllineitems}
\phantomsection\label{\detokenize{code/opihiexarata.__main__:opihiexarata.__main__.__main_execute_arguments}}
\pysigstartsignatures
\pysiglinewithargsret{\sphinxcode{\sphinxupquote{opihiexarata.\_\_main\_\_.}}\sphinxbfcode{\sphinxupquote{\_\_main\_execute\_arguments}}}{\emph{\DUrole{n}{parser}\DUrole{p}{:}\DUrole{w}{  }\DUrole{n}{ArgumentParser}}, \emph{\DUrole{n}{arguments}\DUrole{p}{:}\DUrole{w}{  }\DUrole{n}{Namespace}}}{{ $\rightarrow$ None}}
\pysigstopsignatures
\sphinxAtStartPar
We actually execute the software using the arguments provided in the
\# command line call. GUI’s are started on separate threads
\begin{quote}\begin{description}
\sphinxlineitem{Parameters}
\sphinxAtStartPar
\sphinxstyleliteralstrong{\sphinxupquote{arguments}} (\sphinxstyleliteralemphasis{\sphinxupquote{dict}}) – The parsed arguments from which the interpreted action will use. Note
though that these arguments also has the interpreted actions.

\sphinxlineitem{Return type}
\sphinxAtStartPar
None

\end{description}\end{quote}

\end{fulllineitems}\end{savenotes}

\index{\_\_main\_parse\_arguments() (in module opihiexarata.\_\_main\_\_)@\spxentry{\_\_main\_parse\_arguments()}\spxextra{in module opihiexarata.\_\_main\_\_}}

\begin{savenotes}\begin{fulllineitems}
\phantomsection\label{\detokenize{code/opihiexarata.__main__:opihiexarata.__main__.__main_parse_arguments}}
\pysigstartsignatures
\pysiglinewithargsret{\sphinxcode{\sphinxupquote{opihiexarata.\_\_main\_\_.}}\sphinxbfcode{\sphinxupquote{\_\_main\_parse\_arguments}}}{}{{ $\rightarrow$ tuple\DUrole{p}{{[}}argparse.ArgumentParser\DUrole{p}{,}\DUrole{w}{  }argparse.Namespace\DUrole{p}{{]}}}}
\pysigstopsignatures
\sphinxAtStartPar
The main section for argument parsing. We just have it here to better
organize things.
\begin{quote}\begin{description}
\sphinxlineitem{Parameters}
\sphinxAtStartPar
\sphinxstyleliteralstrong{\sphinxupquote{None}} – 

\sphinxlineitem{Returns}
\sphinxAtStartPar
\begin{itemize}
\item {} 
\sphinxAtStartPar
\sphinxstylestrong{parser} (\sphinxstyleemphasis{ArgumentParser}) – The parser itself. This may not be needed for a lot of things, but
it is still helpful for command processing.

\item {} 
\sphinxAtStartPar
\sphinxstylestrong{parsed\_arguments} (\sphinxstyleemphasis{Namespace}) – The arguments as parsed by the parser. Though technically it is a
Namespace class, it is practically a dictionary.

\end{itemize}


\end{description}\end{quote}

\end{fulllineitems}\end{savenotes}

\index{main() (in module opihiexarata.\_\_main\_\_)@\spxentry{main()}\spxextra{in module opihiexarata.\_\_main\_\_}}

\begin{savenotes}\begin{fulllineitems}
\phantomsection\label{\detokenize{code/opihiexarata.__main__:opihiexarata.__main__.main}}
\pysigstartsignatures
\pysiglinewithargsret{\sphinxcode{\sphinxupquote{opihiexarata.\_\_main\_\_.}}\sphinxbfcode{\sphinxupquote{main}}}{}{{ $\rightarrow$ None}}
\pysigstopsignatures
\sphinxAtStartPar
The main command for argument parsing and figuring out what to do based
on command\sphinxhyphen{}line entries.
\begin{quote}\begin{description}
\sphinxlineitem{Parameters}
\sphinxAtStartPar
\sphinxstyleliteralstrong{\sphinxupquote{None}} – 

\sphinxlineitem{Return type}
\sphinxAtStartPar
None

\end{description}\end{quote}

\end{fulllineitems}\end{savenotes}



\subsubsection{Module contents}
\label{\detokenize{code/opihiexarata:module-opihiexarata}}\label{\detokenize{code/opihiexarata:module-contents}}\index{module@\spxentry{module}!opihiexarata@\spxentry{opihiexarata}}\index{opihiexarata@\spxentry{opihiexarata}!module@\spxentry{module}}
\sphinxAtStartPar
All of the subparts of the OpihiExarata software.
\begin{itemize}
\item {} 
\sphinxAtStartPar
\DUrole{xref,std,std-ref}{genindex}

\item {} 
\sphinxAtStartPar
\DUrole{xref,std,std-ref}{modindex}

\item {} 
\sphinxAtStartPar
\DUrole{xref,std,std-ref}{search}

\end{itemize}


\renewcommand{\indexname}{Python Module Index}
\begin{sphinxtheindex}
\let\bigletter\sphinxstyleindexlettergroup
\bigletter{o}
\item\relax\sphinxstyleindexentry{opihiexarata}\sphinxstyleindexpageref{code/opihiexarata:\detokenize{module-opihiexarata}}
\item\relax\sphinxstyleindexentry{opihiexarata.\_\_main\_\_}\sphinxstyleindexpageref{code/opihiexarata.__main__:\detokenize{module-opihiexarata.__main__}}
\item\relax\sphinxstyleindexentry{opihiexarata.astrometry}\sphinxstyleindexpageref{code/opihiexarata.astrometry:\detokenize{module-opihiexarata.astrometry}}
\item\relax\sphinxstyleindexentry{opihiexarata.astrometry.solution}\sphinxstyleindexpageref{code/opihiexarata.astrometry.solution:\detokenize{module-opihiexarata.astrometry.solution}}
\item\relax\sphinxstyleindexentry{opihiexarata.astrometry.webclient}\sphinxstyleindexpageref{code/opihiexarata.astrometry.webclient:\detokenize{module-opihiexarata.astrometry.webclient}}
\item\relax\sphinxstyleindexentry{opihiexarata.ephemeris}\sphinxstyleindexpageref{code/opihiexarata.ephemeris:\detokenize{module-opihiexarata.ephemeris}}
\item\relax\sphinxstyleindexentry{opihiexarata.ephemeris.jplhorizons}\sphinxstyleindexpageref{code/opihiexarata.ephemeris.jplhorizons:\detokenize{module-opihiexarata.ephemeris.jplhorizons}}
\item\relax\sphinxstyleindexentry{opihiexarata.ephemeris.solution}\sphinxstyleindexpageref{code/opihiexarata.ephemeris.solution:\detokenize{module-opihiexarata.ephemeris.solution}}
\item\relax\sphinxstyleindexentry{opihiexarata.gui}\sphinxstyleindexpageref{code/opihiexarata.gui:\detokenize{module-opihiexarata.gui}}
\item\relax\sphinxstyleindexentry{opihiexarata.gui.automatic}\sphinxstyleindexpageref{code/opihiexarata.gui.automatic:\detokenize{module-opihiexarata.gui.automatic}}
\item\relax\sphinxstyleindexentry{opihiexarata.gui.functions}\sphinxstyleindexpageref{code/opihiexarata.gui.functions:\detokenize{module-opihiexarata.gui.functions}}
\item\relax\sphinxstyleindexentry{opihiexarata.gui.manual}\sphinxstyleindexpageref{code/opihiexarata.gui.manual:\detokenize{module-opihiexarata.gui.manual}}
\item\relax\sphinxstyleindexentry{opihiexarata.gui.name}\sphinxstyleindexpageref{code/opihiexarata.gui.name:\detokenize{module-opihiexarata.gui.name}}
\item\relax\sphinxstyleindexentry{opihiexarata.gui.qtui}\sphinxstyleindexpageref{code/opihiexarata.gui.qtui:\detokenize{module-opihiexarata.gui.qtui}}
\item\relax\sphinxstyleindexentry{opihiexarata.gui.qtui.qtui\_automatic}\sphinxstyleindexpageref{code/opihiexarata.gui.qtui.qtui_automatic:\detokenize{module-opihiexarata.gui.qtui.qtui_automatic}}
\item\relax\sphinxstyleindexentry{opihiexarata.gui.qtui.qtui\_manual}\sphinxstyleindexpageref{code/opihiexarata.gui.qtui.qtui_manual:\detokenize{module-opihiexarata.gui.qtui.qtui_manual}}
\item\relax\sphinxstyleindexentry{opihiexarata.gui.qtui.qtui\_selector}\sphinxstyleindexpageref{code/opihiexarata.gui.qtui.qtui_selector:\detokenize{module-opihiexarata.gui.qtui.qtui_selector}}
\item\relax\sphinxstyleindexentry{opihiexarata.gui.selector}\sphinxstyleindexpageref{code/opihiexarata.gui.selector:\detokenize{module-opihiexarata.gui.selector}}
\item\relax\sphinxstyleindexentry{opihiexarata.library}\sphinxstyleindexpageref{code/opihiexarata.library:\detokenize{module-opihiexarata.library}}
\item\relax\sphinxstyleindexentry{opihiexarata.library.config}\sphinxstyleindexpageref{code/opihiexarata.library.config:\detokenize{module-opihiexarata.library.config}}
\item\relax\sphinxstyleindexentry{opihiexarata.library.conversion}\sphinxstyleindexpageref{code/opihiexarata.library.conversion:\detokenize{module-opihiexarata.library.conversion}}
\item\relax\sphinxstyleindexentry{opihiexarata.library.engine}\sphinxstyleindexpageref{code/opihiexarata.library.engine:\detokenize{module-opihiexarata.library.engine}}
\item\relax\sphinxstyleindexentry{opihiexarata.library.error}\sphinxstyleindexpageref{code/opihiexarata.library.error:\detokenize{module-opihiexarata.library.error}}
\item\relax\sphinxstyleindexentry{opihiexarata.library.fits}\sphinxstyleindexpageref{code/opihiexarata.library.fits:\detokenize{module-opihiexarata.library.fits}}
\item\relax\sphinxstyleindexentry{opihiexarata.library.hint}\sphinxstyleindexpageref{code/opihiexarata.library.hint:\detokenize{module-opihiexarata.library.hint}}
\item\relax\sphinxstyleindexentry{opihiexarata.library.http}\sphinxstyleindexpageref{code/opihiexarata.library.http:\detokenize{module-opihiexarata.library.http}}
\item\relax\sphinxstyleindexentry{opihiexarata.library.image}\sphinxstyleindexpageref{code/opihiexarata.library.image:\detokenize{module-opihiexarata.library.image}}
\item\relax\sphinxstyleindexentry{opihiexarata.library.json}\sphinxstyleindexpageref{code/opihiexarata.library.json:\detokenize{module-opihiexarata.library.json}}
\item\relax\sphinxstyleindexentry{opihiexarata.library.mpcrecord}\sphinxstyleindexpageref{code/opihiexarata.library.mpcrecord:\detokenize{module-opihiexarata.library.mpcrecord}}
\item\relax\sphinxstyleindexentry{opihiexarata.library.path}\sphinxstyleindexpageref{code/opihiexarata.library.path:\detokenize{module-opihiexarata.library.path}}
\item\relax\sphinxstyleindexentry{opihiexarata.library.phototable}\sphinxstyleindexpageref{code/opihiexarata.library.phototable:\detokenize{module-opihiexarata.library.phototable}}
\item\relax\sphinxstyleindexentry{opihiexarata.library.temporary}\sphinxstyleindexpageref{code/opihiexarata.library.temporary:\detokenize{module-opihiexarata.library.temporary}}
\item\relax\sphinxstyleindexentry{opihiexarata.opihi}\sphinxstyleindexpageref{code/opihiexarata.opihi:\detokenize{module-opihiexarata.opihi}}
\item\relax\sphinxstyleindexentry{opihiexarata.opihi.preprocess}\sphinxstyleindexpageref{code/opihiexarata.opihi.preprocess:\detokenize{module-opihiexarata.opihi.preprocess}}
\item\relax\sphinxstyleindexentry{opihiexarata.opihi.solution}\sphinxstyleindexpageref{code/opihiexarata.opihi.solution:\detokenize{module-opihiexarata.opihi.solution}}
\item\relax\sphinxstyleindexentry{opihiexarata.orbit}\sphinxstyleindexpageref{code/opihiexarata.orbit:\detokenize{module-opihiexarata.orbit}}
\item\relax\sphinxstyleindexentry{opihiexarata.orbit.custom}\sphinxstyleindexpageref{code/opihiexarata.orbit.custom:\detokenize{module-opihiexarata.orbit.custom}}
\item\relax\sphinxstyleindexentry{opihiexarata.orbit.orbfit}\sphinxstyleindexpageref{code/opihiexarata.orbit.orbfit:\detokenize{module-opihiexarata.orbit.orbfit}}
\item\relax\sphinxstyleindexentry{opihiexarata.orbit.solution}\sphinxstyleindexpageref{code/opihiexarata.orbit.solution:\detokenize{module-opihiexarata.orbit.solution}}
\item\relax\sphinxstyleindexentry{opihiexarata.photometry}\sphinxstyleindexpageref{code/opihiexarata.photometry:\detokenize{module-opihiexarata.photometry}}
\item\relax\sphinxstyleindexentry{opihiexarata.photometry.panstarrs}\sphinxstyleindexpageref{code/opihiexarata.photometry.panstarrs:\detokenize{module-opihiexarata.photometry.panstarrs}}
\item\relax\sphinxstyleindexentry{opihiexarata.photometry.solution}\sphinxstyleindexpageref{code/opihiexarata.photometry.solution:\detokenize{module-opihiexarata.photometry.solution}}
\item\relax\sphinxstyleindexentry{opihiexarata.propagate}\sphinxstyleindexpageref{code/opihiexarata.propagate:\detokenize{module-opihiexarata.propagate}}
\item\relax\sphinxstyleindexentry{opihiexarata.propagate.polynomial}\sphinxstyleindexpageref{code/opihiexarata.propagate.polynomial:\detokenize{module-opihiexarata.propagate.polynomial}}
\item\relax\sphinxstyleindexentry{opihiexarata.propagate.solution}\sphinxstyleindexpageref{code/opihiexarata.propagate.solution:\detokenize{module-opihiexarata.propagate.solution}}
\end{sphinxtheindex}

\renewcommand{\indexname}{Index}
\printindex
\end{document}